\documentclass[UTF8,titlepage,oneside]{ctexbook}
\usepackage[hidelinks]{hyperref}
% \usepackage{fancyhdr}
\usepackage[toc]{multitoc}

\usepackage[a4paper,top=2.54cm,bottom=2.54cm,left=3.18cm,right=3.18cm]{geometry}

\pagestyle{plain}
\title{\Huge \textbf{文言文合集}}
\date{}
% \widowpenalty=10000
% \clubpenalty=10000

\begin{document}
\maketitle

\tableofcontents

\newpage



\chapter*{出师表}
\addcontentsline{toc}{chapter}{出师表}
\begin{center}
	\textbf{[三国]诸葛亮}
\end{center}


臣亮言:先帝创业未半而中道崩殂,今天下三分,益州疲弊,此诚危急存亡之秋也。然侍卫之臣不懈于内,忠志之士忘身于外者,盖追先帝之殊遇,欲报之于陛下也。诚宜开张圣听,以光先帝遗德,恢弘志士之气,不宜妄自菲薄,引喻失义,以塞忠谏之路也。


宫中府中,俱为一体;陟罚臧否,不宜异同:若有作奸犯科及为忠善者,宜付有司论其刑赏,以昭陛下平明之理;不宜偏私,使内外异法也。


侍中、侍郎郭攸之、费祎、董允等,此皆良实,志虑忠纯,是以先帝简拔以遗陛下:愚以为宫中之事,事无大小,悉以咨之,然后施行,必能裨补阙漏,有所广益。


将军向宠,性行淑均,晓畅军事,试用于昔日,先帝称之曰“能”,是以众议举宠为督:愚以为营中之事,悉以咨之,必能使行阵和睦,优劣得所。


亲贤臣,远小人,此先汉所以兴隆也;亲小人,远贤臣,此后汉所以倾颓也。先帝在时,每与臣论此事,未尝不叹息痛恨于桓、灵也。侍中、尚书、长史、参军,此悉贞良死节之臣,愿陛下亲之信之,则汉室之隆,可计日而待也。


臣本布衣,躬耕于南阳,苟全性命于乱世,不求闻达于诸侯。先帝不以臣卑鄙,猥自枉屈,三顾臣于草庐之中,咨臣以当世之事,由是感激,遂许先帝以驱驰。后值倾覆,受任于败军之际,奉命于危难之间:尔来二十有一年矣。


先帝知臣谨慎,故临崩寄臣以大事也。受命以来,夙夜忧叹,恐托付不效,以伤先帝之明;故五月渡泸,深入不毛。今南方已定,兵甲已足,当奖率三军,北定中原,庶竭驽钝,攘除奸凶,兴复汉室,还于旧都。此臣所以报先帝而忠陛下之职分也。至于斟酌损益,进尽忠言,则攸之、祎、允之任也。


愿陛下托臣以讨贼兴复之效,不效,则治臣之罪,以告先帝之灵。若无兴德之言,则责攸之、祎、允等之慢,以彰其咎;陛下亦宜自谋,以咨诹善道,察纳雅言,深追先帝遗诏。臣不胜受恩感激。


今当远离,临表涕零,不知所言。


\chapter*{岳阳楼记}
\addcontentsline{toc}{chapter}{岳阳楼记}
\begin{center}
	\textbf{[宋朝]范仲淹}
\end{center}

庆历四年春,滕子京谪守巴陵郡。越明年,政通人和,百废具兴。乃重修岳阳楼,增其旧制,刻唐贤今人诗赋于其上。属予作文以记之。


予观夫巴陵胜状,在洞庭一湖。衔远山,吞长江,浩浩汤汤,横无际涯;朝晖夕阴,气象万千。此则岳阳楼之大观也,前人之述备矣。然则北通巫峡,南极潇湘,迁客骚人,多会于此,览物之情,得无异乎?


若夫淫雨霏霏,连月不开,阴风怒号,浊浪排空;日星隐曜,山岳潜形;商旅不行,樯倾楫摧;薄暮冥冥,虎啸猿啼。登斯楼也,则有去国怀乡,忧谗畏讥,满目萧然,感极而悲者矣。


至若春和景明,波澜不惊,上下天光,一碧万顷;沙鸥翔集,锦鳞游泳;岸芷汀兰,郁郁青青。而或长烟一空,皓月千里,浮光跃金,静影沉璧,渔歌互答,此乐何极!登斯楼也,则有心旷神怡,宠辱偕忘,把酒临风,其喜洋洋者矣。


嗟夫!予尝求古仁人之心,或异二者之为。何哉?不以物喜,不以己悲;居庙堂之高则忧其民;处江湖之远则忧其君。是进亦忧,退亦忧。然则何时而乐耶?其必曰:“先天下之忧而忧,后天下之乐而乐”乎。噫!微斯人,吾谁与归?


时六年九月十五日。



\chapter*{逍遥游}
\addcontentsline{toc}{chapter}{逍遥游}
\begin{center}
	\textbf{[春秋战国]庄周}
\end{center}

北冥有鱼,其名为鲲。鲲之大,不知其几千里也。化而为鸟,其名为鹏。鹏之背,不知其几千里也,怒而飞,其翼若垂天之云。是鸟也,海运则将徙于南冥。南冥者,天池也。《齐谐》者,志怪者也。《谐》之言曰:“鹏之徙于南冥也,水击三千里,抟扶摇而上者九万里,去以六月息者也。”野马也,尘埃也,生物之以息相吹也。天之苍苍,其正色邪?其远而无所至极邪?其视下也,亦若是则已矣。且夫水之积也不厚,则其负大舟也无力。覆杯水于坳堂之上,则芥为之舟;置杯焉则胶,水浅而舟大也。风之积也不厚,则其负大翼也无力。故九万里,则风斯在下矣,而后乃今培风;背负青天而莫之夭阏者,而后乃今将图南。

蜩与学鸠笑之曰:“我决起而飞,抢榆枋而止,时则不至,而控于地而已矣,奚以之九万里而南为?”适莽苍者,三餐而反,腹犹果然;适百里者宿舂粮,适千里者,三月聚粮。之二虫又何知?(抢榆枋一作:枪榆枋)

小知不及大知,小年不及大年。奚以知其然也?朝菌不知晦朔,蟪蛄不知春秋,此小年也。楚之南有冥灵者,以五百岁为春,五百岁为秋。上古有大椿者,以八千岁为春,八千岁为秋。此大年也。而彭祖乃今以久特闻,众人匹之。不亦悲乎!

汤之问棘也是已:“穷发之北有冥海者,天池也。有鱼焉,其广数千里,未有知其修者,其名为鲲。有鸟焉,其名为鹏。背若泰山,翼若垂天之云。抟扶摇羊角而上者九万里,绝云气,负青天,然后图南,且适南冥也。斥鷃笑之曰:‘彼且奚适也?我腾跃而上,不过数仞而下,翱翔蓬蒿之间,此亦飞之至也。而彼且奚适也?’”此小大之辩也。

故夫知效一官,行比一乡,德合一君,而征一国者,其自视也亦若此矣。而宋荣子犹然笑之。且举世誉之而不加劝,举世非之而不加沮,定乎内外之分,辩乎荣辱之境,斯已矣。彼其于世,未数数然也。虽然,犹有未树也。夫列子御风而行,泠然善也。旬有五日而后反。彼于致福者,未数数然也。此虽免乎行,犹有所待者也。若夫乘天地之正,而御六气之辩,以游无穷者,彼且恶乎待哉?故曰:至人无己,神人无功,圣人无名。


\chapter*{滥竽充数}
\addcontentsline{toc}{chapter}{滥竽充数}
\begin{center}
	\textbf{[春秋战国]韩非}
\end{center}

齐宣王使人吹竽,必三百人。南郭处士请为王吹竽,宣王说之,廪食以数百人。宣王死,湣王立,好一一听之,处士逃。


\chapter*{醉翁亭记}
\addcontentsline{toc}{chapter}{醉翁亭记}
\begin{center}
	\textbf{[宋朝]欧阳修}
\end{center}

环滁、皆山也。其西南诸峰,林壑尤美。望之蔚然而深秀者,琅琊也。山行六七里,渐闻水声潺潺而泻出于两峰之间者,酿泉也。峰回路转,有亭翼然、临于泉上者,醉翁亭也。作亭者谁?山之僧智仙也。名之者谁?太守自谓也。太守与客来饮于此,饮少辄醉,而年又最高,故自号曰醉翁也。醉翁之意不在酒,在乎山水之间也。山水之乐,得之心而寓之酒也。


若夫日出而林霏开,云归而岩穴暝,晦明变化者,山间之朝暮也。野芳发而幽香,佳木秀而繁阴、,风霜高洁,水落而石出者,山间之四时也。朝而往,暮而归,四时之景不同,而乐亦无穷也。


至于负者歌于途,行者休于树,前者呼,后者应,伛偻提携,往来而不绝者,滁人游也。临溪而渔,溪深而鱼肥。酿泉为酒,泉香而酒洌;山肴野蔌,杂然而前陈者,太守宴也。宴酣之乐,非丝非竹,射者中,弈者胜,觥筹交错,起坐而喧哗者,众宾欢也。苍颜白发,颓然乎其间者,太守醉也。


已而夕阳在山,人影散乱,太守归而宾客从也。树林阴翳、,鸣声上下,游人去而禽鸟乐也。然而禽鸟知山林之乐,而不知人之乐;人知从太守游而乐,而不知太守之乐其乐也。醉能同其乐,醒能述以文者,太守也。太守谓谁?庐陵欧阳修也。



\chapter*{爱莲说}
\addcontentsline{toc}{chapter}{爱莲说}
\begin{center}
	\textbf{[宋朝]周敦颐}
\end{center}

水陆草木之花,可爱者甚蕃。晋陶渊明独爱菊。自李唐来,世人甚爱牡丹。予独爱莲之出淤泥而不染,濯清涟而不妖,中通外直,不蔓不枝,香远益清,亭亭净植,可远观而不可亵玩焉。(甚爱一作:盛爱)

予谓菊,花之隐逸者也;牡丹,花之富贵者也;莲,花之君子者也。噫!菊之爱,陶后鲜有闻。莲之爱,同予者何人?牡丹之爱,宜乎众矣。


\chapter*{兰亭集序}
\addcontentsline{toc}{chapter}{兰亭集序}
\begin{center}
	\textbf{[晋朝]王羲之}
\end{center}

永和九年,岁在癸丑,暮春之初,会于会稽山阴之兰亭,修禊事也。群贤毕至,少长咸集。此地有崇山峻岭,茂林修竹,又有清流激湍,映带左右。引以为流觞曲水,列坐其次。虽无丝竹管弦之盛,一觞一咏,亦足以畅叙幽情。


是日也,天朗气清,惠风和畅。仰观宇宙之大,俯察品类之盛,所以游目骋怀,足以极视听之娱,信可乐也。


夫人之相与,俯仰一世。或取诸怀抱,悟言一室之内;或因寄所托,放浪形骸之外。虽趣舍万殊,静躁不同,当其欣于所遇,暂得于己,快然自足,不知老之将至。及其所之既倦,情随事迁,感慨系之矣。向之所欣,俯仰之间,已为陈迹,犹不能不以之兴怀。况修短随化,终期于尽。古人云:“死生亦大矣!”岂不痛哉!


每览昔人兴感之由,若合一契,未尝不临文嗟悼,不能喻之于怀。固知一死生为虚诞,齐彭殇为妄作。后之视今,亦犹今之视昔,悲夫!故列叙时人,录其所述。虽世殊事异,所以兴怀,其致一也。后之览者,亦将有感于斯文。



\chapter*{隆中对}
\addcontentsline{toc}{chapter}{隆中对}
\begin{center}
	\textbf{[晋朝]陈寿}
\end{center}

亮躬耕陇亩,好(hào)为《梁父(fǔ)吟》。身长八尺,每自比于管仲、乐(yuè)毅,时人莫之许也。惟博陵崔州平、颍(yǐng)川徐庶元直与亮友善,谓为信然。


时先主屯新野。徐庶见先主,先主器之,谓先主曰:"诸葛孔明者,卧龙也,将军岂愿见之乎?"先主曰:“君与俱来。”庶曰:“此人可就见,不可屈致也。将军宜枉驾顾之。”


由是先主遂诣亮,凡三往,乃见。因屏(bǐng)人曰:“汉室倾颓,奸臣窃命,主上蒙尘。孤不度(duó)德量力,欲信(sheng)大义于天下;而智术浅短,遂用猖蹶(chāngjué),至于今日。然志犹(yōu)未已,君谓计将(jiàng)安出"


亮答曰:“自董卓已来,豪杰并起,跨州连郡者不可胜数。曹操比于袁绍,则名微而众寡。然操遂能克绍,以弱为强者,非惟天时,抑亦人谋也。今操已拥百万之众,挟天子而令诸侯,此诚不可与争锋。孙权据有江东,已历三世,国险而民附,贤能为之用,此可以为援而不可图也。荆州北据汉、沔(miǎn),利尽南海,东连吴会(kuài),西通巴蜀,此用武之国,而其主不能守,此殆天所以资将军,将军岂有意乎?益州险塞,沃野千里,天府之土,高祖因之以成帝业。刘璋暗弱,张鲁在北,民殷国富而不知存恤,智能之士思得明君。将军既帝室之胄[zhòu],信义著于四海,总揽英雄,思贤如渴,若跨有荆、益,保其岩阻,西和诸戎,南抚夷越,外结好孙权,内修政理;天下有变,则命一上将(jiàng)将(jiāng)荆州之军以向宛、洛,将军身率益州之众出于秦川,百姓孰(shū)敢不箪(dān)食壶浆,以迎将军者乎?诚如是,则霸业可成,汉室可兴矣。”


先主曰:“善!”于是与亮情好日密。关羽、张飞等不悦,先主解之曰:“孤之有孔明,犹鱼之有水也。愿诸君勿复言。”羽、飞乃止。



\chapter*{核舟记}
\addcontentsline{toc}{chapter}{核舟记}
\begin{center}
	\textbf{[明朝]魏学洢}
\end{center}

明有奇巧人曰王叔远,能以径寸之木,为宫室、器皿、人物,以至鸟兽、木石,罔不因势象形,各具情态。尝贻余核舟一,盖大苏泛赤壁云。


舟首尾长约八分有奇,高可二黍许。中轩敞者为舱,箬篷覆之。旁开小窗,左右各四,共八扇。启窗而观,雕栏相望焉。闭之,则右刻“山高月小,水落石出”,左刻“清风徐来,水波不兴”,石青糁之。


船头坐三人,中峨冠而多髯者为东坡,佛印居右,鲁直居左。苏、黄共阅一手卷。东坡右手执卷端,左手抚鲁直背。鲁直左手执卷末,右手指卷,如有所语。东坡现右足,鲁直现左足,各微侧,其两膝相比者,各隐卷底衣褶中。佛印绝类弥勒,袒胸露乳,矫首昂视,神情与苏黄不属。卧右膝,诎右臂支船,而竖其左膝,左臂挂念珠倚之,珠可历历数也。


舟尾横卧一楫。楫左右舟子各一人。居右者椎髻仰面,左手倚一衡木,右手攀右趾,若啸呼状。居左者右手执蒲葵扇,左手抚炉,炉上有壶,其人视端容寂,若听茶声然。


其船背稍夷,则题名其上,文曰“天启壬戌秋日,虞山王毅叔远甫刻”,细若蚊足,钩画了了,其色墨。又用篆章一,文曰“初平山人”,其色丹。


通计一舟,为人五,为窗八,为篛篷,为楫,为炉,为壶,为手卷,为念珠各一;对联、题名并篆文,为字共三十有四。而计其长,曾不盈寸。盖简桃核修狭者为之。


魏子详瞩既毕,诧曰:嘻,技亦灵怪矣哉!《庄》、《列》所载,称惊犹鬼神者良多,然谁有游削于不寸之质,而须麋瞭然者?假有人焉,举我言以复于我,亦必疑其诳。乃今亲睹之。繇斯以观,棘刺之端,未必不可为母猴也。嘻,技亦灵怪矣哉!



\chapter*{送东阳马生序}
\addcontentsline{toc}{chapter}{送东阳马生序}
\begin{center}
	\textbf{[明朝]宋濂}
\end{center}

余幼时即嗜学。家贫,无从致书以观,每假借于藏书之家,手自笔录,计日以还。天大寒,砚冰坚,手指不可屈伸,弗之怠。录毕,走送之,不敢稍逾约。以是人多以书假余,余因得遍观群书。既加冠,益慕圣贤之道,又患无硕师、名人与游,尝趋百里外,从乡之先达执经叩问。先达德隆望尊,门人弟子填其室,未尝稍降辞色。余立侍左右,援疑质理,俯身倾耳以请;或遇其叱咄,色愈恭,礼愈至,不敢出一言以复;俟其欣悦,则又请焉。故余虽愚,卒获有所闻。


当余之从师也,负箧曳屣,行深山巨谷中,穷冬烈风,大雪深数尺,足肤皲裂而不知。至舍,四支僵劲不能动,媵人持汤沃灌,以衾拥覆,久而乃和。寓逆旅,主人日再食,无鲜肥滋味之享。同舍生皆被绮绣,戴朱缨宝饰之帽,腰白玉之环,左佩刀,右备容臭,烨然若神人;余则缊袍敝衣处其间,略无慕艳意。以中有足乐者,不知口体之奉不若人也。盖余之勤且艰若此。


今虽耄老,未有所成,犹幸预君子之列,而承天子之宠光,缀公卿之后,日侍坐备顾问,四海亦谬称其氏名,况才之过于余者乎?


今诸生学于太学,县官日有廪稍之供,父母岁有裘葛之遗,无冻馁之患矣;坐大厦之下而诵《诗》《书》,无奔走之劳矣;有司业、博士为之师,未有问而不告,求而不得者也;凡所宜有之书,皆集于此,不必若余之手录,假诸人而后见也。其业有不精,德有不成者,非天质之卑,则心不若余之专耳,岂他人之过哉!


东阳马生君则,在太学已二年,流辈甚称其贤。余朝京师,生以乡人子谒余,撰长书以为贽,辞甚畅达,与之论辩,言和而色夷。自谓少时用心于学甚劳,是可谓善学者矣!其将归见其亲也,余故道为学之难以告之。谓余勉乡人以学者,余之志也;诋我夸际遇之盛而骄乡人者,岂知余者哉!



\chapter*{记承天寺夜游}
\addcontentsline{toc}{chapter}{记承天寺夜游}
\begin{center}
	\textbf{[宋朝]苏轼}
\end{center}

元丰六年十月十二日夜,解衣欲睡,月色入户,欣然起行。念无与为乐者,遂至承天寺寻张怀民。怀民亦未寝,相与步于中庭。

庭下如积水空明,水中藻荇交横,盖竹柏影也。何夜无月?何处无竹柏?但少闲人如吾两人者耳。


\chapter*{湖心亭看雪}
\addcontentsline{toc}{chapter}{湖心亭看雪}
\begin{center}
	\textbf{[明朝]张岱}
\end{center}

崇祯五年十二月,余住西湖。大雪三日,湖中人鸟声俱绝。是日更定矣,余拏一小舟,拥毳衣炉火,独往湖心亭看雪。雾凇沆砀,天与云与山与水,上下一白。湖上影子,惟长堤一痕、湖心亭一点、与余舟一芥,舟中人两三粒而已。

到亭上,有两人铺毡对坐,一童子烧酒,炉正沸。见余,大喜曰:“湖中焉得更有此人?”拉余同饮。余强饮三大白而别。问其姓氏,是金陵人,客此。及下船,舟子喃喃曰:“莫说相公痴,更有痴似相公者。”


\chapter*{孟母三迁}
\addcontentsline{toc}{chapter}{孟母三迁}
\begin{center}
	\textbf{[汉朝]刘向}
\end{center}

昔孟子少时,父早丧,母仉氏守节。居住之所近于墓,孟子学为丧葬,躄[bì],踊痛哭之事。母曰:“此非所以处子也。”乃去,遂迁居市旁,孟子又嬉为贾人炫卖之事,母曰:“此又非所以处子也。”舍市,近于屠,学为买卖屠杀之事。母又曰:“是亦非所以处子矣。”继而迁于学宫之旁。每月朔望,官员入文庙,行礼跪拜,揖让进退,孟子见了,一一习记。孟母曰:“此真可以处子也。”遂居于此。



\chapter*{桃花源记}
\addcontentsline{toc}{chapter}{桃花源记}
\begin{center}
	\textbf{[晋朝]陶渊明}
\end{center}

晋太元中,武陵人捕鱼为业。缘溪行,忘路之远近。忽逢桃花林,夹岸数百步,中无杂树,芳草鲜美,落英缤纷,渔人甚异之。复前行,欲穷其林。


林尽水源,便得一山,山有小口,仿佛若有光。便舍船,从口入。初极狭,才通人。复行数十步,豁然开朗。土地平旷,屋舍俨然,有良田美池桑竹之属。阡陌交通,鸡犬相闻。其中往来种作,男女衣着,悉如外人。黄发垂髫,并怡然自乐。


见渔人,乃大惊,问所从来。具答之。便要还家,设酒杀鸡作食。村中闻有此人,咸来问讯。自云先世避秦时乱,率妻子邑人来此绝境,不复出焉,遂与外人间隔。问今是何世,乃不知有汉,无论魏晋。此人一一为具言所闻,皆叹惋。余人各复延至其家,皆出酒食。停数日,辞去。此中人语云:“不足为外人道也。”


既出,得其船,便扶向路,处处志之。及郡下,诣太守,说如此。太守即遣人随其往,寻向所志,遂迷,不复得路。


南阳刘子骥,高尚士也,闻之,欣然规往。未果,寻病终,后遂无问津者。



\chapter*{两小儿辩日}
\addcontentsline{toc}{chapter}{两小儿辩日}
\begin{center}
	\textbf{[春秋战国]列子}
\end{center}

孔子东游,见两小儿辩日,问其故。(辩日一作:辩斗)

一儿曰:“我以日始出时去人近,而日中时远也。”

一儿以日初出远,而日中时近也。

一儿曰:“日初出大如车盖,及日中则如盘盂,此不为远者小而近者大乎?”

一儿曰:“日初出沧沧凉凉,及其日中如探汤,此不为近者热而远者凉乎?”

孔子不能决也。

两小儿笑曰:“孰为汝多知乎?”


\chapter*{少年中国说}
\addcontentsline{toc}{chapter}{少年中国说}
\begin{center}
	\textbf{[近现代]梁启超}
\end{center}

日本人之称我中国也,一则曰老大帝国,再则曰老大帝国。是语也,盖袭译欧西人之言也。呜呼!我中国其果老大矣乎?梁启超曰:恶!是何言!是何言!吾心目中有一少年中国在!

欲言国之老少,请先言人之老少。老年人常思既往,少年人常思将来。惟思既往也,故生留恋心;惟思将来也,故生希望心。惟留恋也,故保守;惟希望也,故进取。惟保守也,故永旧;惟进取也,故日新。惟思既往也,事事皆其所已经者,故惟知照例;惟思将来也,事事皆其所未经者,故常敢破格。老年人常多忧虑,少年人常好行乐。惟多忧也,故灰心;惟行乐也,故盛气。惟灰心也,故怯懦;惟盛气也,故豪壮。惟怯懦也,故苟且;惟豪壮也,故冒险。惟苟且也,故能灭世界;惟冒险也,故能造世界。老年人常厌事,少年人常喜事。惟厌事也,故常觉一切事无可为者;惟好事也,故常觉一切事无不可为者。老年人如夕照,少年人如朝阳;老年人如瘠牛,少年人如乳虎。老年人如僧,少年人如侠。老年人如字典,少年人如戏文。老年人如鸦片烟,少年人如泼兰地酒。老年人如别行星之陨石,少年人如大洋海之珊瑚岛。老年人如埃及沙漠之金字塔,少年人如西比利亚之铁路;老年人如秋后之柳,少年人如春前之草。老年人如死海之潴为泽,少年人如长江之初发源。此老年与少年性格不同之大略也。任公曰:人固有之,国亦宜然。

梁启超曰:伤哉,老大也!浔阳江头琵琶妇,当明月绕船,枫叶瑟瑟,衾寒于铁,似梦非梦之时,追想洛阳尘中春花秋月之佳趣。西宫南内,白发宫娥,一灯如穗,三五对坐,谈开元、天宝间遗事,谱《霓裳羽衣曲》。青门种瓜人,左对孺人,顾弄孺子,忆侯门似海珠履杂遝之盛事。拿破仑之流于厄蔑,阿剌飞之幽于锡兰,与三两监守吏,或过访之好事者,道当年短刀匹马驰骋中原,席卷欧洲,血战海楼,一声叱咤,万国震恐之丰功伟烈,初而拍案,继而抚髀,终而揽镜。呜呼,面皴齿尽,白发盈把,颓然老矣!若是者,舍幽郁之外无心事,舍悲惨之外无天地,舍颓唐之外无日月,舍叹息之外无音声,舍待死之外无事业。美人豪杰且然,而况寻常碌碌者耶?生平亲友,皆在墟墓;起居饮食,待命于人。今日且过,遑知他日?今年且过,遑恤明年?普天下灰心短气之事,未有甚于老大者。于此人也,而欲望以拏云之手段,回天之事功,挟山超海之意气,能乎不能?

呜呼!我中国其果老大矣乎?立乎今日以指畴昔,唐虞三代,若何之郅治;秦皇汉武,若何之雄杰;汉唐来之文学,若何之隆盛;康乾间之武功,若何之烜赫。历史家所铺叙,词章家所讴歌,何一非我国民少年时代良辰美景、赏心乐事之陈迹哉!而今颓然老矣!昨日割五城,明日割十城,处处雀鼠尽,夜夜鸡犬惊。十八省之土地财产,已为人怀中之肉;四百兆之父兄子弟,已为人注籍之奴,岂所谓“老大嫁作商人妇”者耶?呜呼!凭君莫话当年事,憔悴韶光不忍看!楚囚相对,岌岌顾影,人命危浅,朝不虑夕。国为待死之国,一国之民为待死之民。万事付之奈何,一切凭人作弄,亦何足怪!

任公曰:我中国其果老大矣乎?是今日全地球之一大问题也。如其老大也,则是中国为过去之国,即地球上昔本有此国,而今渐澌灭,他日之命运殆将尽也。如其非老大也,则是中国为未来之国,即地球上昔未现此国,而今渐发达,他日之前程且方长也。欲断今日之中国为老大耶?为少年耶?则不可不先明“国”字之意义。夫国也者,何物也?有土地,有人民,以居于其土地之人民,而治其所居之土地之事,自制法律而自守之;有主权,有服从,人人皆主权者,人人皆服从者。夫如是,斯谓之完全成立之国,地球上之有完全成立之国也,自百年以来也。完全成立者,壮年之事也。未能完全成立而渐进于完全成立者,少年之事也。故吾得一言以断之曰:欧洲列邦在今日为壮年国,而我中国在今日为少年国。

夫古昔之中国者,虽有国之名,而未成国之形也。或为家族之国,或为酋长之国,或为诸侯封建之国,或为一王专制之国。虽种类不一,要之,其于国家之体质也,有其一部而缺其一部。正如婴儿自胚胎以迄成童,其身体之一二官支,先行长成,此外则全体虽粗具,然未能得其用也。故唐虞以前为胚胎时代,殷周之际为乳哺时代,由孔子而来至于今为童子时代。逐渐发达,而今乃始将入成童以上少年之界焉。其长成所以若是之迟者,则历代之民贼有窒其生机者也。譬犹童年多病,转类老态,或且疑其死期之将至焉,而不知皆由未完成未成立也。非过去之谓,而未来之谓也。

且我中国畴昔,岂尝有国家哉?不过有朝廷耳!我黄帝子孙,聚族而居,立于此地球之上者既数千年,而问其国之为何名,则无有也。夫所谓唐、虞、夏、商、周、秦、汉、魏、晋、宋、齐、梁、陈、隋、唐、宋、元、明、清者,则皆朝名耳。朝也者,一家之私产也。国也者,人民之公产也。朝有朝之老少,国有国之老少。朝与国既异物,则不能以朝之老少而指为国之老少明矣。文、武、成、康,周朝之少年时代也。幽、厉、桓、赧,则其老年时代也。高、文、景、武,汉朝之少年时代也。元、平、桓、灵,则其老年时代也。自余历朝,莫不有之。凡此者谓为一朝廷之老也则可,谓为一国之老也则不可。一朝廷之老旦死,犹一人之老且死也,于吾所谓中国者何与焉。然则,吾中国者,前此尚未出现于世界,而今乃始萌芽云尔。天地大矣,前途辽矣。美哉我少年中国乎!

玛志尼者,意大利三杰之魁也。以国事被罪,逃窜异邦。乃创立一会,名曰“少年意大利”。举国志士,云涌雾集以应之。卒乃光复旧物,使意大利为欧洲之一雄邦。夫意大利者,欧洲之第一老大国也。自罗马亡后,土地隶于教皇,政权归于奥国,殆所谓老而濒于死者矣。而得一玛志尼,且能举全国而少年之,况我中国之实为少年时代者耶!堂堂四百余州之国土,凛凛四百余兆之国民,岂遂无一玛志尼其人者!

龚自珍氏之集有诗一章,题曰《能令公少年行》。吾尝爱读之,而有味乎其用意之所存。我国民而自谓其国之老大也,斯果老大矣;我国民而自知其国之少年也,斯乃少年矣。西谚有之曰:“有三岁之翁,有百岁之童。”然则,国之老少,又无定形,而实随国民之心力以为消长者也。吾见乎玛志尼之能令国少年也,吾又见乎我国之官吏士民能令国老大也。吾为此惧!夫以如此壮丽浓郁翩翩绝世之少年中国,而使欧西日本人谓我为老大者,何也?则以握国权者皆老朽之人也。非哦几十年八股,非写几十年白折,非当几十年差,非捱几十年俸,非递几十年手本,非唱几十年喏,非磕几十年头,非请几十年安,则必不能得一官、进一职。其内任卿贰以上,外任监司以上者,百人之中,其五官不备者,殆九十六七人也。非眼盲则耳聋,非手颤则足跛,否则半身不遂也。彼其一身饮食步履视听言语,尚且不能自了,须三四人左右扶之捉之,乃能度日,于此而乃欲责之以国事,是何异立无数木偶而使治天下也!且彼辈者,自其少壮之时既已不知亚细亚、欧罗巴为何处地方,汉祖唐宗是那朝皇帝,犹嫌其顽钝腐败之未臻其极,又必搓磨之,陶冶之,待其脑髓已涸,血管已塞,气息奄奄,与鬼为邻之时,然后将我二万里山河,四万万人命,一举而界于其手。呜呼!老大帝国,诚哉其老大也!而彼辈者,积其数十年之八股、白折、当差、捱俸、手本、唱喏、磕头、请安,千辛万苦,千苦万辛,乃始得此红顶花翎之服色,中堂大人之名号,乃出其全副精神,竭其毕生力量,以保持之。如彼乞儿拾金一锭,虽轰雷盘旋其顶上,而两手犹紧抱其荷包,他事非所顾也,非所知也,非所闻也。于此而告之以亡国也,瓜分也,彼乌从而听之,乌从而信之!即使果亡矣,果分矣,而吾今年七十矣,八十矣,但求其一两年内,洋人不来,强盗不起,我已快活过了一世矣!若不得已,则割三头两省之土地奉申贺敬,以换我几个衙门;卖三几百万之人民作仆为奴,以赎我一条老命,有何不可?有何难办?呜呼!今之所谓老后、老臣、老将、老吏者,其修身齐家治国平天下之手段,皆具于是矣。西风一夜催人老,凋尽朱颜白尽头。使走无常当医生,携催命符以祝寿,嗟乎痛哉!以此为国,是安得不老且死,且吾恐其未及岁而殇也。

任公曰:造成今日之老大中国者,则中国老朽之冤业也。制出将来之少年中国者,则中国少年之责任也。彼老朽者何足道,彼与此世界作别之日不远矣,而我少年乃新来而与世界为缘。如僦屋者然,彼明日将迁居他方,而我今日始入此室处。将迁居者,不爱护其窗栊,不洁治其庭庑,俗人恒情,亦何足怪!若我少年者,前程浩浩,后顾茫茫。中国而为牛为马为奴为隶,则烹脔鞭棰之惨酷,惟我少年当之。中国如称霸宇内,主盟地球,则指挥顾盼之尊荣,惟我少年享之。于彼气息奄奄与鬼为邻者何与焉?彼而漠然置之,犹可言也。我而漠然置之,不可言也。使举国之少年而果为少年也,则吾中国为未来之国,其进步未可量也。使举国之少年而亦为老大也,则吾中国为过去之国,其澌亡可翘足而待也。故今日之责任,不在他人,而全在我少年。少年智则国智,少年富则国富;少年强则国强,少年独立则国独立;少年自由则国自由;少年进步则国进步;少年胜于欧洲,则国胜于欧洲;少年雄于地球,则国雄于地球。红日初升,其道大光。河出伏流,一泻汪洋。潜龙腾渊,鳞爪飞扬。乳虎啸谷,百兽震惶。鹰隼试翼,风尘翕张。奇花初胎,矞矞皇皇。干将发硎,有作其芒。天戴其苍,地履其黄。纵有千古,横有八荒。前途似海,来日方长。美哉我少年中国,与天不老!壮哉我中国少年,与国无疆!

“三十功名尘与土,八千里路云和月。莫等闲,白了少年头,空悲切。”此岳武穆《满江红》词句也,作者自六岁时即口受记忆,至今喜诵之不衰。自今以往,弃“哀时客”之名,更自名曰“少年中国之少年”。


\chapter*{愚公移山}
\addcontentsline{toc}{chapter}{愚公移山}
\begin{center}
	\textbf{[春秋战国]列子}
\end{center}

太行、王屋二山,方七百里,高万仞。本在冀州之南,河阳之北。

北山愚公者,年且九十,面山而居。惩山北之塞,出入之迂也。聚室而谋曰:“吾与汝毕力平险,指通豫南,达于汉阴,可乎?”杂然相许。其妻献疑曰:“以君之力,曾不能损魁父之丘,如太行、王屋何?且焉置土石?”杂曰:“投诸渤海之尾,隐土之北。”遂率子孙荷担者三夫,叩石垦壤,箕畚运于渤海之尾。邻人京城氏之孀妻有遗男,始龀,跳往助之。寒暑易节,始一反焉。

河曲智叟笑而止之曰:“甚矣,汝之不惠。以残年余力,曾不能毁山之一毛,其如土石何?”北山愚公长息曰:“汝心之固,固不可彻,曾不若孀妻弱子。虽我之死,有子存焉;子又生孙,孙又生子;子又有子,子又有孙;子子孙孙无穷匮也,而山不加增,何苦而不平?”河曲智叟亡以应。

操蛇之神闻之,惧其不已也,告之于帝。帝感其诚,命夸娥氏二子负二山,一厝朔东,一厝雍南。自此,冀之南,汉之阴,无陇断焉。


\chapter*{郑人买履}
\addcontentsline{toc}{chapter}{郑人买履}
\begin{center}
	\textbf{[春秋战国]韩非}
\end{center}

郑人有欲买履者,先自度其足,而置之其坐。至之市,而忘操之。已得履,乃曰:“吾忘持度。”反归取之。及反,市罢,遂不得履。人曰:“何不试之以足?”曰:“宁信度,无自信也。”


\chapter*{马说}
\addcontentsline{toc}{chapter}{马说}
\begin{center}
	\textbf{[唐朝]韩愈}
\end{center}

世有伯乐,然后有千里马。千里马常有,而伯乐不常有。故虽有名马,祇辱于奴隶人之手,骈(pián)死于槽(cáo)枥(lì)之间,不以千里称也。


马之千里者,一食(shí)或尽粟(sù)一石(dàn)。食(sì)马者,不知其能千里而食(sì)也。是马也,虽有千里之能,食(shí)不饱,力不足,才美不外见(xiàn),且欲与常马等不可得,安求其能千里也?


策之不以其道,食(sì)之不能尽其材,鸣之而不能通其意,执策而临之,曰:“天下无马!”呜呼!其真无马邪(yé)?其真不知马也!



\chapter*{诫子书}
\addcontentsline{toc}{chapter}{诫子书}
\begin{center}
	\textbf{[三国]诸葛亮}
\end{center}

夫(fú)君子之行,静以修身,俭以养德。非澹(淡)泊无以明志,非宁静无以致远。夫(fú)学须静也,才须学也,非学无以广才,非志无以成学。淫慢则不能励精,险躁则不能治性。年与时驰,意与日去,遂成枯落,多不接世,悲守穷庐,将复何及!





\chapter*{朱子家训}
\addcontentsline{toc}{chapter}{朱子家训}
\begin{center}
	\textbf{[宋朝]朱熹}
\end{center}

\begin{center}
	\begin{tabular}{l}
		
		黎明即起,洒扫庭除,要内外整洁,\\
		
		
		既昏便息,关锁门户,必亲自检点。\\
		
		
		一粥一饭,当思来之不易;半丝半缕,恒念物力维艰。\\
		
		
		宜未雨而绸缪,毋临渴而掘井。\\
		
		
		自奉必须俭约,宴客切勿流连。\\
		
		
		器具质而洁,瓦缶胜金玉;饮食约而精,园蔬愈珍馐。\\
		
		
		勿营华屋,勿谋良田。\\
		
		
		三姑六婆,实淫盗之媒;婢美妾娇,非闺房之福。\\
		
		
		童仆勿用俊美,妻妾切忌艳妆。\\
		
		
		祖宗虽远,祭祀不可不诚;子孙虽愚,经书不可不读。\\
		
		
		居身务期质朴,教子要有义方。\\
		
		
		莫贪意外之财,莫饮过量之酒。\\
		
		
		与肩挑贸易,毋占便宜;见穷苦亲邻,须加温恤。\\
		
		
		刻薄成家,理无久享;伦常乖舛,立见消亡。\\
		
		
		兄弟叔侄,须分多润寡;长幼内外,宜法肃辞严。\\
		
		
		听妇言,乖骨肉,岂是丈夫;重资财,薄父母,不成人子。\\
		
		
		嫁女择佳婿,毋索重聘;娶媳求淑女,勿计厚奁。\\
		
		
		见富贵而生谄容者,最可耻;遇贫穷而作骄态者,贱莫甚。\\
		
		
		居家戒争讼,讼则终凶;处世戒多言,言多必失。\\
		
		
		勿恃势力而凌逼孤寡;毋贪口腹而恣杀牲禽。\\
		
		
		乖僻自是,悔误必多;颓惰自甘,家道难成。\\
		
		
		狎昵恶少,久必受其累;屈志老成,急则可相依。\\
		
		
		轻听发言,安知非人之谮诉,当忍耐三思;\\
		
		
		因事相争,焉知非我之不是,须平心暗想。\\
		
		
		施惠无念,受恩莫忘。\\
		
		
		凡事当留余地,得意不宜再往。\\
		
		
		人有喜庆,不可生妒忌心;人有祸患,不可生喜幸心。\\
		
		
		善欲人见,不是真善;恶恐人知,便是大恶。\\
		
		
		见色而起淫心,报在妻女;匿怨而用暗箭,祸延子孙。\\
		
		
		家门和顺,虽饔飧不济,亦有余欢;\\
		
		
		国课早完,即囊橐无余,自得至乐。\\
		
		
		读书志在圣贤,非徒科第;为官心存君国,岂计身家。\\
		
		
		守分安命,顺时听天。为人若此,庶乎近焉。\\
		
	\end{tabular}
\end{center}


\chapter*{五代史伶官传序}
\addcontentsline{toc}{chapter}{五代史伶官传序}
\begin{center}
	\textbf{[宋朝]欧阳修}
\end{center}

呜呼!盛衰之理,虽曰天命,岂非人事哉!原庄宗之所以得天下,与其所以失之者,可以知之矣。


世言晋王之将终也,以三矢赐庄宗而告之曰:“梁,吾仇也;燕王,吾所立,契丹,与吾约为兄弟,而皆背晋以归梁。此三者,吾遗恨也。与尔三矢,尔其无忘乃父之志!”庄宗受而藏之于庙。其后用兵,则遣从事以一少牢告庙,请其矢,盛以锦囊,负而前驱,及凯旋而纳之。


方其系燕父子以组,函梁君臣之首,入于太庙,还矢先王,而告以成功,其意气之盛,可谓壮哉!及仇雠已灭,天下已定,一夫夜呼,乱者四应,仑皇东出,未见贼而士卒离散,君臣相顾,不知所归。至于誓天断发,泣下沾襟,何其衰也!岂得之难而失之易欤?抑本其成败之迹,而皆自于人欤?


《书》曰:“满招损,谦得益。”忧劳可以兴国,逸豫可以亡身,自然之理也。故方其盛也,举天下之豪杰莫能与之争;及其衰也,数十伶人困之,而身死国灭,为天下笑。夫祸患常积于忽微,而智勇多困于所溺,岂独伶人也哉!作《伶官传》。


 



\chapter*{后出师表}
\addcontentsline{toc}{chapter}{后出师表}
\begin{center}
	\textbf{[三国]诸葛亮}
\end{center}

先帝虑汉、贼不两立,王业不偏安,故托臣以讨贼也。以先帝之明,量臣之才,故知臣伐贼,才弱敌强也。然不伐贼,王业亦亡;惟坐而待亡,孰与伐之?是故托臣而弗疑也。


臣受命之日,寝不安席,食不甘味。思惟北征。宜先入南。故五月渡泸,深入不毛,并日而食;臣非不自惜也,顾王业不可得偏全于蜀都,故冒危难,以奉先帝之遗意也,而议者谓为非计。今贼适疲于西,又务于东,兵法乘劳,此进趋之时也。谨陈其事如左:


高帝明并日月,谋臣渊深,然涉险被创,危然后安。今陛下未及高帝,谋臣不如良、平,而欲以长计取胜,坐定天下,此臣之未解一也。


刘繇、王朗各据州郡,论安言计,动引圣人,群疑满腹,众难塞胸,今岁不战,明年不征,使孙策坐大,遂并江东,此臣之未解二也。


曹操智计,殊绝于人,其用兵也,仿佛孙、吴,然困于南阳,险于乌巢,危于祁连,逼于黎阳,几败北山,殆死潼关,然后伪定一时耳。况臣才弱,而欲以不危而定之,此臣之未解三也。


曹操五攻昌霸不下,四越巢湖不成,任用李服而李服图之,委任夏侯而夏侯败亡,先帝每称操为能,犹有此失,况臣驽下,何能必胜?此臣之未解四也。


自臣到汉中,中间期年耳,然丧赵云、阳群、马玉、阎芝、丁立、白寿、刘郃、邓铜等及曲长、屯将七十余人,突将、无前、賨叟、青羌、散骑、武骑一千余人。此皆数十年之内所纠合四分之精锐,非一州之所有;若复数年,则损三分之二也,当何以图敌?此臣之未解五也。


今民穷兵疲,而事不可息;事不可息,则住与行劳费正等。而不及今图之,欲以一州之地,与贼持久,此臣之未解六也。


夫难平者,事也。昔先帝败军于楚,当此时,曹操拊手,谓天下以定。然后先帝东连吴越,西取巴蜀,举兵北征,夏侯授首,此操之失计,而汉事将成也。然后吴更违盟,关羽毁败,秭归蹉跌,曹丕称帝。凡事如是,难可逆见。臣鞠躬尽力,死而后已;至于成败利钝,非臣之明所能逆覩也。



\chapter*{精卫填海}
\addcontentsline{toc}{chapter}{精卫填海}
% \begin{center}
% 	\textbf{[]}
% \end{center}

又北二百里,曰发鸠之山,其上多柘木,有鸟焉,其状如乌,文首,白喙,赤足,名曰:“精卫”,其鸣自詨(xiào)。是炎帝之少女,名曰女娃。女娃游于东海,溺而不返,故为精卫,常衔西山之木石,以堙(yīn)于东海。漳水出焉,东流注于河。——《山海经·北山经》



\chapter*{庖丁解牛}
\addcontentsline{toc}{chapter}{庖丁解牛}
\begin{center}
	\textbf{[春秋战国]庄周}
\end{center}

吾生也有涯,而知也无涯。以有涯随无涯,殆已!已而为知者,殆而已矣!为善无近名,为恶无近刑。缘督以为经,可以保身,可以全生,可以养亲,可以尽年。

庖丁为文惠君解牛,手之所触,肩之所倚,足之所履,膝之所踦,砉然向然,奏刀騞然,莫不中音。合于《桑林》之舞,乃中《经首》之会。

文惠君曰:“嘻,善哉!技盖至此乎?”

庖丁释刀对曰:“臣之所好者,道也,进乎技矣。始臣之解牛之时,所见无非牛者。三年之后,未尝见全牛也。方今之时,臣以神遇而不以目视,官知止而神欲行。依乎天理,批大郤,导大窾,因其固然,技经肯綮之未尝,而况大軱乎!良庖岁更刀,割也;族庖月更刀,折也。今臣之刀十九年矣,所解数千牛矣,而刀刃若新发于硎。彼节者有间,而刀刃者无厚;以无厚入有间,恢恢乎其于游刃必有余地矣,是以十九年而刀刃若新发于硎。虽然,每至于族,吾见其难为,怵然为戒,视为止,行为迟。动刀甚微,謋然已解,如土委地。提刀而立,为之四顾,为之踌躇满志,善刀而藏之。”

文惠君曰:“善哉!吾闻庖丁之言,得养生焉。”


\chapter*{郑人买履}
\addcontentsline{toc}{chapter}{郑人买履}
% \begin{center}
% 	\textbf{[]}
% \end{center}

郑人有欲买履者,先自度其足,而置之其坐。至之市,而忘操之。已得履,乃曰:“吾忘持度。”反归取之。及反,市罢,遂不得履。人曰:“何不试之以足?”曰:“宁信度,无自信也。”



\chapter*{石钟山记}
\addcontentsline{toc}{chapter}{石钟山记}
\begin{center}
	\textbf{[宋朝]苏轼}
\end{center}

《水经》云:“彭蠡之口有石钟山焉。”郦元以为下临深潭,微风鼓浪,水石相搏,声如洪钟。是说也,人常疑之。今以钟磬置水中,虽大风浪不能鸣也,而况石乎!至唐李渤始访其遗踪,得双石于潭上,扣而聆之,南声函胡,北音清越,桴止响腾,余韵徐歇。自以为得之矣。然是说也,余尤疑之。石之铿然有声者,所在皆是也,而此独以钟名,何哉?


元丰七年六月丁丑,余自齐安舟行适临汝,而长子迈将赴饶之德兴尉,送之至湖口,因得观所谓石钟者。寺僧使小童持斧,于乱石间择其一二扣之,硿硿焉,余固笑而不信也。至莫夜月明,独与迈乘小舟,至绝壁下。大石侧立千尺,如猛兽奇鬼,森然欲搏人;而山上栖鹘,闻人声亦惊起,磔磔云霄间;又有若老人咳且笑于山谷中者,或曰此鹳鹤也。余方心动欲还,而大声发于水上,噌吰如钟鼓不绝。舟人大恐。徐而察之,则山下皆石穴罅,不知其浅深,微波入焉,涵澹澎湃而为此也。舟回至两山间,将入港口,有大石当中流,可坐百人,空中而多窍,与风水相吞吐,有窾坎镗鞳之声,与向之噌吰者相应,如乐作焉。因笑谓迈曰:“汝识之乎?噌吰者,周景王之无射也;窾坎镗鞳者,魏庄子之歌钟也。古之人不余欺也!”


事不目见耳闻,而臆断其有无,可乎?郦元之所见闻,殆与余同,而言之不详;士大夫终不肯以小舟夜泊绝壁之下,故莫能知;而渔工水师虽知而不能言。此世所以不传也。而陋者乃以斧斤考击而求之,自以为得其实。余是以记之,盖叹郦元之简,而笑李渤之陋也。



\chapter*{黄州快哉亭记}
\addcontentsline{toc}{chapter}{黄州快哉亭记}
\begin{center}
	\textbf{[宋朝]苏辙}
\end{center}

江出西陵,始得平地。其流奔放肆大,南合沅、湘,北合汉沔,其势益张。至于赤壁之下,波流浸灌,与海相若。清河张君梦得,谪居齐安,即其庐之西南为亭,以览观江流之胜,而余兄子瞻名之曰“快哉”。


盖亭之所见,南北百里,东西一舍。涛澜汹涌,风云开阖。昼则舟楫出没于其前,夜则鱼龙悲啸于其下,变化倏忽,动心骇目,不可久视。今乃得玩之几席之上,举目而足。西望武昌诸山,冈陵起伏,草木行列,烟消日出。渔夫樵父之舍皆可指数。此其所以为“快哉”者也。至于长洲之滨,故城之墟,曹孟德、孙仲谋之所睥睨,周瑜、陆逊之所骋骛,其流风遗迹,亦足以称快世俗。


昔楚襄王从宋玉、景差于兰台之宫,有风飒然至者,王披襟当之,曰:“快哉,此风!寡人所与庶人共者耶?”宋玉曰:“此独大王之雄风耳,庶人安得共之!”玉之言,盖有讽焉。夫风无雌雄之异,而人有遇不遇之变。楚王之所以为乐,与庶人之所以为忧,此则人之变也,而风何与焉?士生于世,使其中不自得,将何往而非病?使其中坦然,不以物伤性,将何适而非快?


今张君不以谪为患,窃会计之余功,而自放山水之间,此其中宜有以过人者。将蓬户瓮牖无所不快,而况乎濯长江之清流,揖西山之白云,穷耳目之胜以自适也哉!不然,连山绝壑,长林古木,振之以清风,照之以明月,此皆骚人思士之所以悲伤憔悴而不能胜者,乌睹其为快也哉!


元丰六年十一月朔日,赵郡苏辙记。



\chapter*{五人墓碑记}
\addcontentsline{toc}{chapter}{五人墓碑记}
\begin{center}
	\textbf{[明朝]张溥}
\end{center}

五人者,盖当蓼洲周公之被逮,激于义而死焉者也。至于今,郡之贤士大夫请于当道,即除魏阉废祠之址以葬之;且立石于其墓之门,以旌其所为。呜呼,亦盛矣哉!


夫五人之死,去今之墓而葬焉,其为时止十有一月耳。夫十有一月之中,凡富贵之子,慷慨得志之徒,其疾病而死,死而湮没不足道者,亦已众矣;况草野之无闻者欤!独五人之皦皦,何也?


予犹记周公之被逮,在丁卯三月之望。吾社之行为士先者,为之声义,敛赀财以送其行,哭声震动天地。缇骑按剑而前,问:“谁为哀者?”众不能堪,抶而仆之。是时以大中丞抚吴者为魏之私人,周公之逮所由使也;吴之民方痛心焉,于是乘其厉声以呵,则噪而相逐。中丞匿于溷藩以免。既而以吴民之乱请于朝,按诛五人,曰颜佩韦、杨念如、马杰、沈扬、周文元,即今之傫然在墓者也。


然五人之当刑也,意气扬扬,呼中丞之名而詈之,谈笑以死。断头置城上,颜色不少变。有贤士大夫发五十金,买五人之脰而函之,卒与尸合。故今之墓中全乎为五人也。


嗟夫!大阉之乱,缙绅而能不易其志者,四海之大,有几人欤?而五人生于编伍之间,素不闻《诗》、《书》之训,激昂大义,蹈死不顾,亦曷故哉?且矫诏纷出,钩党之捕遍于天下,卒以吾郡之发愤一击,不敢复有株治;大阉亦逡巡畏义,非常之谋,难于猝发,待圣人之出而投缳道路:不可谓非五人之力也!


由是观之,则今之高爵显位,一旦抵罪,或脱身以逃,不能容于远近,而又有剪发杜门、佯狂不知所之者,其辱人贱行,视五人之死,轻重固何如哉?是以蓼洲周公忠义暴于朝廷,赠谥美显,荣于身后;而五人亦得以加其土封,列其姓名于大堤之上,凡四方之士无有不过而拜且泣者,斯固百世之遇也。不然,令五人者保其首领,以老于户牖之下,则尽其天年,人皆得以隶使之,安能屈豪杰之流,扼腕墓道,发其志士之悲哉!故予与同社诸君子哀斯墓之徒有其石也而为之记,亦以明死生之大,匹夫之有重于社稷也。


贤士大夫者,冏卿因之吴公,太史文起文公、孟长姚公也。



\chapter*{卖油翁}
\addcontentsline{toc}{chapter}{卖油翁}
\begin{center}
	\textbf{[宋朝]欧阳修}
\end{center}

陈康肃公尧咨(zī)善射,当世无双,公亦以此自矜(jīn)。尝射于家圃(pǔ),有卖油翁释担(dàn)而立,睨(nì)之,久而不去。见其发矢(shǐ)十中八九,但微颔(hàn)之。

康肃问曰:”汝(rǔ)亦知射乎?吾射不亦精乎?”翁曰:”无他,但手熟(shú)尔。”康肃忿(fèn)然曰:”尔安敢轻吾射!”翁曰:”以我酌(zhuó)油知之。”乃取一葫芦置于地,以钱覆其口,徐以杓(sháo)酌油沥(lì)之,自钱孔入,而钱不湿。因曰:”我亦无他,唯手熟(shú)尔。”康肃笑而遣(qiǎn)之。

此与庄生所谓解牛斫轮者何异?


\chapter*{郑伯克段于鄢}
\addcontentsline{toc}{chapter}{郑伯克段于鄢}
\begin{center}
	\textbf{[春秋战国]左丘明}
\end{center}

初,郑武公娶于申,曰武姜,生庄公及共叔段。庄公寤生,惊姜氏,故名曰寤生,遂恶之。爱共叔段,欲立之。亟请于武公,公弗许。

及庄公即位,为之请制。公曰:“制,岩邑也,虢叔死焉。佗邑唯命。”请京,使居之,谓之京城大叔。祭仲曰:“都城过百雉,国之害也。先王之制:大都不过参国之一,中五之一,小九之一。今京不度,非制也,君将不堪。”公曰:“姜氏欲之,焉辟害?”对曰:“姜氏何厌之有!不如早为之所,无使滋蔓,蔓难图也。蔓草犹不可除,况君之宠弟乎!”公曰:“多行不义,必自毙,子姑待之。”

既而大叔命西鄙北鄙贰于己。公子吕曰:“国不堪贰,君将若之何?欲与大叔,臣请事之;若弗与,则请除之。无生民心。”公曰:“无庸,将自及。”大叔又收贰以为己邑,至于廪延。子封曰:“可矣,厚将得众。”公曰:“不义,不暱,厚将崩。”

大叔完聚,缮甲兵,具卒乘,将袭郑。夫人将启之。公闻其期,曰:“可矣!”命子封帅车二百乘以伐京。京叛大叔段,段入于鄢,公伐诸鄢。五月辛丑,大叔出奔共。

书曰:“郑伯克段于鄢。”段不弟,故不言弟;如二君,故曰克;称郑伯,讥失教也;谓之郑志。不言出奔,难之也。

遂寘姜氏于城颍,而誓之曰:“不及黄泉,无相见也。”既而悔之。颍考叔为颍谷封人,闻之,有献于公,公赐之食,食舍肉。公问之,对曰:“小人有母,皆尝小人之食矣,未尝君之羹,请以遗之。”公曰:“尔有母遗,繄我独无!”颍考叔曰:“敢问何谓也?”公语之故,且告之悔。对曰:“君何患焉?若阙地及泉,隧而相见,其谁曰不然?”公从之。公入而赋:“大隧之中,其乐也融融!”姜出而赋:“大隧之外,其乐也洩洩。”遂为母子如初。

君子曰:“颍考叔,纯孝也,爱其母,施及庄公。《诗》曰:‘孝子不匮,永锡尔类。’其是之谓乎!”


\chapter*{留侯论}
\addcontentsline{toc}{chapter}{留侯论}
\begin{center}
	\textbf{[宋朝]苏轼}
\end{center}

古之所谓豪杰之士,必有过人之节。人情有所不能忍者,匹夫见辱,拔剑而起,挺身而斗,此不足为勇也。天下有大勇者,卒然临之而不惊,无故加之而不怒。此其所挟持者甚大,而其志甚远也。


夫子房受书于圯上之老人也,其事甚怪;然亦安知其非秦之世,有隐君子者出而试之。观其所以微见其意者,皆圣贤相与警戒之义;而世不察,以为鬼物,亦已过矣。且其意不在书。当韩之亡,秦之方盛也,以刀锯鼎镬待天下之士。其平居无罪夷灭者,不可胜数。虽有贲、育,无所复施。夫持法太急者,其锋不可犯,而其势未可乘。子房不忍忿忿之心,以匹夫之力,而逞于一击之间;当此之时,子房之不死者,其间不能容发,盖亦已危矣。千金之子,不死于盗贼,何者?其身之可爱,而盗贼之不足以死也。子房以盖世之才,不为伊尹、太公之谋,而特出于荆轲、聂政之计,以侥幸于不死,此圯上老人所为深惜者也。是故倨傲鲜腆而深折之。彼其能有所忍也,然后可以就大事,故曰:“孺子可教也。”


楚庄王伐郑,郑伯肉袒牵羊以逆;庄王曰:“其君能下人,必能信用其民矣。”遂舍之。勾践之困于会稽,而归臣妾于吴者,三年而不倦。且夫有报人之志,而不能下人者,是匹夫之刚也。夫老人者,以为子房才有余,而忧其度量之不足,故深折其少年刚锐之气,使之忍小忿而就大谋。何则?非有生平之素,卒然相遇于草野之间,而命以仆妾之役,油然而不怪者,此固秦皇之所不能惊,而项籍之所不能怒也。


观夫高祖之所以胜,而项籍之所以败者,在能忍与不能忍之间而已矣。项籍唯不能忍,是以百战百胜而轻用其锋;高祖忍之,养其全锋而待其弊,此子房教之也。当淮阴破齐而欲自王,高祖发怒,见于词色。由此观之,犹有刚强不忍之气,非子房其谁全之?太史公疑子房以为魁梧奇伟,而其状貌乃如妇人女子,不称其志气。呜呼!此其所以为子房欤!



\chapter*{墨池记}
\addcontentsline{toc}{chapter}{墨池记}
\begin{center}
	\textbf{[宋朝]曾巩}
\end{center}

临川之城东,有地隐然而高,以临于溪,曰新城。


新城之上,有池洼然而方以长,曰王羲之之墨池者。荀伯子《临川记》云也。羲之尝慕张芝,临池学书,池水尽黑,此为其故迹,岂信然邪?


方羲之之不可强以仕,而尝极东方,出沧海,以娱其意于山水之间。岂有徜徉肆恣,而又尝自休于此邪?羲之之书晚乃善,则


其所能,盖亦以精力自致者,非天成也。然后世未有能及者,岂其学不如彼邪?则学固岂可以少哉!况欲深造道德者邪?


墨池之上,今为州学舍。教授王君盛恐其不章也,书“晋王右军墨池”之六字于楹间以揭之,又告于巩曰:“愿有记。”推王君之心,岂爱人之善,虽一能不以废,而因以及乎其迹邪?其亦欲推其事,以勉其学者邪?夫人之有一能,而使后人尚之如此,况仁人庄士之遗风余思,被于来世者何如哉!


庆历八年九月十二日,曾巩记。



\chapter*{陈情表}
\addcontentsline{toc}{chapter}{陈情表}
\begin{center}
	\textbf{[晋朝]李密}
\end{center}

臣密言:臣以险衅,夙遭闵凶。生孩六月,慈父见背;行年四岁,舅夺母志。祖母刘愍臣孤弱,躬亲抚养。臣少多疾病,九岁不行,零丁孤苦,至于成立。既无伯叔,终鲜兄弟,门衰祚薄,晚有儿息。外无期功强近之亲,内无应门五尺之僮,茕茕孑立,形影相吊。而刘夙婴疾病,常在床蓐,臣侍汤药,未曾废离。(愍一作:悯茕茕孑立一作:独立)

逮奉圣朝,沐浴清化。前太守臣逵察臣孝廉;后刺史臣荣举臣秀才。臣以供养无主,辞不赴命。诏书特下,拜臣郎中,寻蒙国恩,除臣洗马。猥以微贱,当侍东宫,非臣陨首所能上报。臣具以表闻,辞不就职。诏书切峻,责臣逋慢;郡县逼迫,催臣上道;州司临门,急于星火。臣欲奉诏奔驰,则刘病日笃,欲苟顺私情,则告诉不许。臣之进退,实为狼狈。

伏惟圣朝以孝治天下,凡在故老,犹蒙矜育,况臣孤苦,特为尤甚。且臣少仕伪朝,历职郎署,本图宦达,不矜名节。今臣亡国贱俘,至微至陋,过蒙拔擢,宠命优渥,岂敢盘桓,有所希冀!但以刘日薄西山,气息奄奄,人命危浅,朝不虑夕。臣无祖母,无以至今日,祖母无臣,无以终余年。母孙二人,更相为命,是以区区不能废远。

臣密今年四十有四,祖母今年九十有六,是臣尽节于陛下之日长,报养刘之日短也。乌鸟私情,愿乞终养。臣之辛苦,非独蜀之人士及二州牧伯所见明知,皇天后土,实所共鉴。愿陛下矜悯愚诚,听臣微志,庶刘侥幸,保卒余年。臣生当陨首,死当结草。臣不胜犬马怖惧之情,谨拜表以闻。(祖母一作:祖母刘)


\chapter*{祭妹文}
\addcontentsline{toc}{chapter}{祭妹文}
\begin{center}
	\textbf{[清朝]袁枚}
\end{center}

乾隆丁亥冬,葬三妹素文于上元之羊山,而奠以文曰:


呜呼!汝生于浙,而葬于斯,离吾乡七百里矣;当时虽觭梦幻想,宁知此为归骨所耶?


汝以一念之贞,遇人仳离,致孤危托落,虽命之所存,天实为之;然而累汝至此者,未尝非予之过也。予幼从先生授经,汝差肩而坐,爱听古人节义事;一旦长成,遽躬蹈之。呜呼!使汝不识《诗》、《书》,或未必艰贞若是。


余捉蟋蟀,汝奋臂出其间;岁寒虫僵,同临其穴。今予殓汝葬汝,而当日之情形,憬然赴目。予九岁,憩书斋,汝梳双髻,披单缣来,温《缁衣》一章;适先生奓户入,闻两童子音琅琅然,不觉莞尔,连呼“则则”,此七月望日事也。汝在九原,当分明记之。予弱冠粤行,汝掎裳悲恸。逾三年,予披宫锦还家,汝从东厢扶案出,一家瞠视而笑,不记语从何起,大概说长安登科、函使报信迟早云尔。凡此琐琐,虽为陈迹,然我一日未死,则一日不能忘。旧事填膺,思之凄梗,如影历历,逼取便逝。悔当时不将嫛婗情状,罗缕记存;然而汝已不在人间,则虽年光倒流,儿时可再,而亦无与为证印者矣。


汝之义绝高氏而归也,堂上阿奶,仗汝扶持;家中文墨,眣汝办治。尝谓女流中最少明经义、谙雅故者。汝嫂非不婉嫕,而于此微缺然。故自汝归后,虽为汝悲,实为予喜。予又长汝四岁,或人间长者先亡,可将身后托汝;而不谓汝之先予以去也!


前年予病,汝终宵刺探,减一分则喜,增一分则忧。后虽小差,犹尚殗殜,无所娱遣;汝来床前,为说稗官野史可喜可愕之事,聊资一欢。呜呼!今而后,吾将再病,教从何处呼汝耶?


汝之疾也,予信医言无害,远吊扬州;汝又虑戚吾心,阻人走报;及至绵惙已极,阿奶问:“望兄归否?”强应曰:“诺。”已予先一日梦汝来诀,心知不祥,飞舟渡江,果予以未时还家,而汝以辰时气绝;四支犹温,一目未瞑,盖犹忍死待予也。呜呼痛哉!早知诀汝,则予岂肯远游?即游,亦尚有几许心中言要汝知闻、共汝筹画也。而今已矣!除吾死外,当无见期。吾又不知何日死,可以见汝;而死后之有知无知,与得见不得见,又卒难明也。然则抱此无涯之憾,天乎人乎!而竟已乎!


汝之诗,吾已付梓;汝之女,吾已代嫁;汝之生平,吾已作传;惟汝之窀穸,尚未谋耳。先茔在杭,江广河深,势难归葬,故请母命而宁汝于斯,便祭扫也。其傍,葬汝女阿印;其下两冢:一为阿爷侍者朱氏,一为阿兄侍者陶氏。羊山旷渺,南望原隰,西望栖霞,风雨晨昏,羁魂有伴,当不孤寂。所怜者,吾自戊寅年读汝哭侄诗后,至今无男;两女牙牙,生汝死后,才周睟耳。予虽亲在未敢言老,而齿危发秃,暗里自知;知在人间,尚复几日?阿品远官河南,亦无子女,九族无可继者。汝死我葬,我死谁埋?汝倘有灵,可能告我?


呜呼!生前既不可想,身后又不可知;哭汝既不闻汝言,奠汝又不见汝食。纸灰飞扬,朔风野大,阿兄归矣,犹屡屡回头望汝也。呜呼哀哉!呜呼哀哉!



\chapter*{丰乐亭记}
\addcontentsline{toc}{chapter}{丰乐亭记}
\begin{center}
	\textbf{[宋朝]欧阳修}
\end{center}

修既治滁之明年,夏,始饮滁水而甘。问诸滁人,得于州南百步之近。其上则丰山,耸然而特立;下则幽谷,窈然而深藏;中有清泉,滃然而仰出。俯仰左右,顾而乐之。于是疏泉凿石,辟地以为亭,而与滁人往游其间。


滁于五代干戈之际,用武之地也。昔太祖皇帝,尝以周师破李景兵十五万于清流山下,生擒其皇甫辉、姚凤于滁东门之外,遂以平滁。修尝考其山川,按其图记,升高以望清流之关,欲求辉、凤就擒之所。而故老皆无在也,盖天下之平久矣。自唐失其政,海内分裂,豪杰并起而争,所在为敌国者,何可胜数?及宋受天命,圣人出而四海一。向之凭恃险阻,铲削消磨。百年之间,漠然徒见山高而水清;欲问其事,而遗老尽矣。今滁介江淮之间,舟车商贾、四方宾客之所不至,民生不见外事,而安于畎亩衣食,以乐生送死。而孰知上之功德,休养生息,涵煦于百年之深也。


修之来此,乐其地僻而事简,又爱其俗之安闲。既得斯泉于山谷之间,乃日与滁人仰而望山,俯而听泉;掇幽芳而荫乔木,风霜冰雪,刻露清秀,四时之景,无不可爱。又幸其民乐其岁物之丰成,而喜与予游也。因为本其山川,道其风俗之美,使民知所以安此丰年之乐者,幸生无事之时也。夫宣上恩德,以与民共乐,刺史之事也。遂书以名其亭焉。



\chapter*{进学解}
\addcontentsline{toc}{chapter}{进学解}
\begin{center}
	\textbf{[唐朝]韩愈}
\end{center}

国子先生晨入太学,招诸生立馆下,诲之曰:“业精于勤,荒于嬉;行成于思,毁于随。方今圣贤相逢,治具毕张。拔去凶邪,登崇畯良。占小善者率以录,名一艺者无不庸。爬罗剔抉,刮垢磨光。盖有幸而获选,孰云多而不扬?诸生业患不能精,无患有司之不明;行患不能成,无患有司之不公。”


言未既,有笑于列者曰:“先生欺余哉!弟子事先生,于兹有年矣。先生口不绝吟于六艺之文,手不停披于百家之编。记事者必提其要,纂言者必钩其玄。贪多务得,细大不捐。焚膏油以继晷,恒兀兀以穷年。先生之业,可谓勤矣。


觝排异端,攘斥佛老。补苴罅漏,张皇幽眇。寻坠绪之茫茫,独旁搜而远绍。障百川而东之,回狂澜于既倒。先生之于儒,可谓有劳矣。


沉浸醲郁,含英咀华,作为文章,其书满家。上规姚姒,浑浑无涯;周诰、殷《盘》,佶屈聱牙;《春秋》谨严,《左氏》浮夸;《易》奇而法,《诗》正而葩;下逮《庄》、《骚》,太史所录;子云,相如,同工异曲。先生之于文,可谓闳其中而肆其外矣。


少始知学,勇于敢为;长通于方,左右具宜。先生之于为人,可谓成矣。


然而公不见信于人,私不见助于友。跋前踬后,动辄得咎。暂为御史,遂窜南夷。三年博士,冗不见治。命与仇谋,取败几时。冬暖而儿号寒,年丰而妻啼饥。头童齿豁,竟死何裨。不知虑此,而反教人为?”


先生曰:“吁,子来前!夫大木为杗,细木为桷,欂栌、侏儒,椳、闑、扂、楔,各得其宜,施以成室者,匠氏之工也。玉札、丹砂,赤箭、青芝,牛溲、马勃,败鼓之皮,俱收并蓄,待用无遗者,医师之良也。登明选公,杂进巧拙,纡馀为妍,卓荦为杰,校短量长,惟器是适者,宰相之方也。昔者孟轲好辩,孔道以明,辙环天下,卒老于行。荀卿守正,大论是弘,逃谗于楚,废死兰陵。是二儒者,吐辞为经,举足为法,绝类离伦,优入圣域,其遇于世何如也?今先生学虽勤而不繇其统,言虽多而不要其中,文虽奇而不济于用,行虽修而不显于众。犹且月费俸钱,岁靡廪粟;子不知耕,妇不知织;乘马从徒,安坐而食。踵常途之役役,窥陈编以盗窃。然而圣主不加诛,宰臣不见斥,兹非其幸欤?动而得谤,名亦随之。投闲置散,乃分之宜。若夫商财贿之有亡,计班资之崇庳,忘己量之所称,指前人之瑕疵,是所谓诘匠氏之不以杙为楹,而訾医师以昌阳引年,欲进其豨苓也。



\chapter*{项脊轩志}
\addcontentsline{toc}{chapter}{项脊轩志}
\begin{center}
	\textbf{[明朝]归有光}
\end{center}

项脊轩,旧南阁子也。室仅方丈,可容一人居。百年老屋,尘泥渗漉,雨泽下注;每移案,顾视,无可置者。又北向,不能得日,日过午已昏。余稍为修葺,使不上漏。前辟四窗,垣墙周庭,以当南日,日影反照,室始洞然。又杂植兰桂竹木于庭,旧时栏楯,亦遂增胜。借书满架,偃仰啸歌,冥然兀坐,万籁有声;而庭堦寂寂,小鸟时来啄食,人至不去。三五之夜,明月半墙,桂影斑驳,风移影动,珊珊可爱。(堦寂寂一作:阶寂寂)

然余居于此,多可喜,亦多可悲。先是庭中通南北为一。迨诸父异爨,内外多置小门,墙往往而是。东犬西吠,客逾庖而宴,鸡栖于厅。庭中始为篱,已为墙,凡再变矣。家有老妪,尝居于此。妪,先大母婢也,乳二世,先妣抚之甚厚。室西连于中闺,先妣尝一至。妪每谓余曰:”某所,而母立于兹。”妪又曰:”汝姊在吾怀,呱呱而泣;娘以指叩门扉曰:‘儿寒乎?欲食乎?’吾从板外相为应答。”语未毕,余泣,妪亦泣。余自束发,读书轩中,一日,大母过余曰:”吾儿,久不见若影,何竟日默默在此,大类女郎也?”比去,以手阖门,自语曰:”吾家读书久不效,儿之成,则可待乎!”顷之,持一象笏至,曰:”此吾祖太常公宣德间执此以朝,他日汝当用之!”瞻顾遗迹,如在昨日,令人长号不自禁。

轩东,故尝为厨,人往,从轩前过。余扃牖而居,久之,能以足音辨人。轩凡四遭火,得不焚,殆有神护者。

项脊生曰:“蜀清守丹穴,利甲天下,其后秦皇帝筑女怀清台;刘玄德与曹操争天下,诸葛孔明起陇中。方二人之昧昧于一隅也,世何足以知之,余区区处败屋中,方扬眉、瞬目,谓有奇景。人知之者,其谓与坎井之蛙何异?”(人教版《中国古代诗歌散文欣赏》中无此段文字;沪教版无此段。)

余既为此志,后五年,吾妻来归,时至轩中,从余问古事,或凭几学书。吾妻归宁,述诸小妹语曰:”闻姊家有阁子,且何谓阁子也?”其后六年,吾妻死,室坏不修。其后二年,余久卧病无聊,乃使人复葺南阁子,其制稍异于前。然自后余多在外,不常居。

庭有枇杷树,吾妻死之年所手植也,今已亭亭如盖矣。


\chapter*{陆游家训}
\addcontentsline{toc}{chapter}{陆游家训}
\begin{center}
	\textbf{[宋朝]陆游}
\end{center}

后生才锐者,最易坏。若有之,父兄当以为忧,不可以为喜也。切须常加简束,令熟读经学,训以宽厚恭谨,勿令与浮薄者游处。自此十许年,志趣自成。不然,其可虑之事,盖非一端。吾(wú)此言,后生之药石也,各须谨之,毋(wú)贻(yí)后悔。



\chapter*{范增论}
\addcontentsline{toc}{chapter}{范增论}
\begin{center}
	\textbf{[宋朝]苏轼}
\end{center}

汉用陈平计,间疏楚君臣,项羽疑范增与汉有私,稍夺其权。增大怒曰:“天下事大定矣,君王自为之,愿赐骸骨,归卒伍。”未至彭城,疽发背,死。


苏子曰:“增之去,善矣。不去,羽必杀增。独恨其不早尔。”然则当以何事去?增劝羽杀沛公,羽不听,终以此失天下,当以是去耶?曰:“否。增之欲杀沛公,人臣之分也;羽之不杀,犹有君人之度也。增曷为以此去哉?《易》曰:‘知几其神乎!’《诗》曰:‘如彼雨雪,先集为霰。’增之去,当于羽杀卿子冠军时也。”


陈涉之得民也,以项燕。项氏之兴也,以立楚怀王孙心;而诸侯之叛之也,以弑义帝。且义帝之立,增为谋主矣。义帝之存亡,岂独为楚之盛衰,亦增之所与同祸福也;未有义帝亡而增独能久存者也。羽之杀卿子冠军也,是弑义帝之兆也。其弑义帝,则疑增之本也,岂必待陈平哉?物必先腐也,而后虫生之;人必先疑也,而后谗入之。陈平虽智,安能间无疑之主哉?


吾尝论义帝,天下之贤主也。独遣沛公入关,而不遣项羽;识卿子冠军于稠人之中,而擢为上将,不贤而能如是乎?羽既矫杀卿子冠军,义帝必不能堪,非羽弑帝,则帝杀羽,不待智者而后知也。增始劝项梁立义帝,诸侯以此服从。中道而弑之,非增之意也。夫岂独非其意,将必力争而不听也。不用其言,而杀其所立,羽之疑增必自此始矣。


方羽杀卿子冠军,增与羽比肩而事义帝,君臣之分未定也。为增计者,力能诛羽则诛之,不能则去之,岂不毅然大丈夫也哉?增年七十,合则留,不合即去,不以此时明去就之分,而欲依羽以成功名,陋矣!虽然,增,高帝之所畏也;增不去,项羽不亡。亦人杰也哉!



\chapter*{晁错论}
\addcontentsline{toc}{chapter}{晁错论}
\begin{center}
	\textbf{[宋朝]苏轼}
\end{center}

天下之患,最不可为者,名为治平无事,而其实有不测之忧。坐观其变,而不为之所,则恐至于不可救;起而强为之,则天下狃于治平之安而不吾信。惟仁人君子豪杰之士,为能出身为天下犯大难,以求成大功;此固非勉强期月之间,而苟以求名之所能也。天下治平,无故而发大难之端;吾发之,吾能收之,然后有辞于天下。事至而循循焉欲去之,使他人任其责,则天下之祸,必集于我。


昔者晁错尽忠为汉,谋弱山东之诸侯,山东诸侯并起,以诛错为名;而天子不以察,以错为之说。天下悲错之以忠而受祸,不知错有以取之也。


古之立大事者,不惟有超世之才,亦必有坚忍不拔之志。昔禹之治水,凿龙门,决大河而放之海。方其功之未成也,盖亦有溃冒冲突可畏之患;惟能前知其当然,事至不惧,而徐为之图,是以得至于成功。


夫以七国之强,而骤削之,其为变,岂足怪哉?错不于此时捐其身,为天下当大难之冲,而制吴楚之命,乃为自全之计,欲使天子自将而己居守。且夫发七国之难者,谁乎?己欲求其名,安所逃其患。以自将之至危,与居守至安;己为难首,择其至安,而遣天子以其至危,此忠臣义士所以愤怨而不平者也。当此之时,虽无袁盎,错亦未免于祸。何者?己欲居守,而使人主自将。以情而言,天子固已难之矣,而重违其议。是以袁盎之说,得行于其间。使吴楚反,错已身任其危,日夜淬砺,东向而待之,使不至于累其君,则天子将恃之以为无恐,虽有百盎,可得而间哉?


嗟夫!世之君子,欲求非常之功,则无务为自全之计。使错自将而讨吴楚,未必无功,惟其欲自固其身,而天子不悦。奸臣得以乘其隙,错之所以自全者,乃其所以自祸欤!



\chapter*{贾谊论}
\addcontentsline{toc}{chapter}{贾谊论}
\begin{center}
	\textbf{[宋朝]苏轼}
\end{center}

非才之难,所以自用者实难。惜乎!贾生,王者之佐,而不能自用其才也。


夫君子之所取者远,则必有所待;所就者大,则必有所忍。古之贤人,皆负可致之才,而卒不能行其万一者,未必皆其时君之罪,或者其自取也。


愚观贾生之论,如其所言,虽三代何以远过?得君如汉文,犹且以不用死。然则是天下无尧、舜,终不可有所为耶?仲尼圣人,历试于天下,苟非大无道之国,皆欲勉强扶持,庶几一日得行其道。将之荆,先之以冉有,申之以子夏。君子之欲得其君,如此其勤也。孟子去齐,三宿而后出昼,犹曰:“王其庶几召我。”君子之不忍弃其君,如此其厚也。公孙丑问曰:“夫子何为不豫?”孟子曰:“方今天下,舍我其谁哉?而吾何为不豫?”君子之爱其身,如此其至也。夫如此而不用,然后知天下果不足与有为,而可以无憾矣。若贾生者,非汉文之不能用生,生之不能用汉文也。


夫绛侯亲握天子玺而授之文帝,灌婴连兵数十万,以决刘、吕之雌雄,又皆高帝之旧将,此其君臣相得之分,岂特父子骨肉手足哉?贾生,洛阳之少年。欲使其一朝之间,尽弃其旧而谋其新,亦已难矣。为贾生者,上得其君,下得其大臣,如绛、灌之属,优游浸渍而深交之,使天子不疑,大臣不忌,然后举天下而唯吾之所欲为,不过十年,可以得志。安有立谈之间,而遽为人“痛哭”哉!观其过湘为赋以吊屈原,纡郁愤闷,趯然有远举之志。其后以自伤哭泣,至于夭绝。是亦不善处穷者也。夫谋之一不见用,则安知终不复用也?不知默默以待其变,而自残至此。呜呼!贾生志大而量小,才有余而识不足也。


古之人,有高世之才,必有遗俗之累。是故非聪明睿智不惑之主,则不能全其用。古今称苻坚得王猛于草茅之中,一朝尽斥去其旧臣,而与之谋。彼其匹夫略有天下之半,其以此哉!愚深悲生之志,故备论之。亦使人君得如贾生之臣,则知其有狷介之操,一不见用,则忧伤病沮,不能复振。而为贾生者,亦谨其所发哉!



\chapter*{指喻}
\addcontentsline{toc}{chapter}{指喻}
\begin{center}
	\textbf{[明朝]方孝孺}
\end{center}

浦阳郑君仲辨,其容阗然,其色渥然,其气充然,未尝有疾也。左手之拇有疹焉,隆起而粟。君疑之,以示人,人大笑,以为不足患。既三日,聚而如钱。忧之滋甚,又以示人,笑者如初。又三日,拇指大盈握,近拇之指皆为之痛,若剟刺状,肢体心膂无不病者。惧而谋诸医,医视之,惊曰:“此疾之奇者,虽病在指,其实一身病也,不速治,且能伤身。然始发之时,终日可愈;三日,越旬可愈;今疾且成,已非三月不能瘳[。终日可愈,艾可治也;越旬而愈,药可治也;至于既成,甚将延乎肝膈,否亦将为一臂之忧。非有以御其内,其势不止;非有以治其外,疾未易为也。”君从其言,日服汤剂,而傅以善药,果至二月而后瘳,三月而神色始复。


余因是思之:天下之事,常发于至微,而终为大患;始以为不足治,而终至于不可为。当其易也,惜旦夕之力,忽之而不顾;及其既成也,积岁月,疲思虑,而仅克之,如此指者多矣。


盖众人之所可知者,众人之所能治也,其势虽危,而未足深畏。惟萌于不必忧之地,而寓于不可见之初,众人笑而忽之者,此则君子之所深畏也。


昔之天下,有如君之盛壮无疾者乎?爱天下者,有如君之爱身者乎?而可以为天下患者,岂特疮痏之于指乎?君未尝敢忽之,特以不早谋于医,而几至于甚病。况乎视之以至疏之势,重之以疲敝之余,吏之戕摩剥削以速其疾者亦甚;幸其未发,以为无虞而不知畏,此真可谓智也与哉?


余贱不敢谋国,而君虑周行果,非久于布衣者也。《传》不云乎“三折肱而成良医”?君诚有位于时,则宜以拇病为戒。洪武辛酉九月二十六日述。



\chapter*{严先生祠堂记}
\addcontentsline{toc}{chapter}{严先生祠堂记}
\begin{center}
	\textbf{[宋朝]范仲淹}
\end{center}

先生,汉光武之故人也。相尚以道。及帝握《赤符》,乘六龙,得圣人之时,臣妾亿兆,天下孰加焉?惟先生以节高之。既而动星象,归江湖,得圣人之清。泥涂轩冕,天下孰加焉?惟光武以礼下之。


在《蛊》之上九,众方有为,而独“不事王侯,高尚其事”,先生以之。在《屯》之初九,阳德方亨,而能“以贵下贱,大得民也”,光武以之。盖先生之心,出乎日月之上;光武之量,包乎天地之外。微先生,不能成光武之大,微光武,岂能遂先生之高哉?而使贪夫廉,懦夫立,是大有功于名教也。


仲淹来守是邦,始构堂而奠焉,乃复为其后者四家,以奉祠事。又从而歌曰∶“云山苍苍,江水泱泱,先生之风,山高水长!”



\chapter*{捕蛇者说}
\addcontentsline{toc}{chapter}{捕蛇者说}
\begin{center}
	\textbf{[唐朝]柳宗元}
\end{center}

永州之野产异蛇:黑质而白章,触草木尽死;以啮人,无御之者。然得而腊之以为饵,可以已大风、挛踠、瘘疠,去死肌,杀三虫。其始太医以王命聚之,岁赋其二。募有能捕之者,当其租入。永之人争奔走焉。

有蒋氏者,专其利三世矣。问之,则曰:“吾祖死于是,吾父死于是,今吾嗣为之十二年,几死者数矣。”言之貌若甚戚者。余悲之,且曰:“若毒之乎?余将告于莅事者,更若役,复若赋,则何如?”蒋氏大戚,汪然出涕,曰:“君将哀而生之乎?则吾斯役之不幸,未若复吾赋不幸之甚也。向吾不为斯役,则久已病矣。自吾氏三世居是乡,积于今六十岁矣。而乡邻之生日蹙,殚其地之出,竭其庐之入。号呼而转徙,饥渴而顿踣。触风雨,犯寒暑,呼嘘毒疠,往往而死者,相藉也。曩与吾祖居者,今其室十无一焉。与吾父居者,今其室十无二三焉。与吾居十二年者,今其室十无四五焉。非死则徙尔,而吾以捕蛇独存。悍吏之来吾乡,叫嚣乎东西,隳突乎南北;哗然而骇者,虽鸡狗不得宁焉。吾恂恂而起,视其缶,而吾蛇尚存,则弛然而卧。谨食之,时而献焉。退而甘食其土之有,以尽吾齿。盖一岁之犯死者二焉,其余则熙熙而乐,岂若吾乡邻之旦旦有是哉。今虽死乎此,比吾乡邻之死则已后矣,又安敢毒耶?”

余闻而愈悲,孔子曰:“苛政猛于虎也!”吾尝疑乎是,今以蒋氏观之,犹信。呜呼!孰知赋敛之毒有甚是蛇者乎!故为之说,以俟夫观人风者得焉。

(饥渴而顿踣一作:饿渴)


\chapter*{扁鹊见蔡桓公}
\addcontentsline{toc}{chapter}{扁鹊见蔡桓公}
\begin{center}
	\textbf{[春秋战国]韩非}
\end{center}

扁鹊见蔡桓公,立有间,扁鹊曰:“君有疾在腠理,不治将恐深。”桓侯曰:“寡人无疾。”扁鹊出,桓侯曰:“医之好治不病以为功!”

居十日,扁鹊复见,曰:“君之病在肌肤,不治将益深。”桓侯不应。扁鹊出,桓侯又不悦。

居十日,扁鹊复见,曰:“君之病在肠胃,不治将益深。”桓侯又不应。扁鹊出,桓侯又不悦。

居十日,扁鹊望桓侯而还走。桓侯故使人问之,扁鹊曰:“疾在腠理,汤熨之所及也;在肌肤,针石之所及也;在肠胃,火齐之所及也;在骨髓,司命之所属,无奈何也。今在骨髓,臣是以无请也。”

居五日,桓侯体痛,使人索扁鹊,已逃秦矣。桓侯遂死。


\chapter*{口技}
\addcontentsline{toc}{chapter}{口技}
\begin{center}
	\textbf{[清朝]林嗣环}
\end{center}

京中有善口技者。会宾客大宴,于厅事之东北角,施八尺屏障,口技人坐屏障中,一桌、一椅、一扇、一抚尺而已。众宾团坐。少顷,但闻屏障中抚尺一下,满坐寂然,无敢哗者。

遥闻深巷中犬吠,便有妇人惊觉欠伸,其夫呓语。既而儿醒,大啼。夫亦醒。妇抚儿乳,儿含乳啼,妇拍而呜之。又一大儿醒,絮絮不止。当是时,妇手拍儿声,口中呜声,儿含乳啼声,大儿初醒声,夫叱大儿声,一时齐发,众妙毕备。满坐宾客无不伸颈,侧目,微笑,默叹,以为妙绝。

未几,夫齁声起,妇拍儿亦渐拍渐止。微闻有鼠作作索索,盆器倾侧,妇梦中咳嗽。宾客意少舒,稍稍正坐。

忽一人大呼:“火起”,夫起大呼,妇亦起大呼。两儿齐哭。俄而百千人大呼,百千儿哭,百千犬吠。中间力拉崩倒之声,火爆声,呼呼风声,百千齐作;又夹百千求救声,曳屋许许声,抢夺声,泼水声。凡所应有,无所不有。虽人有百手,手有百指,不能指其一端;人有百口,口有百舌,不能名其一处也。于是宾客无不变色离席,奋袖出臂,两股战战,几欲先走。

忽然抚尺一下,群响毕绝。撤屏视之,一人、一桌、一椅、一扇、一抚尺而已。


\chapter*{种树郭橐驼传}
\addcontentsline{toc}{chapter}{种树郭橐驼传}
\begin{center}
	\textbf{[唐朝]柳宗元}
\end{center}

郭橐驼,不知始何名。病偻,隆然伏行,有类橐驼者,故乡人号之“驼”。驼闻之,曰:“甚善。名我固当。”因舍其名,亦自谓橐驼云。

其乡曰丰乐乡,在长安西。驼业种树,凡长安豪富人为观游及卖果者,皆争迎取养。视驼所种树,或移徙,无不活,且硕茂,早实以蕃。他植者虽窥伺效慕,莫能如也。

有问之,对曰:“橐驼非能使木寿且孳也,能顺木之天,以致其性焉尔。凡植木之性,其本欲舒,其培欲平,其土欲故,其筑欲密。既然已,勿动勿虑,去不复顾。其莳也若子,其置也若弃,则其天者全而其性得矣。故吾不害其长而已,非有能硕茂之也;不抑耗其实而已,非有能早而蕃之也。他植者则不然,根拳而土易,其培之也,若不过焉则不及。苟有能反是者,则又爱之太恩,忧之太勤,旦视而暮抚,已去而复顾,甚者爪其肤以验其生枯,摇其本以观其疏密,而木之性日以离矣。虽曰爱之,其实害之;虽曰忧之,其实仇之,故不我若也。吾又何能为哉!”

问者曰:“以子之道,移之官理,可乎?”驼曰:“我知种树而已,官理,非吾业也。然吾居乡,见长人者好烦其令,若甚怜焉,而卒以祸。旦暮吏来而呼曰:‘官命促尔耕,勖尔植,督尔获,早缫而绪,早织而缕,字而幼孩,遂而鸡豚。’鸣鼓而聚之,击木而召之。吾小人辍飧饔以劳吏者,且不得暇,又何以蕃吾生而安吾性耶?故病且怠。若是,则与吾业者其亦有类乎?”

问者曰:“嘻,不亦善夫!吾问养树,得养人术。”传其事以为官戒。


\chapter*{黔之驴}
\addcontentsline{toc}{chapter}{黔之驴}
\begin{center}
	\textbf{[唐朝]柳宗元}
\end{center}

黔无驴,有好事者船载以入。至则无可用,放之山下。虎见之,庞然大物也,以为神,蔽林间窥之。稍出近之,慭慭然,莫相知。

他日,驴一鸣,虎大骇,远遁;以为且噬己也,甚恐。然往来视之,觉无异能者;益习其声,又近出前后,终不敢搏。稍近,益狎,荡倚冲冒。驴不胜怒,蹄之。虎因喜,计之曰:“技止此耳!”因跳踉大㘎,断其喉,尽其肉,乃去。

噫!形之庞也类有德,声之宏也类有能。向不出其技,虎虽猛,疑畏,卒不敢取。今若是焉,悲夫!


\chapter*{道山亭}
\addcontentsline{toc}{chapter}{道山亭}
\begin{center}
	\textbf{[宋朝]曾巩}
\end{center}

闽,故隶周者也。至秦,开其地,列于中国,始并为闽中郡。自粤之太末,与吴之豫章,为其通路。


其路在闽者,陆出则阸于两山之间,山相属无间断,累数驿乃一得平地,小为县,大为州,然其四顾亦山也。其途或逆坂如缘絙,或垂崖如一发,或侧径钩出于不测之溪上:皆石芒峭发,择然后可投步。负戴者虽其土人,犹侧足然后能进。非其土人,罕不踬也。其溪行,则水皆自高泻下,石错出其间,如林立,如士骑满野,千里下上,不见首尾。水行其隙间,或衡缩蟉糅,或逆走旁射,其状若蚓结,若虫镂,其旋若轮,其激若矢。舟溯沿者,投便利,失毫分,辄破溺。虽其土长川居之人,非生而习水事者,不敢以舟揖自任也。其水陆之险如此。汉尝处其众江淮之间而虚其地,盖以其陿多阻,岂虚也哉?


福州治侯官,于闽为土中,所谓闽中也。其地于闽为最平以广,四出之山皆远,而长江(闽江)其南,大海在其东,其城之内外皆涂,旁有沟,沟通潮汐,舟载者昼夜属于门庭。麓多桀木,而匠多良能,人以屋室巨丽相矜,虽下贫必丰其居,而佛、老子之徒,其宫又特盛。城之中三山,西曰闽山,东曰九仙山,北曰粤王山,三山者鼎趾立。其附山,盖佛、老子之宫以数十百,其瑰诡殊绝之状,盖已尽人力。


光禄卿、直昭文馆程公为是州,得闽山嵚崟之际,为亭于其处,其山川之胜,城邑之大,宫室之荣,不下簟席而尽于四瞩。程公以谓在江海之上,为登览之观,可比于道家所谓蓬莱、方丈、瀛州之山,故名之曰“道山之亭”。闽以险且远,故仕者常惮往,程公能因其地之善,以寓其耳目之乐,非独忘其远且险,又将抗其思于埃壒之外,其志壮哉!


程公于是州以治行闻,既新其城,又新其学,而其余功又及于此。盖其岁满就更广州,拜谏议大夫,又拜给事中、集贤殿修撰,今为越州,字公辟,名师孟云。



\chapter*{人间词话七则}
\addcontentsline{toc}{chapter}{人间词话七则}
\begin{center}
	\textbf{[近现代]王国维}
\end{center}

有有我之境,有无我之境。“泪眼问花花不语,乱红飞过秋千去。”“可堪孤馆闭春寒,杜鹃声里斜阳暮。”有我之境也。“采菊东篱下,悠然见南山。”“寒波澹澹起,白鸟悠悠下。”无我之境也。有我之境,以我观物,故物我皆著我之色彩。无我之境,以物观物,故不知何者为我,何者为物。古人为词,写有我之境者为多,然未始不能写无我之境,此在豪杰之士能自树立耳。

境非独谓景物也。喜怒哀乐,亦人心中之一境界。故能写真景物,真感情者,谓之有境界。否则谓之无境界。

境界有大小,不以是而分优劣。“细雨鱼儿出,微风燕子斜”何遽不若“落日照大旗,马鸣风萧萧”。“宝帘闲挂小银钩”何遽不若“雾失楼台,月迷津渡”也。

词至李后主而眼界始大,感慨遂深,遂变伶工之词而为士大夫之词。周介存置诸温韦之下,可为颠倒黑白矣。“自是人生长恨水长东”、“流水落花春去也,天上人间”,《金荃》《浣花》,能有此气象耶?

古今之成大事业、大学问者,罔不经过三种之境界:“昨夜西风凋碧树。独上高楼,望尽天涯路。”此第一境界也。“衣带渐宽终不悔,为伊消得人憔悴。”此第二境界也。“众里寻他千百度,蓦然回首,那人却在,灯火阑珊处。”此第三境界也。此等语皆非大词人不能道。然遽以此意解释诸词,恐为晏欧诸公所不许也。

大家之作,其言情也必沁人心脾,其写景也必豁人耳目。其辞脱口而出,无矫揉妆束之态。以其所见者真,所知者深也。诗词皆然。持此以衡古今之作者,可无大误也。

诗人对宇宙人生,须入乎其内,又须出乎其外。入乎其内,故能写之。出乎其外,故能观之。入乎其内,故有生气。出乎其外,故有高致。美成能入而不出。白石以降,于此二事皆未梦见。


\chapter*{蚊对}
\addcontentsline{toc}{chapter}{蚊对}
\begin{center}
	\textbf{[明朝]方孝孺}
\end{center}

天台生困暑,夜卧絺帷中,童子持翣扬于前,适甚,就睡。久之,童子亦睡,投翣倚床,其音如雷。生惊寤,以为风雨且至也,抱膝而坐。


俄而耳旁闻有飞鸣声,如歌如诉,如怨如慕,拂肱刺肉,扑股噆面,毛发尽竖,肌肉欲颤。两手交拍,掌湿如汗,引而嗅之,赤血腥然也。大愕,不知所为。蹴童子,呼曰:“吾为物所苦,亟起索烛照!”烛至,絺帷尽张,蚊数千皆集帷旁,见烛乱散,如蚁如蝇,利嘴饫腹,充赤圆红。生骂童子曰:“此非噆吾血者耶?皆尔不谨,褰帷而放之入!且彼异类也,防之苟至,乌能为人害?”童子拔蒿束之,置火于端,其烟勃郁,左麾右旋,绕床数匝,逐蚊出门。复于生曰:“可以寝矣,蚊已去矣!”


生乃拂席将寝,呼天而叹曰:“天胡产此微物而毒人乎?”童子闻之,哑尔笑曰:“子何待己之太厚,而尤天之太固也!夫覆载之间,二气絪缊,赋形受质,人物是分。大之为犀象,怪之为蛟龙,暴之为虎豹,驯之为糜鹿与庸狨,羽毛而为禽为兽,裸身而为人为虫,莫不皆有所养。虽巨细修短之不同,然寓形于其中,则一也。自我而观之,则人贵而物贱;自天地而观之,果孰贵而孰贱耶?今人乃自贵其贵,号为长雄;水陆之物,有生之类,莫不高罗而卑网(,山贡而海供,蛙黾莫逃其命,鸿雁莫匿其踪。其食乎物者,可谓泰矣,而物独不可食于人耶?兹夕蚊一举喙,即号天而诉之;使物为人所食者,亦皆呼号告于天,则天之罚人,又当何如耶?且物之食于人,人之食于物,异类也,犹可言也。而蚊且犹畏谨恐惧,白昼不敢露其形,瞰人之不见,乘人之困怠,而后有求焉。今有同类者,啜粟而饮汤,同也;畜妻而育子,同也;衣冠仪貌,无不同者。白昼俨然乘其同类之间而陵之,吮其膏而盬其脑,使其饿踣于草野,离流于道路,呼天之声相接也,而且无恤之者。今子一为蚊所噆,而寝辄不安;闻同类之相噆,而若无闻。岂君子先人后身之道耶?”


天台生于是投枕于地,叩心太息,披衣出户,坐以终夕。



\chapter*{诫外甥书}
\addcontentsline{toc}{chapter}{诫外甥书}
\begin{center}
	\textbf{[三国]诸葛亮}
\end{center}

夫志当存高远,慕先贤,绝情欲,弃凝滞。使庶几之志揭然有所存,恻然有所感。忍屈伸,去细碎,广咨询,除嫌吝,虽有淹留,何损于美趣,何患于不济。若志不强毅,意不慷慨,徒碌碌滞于俗,默默束于情,永窜伏于凡庸,不免于下流矣。



\chapter*{深虑论}
\addcontentsline{toc}{chapter}{深虑论}
\begin{center}
	\textbf{[明朝]方孝孺}
\end{center}

虑天下者,常图其所难而忽其所易;备其所可畏而遗其所不疑。然而祸常发于所忽之中,而乱常起于不足疑之事。岂其虑之未周与?盖虑之所能及者,人事之宜然,而出于智力之所不及者,天道也。


当秦之世,而灭诸侯,一天下。而其心以为周之亡在乎诸侯之强耳,变封建而为郡县。方以为兵革不可复用,天子之位可以世守,而不知汉帝起陇亩之中,而卒亡秦之社稷。汉惩秦之孤立,于是大建庶孽而为诸侯,以为同姓之亲,可以相继而无变,而七国萌篡弑之谋。武、宣以后,稍削析之而分其势,以为无事矣,而王莽卒移汉祚。光武之惩哀、平,魏之惩汉,晋之惩魏,各惩其所由亡而为之备。而其亡也,盖出于所备之外。唐太宗闻武氏之杀其子孙,求人于疑似之际而除之,而武氏日侍其左右而不悟。宋太祖见五代方镇之足以制其君尽释其兵权,使力弱而易制,而不知子孙卒困于敌国。


此其人皆有出人之智、盖世之才,其于治乱存亡之几,思之详而备之审矣。虑切于此而祸兴于彼,终至乱亡者,何哉?盖智可以谋人,而不可以谋天。良医之子,多死于病;良巫之子,多死于鬼。岂工于活人,而拙于谋子也哉?乃工于谋人,而拙于谋天也。古之圣人,知天下后世之变,非智虑之所能周,非法术之所能制,不敢肆其私谋诡计,而唯积至诚,用大德以结乎天心,使天眷其德,若慈母之保赤子而不忍释。故其子孙,虽有至愚不肖者足以亡国,而天卒不忍遽亡之。此虑之远者也。


夫苟不能自结于天,而欲以区区之智笼络当世之务,而必后世之无危亡,此理之所必无者,而岂天道哉!



\chapter*{曾子杀猪}
\addcontentsline{toc}{chapter}{曾子杀猪}
\begin{center}
	\textbf{[春秋战国]韩非}
\end{center}

曾子之妻之市,其子随之而泣。其母曰:“汝还,顾反为女杀彘。”妻适市来,曾子欲捕彘杀之。妻止之曰:“特与婴儿戏耳。”曾子曰:“婴儿非与戏也。婴儿非有知也,待父母而学者也,听父母之教。今子欺之,是教子欺也。母欺子,子而不信其母,非所以成教也。”遂烹彘也。


\chapter*{智子疑邻}
\addcontentsline{toc}{chapter}{智子疑邻}
\begin{center}
	\textbf{[春秋战国]韩非}
\end{center}

宋有富人,天雨墙坏。其子曰:“不筑,必将有盗。”其邻人之父亦云。暮而果大亡其财,其家甚智其子,而疑邻人之父。


\chapter*{登泰山记}
\addcontentsline{toc}{chapter}{登泰山记}
\begin{center}
	\textbf{[清朝]姚鼐}
\end{center}

泰山之阳,汶水西流;其阴,济水东流。阳谷皆入汶,阴谷皆入济。当其南北分者,古长城也。最高日观峰,在长城南十五里。

余以乾隆三十九年十二月,自京师乘风雪,历齐河、长清,穿泰山西北谷,越长城之限,至于泰安。是月丁未,与知府朱孝纯子颍由南麓登。四十五里,道皆砌石为磴,其级七千有余。

泰山正南面有三谷。中谷绕泰安城下,郦道元所谓环水也。余始循以入,道少半,越中岭,复循西谷,遂至其巅。古时登山,循东谷入,道有天门。东谷者,古谓之天门溪水,余所不至也。今所经中岭及山巅崖限当道者,世皆谓之天门云。道中迷雾冰滑,磴几不可登。及既上,苍山负雪,明烛天南;望晚日照城郭,汶水、徂徕如画,而半山居雾若带然。

戊申晦,五鼓,与子颖坐日观亭,待日出。大风扬积雪击面。亭东自足下皆云漫。稍见云中白若摴蒱数十立者,山也。极天云一线异色,须臾成五彩。日上,正赤如丹,下有红光,动摇承之。或曰,此东海也。回视日观以西峰,或得日,或否,绛皓驳色,而皆若偻。

亭西有岱祠,又有碧霞元君祠;皇帝行宫在碧霞元君祠东。是日,观道中石刻,自唐显庆以来,其远古刻尽漫失。僻不当道者,皆不及往。

山多石,少土;石苍黑色,多平方,少圜。少杂树,多松,生石罅,皆平顶。冰雪,无瀑水,无鸟兽音迹。至日观数里内无树,而雪与人膝齐。

桐城姚鼐记。


\chapter*{永州八记}
\addcontentsline{toc}{chapter}{永州八记}
\begin{center}
	\textbf{[唐朝]柳宗元}
\end{center}

\section*{始得西山宴游记}
\addcontentsline{toc}{section}{始得西山宴游记}


自余为僇人,居是州。恒惴慄。时隙也,则施施而行,漫漫而游。日与其徒上高山,入深林,穷回溪,幽泉怪石,无远不到。到则披草而坐,倾壶而醉。醉则更相枕以卧,卧而梦。意有所极,梦亦同趣。觉而起,起而归。以为凡是州之山水有异态者,皆我有也,而未始知西山之怪特。

今年九月二十八日,因坐法华西亭,望西山,始指异之。遂命仆人过湘江,缘染溪,斫榛莽,焚茅茷,穷山之高而上。攀援而登,箕踞而遨,则凡数州之土壤,皆在衽席之下。其高下之势,岈然洼然,若垤若穴,尺寸千里,攒蹙累积,莫得遁隐。萦青缭白,外与天际,四望如一。然后知是山之特立,不与培塿为类,悠悠乎与颢气俱,而莫得其涯;洋洋乎与造物者游,而不知其所穷。引觞满酌,颓然就醉,不知日之入。苍然暮色,自远而至,至无所见,而犹不欲归。心凝形释,与万化冥合。然后知吾向之未始游,游于是乎始,故为之文以志。是岁,元和四年也。



\section*{钴鉧潭记}
\addcontentsline{toc}{section}{钴鉧潭记}


钴鉧潭,在西山西。其始盖冉水自南奔注,抵山石,屈折东流;其颠委势峻,荡击益暴,啮其涯,故旁广而中深,毕至石乃止;流沫成轮,然后徐行。其清而平者,且十亩。有树环焉,有泉悬焉。

其上有居者,以予之亟游也,一旦款门来告曰:“不胜官租、私券之委积,既芟山而更居,愿以潭上田贸财以缓祸。”

予乐而如其言。则崇其台,延其槛,行其泉于高者而坠之潭,有声潀然。尤与中秋观月为宜,于以见天之高,气之迥。孰使予乐居夷而忘故土者,非兹潭也欤?



\section*{钴鉧潭西小丘记}
\addcontentsline{toc}{section}{钴鉧潭西小丘记}


得西山后八日,寻山口西北道二百步,有得钴鉧潭。潭西二十五步,当湍而浚者为鱼梁。梁之上有丘焉,生竹树。其石之突怒偃蹇,负土而出,争为奇状者,殆不可数。其嵚然相累而下者,若牛马之饮于溪;其冲然角列而上者,若熊罴之登于山。丘之小不能一亩,可以笼而有之。问其主,曰:“唐氏之弃地,货而不售。”问其价,曰:“止四百。”余怜而售之。李深源、元克已时同游,皆大喜,出自意外。即更取器用,铲刈秽草,伐去恶木,烈火而焚之。嘉木立,美竹露。奇石显。由其中以望,则山之高,云之浮,溪之流,鸟兽之遨游,举熙熙然回巧献技,以效兹丘之下。枕席而卧,则清冷冷状与目谋,瀯瀯之声与耳谋,悠然而虚者与神谋,渊然而静者与心谋。不匝旬而得异地者二,虽古好古之士,或未能至焉。

噫!以兹丘之胜,致之沣、镐、、杜,则贵游之士争买者,日增千金而愈不可得。今弃是州也,农夫渔父过而陋之,贾四百,连岁不能售。而我与深源、克已独喜得之,是其果有遭乎!书于石,所以贺兹丘之遭也。



\section*{至小丘西小石潭记}
\addcontentsline{toc}{section}{至小丘西小石潭记}


从小丘西行百二十步,隔篁竹,闻水声,如鸣佩环,心乐之。伐竹取道,下见小潭,水尤清冽。全石以为底,近岸,卷石底以出,为坻,为屿,为嵁,为岩。青树翠蔓,蒙络摇缀,参差披拂。

潭中鱼可百许头,皆若空游无所依。日光下澈,影布石上,佁然不动;俶尔远逝,往来翕忽,似与游者相乐。

潭西南而望,斗折蛇行,明灭可见。其岸势犬牙差互,不可知其源。

坐潭上,四面竹树环合,寂寥无人,凄神寒骨,悄怆幽邃。以其境过清,不可久居,乃记之而去。

同游者:吴武陵,龚古,余弟宗玄。隶而从者,崔氏二小生:曰恕己,曰奉壹。



\section*{袁家渴记}
\addcontentsline{toc}{section}{袁家渴记}


由冉溪西南水行十里,山水之可取者五,莫若钴鉧潭。由溪口而西,陆行,可取者八九,莫若西山。由朝阳岩东南水行,至芜江,可取者三,莫若袁家渴。皆永中幽丽奇处也。

楚越之间方言,谓水之反流为“渴”。渴上与南馆高嶂合,下与百家濑合。其中重洲小溪,澄潭浅渚,间厕曲折,平者深墨,峻者沸白。舟行若穷,忽而无际。

有小山出水中,皆美石,上生青丛,冬夏常蔚然。其旁多岩词,其下多白砾,其树多枫柟石楠,樟柚,草则兰芷。又有奇卉,类合欢而蔓生,轇轕水石。

每风自四山而下,振动大木,掩苒众草,纷红骇绿,蓊葧香气,冲涛旋濑,退贮溪谷,摇飃葳蕤,与时推移。其大都如此,余无以穷其状。

永之人未尝游焉,余得之不敢专焉,出而传于世。其地主袁氏。故以名焉。



\section*{石渠记}
\addcontentsline{toc}{section}{石渠记}


自渴西南行不能百步,得石渠,民桥其上。有泉幽幽然,其鸣乍大乍细。渠之广或咫尺,或倍尺,其长可十许步。其流抵大石,伏出其下。踰石而往,有石泓,昌蒲被之,青鲜环周。又折西行,旁陷岩石下,北堕小潭。潭幅员减百尺,清深多倏鱼。又北曲行纡余,睨若无穷,然卒入于渴。其侧皆诡石、怪木、奇卉、美箭,可列坐而庥焉。风摇其巅,韵动崖谷。视之既静,其听始远。

予从州牧得之。揽去翳朽,决疏土石,既崇而焚,既釃釃而盈。惜其未始有传焉者,故累记其所属,遗之其人,书之其阳,俾后好事者求之得以易。

元和七年正月八日,鷁渠至大石。十月十九日,踰石得石泓小潭,渠之美于是始穷也。



\section*{石涧记}
\addcontentsline{toc}{section}{石涧记}


石渠之事既穷,上由桥西北下土山之阴,民又桥焉。其水之大,倍石渠三之一,亘石为底,达于两涯。若床若堂,若陈筳席,若限阃奥。水平布其上,流若织文,响若操琴。揭跣而往,折竹扫陈叶,排腐木,可罗胡床十八九居之。交络之流,触激之音,皆在床下;翠羽之水,龙鳞之石,均荫其上。古之人其有乐乎此耶?后之来者有能追予之践履耶?得之日,与石渠同。

由渴而来者,先石渠,后石涧;由百家濑上而来者,先石涧,后石渠。涧之可穷者,皆出石城村东南,其间可乐者数焉。其上深山幽林逾峭险,道狭不可穷也。



\section*{小石城山记}
\addcontentsline{toc}{section}{小石城山记}


自西山道口径北踰黄茅岭而下,有二道:其一西出,寻之无所得;其一少北而东,不过四十丈,土断二川分,有积石横当其垠。其上为睥睨梁欐之形;其旁出堡坞,有若门焉,窥之正黑,投以小石,洞然有水声,其响之激越,良久乃已。环之可上,望甚远。无土壤而生嘉树美箭,益奇而坚,奇疏数偃仰,类智者所施也。

噫!吾疑造物者之有无久矣,及是,愈以为诚有。又怪其不为之中州而列是夷狄,更千百年不得一售其伎,是固劳而无用,神者倘不宜如是,则其果无乎?或曰:以慰夫贤而辱于此者。或曰:其气之灵,不为伟人而独为是物,故楚之南少人而多石。是二者余未信之。


\chapter*{临江之麋}
\addcontentsline{toc}{chapter}{临江之麋}
\begin{center}
	\textbf{[唐朝]柳宗元}
\end{center}

临江之人,畋得麋麑,畜之。入门,群犬垂涎,扬尾皆来。其人怒,怛之。自是日抱就犬,习示之,使勿动。稍使与之戏。积久,犬皆如人意。

麋麑稍大,忘己之麋也,以为犬良我友,抵触偃仆,益狎。犬畏主人,于之俯仰甚善,然时啖其舌。

三年,麋出门,见外犬在道甚众,走欲与为戏。外犬见而喜且怒,共杀食之,狼藉道上。麋至死不悟。



\chapter*{促织}
\addcontentsline{toc}{chapter}{促织}
\begin{center}
	\textbf{[清朝]蒲松龄}
\end{center}

宣德间,宫中尚促织之戏,岁征民间。此物故非西产;有华阴令欲媚上官,以一头进,试使斗而才,因责常供。令以责之里正。市中游侠儿得佳者笼养之,昂其直,居为奇货。里胥猾黠,假此科敛丁口,每责一头,辄倾数家之产。

邑有成名者,操童子业,久不售。为人迂讷,遂为猾胥报充里正役,百计营谋不能脱。不终岁,薄产累尽。会征促织,成不敢敛户口,而又无所赔偿,忧闷欲死。妻曰:“死何裨益?不如自行搜觅,冀有万一之得。”成然之。早出暮归,提竹筒丝笼,于败堵丛草处,探石发穴,靡计不施,迄无济。即捕得三两头,又劣弱不中于款。宰严限追比,旬余,杖至百,两股间脓血流离,并虫亦不能行捉矣。转侧床头,惟思自尽。

时村中来一驼背巫,能以神卜。成妻具资诣问。见红女白婆,填塞门户。入其舍,则密室垂帘,帘外设香几。问者爇香于鼎,再拜。巫从旁望空代祝,唇吻翕辟,不知何词。各各竦立以听。少间,帘内掷一纸出,即道人意中事,无毫发爽。成妻纳钱案上,焚拜如前人。食顷,帘动,片纸抛落。拾视之,非字而画:中绘殿阁,类兰若;后小山下,怪石乱卧,针针丛棘,青麻头伏焉;旁一蟆,若将跃舞。展玩不可晓。然睹促织,隐中胸怀。折藏之,归以示成。

成反复自念,得无教我猎虫所耶?细瞻景状,与村东大佛阁逼似。乃强起扶杖,执图诣寺后,有古陵蔚起。循陵而走,见蹲石鳞鳞,俨然类画。遂于蒿莱中侧听徐行,似寻针芥。而心目耳力俱穷,绝无踪响。冥搜未已,一癞头蟆猝然跃去。成益愕,急逐趁之,蟆入草间。蹑迹披求,见有虫伏棘根。遽扑之,入石穴中。掭以尖草,不出;以筒水灌之,始出,状极俊健。逐而得之。审视,巨身修尾,青项金翅。大喜,笼归,举家庆贺,虽连城拱璧不啻也。上于盆而养之,蟹白栗黄,备极护爱,留待限期,以塞官责。

成有子九岁,窥父不在,窃发盆。虫跃掷径出,迅不可捉。及扑入手,已股落腹裂,斯须就毙。儿惧,啼告母。母闻之,面色灰死,大惊曰:“业根,死期至矣!而翁归,自与汝复算耳!”儿涕而去。

未几,成归,闻妻言,如被冰雪。怒索儿,儿渺然不知所往。既而得其尸于井,因而化怒为悲,抢呼欲绝。夫妻向隅,茅舍无烟,相对默然,不复聊赖。日将暮,取儿藁葬。近抚之,气息惙然。喜置榻上,半夜复苏。夫妻心稍慰,但儿神气痴木,奄奄思睡。成顾蟋蟀笼虚,则气断声吞,亦不复以儿为念,自昏达曙,目不交睫。东曦既驾,僵卧长愁。忽闻门外虫鸣,惊起觇视,虫宛然尚在。喜而捕之,一鸣辄跃去,行且速。覆之以掌,虚若无物;手裁举,则又超忽而跃。急趋之,折过墙隅,迷其所在。徘徊四顾,见虫伏壁上。审谛之,短小,黑赤色,顿非前物。成以其小,劣之。惟彷徨瞻顾,寻所逐者。壁上小虫忽跃落襟袖间,视之,形若土狗,梅花翅,方首,长胫,意似良。喜而收之。将献公堂,惴惴恐不当意,思试之斗以觇之。

村中少年好事者,驯养一虫,自名“蟹壳青”,日与子弟角,无不胜。欲居之以为利,而高其直,亦无售者。径造庐访成,视成所蓄,掩口胡卢而笑。因出己虫,纳比笼中。成视之,庞然修伟,自增惭怍,不敢与较。少年固强之。顾念蓄劣物终无所用,不如拼博一笑,因合纳斗盆。小虫伏不动,蠢若木鸡。少年又大笑。试以猪鬣毛撩拨虫须,仍不动。少年又笑。屡撩之,虫暴怒,直奔,遂相腾击,振奋作声。俄见小虫跃起,张尾伸须,直龁敌领。少年大骇,急解令休止。虫翘然矜鸣,似报主知。成大喜。方共瞻玩,一鸡瞥来,径进以啄。成骇立愕呼,幸啄不中,虫跃去尺有咫。鸡健进,逐逼之,虫已在爪下矣。成仓猝莫知所救,顿足失色。旋见鸡伸颈摆扑,临视,则虫集冠上,力叮不释。成益惊喜,掇置笼中。

翼日进宰,宰见其小,怒呵成。成述其异,宰不信。试与他虫斗,虫尽靡。又试之鸡,果如成言。乃赏成,献诸抚军。抚军大悦,以金笼进上,细疏其能。既入宫中,举天下所贡蝴蝶、螳螂、油利挞、青丝额一切异状遍试之,莫出其右者。每闻琴瑟之声,则应节而舞。益奇之。上大嘉悦,诏赐抚臣名马衣缎。抚军不忘所自,无何,宰以卓异闻。宰悦,免成役。又嘱学使俾入邑庠。后岁余,成子精神复旧,自言身化促织,轻捷善斗,今始苏耳。抚军亦厚赉成。不数年,田百顷,楼阁万椽,牛羊蹄躈各千计;一出门,裘马过世家焉。

异史氏曰:“天子偶用一物,未必不过此已忘;而奉行者即为定例。加以官贪吏虐,民日贴妇卖儿,更无休止。故天子一跬步,皆关民命,不可忽也。独是成氏子以蠹贫,以促织富,裘马扬扬。当其为里正,受扑责时,岂意其至此哉!天将以酬长厚者,遂使抚臣、令尹,并受促织恩荫。闻之:一人飞升,仙及鸡犬。信夫!”


\chapter*{龙说}
\addcontentsline{toc}{chapter}{龙说}
\begin{center}
	\textbf{[唐朝]韩愈}
\end{center}

龙嘘气成云,云固弗灵于龙也。然龙乘是气,茫洋穷乎玄间,薄日月,伏光景,感震电,神变化,水下土,汩陵谷,云亦灵怪矣哉。


云,龙之所能使为灵也。若龙之灵,则非云之所能使为灵也。然龙弗得云,无以神其灵矣。失其所凭依,信不可欤。异哉!其所凭依,乃其所自为也。


易曰:“云从龙。”既曰:“龙,云从之矣。”





\chapter*{黄冈竹楼记}
\addcontentsline{toc}{chapter}{黄冈竹楼记}
\begin{center}
	\textbf{[宋朝]王禹偁}
\end{center}

黄冈之地多竹,大者如椽。竹工破之,刳去其节,用代陶瓦。比屋皆然,以其价廉而工省也。

子城西北隅,雉堞圮毁,蓁莽荒秽,因作小楼二间,与月波楼通。远吞山光,平挹江濑,幽阒辽夐,不可具状。夏宜急雨,有瀑布声;冬宜密雪,有碎玉声。宜鼓琴,琴调虚畅;宜咏诗,诗韵清绝;宜围棋,子声丁丁然;宜投壶,矢声铮铮然;皆竹楼之所助也。

公退之暇,被鹤氅衣,戴华阳巾,手执《周易》一卷,焚香默坐,消遣世虑。江山之外,第见风帆沙鸟,烟云竹树而已。待其酒力醒,茶烟歇,送夕阳,迎素月,亦谪居之胜概也。彼齐云、落星,高则高矣;井干、丽谯,华则华矣;止于贮妓女,藏歌舞,非骚人之事,吾所不取。

吾闻竹工云:“竹之为瓦,仅十稔;若重覆之,得二十稔。”噫!吾以至道乙未岁,自翰林出滁上,丙申,移广陵;丁酉又入西掖;戊戌岁除日岁除日,新旧岁之交,即除夕。,有齐安之命;己亥闰三月到郡。四年之间,奔走不暇;未知明年又在何处,岂惧竹楼之易朽乎!幸后之人与我同志,嗣而葺之,庶斯楼之不朽也!

咸平二年八月十五日记。


\chapter*{醒心亭记}
\addcontentsline{toc}{chapter}{醒心亭记}
\begin{center}
	\textbf{[宋朝]曾巩}
\end{center}

滁州之西南,泉水之涯,欧阳公作州之二年,构亭曰“丰乐”,自为记,以见其名义。既又直丰乐之东几百步,得山之高,构亭曰“醒心”,使巩记之。


凡公与州之宾客者游焉,则必即丰乐以饮。或醉且劳矣,则必即醒心而望,以见夫群山之相环,云烟之相滋,旷野之无穷,草树众而泉石嘉,使目新乎其所睹,耳新乎其所闻,则其心洒然而醒,更欲久而忘归也,故即其事之所以然而为名,取韩子退之《北湖》之诗云。噫!其可谓善取乐于山泉之间,而名之以见其实,又善者矣。


虽然,公之作乐,吾能言之,吾君优游而无为于上,吾民给足而无憾于下。天下之学者,皆为才且良;夷狄鸟兽草木之生者,皆得其宜,公乐也。一山之隅,一泉之旁,岂公乐哉?乃公所寄意于此也。


若公之贤,韩子殁数百年而始有之。今同游之宾客,尚未知公之难遇也。后百千年,有慕公之为人,而览公之迹,思欲见之,有不可及之叹,然后知公之难遇也。则凡同游于此者,其可不喜且幸欤!而巩也,又得以文词托名于公文之次,其又不喜且幸欤!


庆历七年八月十五日记.



\chapter*{大人先生传}
\addcontentsline{toc}{chapter}{大人先生传}
\begin{center}
	\textbf{[三国]阮籍}
\end{center}

大人先生盖老人也,不知姓字。陈天地之始,言神农黄帝之事,昭然也;莫知其生年之数。尝居苏门之山,故世或谓之闲。养性延寿,与自然齐光。其视尧、舜之所事,若手中耳。以万里为一步,以千岁为一朝。行不赴而居不处,求乎大道而无所寓。先生以应变顺和,天地为家,运去势颓,魁然独存。自以为能足与造化推移,故默探道德,不与世同。自好者非之,无识者怪之,不知其变化神微也。而先生不以世之非怪而易其务也。先生以为中区之在天下,曾不若蝇蚊之著帷,故终不以为事,而极意乎异方奇域,游览观乐非世所见,徘徊无所终极。遗其书於苏门之山而去。天下莫知其所如往也。


或遗大人先生书,曰:“天下之贵,莫贵於君子。服有常色,貌有常则,言有常度,行有常式。立则磬折,拱若抱鼓。动静有节,趋步商羽,进退周旋,咸有规矩。心若怀冰,战战栗栗。束身修行,日慎一日。择地而行,唯恐遗失。颂周、孔之遗训,叹唐、虞之道德,唯法是修,为礼是克。手执珪璧,足履绳墨,行欲为目前检,言欲为无穷则。少称乡闾,长闻邦国,上欲图三公,下不失九州牧。故挟金玉,垂文组,享尊位,取茅土。扬声名於后世,齐功德於往古。奉事君上,牧养百姓。退营私家,育长妻子。卜吉宅,虑乃亿祉。远祸近福,永坚固己。此诚士君子之高致,古今不易之美行也,今先生乃披发而居巨海之中,与若君子者远,吾恐世之叹先生而非之也。行为世所笑,身无自由达,则可谓耻辱矣。身处困苦之地,而行为世俗之所笑,吾为先生不取也。”


於是大人先生乃逌然而叹,假云霓而应之曰:“若之云尚何通哉!夫大人者,乃与造物同体,天地并生,逍遥浮世,与道俱成,变化散聚,不常其形。天地制域於内,而浮明开达於外。天地之永,固非世俗之所及也。吾将为汝言之。


“往者天尝在下,地尝在上,反覆颠倒,未之安固。焉得不失度式而常之?天因地动,山陷川起,云散震坏,六合失理,汝又焉得择地而行,趋步商羽?往者群气争存,万物死虑,支体不从,身为泥土,根拔枝殊,咸失其所,汝又焉得束身修行,磬折抱鼓?李牧功而身死,伯宗忠而世绝,进求利而丧身,营爵赏而家灭,汝又焉得挟金玉万亿,只奉君上,而全妻子乎?


“且汝独不见夫虱之处於褌中,逃乎深缝,匿乎坏絮,自以为吉宅也。行不敢离缝际,动不敢出褌裆,自以为得绳墨也。饥则啮人,自以为无穷食也。然炎丘火流,焦邑灭都,群虱死於褌中而不能出。汝君子之处区内,亦何异夫虱之处褌中乎?悲夫!而乃自以为远祸近幅,坚无穷也。亦观夫阳乌游於尘外,而鹪鹩戏于蓬艾,小大固不相及,汝又何以为若君子闻於余乎?


“且近者,夏丧於商,周播之刘,耿薄为墟,丰、镐成丘。至人未一顾,而世代相酬。厥居未定,他人已有。汝之茅土,谁将与久?是以至人不处而居,不修而治,日月为正,阴阳为期,岂吝情乎世,系累於一时,乘东云,驾西风,与阴守雌,据阳为雄。志得欲从,物莫之穷。又何不能自达而畏夫世笑哉?


“昔者天地开辟,万物并生。大者恬其性,细者静其形。阴藏其气,阳发其精,害无所避,利无所争。放之不失,收之不盈;亡不为夭,存不为寿。福无所得,祸无所咎;各从其命,以度相守。明者不以智胜,暗者不以愚败,弱者不以迫畏,强者不以力尽。盖无君而庶物定,无臣而万事理,保身修性,不违其纪。惟兹若然,故能长久。今汝造音以乱声,作色以诡形,外易其貌,内隐其情。怀欲以求多,诈伪以要名;君立而虐兴,臣设而贼生。坐制礼法,束缚下民。欺愚诳拙,藏智自神。强者睽视而凌暴,弱者憔悴而事人。假廉而成贪,内险而外仁,罪至不悔过,幸遇则自矜。驰此以奏除,故循滞而不振。


“夫无贵则贱者不怨,无富则贫者不争,各足於身而无所求也。恩泽无所归,则死败无所仇。奇声不作,则耳不易听;淫色不显,则目不改视。耳目不相易改,则无以乱其神矣。此先世之所至止也。今汝尊贤以相高,竞能以相尚,争势以相君,宠贵以相加,趋天下以趣之,此所以上下相残也。竭天地万物之至,以奉声色无穷之欲,此非所以养百姓也。於是惧民之知其然,故重赏以喜之,严刑以威之。财匮而赏不供,刑尽而罚不行,乃始有亡国、戮君、溃败之祸。此非汝君子之为乎?汝君子之礼法,诚天下残贼、乱危、死亡之术耳!而乃目以为美行不易之道,不亦过乎!


“今吾乃飘颻於天地之外,与造化为友,朝飧汤谷,夕饮西海,将变化迁易,与道周始。此之於万物,岂不厚哉!故不通於自然者,不足以言道;暗於昭昭者不足与达明,子之谓也。”


先生既申若言,天下之喜奇者异之,慷忾者高之。其不知其体,不见其情,猜耳其道,虚伪之名。莫识其真,弗达其情,虽异而高之,与向之非怪者,蔑如也。至人者,不知乃贵,不见乃神。神贵之道存乎内,而万物运於天外矣。故天下终而不知其用也。


逌乎有宋,扶摇之野。有隐士焉,见之而喜,自以为均志同行也。曰:“善哉!吾得之见而舒愤也。上古质朴纯厚之道已废,而末枝遗华并兴。豺虎贪虐,群物无辜,以害为利,殒性亡驱。吾不忍见也,故去而处兹。人不可与为俦,不若与木石为邻。安期逃乎蓬山,用李潜乎丹水,鲍焦立以枯槁,莱维去而逌死。亦由兹夫!吾将抗志显高,遂终於斯。禽生而兽死,埋形而遗骨,不复返余之生乎!夫志均者相求,好合者齐颜,与夫子同之。”


於是,先生乃舒虹霓以蕃尘,倾雪盖以蔽明,倚瑶厢而徘徊,总众辔而安行,顾而谓之曰:“泰初真人,唯大之根。专气一志,万物以存。退不见后,进不睹先,发西北而造制,启东南以为门。微道德以久娱,跨天地而处尊。夫然成吾体也。是以不避物而处,所赌则宁;不以物为累,所逌则成。彷徉是以舒其意,浮腾足以逞其情。故至人无宅,天地为客;至人无主,天地为所;至人无事,天地为故。无是非之别,无善恶之异。故天下被其泽,而万物所以炽也。若夫恶彼而好我,自是而非人,忿激以争求,贵志而贱身,伊禽生而兽死,尚何显而获荣?悲夫!子之用心也!薄安利以忘生,要求名以丧体,诚与彼其无诡,何枯槁而逌死?子之所好,何足言哉?吾将去子矣。”乃扬眉而荡目,振袖而抚裳,令缓辔而纵策,遂风起而云翔。彼人者瞻之而垂泣,自痛其志;衣草木之皮,伏於岩石之下,惧不终夕而死。


先生过神宫而息,漱吾泉而行,回乎逌而游览焉,见薪於阜者,叹曰:“汝将焉以是终乎哉?”


薪者曰:“是终我乎?不以是终我乎?且圣人无怀,何其哀?盛衰变化,常不於兹?藏器於身,伏以俟时,孙刖足以擒庞,睢折胁而乃休,百里困而相嬴,牙既老而弼周。既颠倒而更来兮,固先穷而后收。秦破六国,兼并其地,夷灭诸侯,南面称帝。姱盛色,崇靡丽。凿南山以为阙,表东海以为门,门万室而不绝,图无穷而永存。美宫室而盛帷□,击钟鼓而扬其章。广苑囿而深池沼,兴渭北而建咸阳。骊木曾未及成林,而荆棘已丛乎阿房。时代存而迭处,故先得而后亡。山东之徒虏,遂起而王天下。由此视之,穷达讵可知耶?且圣人以道德为心,不以富贵为志;以无为用,不以人物为事。尊显不加重,贫贱不自轻,失不自以为辱,得不自以为荣。木根挺而枝远,叶繁茂而华零。无穷之死,犹一朝之生。身之多少,又何足营?”


因叹曰而歌曰:

\begin{center}
	
	“日没不周方,月出丹渊中。
	
	
	阳精蔽不见,阴光大为雄。
	
	
	亭亭在须臾,厌厌将复东。
	
	
	离合云雾兮,往来如飘风。
	
	
	富贵俛仰间,贫贱何必终?
	
	
	留侯起亡虏,威武赫夷荒。
	
	
	召平封东陵,一旦为布衣。
	
	
	枝叶托根柢,死生同盛衰。
	
	
	得志从命生,失势与时颓。
	
	
	寒暑代征迈,变化更相推。
	
	
	祸福无常主,何忧身无归?
	
	
	推兹由斯理,负薪又何哀?”
	
\end{center}

先生闻之,笑曰:“虽不及大,庶免小也。”乃歌曰:“天地解兮六和开,星辰霄兮日月颓,我腾而上将何怀?衣弗袭而服美,佩弗饰而自章,上下徘徊兮谁识吾常?”遂去而遐浮,肆云轝,兴气盖,徜徉回翔兮漭漾之外。建长星以为旗兮,击雷霆之康盖。开不周而出车兮,出九野之夷泰。坐中州而一顾兮,望崇山而回迈。端余节而飞旃兮,纵心虑乎荒裔,释前者而弗修兮,驰蒙间而远逌。弃世务之众为兮,何细事之足赖?虚形体而轻举兮,精微妙而神丰。命夷羿使宽日兮,召忻来使缓风。攀扶桑之长枝兮,登扶摇之隆崇。跃潜飘之冥昧兮。洗光曜之昭明。遗衣裳而弗服兮,服云气而遂行。朝造驾乎汤谷兮,夕息马乎长泉。时崦嵫而易气兮,挥若华以照冥。左朱阳以举麾兮,右玄阴以建旗,变容饰而改度,遂腾窃以修征。


阴阳更而代迈,四时奔而相逌,惟仙化之倏忽兮,心不乐乎久留。惊风奋而遗乐兮,虽云起而忘忧,忽电消而神逌兮,历寥廓而遐游。佩日月以舒光兮,登徜徉而上浮,压前进於彼逌道兮,将步足乎虚州。扫紫宫而陈席兮,坐帝室而忽会酬。萃众音而奏乐兮,声惊渺而悠悠。五帝舞而再属兮,六神歌而代周。乐啾啾肃肃,洞心达神,超遥茫茫,心往而忘返,虑大而志矜。


“粤大人微而弗复兮,扬云气而上陈。召大幽之玉女兮,接上王之美人。体云气之逌畅兮,服太清之淑贞。合欢情而微授兮,先艳溢其若神。华兹烨以俱发兮,采色焕其并振。倾玄麾而垂鬓兮,曜红颜而自新。时暧靆而将逝兮,风飘颻而振衣。云气解而雾离兮,霭奔散而永归。心惝惘而遥思兮,眇回目而弗晞。


“扬清风以为旟兮,翼旋轸而反衍。腾炎阳而出疆兮,命祝融而使遣。驱玄冥以摄坚兮,蓐收秉而先戈。勾芒奉毂,浮惊朝霞,寥廓茫茫而靡都兮,邈无俦而独立。倚瑶厢而一顾兮,哀下土之憔悴。分是非以为行兮,又何足与比类?霓旌飘兮云旗蔼,乐游兮出天外。”


大人先生披发飞鬓,衣方离之衣,绕绂阳之带。含奇芝,嚼甘华,吸浮雾,餐霄霞,兴朝云,颺春风。奋乎太极之东,游乎昆仑之西,遗辔颓策,流盼乎唐、虞之都。惘然而思,怅尔若忘,慨然而叹曰:


“呜呼!时不若岁,岁不若天,天不若道,道不若神。神者,自然之根也。彼勾勾者自以为贵夫世矣,而恶知夫世之贱乎兹哉?故与世争贵,贵不足尊;与世争富,富不足先。必超世而绝群,遗俗而独往,登乎太始之前,览乎忽漠之初,虑周流於无外,志浩荡而自舒,飘颻於四运,翻翱翔乎八隅。欲从而彷佛,洸漾而靡拘,细行不足以为毁,圣贤不足以为誉。变化移易,与神明扶。廓无外以为宅,周宇宙以为庐,强八维而处安,据制物以永居。夫如是,则可谓富贵矣。是故不与尧、舜齐德,不与汤、武并功,王、许不足以为匹,杨、丘岂能与比纵?天地且不能越其寿,广成子曾何足与并容?激八风以扬声,蹑元吉之高踪,被九天以开除兮,来云气以驭飞龙,专上下以制统兮,殊古今而靡同。夫世之名利,胡足以累之哉?故提齐而踧楚,掣赵而蹈秦,不满一朝而天下无人,东西南北莫之与邻。悲夫!子之修饰,以余观之,将焉存乎於兹?”


先生乃去之,纷泱莽,轨汤洋,流衍溢,历度重渊,跨青天,顾而逌览焉。则有逍遥以永年,无存忽合,散而上臻。霍分离荡,漾漾洋洋,飙涌云浮,达於摇光。直驰骛乎太初之中,而休息乎无为之宫。太初何如?无后无先。莫究其极,谁识其根。邈渺绵绵,乃反覆乎大道之所存。莫畅其究,谁晓其根。辟九灵而求索,曾何足以自隆?登其万天而通观,浴太始之和风。漂逍遥以远游,遵大路之无穷。遣太乙而弗使,陵天地而径行。超蒙鸿而远迹,左荡莽而无涯,右幽悠而无方,上遥听而无声,下修视而无章。施无有而宅神,永太清乎敖翔。


崔魏高山勃玄云,朔风横厉白雪纷,积水若陵寒伤人。阴阳失位日月颓,地坼石裂林木摧,火冷阳凝寒伤怀。阳和微弱隆阴竭,海冻不流绵絮折,呼吸不通寒伤裂。气并代动变如神,寒倡热随害伤人。熙与真人怀太清,精神专一用意平,寒暑勿伤莫不惊,忧患靡由素气宁。浮雾凌天恣所经,往来微妙路无倾,好乐非世又何争。人且皆死我独生。


真人游,驾八龙,曜日月,载云旗。徘徊逌,乐所之。真人游,太阶夷,□原辟,天地开。雨蒙蒙、风浑浑。登黄山,出栖迟。江河清,洛无埃,云气消,真人来,惟乐哉!时世易,好乐颓,真人去,与天回。反未央,延年寿,独敖世。望我□,何时反?超漫漫,路日远。


先生从此去矣,天下莫知其所终极。盖陵天地而与浮明遨游无始终,自然之至真也。鸲鹆不逾济,貉不度汶,世之常人,亦由此矣。曾不通区域,又况四海之表、天地之外哉!若先生者,以天地为卵耳。如小物细人欲论其长短,议其是非,岂不哀也哉!



\chapter*{吊古战场文}
\addcontentsline{toc}{chapter}{吊古战场文}
\begin{center}
	\textbf{[唐朝]李华}
\end{center}

浩浩乎,平沙无垠,夐不见人。河水萦带,群山纠纷。黯兮惨悴,风悲日曛。蓬断草枯,凛若霜晨。鸟飞不下,兽铤亡群。亭长告余曰:“此古战场也,常覆三军。往往鬼哭,天阴则闻。”伤心哉!秦欤汉欤?将近代欤?

吾闻夫齐魏徭戍,荆韩召募。万里奔走,连年暴露。沙草晨牧,河冰夜渡。地阔天长,不知归路。寄身锋刃,腷臆谁愬?秦汉而还,多事四夷,中州耗斁,无世无之。古称戎夏,不抗王师。文教失宣,武臣用奇。奇兵有异于仁义,王道迂阔而莫为。呜呼噫嘻!

吾想夫北风振漠,胡兵伺便。主将骄敌,期门受战。野竖旌旗,川回组练。法重心骇,威尊命贱。利镞穿骨,惊沙入面,主客相搏,山川震眩。声析江河,势崩雷电。至若穷阴凝闭,凛冽海隅,积雪没胫,坚冰在须。鸷鸟休巢,征马踟蹰。缯纩无温,堕指裂肤。当此苦寒,天假强胡,凭陵杀气,以相剪屠。径截辎重,横攻士卒。都尉新降,将军复没。尸踣巨港之岸,血满长城之窟。无贵无贱,同为枯骨。可胜言哉!鼓衰兮力竭,矢尽兮弦绝,白刃交兮宝刀折,两军蹙兮生死决。降矣哉,终身夷狄;战矣哉,暴骨沙砾。鸟无声兮山寂寂,夜正长兮风淅淅。魂魄结兮天沉沉,鬼神聚兮云幂幂。日光寒兮草短,月色苦兮霜白。伤心惨目,有如是耶!

吾闻之:牧用赵卒,大破林胡,开地千里,遁逃匈奴。汉倾天下,财殚力痡。任人而已,岂在多乎!周逐猃狁,北至太原。既城朔方,全师而还。饮至策勋,和乐且闲。穆穆棣棣,君臣之间。秦起长城,竟海为关。荼毒生民,万里朱殷。汉击匈奴,虽得阴山,枕骸徧野,功不补患。

苍苍蒸民,谁无父母?提携捧负,畏其不寿。谁无兄弟?如足如手。谁无夫妇?如宾如友。生也何恩,杀之何咎?其存其没,家莫闻知。人或有言,将信将疑。悁悁心目,寤寐见之。布奠倾觞,哭望天涯。天地为愁,草木凄悲。吊祭不至,精魂无依。必有凶年,人其流离。呜呼噫嘻!时耶命耶?从古如斯!为之奈何?守在四夷。


\chapter*{杨布打狗}
\addcontentsline{toc}{chapter}{杨布打狗}
\begin{center}
	\textbf{[春秋战国]列子}
\end{center}

杨朱之弟曰布,衣素衣而出。天雨,解素衣,衣缁衣而返。其狗不知,迎而吠之。杨布怒,将扑之。杨朱曰:“子无扑矣,子亦犹是也。向者使汝狗白而往黑而来,岂能无怪哉?”


\chapter*{与唐处士书}
\addcontentsline{toc}{chapter}{与唐处士书}
\begin{center}
	\textbf{[宋朝]范仲淹}
\end{center}

仲淹谨再拜致书于处士唐君:盖闻圣人之作琴也,鼓天下之和而和天下。斫琴之道大哉。秦祚之后,礼乐失驭。予嗟乎琴散久矣,后之传者,妙指美声,巧以相尚,丧其大,矜其细,人以艺观焉。皇宋文明之运,宜建大雅东宫故谕德。崔公其人也,得琴之道于斯,乐于斯垂五十年,清净平和,惟与琴会,著琴笺,而自然之义在矣。予尝游于门下,一日请曰:琴何谓是?公曰:清丽而静,和润而远。予拜而退,思释之曰:清丽而弗静其失也躁。和润而弗远其失也佞。不躁不佞其中和之道欤。一日请曰:今之能琴谁可与先生和者?曰:唐处士可矣。予拜而退,美而歌曰:有人焉,有人焉,且将师其一二属远仁乎。千里未获所存令复选于上京,崔公既没琴不在于君乎。君将怜其意,授之一二,使得操尧舜之音,游羲皇之域,其赐也,岂不大哉。又先生之琴,传传而不穷,上圣之风,存存乎盛时其音也,岂不远哉。诚不敢助南风之诗,以为天富寿庶几宣三乐之情以美生平,而可乎狂率之咎亦冀舍旃。



\chapter*{答手诏条陈十事}
\addcontentsline{toc}{chapter}{答手诏条陈十事}
\begin{center}
	\textbf{[宋朝]范仲淹}
\end{center}

伏奉手诏“今来用韩琦、范仲淹、富弼,皆是中外人望,不次拔擢。韩琦暂往陕西,范仲淹、富弼皆在两地,所宜尽心为国家,诸事建明,不得顾避。兼章得象等同心忧国,足得商量。如有当世急务可以施行者,并须条列闻奏,副朕拔擢之意”者。臣智不逮人,术不通古,岂足以奉大对。然臣蒙陛下不次之擢,预闻政事,又诏意丁宁,臣战汗惶怖,曾不获让。


臣闻历代之政,久皆有弊。弊而不救,祸乱必生。何哉?纲纪浸隳。制度日削,恩赏不节,赋敛无度,人情惨怨,天祸暴起。惟尧舜能通其变,使民不倦。


《易》曰:“穷则变.变则通,通则久。”此言天下之理有所穷塞,则思变通之道。既能变通,则成长久之业。我国家革五代之乱,富有四海,垂八十年,纲纪制度,日削月侵,官壅于下,民困于外,夷狄骄盛,寇盗横炽,不可不更张以救之。然则欲正其末,必端其本;欲清其流,必澄其源。臣敢约前代帝王之道,求今朝祖宗之烈,釆其可行者条奏。愿陛下顺天下之心,力行此事,庶几法制有立,纲纪再振,则宗社灵长,天下蒙福。


一曰明黜陟。臣观《书》曰:“三载考绩,三考黜陟幽明。”然则尧舜之朝,建官至少,尚乃九载一迁,必求成绩,而天下大化,百世之后,仰为帝范。我祖宗朝,文武百官皆无磨勘之例,惟政能可旌者,擢以不次;无所称者,至老不迁。故人人自励,以求绩效。今文资三年一迁,武职五年一迁,谓之磨勘。不限内外,不问劳逸,贤不肖并进,此岂尧舜黜陟幽明之意耶!假如庶僚中有一贤于众者,理一郡县,领一务局,思兴利去害而有为也,众皆指为生事,必嫉之沮之,非之笑之,稍有差失,随而挤陷。故不肖者素餐尸禄,安然而莫有为也。虽愚暗鄙猥,人莫齿之。而三年一迁,坐至卿监丞郎者,历历皆是,谁肯为陛下兴公家之利,救生民之病,去政事之弊,葺纪纲之坏哉!利而不兴则国虚,病而不救则民怨。弊而不去则小人得志,坏而不葺则王者失。贤不肖混淆,请托侥幸,迁易不已,中外苛且,百事废堕,生民久苦,羣盗渐起。劳陛下旰昃之忧者,岂非官失其正而致其危耶!至若在京百司,金谷浩瀚,权势子弟长为占据,有虚食禀禄,待阙一二年者。暨临事局,挟以势力。岂肯恪恭其职?使祖宗根本之地,纲纪日隳。故在京官司,有一员阙,则争夺者数人。其外任京朝官,则有私居待阙,动逾岁时,往往到职之初,便该磨勘,一无勤效,例蒙迁改。此则人人因循,不复奋励之由也。


臣请特降诏书,今后两地臣僚,有大功大善,则特加爵命;无大功大善,更不非时进秩。其理状寻常而出者,祇守本官,不得更带美职。应京朝官在台省、馆阁职任,及在审刑、大理寺、开封府、两赤县、国子监、诸王府,并因保举及选差监在京重难库务者,并须在任三周年,即与磨勘。若因陈乞,并于中书、审官院愿在京差遣者,与保举选差不同,并须勾当通计及五周年,方得磨勘。如此则权势子弟,肯就外任,各知艰难。亦有俊明之人,因此树立,可以进用。如今日已前受在京差遣已勾当者,且依旧日年限磨勘.其未曾交割勾当。却求外任者,并听其外任。在京朝官到职勾当及三年者与磨勘,内前任勾当年月日及公程日限,并非因陈乞而移任在道月日,及升朝官在京朝请月日,并令通计。其远官近地,劳逸不同,并在假待阙及公程外住滞,或因公事,非时移替。在道月日委有司别行定夺闻奏。如任内有私罪并公罪徒以上者,至该磨勘日,具情理轻重,别取进止。其庶寮中有高才异行,多所荐论,或异略嘉谋,为上信纳者,自有特恩改迁,非磨勘之可滞也。又外任善政著闻,有补风化;或累讼之狱,能辨寃沈;或五次推勘,人无翻讼;或劝课农桑,大获美利;或京城库务,能革大弊,惜费巨万者,仰本辖保明闻奏,下尚书省集议。为众所许,则列状上闻,并与改官,不隔磨勘。或有异同,各以所执取旨,出于圣断。仍请诏下审官院、流内铨、尚书考功,应京朝官选人逐任得替,明具较定考绩、结罪闻奏。内有事状猥滥,并老疾愚昧之人,不堪理民者,别取进止。已上磨勘考绩条件,该说不尽者,有司比类上闻。如此,则因循者拘考绩之限,特达者加不次之赏,然后天下公家之利必兴,生民之病必救,政事之弊必去,纲纪之坏必葺,人人自劝,天下兴治,则前王之业,祖宗之权,复振于陛下之手矣。其武臣磨勘年限,委枢密院比附文资定夺闻奏。


二曰抑侥幸。臣闻先王赏延于世,诸侯有世子袭国,公卿以德而任,有袭爵者,《春秋》讥之。及汉之公卿,有封爵而殁,立一子为后者,未闻余子皆有爵命。其次宠待大臣,赐一子官者有之,未闻每岁有自荐其子弟者。祖宗之朝,亦不过此。自真宗皇帝以太平之乐,与臣下共庆,恩意渐广。大两省至知杂御史以上,每遇南郊并圣节,各奏一子充京官,少卿、监奏一子充试衔。其正郎、带职员外郎,并诸路提点刑狱以上差遣者,每遇南郊,奏一子充斋郎。其大两省等官,既奏得子充京官,明异于庶僚,大示区别,复更每岁奏荐,积成冗官。假有任学士以上官经二十年者,则一家兄弟子孙出京官二十人,仍接次升朝,此滥进之极也。今百姓贫困,冗官至多。授任既轻,政事不举。俸禄既广,刻剥不暇。审官院常患充塞,无阙可补。臣请特降诏书,今后两府并两省官等,遇大礼许奏一子充京官,如奏弟侄骨肉,即与试衔外,每年圣节更不得陈乞。如别有勋劳著闻于外,非时赐一子官者,系自圣恩。其转运使及边任文臣初除授后,合奏得子弟身事者。并候到任二年无遗阙,方许陈乞。如二年内非次移改者,即许通计三年陈乞。三司副使、知杂御史、少卿、监以上,并同两省,遇大礼各奏荐子孙。其正郎、带馆职员外郎,并省府推判官、外任提点刑狱以上,遇大礼合该奏荐子孙者,须是在任及二周年,方得陈乞。已上有该说不尽者,委有司比类闻奏。如此则内外朝臣,各务久于其职,不为苟且之政,兼抑躁动之心。亦免子弟充塞铨曹,与孤寒争路,轻忽郡县,使生民受弊。其武臣入边上差遣,并大礼合奏荐子弟者,乞下枢密院详定比类闻奏。


又国家开文馆,延天下英才,使之直秘庭,览羣书,以待顾问,以养器业,为大用之备。今乃登进士高等者,一任才罢,不以能否,例得召试而补之。两府、两省子弟亲戚,不以贤不肖,辄自陈乞馆阁职事者,亦得进补。太宗皇帝建崇文院、秘阁,自书碑文,重天下贤才也。陛下当思祖宗之意,不宜甚轻之。臣请特降诏书,今后进士三人内及第者,一任回日。许进于教化经术文字十轴,下两制看详,作五等品第。中第一第二等者,即赐召试;试又优等,即补馆阁职事。两府、两省子弟,并不得陈乞馆阁职事及读书之类。御史台画时弹劾,并谏院论奏。如馆阁阙人,即委两地举文有古道、才堪大用之士,进名同举,并两制列署表章,仍上殿称荐,以充其职。如此,则馆阁职事更不轻授,足以起朝廷之风采,绍祖宗之本意,副陛下慎选矣。


三曰精贡举。臣谨按《周礼》卿大夫之职,各教其所治,三年一大比,考其德行道艺,乃献贤能之书于王。贤为有德行,能为有道艺。王再拜受之,登于天府。天府,太庙之宝藏也。盖言王者举贤能,所以上安宗社,故拜受其名,藏于庙中,以重其事也。卿大夫之职,废既久矣。今诸道学校,如得明师,尚可教人六经,传治国治人之道。而国家乃专以辞赋取进士,以墨义取诸科,士皆舍大方而趋小道,虽济济盈庭,求有才有识者十无一二。况天下危困,乏人如此,将何以救?在乎教以经济之业,取以经济之才,庶可救其不逮。或谓救弊之术无乃后时,臣谓四海尚完,朝谋而夕行,庶乎可济,安得晏然不救,坐俟其乱哉!


臣请诸路州郡有学校处,奏举通经有道之士,专于教授,务在兴行。其取士之科,即依贾昌朝等起请,进士先策论而后诗赋;诸科墨义之外,更通经旨。使人不专辞藻,必明理道,则天下讲学必兴,浮薄知劝,最为至要。内欧阳修、蔡襄更乞逐场去留,贵文卷少而考校精。臣谓尽令逐场去留,则恐旧人杆格,不能创习策论,亦不能旋通经旨,皆忧弃遗,别无进路。臣请进士旧人三举以上者,先策论而后诗赋。许将三场文卷通考,互取其长。两举、初举者,皆是少年,足以进学,请逐场去留。诸科中有通经旨者,至终场,别问经旨十道,如不能命辞而对,则于知举官员前,讲说七通者为合格。不会经旨者,三举已上即逐场所对墨义,依自来通粗施行。两举、初举者,至于终场日,须八通者为合格。


又外郡解发进士、诸科人,本乡举里选之式,必先考其履行,然后取以艺业。今乃下求履行,惟以词藻、墨义取之,加用封弥,不见姓字,实非乡里举选之本意也。又南省考试举人,一场试诗赋,一场试策,人皆精意,尽其所能。复考校日久,实少舛谬。及御试之日,诗赋文论共为一场,既声病所拘,意思不远。或音韵中一字有差,虽生平苦辛,即时摈逐。如音韵不失,虽末学浅近,俯拾科级。既乡举之处不考履行,又御试之日更拘声病,以此士之进退,多言命运而不言行业。明君在上,固当使人以行业而进,而乃言命运者,是善恶不辨而归诸天也,岂国家之美事哉!臣请重定外郡发解条约,须是履行无恶、艺业及等者,方得解荐,更不封弥试卷。其南省考试之人,已经本乡询考履行,却须封弥试卷,精考艺业,定夺等第,进入御前。选官覆考,重定等第讫,然后开看南省所定等第,内合同姓名偶有高下者,更不移改。若等第不同者,人数必少,却加封弥,更宣两地参校,然后御前放榜,此为至当。内三人已上,即于高等人中选择,圣意宣放。其考校进士,以策论高、词赋次者为优等,策论平、词赋优者为次等。诸科经旨通者为优等,墨义通者为次等。已上进士、诸科,并以优等及第者放选注官,次等及第者守本科选限。自唐以来,及第人皆守选限。国家以收复诸国,郡邑乏官,其新及第人,权与放选注官。今来选人壅塞,宜有改革,又足以劝学,使其知圣人治身之道,则国家得人,百姓受赐。


四曰择官长。臣闻先王建侯,以共理天下。今之刺史、县令,即古之诸侯。一方舒惨,百姓休戚,实系其人。故历代盛明之时,必重此任。今乃不问贤愚,不较能否,累以资考,升为方面。懦弱者不能检吏,得以蠹民;强干者惟是近名,率多害物。邦国之本,由此凋残。朝廷虽至忧勤,天下何以苏息!其转运使并提点刑狱按察列城,当得贤于众者。臣请特降诏书,委中书、枢密院且各选转运使、提点刑狱共十人,大藩知州十人;委两制共举知州十人;三司副使、判官同举知州五人;御史台中丞、知杂、三院共举知州五人;开封知府、推官共举知州五人;逐路转运使、提点刑狱各同举知州五人,知县、县令共十人;逐州知州、通判同举知县、县令共二人。得前件所举之人,举主多者先次差补。仍指挥审官院、流内铨今日以后所差知州、知县、县令并具合入人历任功过、举主人数闻奏,委中书看详。委得允当,然后引对。如此举择,则诸道官吏庶几得人,为陛下爱惜百姓,均其徭役,宽于赋敛,各获安宁,不召祸乱,天下幸甚。


五曰均公田。臣闻《易》曰:“天地养万物,圣人养贤以及万民。”此言圣人养民之时,必先养贤。养贤之方,必先厚禄。厚禄然后可以责廉隅,安职业也。皇朝之初,承五代乱离之后,民庶凋弊,时物至贱。暨诸国收复,天下郡县之官少人除补,至有经五七年不替罢者。或才罢去,便入见阙。当物价至贱之时,俸禄不辍,士人之家无不自足。咸平已后,民庶渐繁,时物遂贵。入仕门多,得官者众,至有得替守选一二年,又授官待阙一二年者。在天下物贵之后,而俸禄不继,士人家鲜不穷窘,男不得婚,女不得嫁,丧不得葬者,比比有之。复于守选、待阙之日,衣食不足,贷债以苟朝夕。到官之后,必来见逼,至有冒法受赃,赊贷度日⑥,或不耻贾贩,与民争利。既为负罪之人,不守名节,吏有奸赃而不敢发,民有豪猾而不敢制。奸吏豪民得以侵暴,于是贫弱百姓理不得直,寃不得诉,徭役不均,刑罚不正,比屋受弊,无可奈何,由乎制禄之方有所未至。


真宗皇帝思深虑远,复前代职田之制,使中常之士自可守节,婚嫁以时,丧葬以礼,皆国恩也。能守节者,始可制奸赃之吏,镇豪猾之人。法乃不私,民则无枉。近日屡有臣僚乞罢职田,以其有不均之谤,有侵民之害。臣谓职田本欲养贤,缘而侵民者有矣,比之衣食不足,坏其名节,不能奉法,以直为枉,以枉为直,众怨思乱而天下受弊,岂止职田之害耶!又自古常患百官重内而轻外,唐外官月俸尤更丰足,簿尉俸钱尚二十贯。今窘于财用,未暇增复。臣请两地同议外官职田,有不均者均之,有未给者给之,使其衣食得足,婚嫁丧葬之礼不废,然后可以责其廉节,督其善政。有不法者,可废可诛。且使英俊之流,乐于为郡为邑之任,则百姓受赐。又将来升擢,多得曾经郡县之人,深悉民隐,亦致化之本也。惟圣慈深察,天下幸甚。


六曰厚农桑。臣观《书》曰:“德惟善政,政在养民。”此言圣人之德,惟在善政。善政之要,惟在养民;养民之政,必先务农;农政既修,则衣食足;衣食足,则爱肤体;爱肤体,则畏刑罚;畏刑罚,则寇盗自息,祸乱不兴。是圣人之德,发于善政;天下之化,起于农亩。故《诗》有《七月》之篇,陈王业也。今国家不务农桑,粟帛常贵。浙江诸路岁籴米六百万石,其所籴之价与辇运之费,每岁共用钱三百余万贯文。又贫弱之民,困于赋敛,岁伐桑枣,鬻而为薪。劝课之方,有名无实。故粟帛常贵,府库日虚。此而不谋,将何以济!


臣于天下农利之中,粗举二三以言之。且如五代羣雄争霸之时,本国岁饥,则乞籴于邻国,故各兴农利,自至丰足。江南旧有圩田,每一圩方数十里,如大城。中有河渠,外有门闸。旱则开闸引江水之利,涝则闭闸拒江水之害,旱涝不及,为农美利。又浙西地卑,常苦水沴。虽有沟河,可以通海,惟时开导,则潮泥不得而堙之。虽有堤塘,可以御患,惟时修固,则无摧坏。臣知苏州日,点检簿书,一州之田,系出税者三万四千顷。中稔之利,每亩得米二石至三石。计出米七百余万石。东南每岁上供之数六百万石,乃一州所出。臣询访高年,则云曩时两浙未归朝廷,苏州有营田军四都,共七八千人,专为田事,导河筑堤,以减水患。于时民间钱五十文籴白米一石。自皇朝一统,江南不稔则取之浙右,浙右不稔则取之淮南,故慢于农政,不复修举。江南圩田、浙西河塘,大半隳废,失东南之大利。今江浙之米,石不下六七百文足。至一贯文省,比于当时,其贵十倍,而民不得不困,国不得不虚矣。


又京东西路有卑湿积潦之地,早年国家特令开决之后,水患大减。今罢役数年,渐已堙塞,复将为患。臣请每岁之秋,降勅下诸路转运司,令辖下州军吏民各言农桑之间可兴之利、可去之害。或合开河渠,或筑堤堰陂塘之类,并委本州军选官计定工料,每岁于二月间兴役,半月而罢,仍具功绩闻奏。如此不绝。数年之间,农利大兴。下少饥岁,上无贵籴,则东南岁籴辇运之费大可减省。其劝课之法,宜选官讨论古制,取其简约易从之术,颁赐诸路转运使,及面赐一本,付新授知州、知县、县令等。此养民之政、富国之本也。


七曰修武备。臣闻古者天子六军,以宁邦国。唐初京师置十六将军官属,亦六军之义也。诸道则开折冲、果毅府五百七十四,以储兵伍。每岁三时耕稼,一时习武。自贞观至于开元,百三十年,戎臣兵伍,无一逆乱。至开元末,听匪人之言,遂罢府兵。唐衰,兵伍皆市井之徒,无礼义之教,无忠信之心,骄蹇凶逆,至于丧亡。我祖宗以来,罢诸侯权,聚兵京师,衣粮赏赐丰足,经八十年矣。虽已困生灵、虚府库,而难于改作者,所以重京师也。今西北强梗,边备未足,京师卫兵多远戍,或有仓卒,辇毂无备,此大可忧也。远戍者防边陲之患,或缓急抽还,则外御不严,戎狄进奔,便可直趋关辅。新招者聚市井之辈,而轻嚣易动,或财力一屈,请给不充,则必散为羣盗。今生民已困,无可诛求,或连年凶饥,将何以济!赡军之策,可不预图?若因循过时,臣恐急难之际,宗社可忧。


臣请密委两地,以京畿见在军马,同议有无阙数。如六军末整,须议置兵,则请约唐之法,先于畿内并近辅州府召募强壮之人,充京畿卫士。得五万人以助正兵,足为强盛。使三时务农,大省给赡之费;一时教战,自可防虞外患。其召募之法,并将校次第,并先密切定夺闻奏。此实强兵节财之要也。候京畿近辅召募卫兵,已成次第,然后诸道放此⑨,渐可施行。惟圣慈留意。


八曰减徭役。臣闻汉光武建武六年六月诏曰:“夫张官置吏,所以为人也。今户口耗少,而县官吏职,所置尚繁。令司隶州牧各实所部。”二府于是条奏并省四百余县,天下至治。臣又观西京图经,唐会昌中,河南府有户一十九万四千七百余户,置二十县。今河南府主客户七万五千九百余户,仍置一十九县。主户五万七百,客户二万五千二百⑩。巩县七百户,偃师一千一百户,逐县三等而堪役者,不过百家,而所供役人不下二百数。新旧循环,非鳏寡孤独,不能无役。西洛之民,最为穷困。臣请依后汉故事,遣使先往西京并省诸邑为十县。其所废之邑,并改为镇,令本路举文资一员,董榷酤、关征之利兼人烟公事。所废公人,除归农外,有愿居公门者,送所存之邑。其所在邑中役人,却可减省归农,则两不失所。候西京并省稍成伦序,则行于大名府,然后遣使诸道,依此施行。仍先指挥诸道防团州已下,有使、州两院者,皆为一院,公人愿去者,各放归农。职官厅可给本城兵士七人至十人,替人力归农。其乡村耆保地里近者,亦令并合。能并一耆保管?,亦减役十余户。但少徭役,人自耕作,可期富庶。


九曰覃恩信。臣窃覩国家三年一郊,天子斋戒衮冕,谒见宗庙,乃祀上帝。大礼既成,还御端门,肆赦天下,曰:赦书日行五百里,敢以赦前事言者,以其罪罪之,欲其王泽及物之速也如此。今大赦每降,天下欢呼。一两月间,钱谷司存督责如旧,桎梏老幼,籍没家产。至于宽赋敛。减摇役,存恤孤贫,振举滞淹之事,未尝施行,使天子及民之意,尽成空言,有负圣心,损伤和气。臣请特降诏书,今后赦书内宣布恩泽,有所施行,而三司、转运司、州县不切遵禀者,井从违制,徒二年断,情重者,当行刺配。应天禧年以前天下欠负,不问有无侵欺盗用,并与除放,违者仰御史台、提点刑狱司常切觉察纠劾,无令壅遏。臣又闻《易》曰:“先王以省方观民设教。”故有巡狩之礼,察诸侯善恶,观风俗厚薄,此圣人顺动之意。今巡狩之礼不可复行,民隐无穷,天听甚远。臣请降诏中书,今后每遇南郊赦后,精选臣僚往诸路安抚,察官吏能否,求百姓疾苦,使赦书中及民之事,一一施行,天下百姓莫不幸甚。


十曰重命令。臣闻《书》曰:“慎乃出令,令出惟行。”准律文,诸被制书有所施行而违者,徒二年;失错者,杖一百。又监临主司受财而枉法者,十五疋,绞。盖先王重其法令,使无敢动摇,将以行天下之政也。今覩国家每降宣勅条贯,烦而无信,轻而弗禀,上失其威。下受其弊。盖由朝廷采百官起请,率尔颁行,既昧经常,即时更改,此烦而无信之验矣。又海行条贯,虽是故违.皆从失坐,全乖律意,致坏大法,此轻而弗禀之甚矣。臣请特降诏书,今后百官起请条贯,令中书、枢密院看详会议,必可经久,方得施行。如事干刑名者,更于审刑、大理寺勾明会法律官员参详起请之词,删去繁冗,裁为制敕,然后颁行天下,必期遵守。其冲改条贯,并令缴纳,免致错乱,误有施行。仍望别降敕命,今后逐处当职官吏亲被制书,及到职后所受条贯,敢故违者,不以海行,并从违制,徒二年。未到职已前所降条贯,失于检用,情非故违者,并从本条失错科断,杖一百。余人犯海行条贯,不指定违制刑名者,并从失坐。若条贯差失,于事有害,逐处长吏,别见机会,须至便宜而行者,并须具缘由闻奏,委中书、枢密院详察,如合理道,即与放罪。仍便相度,别从更改。



\chapter*{谏逐客书}
\addcontentsline{toc}{chapter}{谏逐客书}
\begin{center}
	\textbf{[秦朝]李斯}
\end{center}

臣闻吏议逐客,窃以为过矣。昔穆公求士,西取由余于戎,东得百里奚于宛,迎蹇叔于宋,来邳豹、公孙支于晋。此五子者,不产于秦,而穆公用之,并国二十,遂霸西戎。孝公用商鞅之法,移风易俗,民以殷盛,国以富强,百姓乐用,诸侯亲服,获楚、魏之师,举地千里,至今治强。惠王用张仪之计,拔三川之地,西并巴、蜀,北收上郡,南取汉中,包九夷,制鄢、郢,东据成皋之险,割膏腴之壤,遂散六国之纵,使之西面事秦,功施到今。昭王得范雎,废穰侯,逐华阳,强公室,杜私门,蚕食诸侯,使秦成帝业。此四君者,皆以客之功。由此观之,客何负于秦哉!向使四君却客而不内,疏士而不用,是使国无富利之实,而秦无强大之名也。

今陛下致昆山之玉,有随和之宝,垂明月之珠,服太阿之剑,乘纤离之马,建翠凤之旗,树灵鼍之鼓。此数宝者,秦不生一焉,而陛下说之,何也?必秦国之所生然后可,则是夜光之璧,不饰朝廷;犀象之器,不为玩好;郑、卫之女不充后宫,而骏良駃騠不实外厩,江南金锡不为用,西蜀丹青不为采。所以饰后宫,充下陈,娱心意,说耳目者,必出于秦然后可,则是宛珠之簪,傅玑之珥,阿缟之衣,锦绣之饰不进于前,而随俗雅化,佳冶窈窕,赵女不立于侧也。夫击瓮叩缶弹筝搏髀,而歌呼呜呜快耳者,真秦之声也;《郑》、《卫》、《桑间》,《韶》、《虞》、《武》、《象》者,异国之乐也。今弃击瓮叩缶而就《郑》、《卫》,退弹筝而取《昭》、《虞》,若是者何也?快意当前,适观而已矣。今取人则不然。不问可否,不论曲直,非秦者去,为客者逐。然则是所重者在乎色乐珠玉,而所轻者在乎人民也。此非所以跨海内、制诸侯之术也。

臣闻地广者粟多,国大者人众,兵强则士勇。是以泰山不让土壤,故能成其大;河海不择细流,故能就其深;王者不却众庶,故能明其德。是以地无四方,民无异国,四时充美,鬼神降福,此五帝三王之所以无敌也。今乃弃黔首以资敌国,却宾客以业诸侯,使天下之士退而不敢西向,裹足不入秦,此所谓“借寇兵而赍盗粮”者也。夫物不产于秦,可宝者多;士不产于秦,而愿忠者众。今逐客以资敌国,损民以益雠,内自虚而外树怨于诸侯,求国无危,不可得也。(泰山一作:太山)


\chapter*{莺莺传}
\addcontentsline{toc}{chapter}{莺莺传}
\begin{center}
	\textbf{[唐朝]元稹}
\end{center}

唐贞元中,有张生者,性温茂,美风容,内秉坚孤,非礼不可入。或朋従游宴,扰杂其间,他人皆汹汹拳拳,若将不及;张生容顺而已,终不能乱。以是年二十三,未尝近女色。知者诘之,谢而言曰:"登徒子非好色者,是有凶行。余真好色者,而适不我值。何以言之?大凡物之尤者,未尝不留连于心,是知其非忘情者也。"诘者识之。

无几何,张生游于蒲,蒲之东十余里,有僧舍曰普救寺,张生寓焉。适有崔氏孀妇,将归长安,路出于蒲,亦止兹寺。崔氏妇,郑女也;张出于郑,绪其亲,乃异派之従母。是岁,浑瑊薨于蒲,有中人丁文雅,不善于军,军人因丧而扰,大掠蒲人。崔氏之家,财产甚厚,多奴仆,旅寓惶骇,不知所托。

先是张与蒲将之党有善,请吏护之,遂不及于难。十余日,廉使杜确将天子命以总戎节,令于军,军由是戢。郑厚张之德甚,因饰馔以命张,中堂宴之。复谓张曰:"姨之孤嫠未亡,提携幼稚,不幸属师徒大溃,实不保其身,弱子幼女,犹君之生,岂可比常恩哉?今俾以仁兄礼奉见,冀所以报恩也。"命其子,曰欢郎,可十余岁,容甚温美。次命女:"出拜尔兄,尔兄活尔。"久之辞疾,郑怒曰:"张兄保尔之命,不然,尔且掳矣,能复远嫌乎?"久之乃至,常服睟容,不加新饰。垂鬟接黛,双脸销红而已,颜色艳异,光辉动人。张惊为之礼,因坐郑旁。以郑之抑而见也,凝睇怨绝,若不胜其体者。问其年纪,郑曰:"今天子甲子岁之七月,终于贞元庚辰,生年十七矣。"张生稍以词导之,不对,终席而罢。

张自是惑之,愿致其情,无由得也。崔之婢曰红娘,生私为之礼者数四,乘间遂道其衷。婢果惊沮,腆然而奔,张生悔之。翼日,婢复至,张生乃羞而谢之,不复云所求矣。婢因谓张曰:"郎之言,所不敢言,亦不敢泄。然而崔之姻族,君所详也,何不因其德而求娶焉?"张曰:"余始自孩提,性不苟合。或时纨绮间居,曾莫流盼。不为当年,终有所蔽。昨日一席间,几不自持。数日来,行忘止,食忘饱,恐不能逾旦暮。若因媒氏而娶,纳采问名,则三数月间,索我于枯鱼之肆矣。尔其谓我何?"婢曰:"崔之贞慎自保,虽所尊不可以非语犯之,下人之谋,固难入矣。然而善属文,往往沈吟章句,怨慕者久之。君试为喻情诗以乱之,不然则无由也。"张大喜,立缀春词二首以授之。是夕,红娘复至,持彩笺以授张曰:"崔所命也。"题其篇曰《明月三五夜》,其词曰:"待月西厢下,迎风户半开。拂墙花影动,疑是玉人来。"张亦微喻其旨,是夕,岁二月旬有四日矣。崔之东有杏花一株,攀援可逾。既望之夕,张因梯其树而逾焉,达于西厢,则户半开矣。红娘寝于床,生因惊之。红娘骇曰:"郎何以至?"张因绐之曰:"崔氏之笺召我也,尔为我告之。"无几,红娘复来,连曰:"至矣!至矣!"张生且喜且骇,必谓获济。及崔至,则端服严容,大数张曰:"兄之恩,活我之家,厚矣。是以慈母以弱子幼女见托。奈何因不令之婢,致淫逸之词,始以护人之乱为义,而终掠乱以求之,是以乱易乱,其去几何?诚欲寝其词,则保人之奸,不义;明之于母,则背人之惠,不祥;将寄与婢仆,又惧不得发其真诚。是用托短章,愿自陈启,犹惧兄之见难,是用鄙靡之词,以求其必至。非礼之动,能不愧心,特愿以礼自持,无及于乱。"言毕,翻然而逝。张自失者久之,复逾而出,于是绝望。

数夕,张生临轩独寝,忽有人觉之。惊骇而起,则红娘敛衾携枕而至。抚张曰:"至矣!至矣!睡何为哉?"并枕重衾而去。张生拭目危坐久之,犹疑梦寐,然而修谨以俟。俄而红娘捧崔氏而至,至则娇羞融冶,力不能运支体,曩时端庄,不复同矣。是夕旬有八日也,斜月晶莹,幽辉半床。张生飘飘然,且疑神仙之徒,不谓従人间至矣。有顷,寺钟鸣,天将晓,红娘促去。崔氏娇啼宛转,红娘又捧之而去,终夕无一言。张生辨色而兴,自疑曰:"岂其梦邪?"及明,睹妆在臂,香在衣,泪光荧荧然,犹莹于茵席而已。是后又十余日,杳不复知。张生赋《会真诗》三十韵,未毕,而红娘适至。因授之,以贻崔氏。自是复容之,朝隐而出,暮隐而入,同安于曩所谓西厢者,几一月矣。张生常诘郑氏之情,则曰:"我不可奈何矣,因欲就成之。"无何,张生将之长安,先以情喻之。崔氏宛无难词,然而愁怨之容动人矣。将行之再夕,不可复见,而张生遂西下。

数月,复游于蒲,会于崔氏者又累月。崔氏甚工刀札,善属文,求索再三,终不可见。往往张生自以文挑,亦不甚睹览。大略崔之出人者,艺必穷极,而貌若不知;言则敏辩,而寡于酬对。待张之意甚厚,然未尝以词继之。时愁艳幽邃,恒若不识;喜愠之容,亦罕形见。异时独夜操琴,愁弄凄恻,张窃听之,求之,则终不复鼓矣。以是愈惑之。张生俄以文调及期,又当西去。当去之夕,不复自言其情,愁叹于崔氏之侧。崔已阴知将诀矣,恭貌怡声,徐谓张曰:"始乱之,终弃之,固其宜矣,愚不敢恨。必也君乱之,君终之,君之惠也;则殁身之誓,其有终矣,又何必深感于此行?然而君既不怿,无以奉宁。君常谓我善鼓琴,向时羞颜,所不能及。今且往矣,既君此诚。"因命拂琴,鼓《霓裳羽衣序》,不数声,哀音怨乱,不复知其是曲也。左右皆唏嘘,张亦遽止之。投琴,泣下流连,趋归郑所,遂不复至。明旦而张行。

明年,文战不胜,张遂止于京,因贻书于崔,以广其意。崔氏缄报之词,粗载于此。曰:捧览来问,抚爱过深,儿女之情,悲喜交集。兼惠花胜一合,口脂五寸,致耀首膏唇之饰。虽荷殊恩,谁复为容?睹物增怀,但积悲叹耳。伏承使于京中就业,进修之道,固在便安。但恨僻陋之人,永以遐弃,命也如此,知复何言?自去秋已来,常忽忽如有所失,于喧哗之下,或勉为语笑,闲宵自处,无不泪零。乃至梦寝之间,亦多感咽。离忧之思,绸缪缱绻,暂若寻常;幽会未终,惊魂已断。虽半衾如暖,而思之甚遥。一昨拜辞,倏逾旧岁。长安行乐之地,触绪牵情,何幸不忘幽微,眷念无斁。鄙薄之志,无以奉酬。至于终始之盟,则固不忒。鄙昔中表相因,或同宴处,婢仆见诱,遂致私诚。儿女之心,不能自固。君子有援琴之挑,鄙人无投梭之拒。及荐寝席,义盛意深,愚陋之情,永谓终托。岂期既见君子,而不能定情,致有自献之羞,不复明侍巾帻。没身永恨,含叹何言?倘仁人用心,俯遂幽眇;虽死之日,犹生之年。如或达士略情,舍小従大,以先配为丑行,以要盟为可欺。则当骨化形销,丹诚不泯;因风委露,犹托清尘。存没之诚,言尽于此;临纸呜咽,情不能申。千万珍重!珍重千万!玉环一枚,是儿婴年所弄,寄充君子下体所佩。玉取其坚润不渝,环取其终始不绝。兼乱丝一絇,文竹茶碾子一枚。此数物不足见珍,意者欲君子如玉之真,弊志如环不解,泪痕在竹,愁绪萦丝,因物达情,永以为好耳。心迩身遐,拜会无期,幽愤所钟,千里神合。千万珍重!春风多厉,强饭为嘉。慎言自保,无以鄙为深念。"张生发其书于所知,由是时人多闻之。

所善杨巨源好属词,因为赋《崔娘诗》一绝云:"清润潘郎玉不如,中庭蕙草雪销初。风流才子多春思,肠断萧娘一纸书。"河南元稹,亦续生《会真诗》三十韵。诗曰:

微月透帘栊,萤光度碧空。遥天初缥缈,低树渐葱胧。龙吹过庭竹,鸾歌拂井桐。罗绡垂薄雾,环佩响轻风。绛节随金母,云心捧玉童。更深人悄悄,晨会雨蒙蒙。珠莹光文履,花明隐绣龙。瑶钗行彩凤,罗帔掩丹虹。言自瑶华浦,将朝碧玉宫。因游洛城北,偶向宋家东。戏调初微拒,柔情已暗通。低鬟蝉影动,回步玉尘蒙。转面流花雪,登床抱绮丛。鸳鸯交颈舞,翡翠合欢笼。眉黛羞偏聚,唇朱暖更融。气清兰蕊馥,肤润玉肌丰。无力佣移腕,多娇爱敛躬。汗流珠点点,发乱绿葱葱。方喜千年会,俄闻五夜穷。留连时有恨,缱绻意难终。慢脸含愁态,芳词誓素衷。赠环明运合,留结表心同。啼粉流宵镜,残灯远暗虫。华光犹苒苒,旭日渐瞳瞳。乘鹜还归洛,吹箫亦上嵩。衣香犹染麝,枕腻尚残红。幂幂临塘草,飘飘思渚蓬。素琴鸣怨鹤,清汉望归鸿。海阔诚难渡,天高不易冲。行云无处所,萧史在楼中。

张之友闻之者,莫不耸异之,然而张志亦绝矣。稹特与张厚,因徵其词。张曰:"大凡天之所命尤物也,不妖其身,必妖于人。使崔氏子遇合富贵,乘宠娇,不为云,不为雨,为蛟为螭,吾不知其所变化矣。昔殷之辛,周之幽,据百万之国,其势甚厚。然而一女子败之,溃其众,屠其身,至今为天下僇笑。予之德不足以胜妖孽,是用忍情。"于时坐者皆为深叹。

后岁余,崔已委身于人,张亦有所娶。适经所居,乃因其夫言于崔,求以外兄见。夫语之,而崔终不为出。张怨念之诚,动于颜色,崔知之,潜赋一章词曰:"自従消瘦减容光,万转千回懒下床。不为旁人羞不起,为郎憔悴却羞郎。"竟不之见。后数日,张生将行,又赋一章以谢绝云:"弃置今何道,当时且自亲。还将旧时意,怜取眼前人。"自是绝不复知矣。时人多许张为善补过者。予常于朋会之中,往往及此意者,夫使知者不为,为之者不惑。贞元岁九月,执事李公垂,宿于予靖安里第,语及于是。公垂卓然称异,遂为《莺莺歌》以传之。崔氏小名莺莺,公垂以命篇。


\chapter*{齐桓公伐楚}
\addcontentsline{toc}{chapter}{齐桓公伐楚}
\begin{center}
	\textbf{[春秋战国]左丘明}
\end{center}

齐侯与蔡姬乘舟于囿,荡公。公惧变色;禁之,不可。公怒,归之,未之绝也。蔡人嫁之。

四年春,齐侯以诸侯之师侵蔡,蔡溃,遂伐楚。楚子使与师言曰:“君处北海,寡人处南海,唯是风马牛不相及也。不虞君之涉吾地也,何故?”管仲对曰:“昔召康公命我先君太公曰:‘五侯九伯,女实征之,以夹辅周室。’赐我先君履:东至于海,西至于河,南至于穆陵,北至于无棣。尔贡包茅不入,王祭不共,无以缩酒,寡人是征;昭王南征而不复,寡人是问。”对曰:“贡之不入,寡君之罪也,敢不共给?昭王之不复,君其问诸水滨。”(徵通征)

师进,次于陉。

夏,楚子使屈完如师。师退,次于召陵。齐侯陈诸侯之师,与屈完乘而观之。齐侯曰:“岂不榖()是为?先君之好是继,与不榖同好,何如?”对曰:“君惠徼福于敝邑之社稷,辱收寡君,寡君之愿也。”齐侯曰:“以此众战,谁能御之!以此攻城,何城不克!”对曰:“君若以德绥诸侯,谁敢不服?君若以力,楚国方城以为城,汉水以为池,虽众,无所用之!”

屈完及诸侯盟。


\chapter*{柳毅传}
\addcontentsline{toc}{chapter}{柳毅传}
\begin{center}
	\textbf{[唐朝]李朝威}
\end{center}

仪凤中,有儒生柳毅者,应举下第,将还湘滨。念乡人有客于泾阳者,遂往告别。至六七里,鸟起马惊,疾逸道左。又六七里,乃止。见有妇人,牧羊于道畔。毅怪视之,乃殊色也。然而蛾脸不舒,巾袖无光,凝听翔立,若有所伺。毅诘之曰:“子何苦而自辱如是?”妇始楚而谢,终泣而对曰:“贱妾不幸,今日见辱问于长者。然而恨贯肌骨,亦何能愧避?幸一闻焉。妾,洞庭龙君小女也。父母配嫁泾川次子,而夫婿乐逸,为婢仆所惑,日以厌薄。既而将诉于舅姑,舅姑爱其子,不能御。迨诉频切,又得罪舅姑。舅姑毁黜以至此。”言讫,歔欷流涕,悲不自胜。又曰:“洞庭于兹,相远不知其几多也?长天茫茫,信耗莫通。心目断尽,无所知哀。闻君将还吴,密通洞庭。或以尺书寄托侍者,未卜将以为可乎?”毅曰:“吾义夫也。闻子之说,气血俱动,恨无毛羽,不能奋飞,是何可否之谓乎!然而洞庭深水也。吾行尘间,宁可致意耶?惟恐道途显晦,不相通达,致负诚托,又乖恳愿。子有何术可导我邪?”女悲泣且谢,曰:“负载珍重,不复言矣。脱获回耗,虽死必谢。君不许,何敢言。既许而问,则洞庭之与京邑,不足为异也。”毅请闻之。女曰:“洞庭之阴,有大橘树焉,乡人谓之‘社橘’。君当解去兹带,束以他物。然后叩树三发,当有应者。因而随之,无有碍矣。幸君子书叙之外,悉以心诚之话倚托,千万无渝!”毅曰:“敬闻命矣。”女遂于襦间解书,再拜以进。东望愁泣,若不自胜。毅深为之戚,乃致书囊中,因复谓曰:“吾不知子之牧羊,何所用哉?神岂宰杀乎?”女曰:“非羊也,雨工也。”“何为雨工?”曰:“雷霆之类也。”毅顾视之,则皆矫顾怒步,饮龁甚异,而大小毛角,则无别羊焉。毅又曰:“吾为使者,他日归洞庭,幸勿相避。”女曰:“宁止不避,当如亲戚耳。”语竟,引别东去。不数十步,回望女与羊,俱亡所见矣。

其夕,至邑而别其友,月余到乡,还家,乃访友于洞庭。洞庭之阴,果有社橘。遂易带向树,三击而止。俄有武夫出于波问,再拜请曰:“贵客将自何所至也?”毅不告其实,曰:“走谒大王耳。”武夫揭水止路,引毅以进。谓毅曰:“当闭目,数息可达矣。”毅如其言,遂至其宫。始见台阁相向,门户千万,奇草珍木,无所不有.夫乃止毅,停于大室之隅,曰:“客当居此以俟焉。”毅曰:“此何所也?”夫曰:“此灵虚殿也。”谛视之,则人间珍宝毕尽于此。柱以白璧,砌以青玉,床以珊瑚,帘以水精,雕琉璃于翠楣,饰琥珀于虹栋。奇秀深杳,不可殚言。然而王久不至。毅谓夫曰:“洞庭君安在哉?”曰:“吾君方幸玄珠阁,与太阳道士讲《火经》,少选当毕。”毅曰:“何谓《火经》?”夫曰:“吾君,龙也。龙以水为神,举一滴可包陵谷。道士,乃人也。人以火为神圣,发一灯可燎阿房。然而灵用不同,玄化各异。太阳道士精于人理,吾君邀以听焉。”语毕而宫门辟,景从云合,而见一人,披紫衣,执青玉。夫跃曰:“此吾君也!”乃至前以告之。

君望毅而问曰:“岂非人间之人乎?”对曰:“然。”毅而设拜,君亦拜,命坐于灵虚之下。谓毅曰:“水府幽深,寡人暗昧,夫子不远千里,将有为乎?”毅曰:“毅,大王之乡人也。长于楚,游学于秦。昨下第,闲驱泾水右涘,见大王爱女牧羊于野,风鬟雨鬓,所不忍睹。毅因诘之,谓毅曰:‘为夫婿所薄,舅姑不念,以至于此’。悲泗淋漓,诚怛人心。遂托书于毅。毅许之,今以至此。”因取书进之。洞庭君览毕,以袖掩面而泣曰:“老父之罪,不能鉴听,坐贻聋瞽,使闺窗孺弱,远罹构害。公,乃陌上人也,而能急之。幸被齿发,何敢负德!”词毕,又哀咤良久。左右皆流涕。时有宦人密侍君者,君以书授之,令达宫中。须臾,宫中皆恸哭。君惊,谓左右曰:“疾告宫中,无使有声,恐钱塘所知。”毅曰:“钱塘,何人也?”曰:“寡人之爱弟,昔为钱塘长,今则致政矣。”毅曰:“何故不使知?”曰:“以其勇过人耳。昔尧遭洪水九年者,乃此子一怒也。近与天将失意,塞其五山。上帝以寡人有薄德于古今,遂宽其同气之罪。然犹縻系于此,故钱塘之人日日候焉。”语未毕,而大声忽发,天拆地裂。宫殿摆簸,云烟沸涌。俄有赤龙长千余尺,电目血舌,朱鳞火鬣,项掣金锁,锁牵玉柱。千雷万霆,激绕其身,霰雪雨雹,一时皆下。乃擘青天而飞去。毅恐蹶仆地。君亲起持之曰:“无惧,固无害。”毅良久稍安,乃获自定。因告辞曰:“愿得生归,以避复来。”君曰:“必不如此。其去则然,其来则不然,幸为少尽缱绻。”因命酌互举,以款人事。

俄而祥风庆云,融融恰怡,幢节玲珑,箫韶以随。红妆千万,笑语熙熙。中有一人,自然蛾眉,明珰满身,绡縠参差。迫而视之,乃前寄辞者。然若喜若悲,零泪如丝。须臾,红烟蔽其左,紫气舒其右,香气环旋,入于宫中。君笑谓毅曰:“泾水之囚人至矣。”君乃辞归宫中。须臾,又闻怨苦,久而不已。有顷,君复出,与毅饮食。又有一人,披紫裳,执青玉,貌耸神溢,立于君左。君谓毅曰:“此钱塘也。”毅起,趋拜之。钱塘亦尽礼相接,谓毅曰:“女侄不幸,为顽童所辱。赖明君子信义昭彰,致达远冤。不然者,是为泾陵之土矣。飨德怀恩,词不悉心。”毅撝退辞谢,俯仰唯唯。然后回告兄曰:“向者辰发灵虚,巳至泾阳,午战于彼,未还于此。中间驰至九天,以告上帝。帝知其冤,而宥其失。前所谴责,因而获免。然而刚肠激发,不遑辞候,惊扰宫中,复忤宾客。愧惕惭惧,不知所失。”因退而再拜。君曰:“所杀几何?”曰:“六十万。”“伤稼乎?”曰:“八百里。”无情郎安在?”曰:“食之矣。”君怃然曰:“顽童之为是心也,诚不可忍,然汝亦太草草。赖上帝显圣,谅其至冤。不然者,吾何辞焉?从此以去,勿复如是。”钱塘君复再拜。是夕,遂宿毅于凝光殿。

明日,又宴毅于凝碧宫。会友戚,张广乐,具以醪醴,罗以甘洁。初,笳角鼙鼓,旌旗剑戟,舞万夫于其右。中有一夫前曰:“此《钱塘破阵乐》。”旌杰气,顾骤悍栗。座客视之,毛发皆竖。复有金石丝竹,罗绮珠翠,舞千女于其左,中有一女前进曰:“此《贵主还宫乐》。”清音宛转,如诉如慕,坐客听下,不觉泪下。二舞既毕,龙君大悦。锡以纨绮,颁于舞人,然后密席贯坐,纵酒极娱。酒酣,洞庭君乃击席而歌曰:“大天苍苍兮,大地茫茫,人各有志兮,何可思量,狐神鼠圣兮,薄社依墙。雷霆一发兮,其孰敢当?荷贞人兮信义长,令骨肉兮还故乡,齐言惭愧兮何时忘!”洞庭君歌罢,钱塘君再拜而歌曰:“上天配合兮,生死有途。此不当妇兮,彼不当夫。腹心辛苦兮,泾水之隅。风霜满鬓兮,雨雪罗襦。赖明公兮引素书,令骨肉兮家如初。永言珍重兮无时无。”钱塘君歌阕,洞庭君俱起,奉觞于毅。毅踧踖而受爵,饮讫,复以二觞奉二君,乃歌曰:“碧云悠悠兮,泾水东流。伤美人兮,雨泣花愁。尺书远达兮,以解君忧。哀冤果雪兮,还处其休。荷和雅兮感甘羞。山家寂寞兮难久留。欲将辞去兮悲绸缪。”歌罢,皆呼万岁。洞庭君因出碧玉箱,贮以开水犀;钱塘君复出红珀盘,贮以照夜玑:皆起进毅,毅辞谢而受。然后宫中之人,咸以绡彩珠璧,投于毅侧。重叠焕赫,须臾埋没前后。毅笑语四顾,愧谢不暇。洎酒阑欢极,毅辞起,复宿于凝光殿。

翌日,又宴毅于清光阁。钱塘因酒作色,踞谓毅曰:“不闻猛石可裂不可卷,义士可杀不可羞耶?愚有衷曲,欲一陈于公。如可,则俱在云霄;如不可,则皆夷粪壤。足下以为何如哉?”毅曰:“请闻之。”钱塘曰:“泾阳之妻,则洞庭君之爱女也。淑性茂质,为九姻所重。不幸见辱于匪人,今则绝矣。将欲求托高义,世为亲戚,使受恩者知其所归,怀爱者知其所付,岂不为君子始终之道者?”毅肃然而作,欻然而笑曰:“诚不知钱塘君孱困如是!毅始闻跨九州,怀五岳,泄其愤怒;复见断金锁,掣玉柱,赴其急难。毅以为刚决明直,无如君者。盖犯之者不避其死,感之者不爱其生,此真丈夫之志。奈何萧管方洽,亲宾正和,不顾其道,以威加人?岂仆人素望哉!若遇公于洪波之中,玄山之间,鼓以鳞须,被以云雨,将迫毅以死,毅则以禽兽视之,亦何恨哉!今体被衣冠,坐谈礼义,尽五常之志性,负百行怖之微旨,虽人世贤杰,有不如者,况江河灵类乎?而欲以蠢然之躯,悍然之性,乘酒假气,将迫于人,岂近直哉!且毅之质,不足以藏王一甲之间。然而敢以不伏之心,胜王不道之气。惟王筹之!”钱塘乃逡巡致谢曰:“寡人生长宫房,不闻正论。向者词述疏狂,妄突高明。退自循顾,戾不容责。幸君子不为此乖问可也。”其夕,复饮宴,其乐如旧。毅与钱塘遂为知心友。

明日,毅辞归。洞庭君夫人别宴毅于潜景殿,男女仆妾等悉出预会。夫人泣谓毅曰:“骨肉受君子深恩,恨不得展愧戴,遂至睽别。”使前泾阳女当席拜毅以致谢。夫人又曰:“此别岂有复相遇之日乎?”毅其始虽不诺钱塘之情,然当此席,殊有叹恨之色。宴罢,辞别,满宫凄然。赠遗珍宝,怪不可述。毅于是复循途出江岸,见从者十余人,担囊以随,至其家而辞去。毅因适广陵宝肆,鬻其所得。百未发一,财已盈兆。故淮右富族,咸以为莫如。遂娶于张氏,亡。又娶韩氏。数月,韩氏又亡。徙家金陵。常以鳏旷多感,或谋新匹。有媒氏告之曰:“有卢氏女,范阳人也。父名曰浩,尝为清流宰。晚岁好道,独游云泉,今则不知所在矣。母曰郑氏。前年适清河张氏,不幸而张夫早亡。母怜其少,惜其慧美,欲择德以配焉。不识何如?”毅乃卜日就礼。既而男女二姓俱为豪族,法用礼物,尽其丰盛。金陵之士,莫不健仰。居月余,毅因晚入户,视其妻,深觉类于龙女,而艳逸丰厚,则又过之。因与话昔事。妻谓毅曰:“人世岂有如是之理乎?”

经岁余,有一子。毅益重之。既产,逾月,乃秾饰换服,召毅于帘室之间,笑谓毅曰:“君不忆余之于昔也?”毅曰:“夙为姻好,何以为忆?”妻曰:“余即洞庭君之女也。泾川之冤,君使得白。衔君之恩,誓心求报。洎钱塘季父论亲不从,遂至睽违。天各一方,不能相问。父母欲配嫁于濯锦小儿某。遂闭户剪发,以明无意。虽为君子弃绝,分见无期。而当初之心,死不自替。他日父母怜其志,复欲驰白于君子。值君子累娶,当娶于张,已而又娶于韩。迨张、韩继卒,君卜居于兹,故余之父母乃喜余得遂报君之意。今日获奉君子,咸善终世,死无恨矣。”因呜咽,泣涕交下。对毅曰:“始不言者,知君无重色之心。今乃言者,知君有感余之意。妇人匪薄,不足以确厚永心,故因君爱子,以托相生。未知君意如何?愁惧兼心,不能自解。君附书之日,笑谓妾曰:‘他日归洞庭,慎无相避。’诚不知当此之际,君岂有意于今日之事乎?其后季父请于君,君固不许。君乃诚将不可邪,抑忿然邪?君其话之。”毅曰:“似有命者。仆始见君子,长泾之隅,枉抑憔悴,诚有不平之志。然自约其心者,达君之冤,余无及也。以言‘慎无相避’者,偶然耳,岂有意哉。洎钱塘逼迫之际,唯理有不可直,乃激人之怒耳。夫始以义行为之志,宁有杀其婿而纳其妻者邪?一不可也。某素以操真为志尚,宁有屈于己而伏于心者乎?二不可也。且以率肆胸臆,酬酢纷纶,唯直是图,不遑避害。然而将别之日。见君有依然之容,心甚恨之。终以人事扼束,无由报谢。吁,今日,君,卢氏也,又家于人间。则吾始心未为惑矣。从此以往,永奉欢好,心无纤虑也。”妻因深感娇泣,良久不已。有顷,谓毅曰:“勿以他类,遂为无心,固当知报耳。夫龙寿万岁,今与君同之。水陆无往不适。君不以为妄也。”毅嘉之曰:“吾不知国客乃复为神仙之饵!”。乃相与觐洞庭。既至,而宾主盛礼,不可具纪。

后居南海仅四十年,其邸第、舆马、珍鲜、服玩,虽侯伯之室,无以加也。毅之族咸遂濡泽。以其春秋积序,容状不衰。南海之人,靡不惊异。

洎开元中,上方属意于神仙之事,精索道术。毅不得安,遂相与归洞庭。凡十余岁,莫知其迹。

至开元末,毅之表弟薛嘏为京畿令,谪官东南。经洞庭,晴昼长望,俄见碧山出于远波。舟人皆侧立,曰:“此本无山,恐水怪耳。”指顾之际,山与舟相逼,乃有彩船自山驰来,迎问于嘏。其中有一人呼之曰:“柳公来候耳。”嘏省然记之,乃促至山下,摄衣疾上。山有宫阙如人世,见毅立于宫室之中,前列丝竹,后罗珠翠,物玩之盛,殊倍人间。毅词理益玄,容颜益少。初迎嘏于砌,持嘏手曰:“别来瞬息,而发毛已黄。”嘏笑曰:“兄为神仙,弟为枯骨,命也。”毅因出药五十丸遗嘏,曰:“此药一丸,可增一岁耳。岁满复来,无久居人世以自苦也。”欢宴毕,嘏乃辞行。自是已后,遂绝影响。嘏常以是事告于人世。殆四纪,嘏亦不知所在。

陇西李朝威叙而叹曰:“五虫之长,必以灵者,别斯见矣。人,裸也,移信鳞虫。洞庭含纳大直,钱塘迅疾磊落,宜有承焉。嘏咏而不载,独可邻其境。愚义之,为斯文。”


\chapter*{矛与盾}
\addcontentsline{toc}{chapter}{矛与盾}
\begin{center}
	\textbf{[春秋战国]韩非}
\end{center}

楚人有鬻盾与矛者,誉之曰:“吾盾之坚,物莫能陷也。”又誉其矛曰:“吾矛之利,于物无不陷也。”或曰:“以子之矛,陷子之盾,何如?”其人弗能应也。夫不可陷之盾与无不陷之矛,不可同世而立。


\chapter*{哀溺文序}
\addcontentsline{toc}{chapter}{哀溺文序}
\begin{center}
	\textbf{[唐朝]柳宗元}
\end{center}

永之氓咸善游。一日,水暴甚,有五、六氓乘小船绝湘水。中济,船破,皆游。其一氓尽力而不能寻常。其侣曰:“汝善游最也,今何后为?”曰:“吾腰千钱,重,是以后。”曰:“何不去之?”不应,摇其首。有顷,益怠。已济者立岸上呼且号曰:“汝愚之甚,蔽之甚,身且死,何以货为?”又摇其首。遂溺死。吾哀之。且若是,得不有大货之溺大氓者乎?于是作《哀溺》。


\chapter*{段太尉逸事状}
\addcontentsline{toc}{chapter}{段太尉逸事状}
\begin{center}
	\textbf{[唐朝]柳宗元}
\end{center}

太尉始为泾州刺史时,汾阳王以副元帅居蒲。王子晞为尚书,领行营节度使,寓军邠州,纵士卒无赖。邠人偷嗜暴恶者,卒以货窜名军伍中,则肆志,吏不得问。日群行丐取于市,不嗛,辄奋击折人手足,椎釜鬲瓮盎盈道上,袒臂徐去,至撞杀孕妇人。邠宁节度使白孝德以王故,戚不敢言。

太尉自州以状白府,愿计事。至则曰:“天子以生人付公理,公见人被暴害,因恬然。且大乱,若何?”孝德曰:“愿奉教。”太尉曰:“某为泾州,甚适,少事;今不忍人无寇暴死,以乱天子边事。公诚以都虞候命某者,能为公已乱,使公之人不得害。”孝德曰:“幸甚!”如太尉请。

既署一月,晞军士十七人入市取酒,又以刃刺酒翁,坏酿器,酒流沟中。太尉列卒取十七人,皆断头注槊上,植市门外。晞一营大噪,尽甲。孝德震恐,召太尉曰:“将奈何?”太尉曰:“无伤也!请辞于军。”孝德使数十人从太尉,太尉尽辞去。解佩刀,选老躄者一人持马,至晞门下。甲者出,太尉笑且入曰:“杀一老卒,何甲也?吾戴吾头来矣!”甲者愕。因谕曰:“尚书固负若属耶?副元帅固负若属耶?奈何欲以乱败郭氏?为白尚书,出听我言。”晞出见太尉。太尉曰:“副元帅勋塞天地,当务始终。今尚书恣卒为暴,暴且乱,乱天子边,欲谁归罪?罪且及副元帅。今邠人恶子弟以货窜名军籍中,杀害人,如是不止,几日不大乱?大乱由尚书出,人皆曰尚书倚副元帅,不戢士。然则郭氏功名,其与存者几何?”

言未毕,晞再拜曰:“公幸教晞以道,恩甚大,愿奉军以从。”顾叱左右曰:“皆解甲散还火伍中,敢哗者死!”太尉曰:“吾未晡食,请假设草具。”既食,曰:“吾疾作,愿留宿门下。”命持马者去,旦日来。遂卧军中。晞不解衣,戒候卒击柝卫太尉。旦,俱至孝德所,谢不能,请改过。邠州由是无祸。

先是,太尉在泾州为营田官。泾大将焦令谌取人田,自占数十顷,给与农,曰:“且熟,归我半。”是岁大旱,野无草,农以告谌。谌曰:“我知入数而已,不知旱也。”督责益急,农且饥死,无以偿,即告太尉。太尉判状辞甚巽,使人求谕谌。谌盛怒,召农者曰:“我畏段某耶?何敢言我!”取判铺背上,以大杖击二十,垂死,舆来庭中。太尉大泣曰:“乃我困汝!”即自取水洗去血,裂裳衣疮,手注善药,旦夕自哺农者,然后食。取骑马卖,市谷代偿,使勿知。

淮西寓军帅尹少荣,刚直士也。入见谌,大骂曰:“汝诚人耶?泾州野如赭,人且饥死;而必得谷,又用大杖击无罪者。段公,仁信大人也,而汝不知敬。今段公唯一马,贱卖市谷入汝,汝又取不耻。凡为人傲天灾、犯大人、击无罪者,又取仁者谷,使主人出无马,汝将何以视天地,尚不愧奴隶耶!”谌虽暴抗,然闻言则大愧流汗,不能食,曰:“吾终不可以见段公!”一夕,自恨死。

及太尉自泾州以司农征,戒其族:“过岐,朱泚幸致货币,慎勿纳。”及过,泚固致大绫三百匹。太尉婿韦晤坚拒,不得命。至都,太尉怒曰:“果不用吾言!”晤谢曰:“处贱无以拒也。”太尉曰:“然终不以在吾第。”以如司农治事堂,栖之梁木上。泚反,太尉终,吏以告泚,泚取视,其故封识具存。

太尉逸事如右。元和九年月日,永州司马员外置同正员柳宗元谨上史馆。

今之称太尉大节者出入,以为武人一时奋不虑死,以取名天下,不知太尉之所立如是。宗元尝出入岐周邠斄间,过真定,北上马岭,历亭障堡戍,窃好问老校退卒,能言其事。太尉为人姁姁,常低首拱手行步,言气卑弱,未尝以色待物;人视之,儒者也。遇不可,必达其志,决非偶然者。会州刺史崔公来,言信行直,备得太尉遗事,覆校无疑,或恐尚逸坠,未集太史氏,敢以状私于执事。谨状。


\chapter*{封禅文}
\addcontentsline{toc}{chapter}{封禅文}
\begin{center}
	\textbf{[汉朝]司马相如}
\end{center}

伊上古之初肇,自昊穹兮生民。历撰列辟,以迄于秦。率迩者踵武,逖(tì)听者风声。纷纶葳蕤,堙灭而不称者,不可胜数也。续《韶》《夏》,崇号谥,略可道者七十有二君。罔若淑而不昌,畴逆失而能存?


轩辕之前,遐哉邈乎,其详不可得闻也。五三《六经》载籍之传,维见可观也。《书》曰:“元首明哉,股肱(gōng))良哉。”因斯以谈,君莫盛于唐尧,臣莫贤于后稷。后稷创业于唐尧,公刘发迹于西戎,文王改制,爰周郅(zhì)隆,大行越成,而后陵夷衰微,千载无声,岂不善始善终哉!然无异端,慎所由于前,谨遗教于后耳。故轨迹夷易,易遵也;湛恩濛涌,易丰也;宪度著明,易则也;垂统理顺,易继也。是以业隆于襁褓而崇冠于二后。揆其所元,终都攸卒,未有殊尤绝迹可考于今者也。然犹蹑梁父,登泰山,建显号,施尊名。大汉之德,烽涌原泉,沕(mì)潏(yù)漫衍,旁魄四塞,云尃(fú)雾散,上畅九垓,下沂八埏(yán)。怀生之类沾濡浸润,协气横流,武节飘逝,迩陕游原,迥阔泳沫,首恶湮没,暗昧昭晰,昆虫凯泽,回首面内。然后囿驺虞之珍群,徼麋鹿之怪兽,噵(dào)一茎六穗于庖,牺双觡(gé)共抵之兽,获周余珍,收龟于岐,招翠黄乘龙于沼。鬼神接灵圉,宾于闲馆。奇物谲(jué)诡,俶(tì)傥(tǎng)穷变。钦哉,符瑞臻兹,犹以为薄,不敢道封禅。盖周跃鱼陨杭,休之以燎,微夫斯之为符也,以登介丘,不亦恧(nǜ)乎!进让之道,其何爽与!


于是大司马进曰:“陛下仁育群生,义征不憓(huì),诸夏乐贡,百蛮执贽,德侔往初,功无与二,休烈浃洽,符瑞众变,期应绍至,不特创见,意者泰山、梁父设坛场望幸。盖号以况荣,上帝垂恩储祉,将以荐成。陛下谦让而弗发也,挈(qiè)三神之欢,缺王道之仪,群臣恧焉。或谓且天为质暗,示珍符固不可辞;若然辞之,是泰山靡记而梁父靡几也。亦各并时而荣,咸济世而屈,说者尚何称于后,而云七十二君乎?夫修德以锡符,奉符以行事,不为进越。故圣王弗替,而修礼地祗,谒欵天神。勒功中岳,以彰至尊,舒盛德,发号荣,受厚福,以浸黎民也。皇皇哉斯事!天下之壮观,王者之丕业,不可贬也。愿陛下全之。而后因杂荐绅先生之略术,使获耀日月之末光绝炎,以展采错事;犹兼正列其义,校饬厥文,作《春秋》一艺,将袭旧六为七,摅(shū)之无穷,俾万世得激清流,扬微波,蜚英声,腾茂实。前圣之所以永保鸿名而常为称首者用此,宜命掌故悉奏其义而览焉。”


于是天子沛然改容,曰:“愉乎,朕其试哉!”乃迁思回虑,总公卿之议,询封禅之事,诗大泽之博,广符瑞之富。乃作颂曰:


自我天覆,云之油油。甘露时雨,阙壤可游。滋液渗漉,何生不育;嘉谷六穗,我穑曷蓄。


非唯雨之,又润泽之;非唯濡之,氾尃护之。万物熙熙,怀而慕思。名山显位,望君之来。君乎君乎,侯不迈哉!


般般之兽,乐我君囿;白质黑章,其仪可嘉;旼(mín)旼睦睦,君子之能。盖闻其声,今观其来。厥途靡踪,天瑞之征。兹亦于舜,虞氏以兴。


濯濯之麟,游彼灵畤(zhì)。孟冬十月,君徂郊祀。驰我君舆,帝以享祉。三代之前,未之尝有。


宛宛黄龙,兴德而升;采色炫耀,熿(huáng)炳辉湟。正阳显见,觉寤黎烝。于传载之,云受命所乘。


厥之有章,不必谆谆。依类讬寓,谕以封峦。


披艺观之,天人之际已交,上下相发允答。圣王之德,兢兢翼翼也。故曰:“兴必虑衰,安必思危。”是以汤武至尊严,不失肃祗;舜在假典,顾省厥遗:此之谓也。



\chapter*{医说}
\addcontentsline{toc}{chapter}{医说}
\begin{center}
	\textbf{[唐朝]韩愈}
\end{center}

善医者,不视人之瘠肥,察其脉之病否而已矣;善计天下者,不视天下之安危,察其纪纲之理乱而已矣。天下者,人也;安危者,肥瘠也;纪纲者,脉也。脉不病,虽瘠不害;脉病而肥者,死矣。通于此说者,其知所以为天下乎!夏、殷、周之衰也,诸侯作而战伐日行矣。传数十王而天下不倾者,纪纲存焉耳。秦之王天下也,无分势于诸侯,聚兵而焚之,传二世而天下倾者,纪纲亡焉耳。是故四支虽无故,不足恃也,脉而已矣;四海虽无事,不足矜也,纪纲而已矣。忧其所可恃,惧其所可矜,善医善计者,谓之天扶与之。《易》曰:“视履考祥。”善医善计者为之。



\chapter*{永某氏之鼠}
\addcontentsline{toc}{chapter}{永某氏之鼠}
\begin{center}
	\textbf{[唐朝]柳宗元}
\end{center}

永有某氏者,畏日,拘忌异甚。以为己生岁直子,鼠,子神也,因爱鼠,不畜猫犬,禁僮勿击鼠。仓廪庖厨,悉以恣鼠不问。由是鼠相告,皆来某氏,饱食而无祸。某氏室无完器,椸无完衣,饮食大率鼠之余也。昼累累与人兼行,夜则窃啮斗暴,其声万状,不可以寝,终不厌。

数岁,某氏徙居他州。后人来居,鼠为态如故。其人曰:“是阴类恶物也,盗暴尤甚。且何以至是乎哉?”假五六猫,阖门,撤瓦,灌穴,购僮罗捕之。杀鼠如丘,弃之隐处,臭数月乃已。

呜呼!彼以其饱食无祸为可恒也哉!


\chapter*{喻巴蜀檄}
\addcontentsline{toc}{chapter}{喻巴蜀檄}
\begin{center}
	\textbf{[汉朝]司马相如}
\end{center}

告巴蜀太守:蛮夷自擅,不讨之日久矣,时侵犯边境,劳士大夫。陛下即位,存抚天下,辑安中国,然后兴师出兵,北征匈奴。单于怖骇,交臂受事,屈膝请和。康居西域,重译请朝,稽首来享。移师东指,闽越相诛;右吊番禺,太子入朝。南夷之君,西僰(bó)之长,常效贡职,不敢怠堕,延颈举踵,喁喁然皆争归义,欲为臣妾;道里辽远,山川阻深,不能自致。夫不顺者已诛,而为善者未赏,故遣中郎将往宾之,发巴蜀士民各五百人,以奉币帛,卫使者不然,靡有兵革之事,战斗之患。今闻其乃发军兴制,警惧子弟,忧患长老,郡又擅为转粟运输,皆非陛下之意也。当行者或亡逃自贼杀,亦非人臣之节也。


夫边郡之士,闻烽举燧燔,皆摄弓而驰,荷兵而走,流汗相属,唯恐居后;触白刃,冒流矢,义不反顾,计不旋踵,人怀怒心,如报私仇。彼岂乐死恶生,非编列之民,而与巴蜀异主哉?计深虑远,急国家之难,而乐尽人臣之道也。故有剖符之封,析珪之爵,位为通侯,居列东第,终则遗显号于后世,传土地于子孙。行事甚忠敬,居位安佚,名声施于无穷,功烈著而不灭。是以贤人君子,肝脑涂中原,膏液润野草而不辞也。今奉币役至南夷,即自贼杀,或亡逃抵诛,身死无名,谥为至愚,耻及父母,为天下笑。人之度量相越,岂不远哉!然此非独行者之罪也,父兄之教不先,子弟之率不谨也,寡廉鲜耻;而俗不长厚也。其被刑戮,不亦宜乎!


陛下患使者有司之若彼,悼不肖愚民之如此,故遣信使晓谕百姓以发卒之事,因数之以不忠死亡之罪,让三老孝悌以不教之过。方今田时,重烦百姓,已亲见近县,恐远所溪谷山泽之民不遍闻,檄到,亟下县道,使咸知陛下之意,唯毋忽也。



\chapter*{朋党论}
\addcontentsline{toc}{chapter}{朋党论}
\begin{center}
	\textbf{[宋朝]欧阳修}
\end{center}

臣闻朋党之说,自古有之,惟幸人君辨其君子、小人而已。


大凡君子与君子以同道为朋,小人与小人以同利为朋,此自然之理也。然臣谓小人无朋,惟君子则有之,其故何哉?小人之所好者,禄利也;所贪者,财货也。当其同利之时,暂相党引以为朋者,伪也;及其见利而争先,或利尽而交疏,则反相贼害,虽其兄弟亲戚不能相保。故臣谓小人无朋,其暂为朋者,伪也。君子则不然,所守者道义,所行者忠信,所惜者名节。以之修身,则同道而相益;以之事国,则同心而共济,始终如一,此君子之朋也,故为人君者,但当退小人之伪朋,用君子之真朋,则天下治矣。


尧之时,小人共工、驩兜等四人为一朋,君子八元、八恺十六人为一朋。舜佐尧,退四凶小人之朋,而进元、恺君子之朋,尧之天下大治。及舜自为天子,而皋、夔、稷、契等二十二人并列于朝,更相称美,更相推让,凡二十二人为一朋,而舜皆用之,天下亦大治。《书》曰:“纣有臣亿万,惟亿万心;周有臣三千,惟一心。”纣之时,亿万人各异心,可谓不为朋矣。然纣以亡国。周武王之臣,三千人为一大朋,而周用以兴。后汉献帝时,尽取天下名士囚禁之,目为党人。及黄巾贼起,汉室大乱,后方悔悟,尽解党人而释之,然已无救矣。唐之晚年,渐起朋党之论,及昭宗时,尽杀朝之名士,或投之黄河,曰:“此辈清流,可投浊流。”而唐遂亡矣。


夫前世之主,能使人人异心不为朋,莫如纣;能禁绝善人为朋,莫如汉献帝;能诛戮清流之朋,莫如唐昭宗之世:然皆乱亡其国。更相称美,推让而不自疑,莫如舜之二十二臣,舜亦不疑而皆用之。然而后世不诮舜为二十二人朋党所欺,而称舜为聪明之圣者,以能辨君子与小人也。周武之世,举其国之臣三千人共为一朋,自古为朋之多且大莫如周,然周用此以兴者,善人虽多而不厌也。


嗟呼!夫兴亡治乱之迹,为人君者可以鉴矣!



\chapter*{亡妻王氏墓志铭}
\addcontentsline{toc}{chapter}{亡妻王氏墓志铭}
\begin{center}
	\textbf{[宋朝]苏轼}
\end{center}

治平二年五月丁亥①,赵郡苏轼之妻王氏卒于京师②。六月甲午③,殡于京城之西④。其明年六月壬午⑤,葬于眉之东北彭山县安镇乡可龙里⑥,先君、先夫人墓之西北八步。轼铭其墓曰:


君讳弗,眉之青神人,乡贡进士方之女。生十有六年而归于轼,有子迈。君之未嫁,事父母;既嫁,事吾先君先夫人⑦,皆以谨肃闻。其始,未尝自言其知书也。见轼读书,则终日不去,亦不知其能通也。其后,轼有所忘,君辄能记之。问其他书,则皆略知之,由是始知其敏而静也。


从轼官于凤翔。轼有所为于外,君未尝不问知其详。曰:“子去亲远,不可以不慎。”日以先君之所以戒轼者相语也。轼与客言于外,君立屏间听之,退必反覆其言,曰:“某人也,言辄持两端,惟子意之所向,子何用与是人言。”有来求与轼亲厚甚者,君曰:“恐不能久,其与人锐,其去人必速。”已而果然。将死之岁,其言多可听,类有识者。其死也,盖年二十有七而已。始死,先君命轼曰:“妇从汝于艰难,不可忘也。他日,汝必葬诸其姑之侧。”未期年而先君没,轼谨以遗令葬之,铭曰:


君得从先夫人于九泉,余不能。呜呼哀哉!余永无所依怙。君虽没,其有与为妇何伤乎。呜呼哀哉!



\chapter*{记游定惠院}
\addcontentsline{toc}{chapter}{记游定惠院}
\begin{center}
	\textbf{[宋朝]苏轼}
\end{center}

黄州定惠院东小山上,有海棠一株,特繁茂。每岁盛开,必携客置酒,已五醉其下矣。今年复与参寥禅师及二三子访焉,则园已易主。主虽市井人,然以予故,稍加培治。山上多老枳木,性瘦韧,筋脉呈露,如老人头颈。花白而圆,如大珠累累,香色皆不凡。此木不为人所喜,稍稍伐去,以予故,亦得不伐。既饮,往憩于尚氏之第。尚氏亦市井人也,而居处修洁,如吴越间人,竹林花圃皆可喜。醉卧小板阁上,稍醒,闻坐客崔成老弹雷氏琴,作悲风晓月,铮铮然,意非人间也。晚乃步出城东,鬻大木盆,意者谓可以注清泉,瀹瓜李,遂夤缘小沟,入何氏、韩氏竹园。时何氏方作堂竹间,既辟地矣,遂置酒竹阴下。有刘唐年主簿者,馈油煎饵,其名为甚酥,味极美。客尚欲饮,而予忽兴尽,乃径归。道过何氏小圃,乞其丛橘,移种雪堂之西。坐客徐君得之将适闽中,以后会未可期,请予记之,为异日拊掌。时参寥独不饮,以枣汤代之。



\chapter*{祭石曼卿文}
\addcontentsline{toc}{chapter}{祭石曼卿文}
\begin{center}
	\textbf{[宋朝]欧阳修}
\end{center}

维治平四年七月日,具官欧阳修,谨遣尚书都省令史李敭,至于太清,以清酌庶羞之奠,致祭于亡友曼卿之墓下,而吊之以文。曰:


呜呼曼卿!生而为英,死而为灵。其同乎万物生死,而复归于无物者,暂聚之形;不与万物共尽,而卓然其不朽者,后世之名。此自古圣贤,莫不皆然,而著在简册者,昭如日星。


呜呼曼卿!吾不见子久矣,犹能仿佛子之平生。其轩昂磊落,突兀峥嵘而埋藏于地下者,意其不化为朽壤,而为金玉之精。不然,生长松之千尺,产灵芝而九茎。奈何荒烟野蔓,荆棘纵横;风凄露下,走磷飞萤!但见牧童樵叟,歌吟而上下,与夫惊禽骇兽,悲鸣踯躅而咿嘤。今固如此,更千秋而万岁兮,安知其不穴藏狐貉与鼯鼪?此自古圣贤亦皆然兮,独不见夫累累乎旷野与荒城!


呜呼曼卿!盛衰之理,吾固知其如此,而感念畴昔,悲凉凄怆,不觉临风而陨涕者,有愧乎太上之忘情。尚飨!



\chapter*{入蜀记}
\addcontentsline{toc}{chapter}{入蜀记}
\begin{center}
	\textbf{[宋朝]陆游}
\end{center}

七月


十四日,晚,晴。开南窗观溪山。溪中绝多鱼,时裂水面跃出,斜日映之,有如银刀。垂钓挽罟者弥望,以故价甚贱,僮使辈日皆餍饫。土人云,此溪水肥,宜鱼。及饮之,水味果甘,岂信以肥故多鱼邪?溪东南数峰如黛,盖青山也。


八月


十四日,晓,雨。过一小石山,自顶直削去半,与余姚江滨之蜀山绝相类。抛大江,遇一木筏,广十余丈,长五十余丈。上有三四十家,妻子鸡犬臼碓皆具,中为阡陌相往来,亦有神祠,素所未睹也。舟人云,此尚其小者耳,大者于筏上铺土作蔬圃,或作酒肆,皆不复能入夹,但行大江而已。是日逆风挽船,自平旦至日昳(dié)才行十五六里。泊刘官矶,旁蕲州界也。儿辈登岸,归云:“得小径,至山后,有陂湖渺然,莲芰甚富。沿湖多木芙蕖,数家夕阳中,芦藩茅舍,宛有幽致,而寂然无人声。有大梨,欲买之,不可得。湖中小艇采菱,呼之亦不应。更欲穷之,会见道旁设机,疑有虎狼,遂不敢往。”刘官矶者,传云汉昭烈入吴尝杈舟于此。晚,观大鼋浮沉水中。


(八月)二十一日。过双柳夹,回望江上,远山重复深秀。自离黄,虽行夹中,亦皆旷远,地形渐高,多种菽粟荞麦之属。晚,泊杨罗,大堤高柳,居民稠众。鱼贱如土,百钱可饱二十口;又皆巨鱼,欲觅小鱼饲猫,不可得。


九月


九日,早,谒后土祠。道旁民屋,苫茅皆厚尺余,整洁无一枝乱。挂帆,抛江行三十里,泊塔子矶,江滨大山也。自离鄂州,至是始见山。买羊置酒。盖村步以重九故,屠一羊,诸舟买之,俄顷而尽。求菊花于江上人家,得数枝,芬馥可爱,为之颓然径醉。夜雨,极寒,始覆絮衾。


十月


二十一日。舟中望石门关,仅通一人行,天下至险也。晚,泊巴东县,江山雄丽,大胜秭归。但井邑极于萧条,邑中才百余户,自令廨而下皆茅茨,了无片瓦。权县事秭归尉右迪功郎王康年、尉兼主簿右迪功郎杜德先来,皆蜀人也。谒寇莱公祠堂,登秋风亭,下临江山。是日重阴微雪,天气,复观亭名,使人怅然,始有流落天涯之叹。遂登双柏堂、白云亭。堂下旧有莱公所植柏,今已槁死。然南山重复,秀丽可爱。白云亭则天下幽奇绝境,群山环拥,层出间见,古木森然,往往二三百年物。栏外双瀑泻石涧中,跳珠溅玉,冷入人骨。其下是为慈溪,奔流与江会。余自吴入楚,行五千余里,过十五州,亭榭之胜无如白云者,而止在县廨听事之后。巴东了无一事,为令者可以寝饭于亭中,其乐无涯,而阙令动辄二三年,无肯补者,何哉?


二十三日,过巫山凝真观,谒妙用真人祠。真人即世所谓巫山神女也。祠正对巫山,峰峦上入霄汉,山脚直插江中。议者谓太、华、衡、庐,皆无此奇。然十二峰者不可悉见,所见八九峰,惟神女峰最为纤丽奇峭,宜为仙真所托。祝史云:“每八月十五夜月明时,有丝竹之音,往来峰顶,山猿皆鸣,达旦方渐止。”庙后,山半有石坛,平旷。传云:“夏禹见神女,授符书于此。”坛上观十二峰,宛如屏障。是日,天宇晴霁,四顾无纤翳,惟神女峰上有白云数片,如鸾鹤翔舞徘徊,久之不散,亦可异也。祠旧有乌数百,送客迎舟。



\chapter*{重巽以申命论}
\addcontentsline{toc}{chapter}{重巽以申命论}
\begin{center}
	\textbf{[宋朝]苏轼}
\end{center}

论曰:昔圣人之始画卦也,皆有以配乎物者也。巽之配于风者,以其发而有所动也。配于木者,以其仁且顺也。夫发而有所动者,不仁则不可以久,不顺则不可以行,故发而仁,动而顺,而巽之道备矣。圣人以为不重,则不可以变,故因而重之,使之动而能变,变而不穷,故曰“重巽以申命”。言天子之号令如此而后可也。


天地之化育,有可以指而言者,有不可以求而得者。今夫日,皆知其所以为暖;雨,皆知其所以为润;雷霆,皆知其所以为震;雪霜,皆知其所以为杀。至於风,悠然布于天地之间,来不知其所自,去不知其所入,嘘而炎,吹而冷,大而鼓乎大山乔岳之上,细而入乎窍空?屋之下,发达万物,而天下不以为德,摧败草木,而天下不以为怒,故曰天地之化育,有不可求而得者。此圣人之所法,以令天下之术也。


圣人在上,天下之民,各得其职。士者皆曰“吾学而仕”,农者皆曰“吾耕而食”,工者皆曰“吾作而用”,贾者皆曰“吾负而贩”,不知圣人之制命令以鼓舞、通变其道,而使之安乎此也。圣人之在上也,天下可由而不可知,可言而不可议,盖得乎巽之道也。易者,圣人之动,而卦者,动之时也。《蛊》之彖曰:“先甲三日,后甲三日。”而《巽》之九五亦曰:“先庚三日,后庚三日。”而说者谓甲庚皆所以申命,而先后者,慎之至也。圣人悯斯民之愚,而不忍使之遽陷于罪戾也,故先三日而令之,后三日而申之,不从而后诛,盖其用心之慎也。以至神之化令天下,使天下不测其端;以至详之法晓天下,使天下明知其所避。天下不测其端,而明知其所避,故靡然相率而不敢议也。上令而下不议,下从而上不诛,顺之至也。故重巽之道,上下顺也。谨论。



\chapter*{越巫}
\addcontentsline{toc}{chapter}{越巫}
\begin{center}
	\textbf{[明朝]方孝孺}
\end{center}

越巫自诡善驱鬼物。人病,立坛场,鸣角振铃,跳掷叫呼,为胡旋舞禳之。病幸已,馔酒食持其赀去,死则诿以他故,终不自信其术之妄。恒夸人曰:“我善治鬼,鬼莫敢我抗。”恶少年愠其诞,瞷其夜归,分五六人栖道旁木上,相去各里所,候巫过下,砂石击之。巫以为真鬼也,即旋其角,且角且走,心大骇,首岑岑加重,行不知足所在。稍前,骇颇定,木间砂乱下如初,又旋而角,角不能成音,走愈急。复至前,复如初,手栗气慑不能角,角坠振其铃,既而铃坠,唯大叫以行。行闻履声及叶鸣谷响,亦皆以为鬼,号求救于人甚哀。夜半抵家,大哭叩门,其妻问故,舌缩不能言,唯指床曰:“亟扶我寝!我遇鬼,今死矣!”扶至床,胆裂死,肤色如蓝。巫至死不知其非鬼。



\chapter*{奉天北伐讨元檄文}
\addcontentsline{toc}{chapter}{奉天北伐讨元檄文}
\begin{center}
	\textbf{[明朝]宋濂}
\end{center}

自古帝王临御天下,皆中国居内以制夷狄,夷狄居外以奉中国,未闻以夷狄居中国而制天下也。自宋祚倾移,元以北夷入主中国,四海以内,罔不臣服,此岂人力,实乃天授。彼时君明臣良,足以纲维天下,然达人志士,尚有冠履倒置之叹。自是以后,元之臣子,不遵祖训,废坏纲常,有如大德废长立幼,泰定以臣弑君,天历以弟鸩兄,至于弟收兄妻,子征父妾,上下相习,恬不为怪,其于父子君臣夫妇长幼之伦,渎乱甚矣。夫人君者斯民之宗主,朝廷者天下之根本,礼仪者御世之大防,其所为如彼,岂可为训于天下后世哉!


及其后嗣沉荒,失君臣之道,又加以宰相专权,宪台抱怨,有司毒虐,于是人心离叛,天下兵起,使我中国之民,死者肝脑涂地,生者骨肉不相保,虽因人事所致,实乃天厌其德而弃之之时也。古云:“胡虏无百年之运”,验之今日,信乎不谬。


当此之时,天运循环,中原气盛,亿兆之中,当降生圣人,驱除胡虏,恢复中华,立纲陈纪,救济斯民。今一纪于兹,未闻有治世安民者,徒使尔等战战兢兢,处于朝秦暮楚之地,诚可矜闵。


方今河、洛、关、陕,虽有数雄:忘中国祖宗之姓,反就胡虏禽兽之名,以为美称,假元号以济私,恃有众以要君,凭陵跋扈,遥制朝权,此河洛之徒也;或众少力微,阻兵据险,贿诱名爵,志在养力,以俟衅隙,此关陕之人也。二者其始皆以捕妖人为名,乃得兵权。及妖人已灭,兵权已得,志骄气盈,无复尊主庇民之意,互相吞噬,反为生民之巨害,皆非华夏之主也。


予本淮右布衣,因天下大乱,为众所推,率师渡江,居金陵形式之地,得长江天堑之险,今十有三年。西抵巴蜀,东连沧海,南控闽越,湖、湘、汉、沔,两淮、徐、邳,皆入版图,奄及南方,尽为我有。民稍安,食稍足,兵稍精,控弦执矢,目视我中原之民,久无所主,深用疚心。予恭承天命,罔敢自安,方欲遣兵北逐胡虏,拯生民于涂炭,复汉官之威仪。虑民人未知,反为我仇,絜家北走,陷溺犹深,故先逾告:兵至,民人勿避。予号令严肃,无秋毫之犯,归我者永安于中华,背我者自窜于塞外。盖我中国之民,天必命我中国之人以安之,夷狄何得而治哉!予恐中土久污膻腥,生民扰扰,故率群雄奋力廓清,志在逐胡虏,除暴乱,使民皆得其所,雪中国之耻,尔民等其体之。


如蒙古、色目,虽非华夏族类,然同生天地之间,有能知礼义,愿为臣民者,与中夏之人抚养无异。故兹告谕,想宜知悉。



\chapter*{金石录序}
\addcontentsline{toc}{chapter}{金石录序}
\begin{center}
	\textbf{[宋朝]李清照}
\end{center}

右《金石录》三十卷者何?赵侯德父所著书也。取上自三代,下迄五季,钟、鼎、甗、鬲、盘、匝、尊、敦之款识,丰碑、大碣,显人、晦士之事迹,凡见于金石刻者二千卷,皆是正讹谬,去取褒贬,上足以合圣人之道,下足以订史氏之失者,皆载之,可谓多矣。


呜呼!自王涯、元载之祸,书画与胡椒无异;长舆、元凯之病,钱癖与传癖何殊。名虽不同,其惑一也。


余建中辛巳,始归赵氏。时先君作礼部员外郎,丞相时作吏部侍郎。侯年二十一,在太学作学生。赵、李族寒,素贫俭。每朔望谒告出,质衣,取半千钱,步入相国寺,市碑文果实。归,相对展玩咀嚼,自谓葛天氏之民也。后二年,出仕宦,便有饭蔬衣綀,穷遐方绝域,尽天下古文奇字之志。日就月将,渐益堆积。丞相居政府,亲旧或在馆阁,多有亡诗、逸史,鲁壁、汲冢所未见之书,遂尽力传写,浸觉有味,不能自已。后或见古今名人书画,一代奇器,亦复脱衣市易。尝记崇宁间,有人持徐熙《牡丹图》,求钱二十万。当时虽贵家子弟,求二十万钱,岂易得耶?留信宿,计无所出而还之,夫妇相向惋怅者数日。


后屏居乡里十年,仰取俯拾,衣食有余。连守两郡,竭其俸入以事铅椠。每获一书,即同共勘校,整集签题;得书画彝鼎,亦摩玩舒卷,指摘疵病,夜尽一烛为率。故能纸札精致,字画完整,冠诸收书家。余性偶强记,每饭罢,坐归来堂烹茶,指堆积书史,言某事在某书某卷第几叶第几行,以中否角胜负,为饮茶先后。中,即举杯大笑,至茶倾覆怀中,反不得饮而起。甘心老是乡矣。故虽处忧患困穷,而志不屈。收书既成,归来堂起书库大橱,簿甲乙,置书册。如要讲读,即请钥上簿,关出卷帙。或少损污,必惩责揩完涂改,不复向时之坦夷也。是欲求适意而反取憀栗。余性不耐,始谋食去重肉,衣去重彩,首无明珠翠羽之饰,室无涂金刺绣之具。遇书史百家,字不刓缺,本不讹谬者,辄市之,储作副本。自来家传《周易》、《左氏传》,故两家者流,文字最备。于是几案罗列,枕席枕藉,意会心谋,目往神授,乐在声色狗马之上。


至靖康丙午岁,侯守淄川,闻金寇犯京师,四顾茫然,盈箱溢箧,且恋恋,且怅怅,知其必不为己物矣!建炎丁未春三月,奔太夫人丧南来,既长物不能尽载,乃先去书之重大印本者,又去画之多幅者,又去古器之无款识者。后又去书之监本者,画之平常者,器之重大者。凡屡减去,尚载书十五车。至东海,连舻渡淮,又渡江,至建康。青州故第,尚锁书册什物,用屋十余间,期明年春再具舟载之。十二月,金人陷青州,凡所谓十余屋者,已皆为煨烬矣。


建炎戊申秋九月,侯起复,知建康府。己酉春三月罢,具舟上芜湖,入姑孰,将卜居赣水上。夏五月,至池阳,被旨知湖州,过阙上殿,遂驻家池阳,独赴召。六月十三日,始负担舍舟,坐岸上,葛衣岸巾,精神如虎,目光烂烂射人,望舟中告别。余意甚恶,呼曰:“如传闻城中缓急,奈何?”戟手遥应曰:“从众。必不得已,先弃辎重,次衣被,次书册卷轴,次古器,独所谓宗器者,可自负抱,与身俱存亡,勿忘之。”遂驰马去。途中奔驰,冒大暑,感疾。至行在,病痁。七月末,书报卧病,余惊怛。念侯性素急,奈何。病痁或热,必服寒药,疾可忧。遂解舟下,一日夜行三百里。比至,果大服柴胡黄芩药,疟且痢,病危在肓。余悲泣仓皇,不忍问后事。八月十八日,遂不起。取笔作诗,绝笔而终,殊无分香卖履之意。


葬毕,余无所之。朝廷已分遣六宫,又传江当禁渡。时犹有书二万卷,金石刻二千卷,器皿茵褥,可待百客,他长物称是。余又大病,仅存喘息。事势日迫,念侯有妹婿任兵部侍郎从卫在洪州,遂遣二故吏先部送行李往投之。冬十二月,金寇陷洪州,遂尽委弃。所谓连舻渡江之书,又散为云烟矣。独余少轻小卷轴书帖,写本李、杜、韩、柳集,《世说》、《盐铁论》,汉唐石刻副本数十轴,三代鼎鼐十数事,南唐写本书数箧,偶病中把玩,搬在卧内者,岿然独存。


上江既不可往,又虏势叵测,有弟迒任敕局删定官,遂往依之。到台,守已遁。之剡,出陆,又弃衣被。走黄岩,雇舟入海,奔行朝。时驻跸章安,从御舟海道之温,又之越。庚戌十二月,放散百官,遂之衢。绍兴辛亥春三月,复赴越。壬子,又赴杭。先侯疾亟时,有张飞卿学士携玉壶过,视侯,便携去,其实珉也。不知何人传道,遂妄言有颁金之语。或传亦有密论列者。余大惶怖,不敢言,亦不敢遂已,尽将家中所有铜器等物,欲赴外庭投进。到越,已移幸四明。不敢留家中,并写本书寄剡。后官军收叛卒取去,闻尽入故李将军家。所谓岿然独存者,无虑十去五六矣。惟有书、画、砚、墨可五、七簏,更不忍置他所,常在卧榻下,手自开阖。在会稽,卜居土民钟氏舍。忽一夕,穴壁负五簏去。余悲恸不已,重立赏收赎。后二日,邻人钟复皓出十八轴求赏。故知其盗不远矣。万计求之,其余遂劳不可出。今知尽为吴说运使贱价得之。所谓岿然独存者,乃十去其七八,所有一二残零,不成部帙书册三数种,平平书帖,犹复爱惜如护头目,何愚也耶!


今日忽阅此书,如见故人。因忆侯在东莱静治堂,装卷初就,芸签缥带,束十卷作一帙。每日晚吏散,辄校勘二卷,跋题一卷。此二千卷,有题跋者五百二卷耳。今手泽如新,而墓木已拱,悲夫!


昔萧绎江陵陷没,不惜国亡,而毁裂书画;杨广江都倾覆,不悲身死,而复取图书。岂人性之所著,死生不能忘之欤?或者天意以余菲薄,不足以享此尤物耶?抑亦死者有知,犹斤斤爱惜,不肯留在人间耶?何得之艰而失之易也!


呜呼!余自少陆机作赋之二年,至过蘧瑗知非之两岁,三十四年之间,忧患得失,何其多也。然有有必有无,有聚必有散,乃理之常。人亡弓,人得之,又胡足道。所以区区记其终始者,亦欲为后世好古博雅者之戒云。绍兴二年玄黓岁壮月朔甲寅易安室题。



\chapter*{论盛孝章书}
\addcontentsline{toc}{chapter}{论盛孝章书}
\begin{center}
	\textbf{[汉朝]孔融}
\end{center}

岁月不居,时节如流。五十之年,忽焉已至。公为始满,融又过二。海内知识,零落殆尽,惟会稽盛孝章尚存。其人困于孙氏,妻孥湮没,单孑独立,孤危愁苦。若使忧能伤人,此子不得复永年矣!

《春秋传》曰:“诸侯有相灭亡者,桓公不能救,则桓公耻之。”今孝章,实丈夫之雄也,天下谈士,依以扬声,而身不免于幽絷,命不期于旦夕,是吾祖不当复论损益之友,而朱穆所以绝交也。公诚能驰一介之使,加咫尺之书,则孝章可致,友道可弘矣。

今之少年,喜谤前辈,或能讥评孝章。孝章要为有天下大名,九牧之人,所共称叹。燕君市骏马之骨,非欲以骋道里,乃当以招绝足也。惟公匡复汉室,宗社将绝,又能正之。正之之术,实须得贤。珠玉无胫而自至者,以人好之也,况贤者之有足乎!昭王筑台以尊郭隗,隗虽小才,而逢大遇,竟能发明主之至心,故乐毅自魏往,剧辛自赵往,邹衍自齐往。向使郭隗倒悬而王不解,临溺而王不拯,则士亦将高翔远引,莫有北首燕路者矣。凡所称引,自公所知,而复有云者,欲公崇笃斯义也。因表不悉。


\chapter*{运命论}
\addcontentsline{toc}{chapter}{运命论}
\begin{center}
	\textbf{[三国]李康}
\end{center}

夫治乱,运也;穷达,命也;贵贱,时也。故运之将隆,必生圣明之君。圣明之君,必有忠贤之臣。其所以相遇也,不求而自合;其所以相亲也,不介而自亲。唱之而必和,谋之而必从,道德玄同,曲折合符,得失不能疑其志,谗构不能离其交,然后得成功也。其所以得然者,岂徒人事哉?授之者天也,告之者神也,成之者运也。

夫黄河清而圣人生,里社鸣而圣人出,群龙见而圣人用。故伊尹,有莘氏之媵臣也,而阿衡于商。太公,渭滨之贱老也,而尚父于周。百里奚在虞而虞亡,在秦而秦霸,非不才于虞而才于秦也。张良受黄石之符,诵三略之说,以游于群雄,其言也,如以水投石,莫之受也;及其遭汉祖,其言也,如以石投水,莫之逆也。非张良之拙说于陈项,而巧言于沛公也。然则张良之言一也,不识其所以合离?合离之由,神明之道也。故彼四贤者,名载于箓图,事应乎天人,其可格之贤愚哉?孔子曰:“清明在躬,气志如神。嗜欲将至,有开必先。天降时雨,山川出云。”诗云:“惟岳降神,生甫及申;惟申及甫,惟周之翰。”运命之谓也。

岂惟兴主,乱亡者亦如之焉。幽王之惑褒女也,祅始于夏庭。曹伯阳之获公孙强也,征发于社宫。叔孙豹之昵竖牛也,祸成于庚宗。吉凶成败,各以数至。咸皆不求而自合,不介而自亲矣。昔者,圣人受命河洛曰:以文命者,七九而衰;以武兴者,六八而谋。及成王定鼎于郏鄏,卜世三十,卜年七百,天所命也。故自幽厉之间,周道大坏,二霸之后,礼乐陵迟。文薄之弊,渐于灵景;辩诈之伪,成于七国。酷烈之极,积于亡秦;文章之贵,弃于汉祖。虽仲尼至圣,颜冉大贤,揖让于规矩之内,訚訚于洙、泗之上,不能遏其端;孟轲、孙卿体二希圣,从容正道,不能维其末,天下卒至于溺而不可援。

夫以仲尼之才也,而器不周于鲁卫;以仲尼之辩也,而言不行于定哀;以仲尼之谦也,而见忌于子西;以仲尼之仁也,而取仇于桓魋;以仲尼之智也,而屈厄于陈蔡;以仲尼之行也,而招毁于叔孙。夫道足以济天下,而不得贵于人;言足以经万世,而不见信于时;行足以应神明,而不能弥纶于俗;应聘七十国,而不一获其主;驱骤于蛮夏之域,屈辱于公卿之门,其不遇也如此。及其孙子思,希圣备体,而未之至,封己养高,势动人主。其所游历诸侯,莫不结驷而造门;虽造门犹有不得宾者焉。其徒子夏,升堂而未入于室者也。退老于家,魏文候师之,西河之人肃然归德,比之于夫子而莫敢间其言。故曰:治乱,运也;穷达,命也;贵贱,时也。而后之君子,区区于一主,叹息于一朝。屈原以之沈湘,贾谊以之发愤,不亦过乎!

然则圣人所以为圣者,盖在乎乐天知命矣。故遇之而不怨,居之而不疑也。其身可抑,而道不可屈;其位可排,而名不可夺。譬如水也,通之斯为川焉,塞之斯为渊焉,升之于云则雨施,沈之于地则土润。体清以洗物,不乱于浊;受浊以济物,不伤于清。是以圣人处穷达如一也。夫忠直之迕于主,独立之负于俗,理势然也。故木秀于林,风必摧之;堆出于岸,流必湍之;行高于人,众必非之。前监不远,覆车继轨。然而志士仁人,犹蹈之而弗悔,操之而弗失,何哉?将以遂志而成名也。求遂其志,而冒风波于险涂;求成其名,而历谤议于当时。彼所以处之,盖有算矣。子夏曰:“死生有命,富贵在天”故道之将行也,命之将贵也,则伊尹吕尚之兴于商周,百里子房之用于秦汉,不求而自得,不徼而自遇矣。道之将废也,命之将贱也,岂独君子耻之而弗为乎?盖亦知为之而弗得矣。

凡希世苟合之士,蘧蒢戚之人,俛仰尊贵之颜,逶迤势利之间,意无是非,赞之如流;言无可否,应之如响。以窥看为精神,以向背为变通。势之所集,从之如归市;势之所去,弃之如脱遗。其言曰:名与身孰亲也?得与失孰贤也?荣与辱孰珍也?故遂絜其衣服,矜其车徒,冒其货贿,淫其声色,脉脉然自以为得矣。盖见龙逢、比干之亡其身,而不惟飞廉、恶来之灭其族也。盖知伍子胥之属镂于吴,而不戒费无忌之诛夷于楚也。盖讥汲黯之白首于主爵,而不惩张汤牛车之祸也。盖笑萧望之跋踬于前,而不惧石显之绞缢于后也。故夫达者之筭也,亦各有尽矣。

曰:凡人之所以奔竞于富贵,何为者哉?若夫立德必须贵乎?则幽厉之为天子,不如仲尼之为陪臣也。必须势乎?则王莽、董贤之为三公,不如杨雄、仲舒之阒其门也。必须富乎?则齐景之千驷,不如颜回、原宪之约其身也。其为实乎?则执杓而饮河者,不过满腹;弃室而洒雨者,不过濡身;过此以往,弗能受也。其为名乎?则善恶书于史册,毁誉流于千载;赏罚悬于天道,吉凶灼乎鬼神,固可畏也。将以娱耳目、乐心意乎?譬命驾而游五都之市,则天下之货毕陈矣。褰裳而涉汶阳之丘,则天下之稼如云矣。椎紒而守敖庾、海陵之仓,则山坻之积在前矣。扱衽而登钟山、蓝田之上,则夜光玙璠之珍可观矣。夫如是也,为物甚众,为己甚寡,不爱其身,而啬其神。风惊尘起,散而不止。六疾待其前,五刑随其后。利害生其左,攻夺出其右,而自以为见身名之亲疏,分荣辱之客主哉。

天地之大德曰生,圣人之大宝曰位,何以守位曰仁,何以正人曰义。故古之王者,盖以一人治天下,不以天下奉一人也。古之仕者,盖以官行其义,不以利冒其官也。古之君子,盖耻得之而弗能治也,不耻能治而弗得也。原乎天人之性,核乎邪正之分,权乎祸福之门,终乎荣辱之算,其昭然矣。故君子舍彼取此。若夫出处不违其时,默语不失其人,天动星回而辰极犹居其所,玑旋轮转,而衡轴犹执其中,既明且哲,以保其身,贻厥孙谋,以燕翼子者,昔吾先友,尝从事于斯矣。


\chapter*{问说}
\addcontentsline{toc}{chapter}{问说}
\begin{center}
	\textbf{[清朝]刘开}
\end{center}

君子之学必好问。问与学,相辅而行者也。非学无以致疑,非问无以广识;好学而不勤问,非真能好学者也。理明矣,而或不达于事;识其大矣,而或不知其细,舍问,其奚决焉?

贤于己者,问焉以破其疑,所谓“就有道而正”也。不如己者,问焉以求一得,所谓“以能问于不能,以多问于寡”也。等于己者,问焉以资切磋,所谓交相问难(nàn),审问而明辨之也。《书》不云乎?“好问则裕。”孟子论:“求放心”,而并称曰“学问之道”,学即继以问也。子思言“尊德性”,而归于“道问学”,问且先于学也。

古之人虚中乐善,不择事而问焉,不择人而问焉,取其有益于身而已。是故狂夫之言,圣人择之,刍荛(ráo)之微,先民询之,舜以天子而询于匹夫,以大知而察及迩言,非苟为谦,诚取善之弘也。三代而下,有学而无问,朋友之交,至于劝善规过足矣,其以义理相咨访,孜孜焉唯进修是急,未之多见也,况流俗乎?

是己而非人,俗之同病。学有未达,强以为知;理有未安,妄以臆度。如是,则终身几无可问之事。贤于己者,忌之而不愿问焉;不如己者,轻之而不屑问焉;等于己者,狎xiá之而不甘问焉,如是,则天下几无可问之人。人不足服矣,事无可疑矣,此唯师心自用耳。夫自用,其小者也;自知其陋而谨护其失,宁使学终不进,不欲虚以下人,此为害于心术者大,而蹈之者常十之八九。

不然,则所问非所学焉:询天下之异文鄙事以快言论;甚且心之所已明者,问之人以试其能,事之至难解者,问之人以穷其短。而非是者,虽有切于身心性命之事,可以收取善之益,求一屈己焉而不可得也。嗟乎!学之所以不能几(jī)于古者,非此之由乎?

且夫不好问者,由心不能虚也;心之不虚,由好学之不诚也。亦非不潜心专力之敌,其学非古人之学,其好亦非古人之好也,不能问宜也。

智者千虑,必有一失。圣人所不知,未必不为愚人之所知也;愚人之所能,未必非圣人之所不能也。理无专在,而学无止境也,然则问可少耶?《周礼》,外朝以询万民,国之政事尚问及庶人,是故贵可以问贱,贤可以问不肖,而老可以问幼,唯道之所成而已矣。

孔文子不耻下问,夫子贤之。古人以问为美德,而并不见其有可耻也,后之君子反争以问为耻,然则古人所深耻者,后世且行之而不以为耻者多矣,悲夫!


\chapter*{与友人书}
\addcontentsline{toc}{chapter}{与友人书}
\begin{center}
	\textbf{[清朝]顾炎武}
\end{center}

人之为学,不日进则日退。独学无友,则孤陋而难成。久处一方,则习染而不自觉。不幸而在穷僻之域,无车马之资,犹当博学审问,古人与稽,以求其是非之所在,庶几可得十之五六。若既不出户,又不读书,则是面墙之士,虽有子羔、原宪之贤,终无济于天下。子曰:“十室之邑,必有忠信如丘者焉,不如丘之好学也。”夫以孔子之圣,犹须好学,今人可不勉乎?


\chapter*{天人三策}
\addcontentsline{toc}{chapter}{天人三策}
\begin{center}
	\textbf{[汉朝]董仲舒}
\end{center}

仲舒对曰:

陛下发德音,下明诏,求天命与情性,皆非愚臣之所能及也。臣谨案《春秋》之中,视前世已行之事,以观天人相与之际,甚可畏也。国家将有失道之败,而天乃先出灾害以谴告之,不知自省,又出怪异以警惧之,尚不知变,而伤败乃至。以此见天心之仁爱人君而欲止其乱也。自非大亡道之世者,天尽欲扶持而全安之,事在强勉而已矣。强勉学习,则闻见博而知益明;强勉行道,则德日起而大有功:此皆可使还至而有效者也。《诗》曰“夙夜匪解”,《书》云“茂哉茂哉!”皆强勉之谓也。

道者,所繇适于治之路也,仁义礼乐皆其具也。故圣王已没,而子孙长久安宁数百岁,此皆礼乐教化之功也。王者未作乐之时,乃用先五之乐宜于世者,而以深入教化于民。教化之情不得,雅颂之乐不成,故王者功成作乐,乐其德也。乐者,所以变民风,化民俗也;其变民也易,其化人也著。故声发于和而本于情,接于肌肤,臧于骨髓。故王道虽微缺,而管弦之声未衰也。夫虞氏之不为政久矣,然而乐颂遗风犹有存者,是以孔子在齐而闻《韶》也。夫人君莫不欲安存而恶危亡,然而政乱国危者甚众,所任者非其人,而所繇者非其道,是以政日以仆灭也。夫周道衰于幽、厉,非道亡也,幽、厉不繇也。至于宣王,思昔先王之德,兴滞补弊,明文、武之功业,周道粲然复兴,诗人美之而作,上天晁之,为生贤佐,后世称通,至今不绝。此夙夜不解行善之所致也。孔子曰“人能弘道,非道弘人”也。故治乱废兴在于己,非天降命不得可反,其所操持誖谬失其统也。

臣闻天之所大奉使之王者,必有非人力所能致而自至者,此受命之符也。天下之人同心归之,若归父母,故天瑞应诚而至。《书》曰“白鱼入于王舟,有火复于王屋,流为乌”,此盖受命之符也。周公曰“复哉复哉”,孔子曰“德不孤,必有邻”,皆积善累德之效也。及至后世,淫佚衰微,不能统理群生,诸侯背畔,残贱良民以争壤土,废德教而任刑罚。刑罚不中,则生邪气;邪气积于下,怨恶畜于上。上下不和,则阴阳缪盭而娇孽生矣。此灾异所缘而起也。

臣闻命者天之令也,性者生之质也,情者人之欲也。或夭或寿,或仁或鄙,陶冶而成之,不能粹美,有治乱之所在,故不齐也。孔子曰:“君子之德风,小人之德草,草上之风必偃。”故尧、舜行德则民仁寿,桀、纣行暴则民鄙夭。未上之化下,下之从上,犹泥之在钧,唯甄者之所为,犹金之在熔,唯冶者之所铸。“绥之斯俫,动之斯和”,此之谓也。

臣谨案《春秋》之文,求王道之端,得之于正。正次王,王次春。春者,天之所为也;正者,王之所为也。其意曰,上承天之所为,而下以正其所为,正王道之端云尔。然则王者欲有所为,宜求其端于天。天道之大者在阴阳。阳为德,阴为刑;刑主杀而德主生。是故阳常居大夏,而以生育养长为事;阴常居大冬,而积于空虚不用之处。以此见天之任德不任刑也。天使阳出布施于上而主岁功,使阴入伏于下而时出佐阳;阳不得阴之助,亦不能独成岁。终阳以成岁为名,此天意也。王者承天意以从事,故任德教而不任刑。刑者不可任以治世,犹阴之不可任以成岁也。为政而任刑,不顺于天,故先王莫之肯为也。今废先王德教之官,而独任执法之吏治民,毋乃任刑之意与!孔子曰:“不教而诛谓之虐。”虐政用于下,而欲德教之被四海,故难成也。

臣谨案《春秋》谓一元之意,一者万物之所从始也,元者辞之所谓大也。谓一为元者,视大始而欲正本也。《春秋》深探其本,而反自贵者始。故为人君者,正心以正朝廷,正朝廷以正百官,正百官以正万民,正万民以正四方。四方正,远近莫敢不壹于正,而亡有邪气奸其间者。是以阴阳调而风雨时,群生和而万民殖,五谷孰而草木茂,天地之间被润泽而大丰美,四海之内闻盛德而皆徕臣,诸福之物,可致之祥,莫不毕至,而王道终矣。

孔子曰:“凤鸟不至,河不出图,吾已矣夫!”自悲可致此物,而身卑贱不得致也。今陛下贵为天子,富有四海,居得致之位,操可致之势,又有能致之资,行高而恩厚,知明而意美,爱民而好士,可谓谊主矣。然而天地未应而美祥莫至者,何也?凡以教化不立而万民不正也。夫万民之从利也,如水之走下,不以教化堤防之,不能止也。是故教化立而奸邪皆止者,其堤防完也;教化废而奸邪并出,刑罚不能胜者,其堤防坏也。古之王者明于此,是故南面而治天下,莫不以教化为大务。立太学以教于国,设痒序以化于邑,渐民以仁,摩民以谊,节民以礼,故其刑罚甚轻而禁不犯者,教化行而习俗美也。

圣王之继乱世也,扫除其迹而悉去之,复修教化而崇起之。教化已明,习俗已成,子孙循之,行五六百岁尚未败也。至周之末世,大为亡道,以失天下。秦继其后,独不能改,又益甚之,重禁文学,不得挟书,弃捐礼谊而恶闻之,其心欲尽灭先圣之道,而颛为自恣苟简之治,故立为天子十四岁而国破亡矣。自古以来,未尝有以乱济乱,大败天下之民如秦者也。其遗毒余烈,至今未灭,使习俗薄恶,人民嚚顽,抵冒殊扞,孰烂如此之甚者也。

孔子曰:“腐朽之木不可雕也,粪土之墙不可圬也。”今汉继秦之后,如朽木、粪墙矣,虽欲善治之,亡可奈何。法出而奸生,令下而诈起,如以汤止沸,抱薪救火,愈甚亡益也。窃譬之琴瑟不调,甚者必解而更张之,乃可鼓也;为政而不行,甚者必变而更化之,乃可理也。当更张而不更张,虽有良工不能善调也:当更化而不更化,虽有大贤不能善治也。故汉得天下以来,常欲善治而至今不可善治者,失之于当更化而不更化也。

古人有言曰:“临渊羡鱼,不如退而结网。”今临政而愿治七十余岁矣,不如退而更化;更化则可善治,善治则灾害日去,福禄日来。《诗》云:“宜民宜人,受禄于人。”为政而宜于民者,固当受禄于天。夫仁、谊、礼、知、信五常之道,王者所当修饬也;五者修饬,故受天之晁,而享鬼神之灵,德施于方外,延及群生也。


\chapter*{世无良猫}
\addcontentsline{toc}{chapter}{世无良猫}
\begin{center}
	\textbf{[清朝]乐钧}
\end{center}

某恶鼠,破家求良猫。厌以腥膏,眠以毡罽。猫既饱且安,率不食鼠,甚者与鼠游戏,鼠以故益暴。某恐,遂不复蓄猫,以为天下无良猫也。是无猫邪,是不会蓄猫也。


\chapter*{文帝议佐百姓诏}
\addcontentsline{toc}{chapter}{文帝议佐百姓诏}
\begin{center}
	\textbf{[汉朝]刘恒}
\end{center}

间者数年比不登,又有水旱疾疫之灾,朕甚忧之。愚而不明,未达其咎。意者朕之政有所失、而行有过与?乃天道有不顺、地利或不得、人事多失和、鬼神废不享与?何以致此?将百官之奉养或费、无用之事或多与?何其民食之寡乏也?

夫度田非益寡,而计民未加益,以口量地,其于古犹有余,而食之甚不足者,其咎安在?无乃百姓之从事于末、以害农者蕃、为酒醪以靡谷者多、六畜之食焉者众与?细大之义,吾未能得其中。其与丞相、列侯、吏二千石、博土议之,有可以佐百姓者,率意远思,无有所隐。


\chapter*{祭十二郎文}
\addcontentsline{toc}{chapter}{祭十二郎文}
\begin{center}
	\textbf{[唐朝]韩愈}
\end{center}

年、月、日,季父愈闻汝丧之七日,乃能衔哀致诚,使建中远具时羞之奠,告汝十二郎之灵:

呜呼!吾少孤,及长,不省所怙,惟兄嫂是依。中年,兄殁南方,吾与汝俱幼,从嫂归葬河阳。既又与汝就食江南。零丁孤苦,未尝一日相离也。吾上有三兄,皆不幸早世。承先人后者,在孙惟汝,在子惟吾。两世一身,形单影只。嫂尝抚汝指吾而言曰:“韩氏两世,惟此而已!”汝时尤小,当不复记忆。吾时虽能记忆,亦未知其言之悲也。

吾年十九,始来京城。其后四年,而归视汝。又四年,吾往河阳省坟墓,遇汝从嫂丧来葬。又二年,吾佐董丞相于汴州,汝来省吾。止一岁,请归取其孥。明年,丞相薨。吾去汴州,汝不果来。是年,吾佐戎徐州,使取汝者始行,吾又罢去,汝又不果来。吾念汝从于东,东亦客也,不可以久;图久远者,莫如西归,将成家而致汝。呜呼!孰谓汝遽去吾而殁乎!吾与汝俱少年,以为虽暂相别,终当久相与处。故舍汝而旅食京师,以求斗斛之禄。诚知其如此,虽万乘之公相,吾不以一日辍汝而就也。

去年,孟东野往。吾书与汝曰:“吾年未四十,而视茫茫,而发苍苍,而齿牙动摇。念诸父与诸兄,皆康强而早世。如吾之衰者,其能久存乎?吾不可去,汝不肯来,恐旦暮死,而汝抱无涯之戚也!”孰谓少者殁而长者存,强者夭而病者全乎!

呜呼!其信然邪?其梦邪?其传之非其真邪?信也,吾兄之盛德而夭其嗣乎?汝之纯明而不克蒙其泽乎?少者、强者而夭殁,长者、衰者而存全乎?未可以为信也。梦也,传之非其真也,东野之书,耿兰之报,何为而在吾侧也?呜呼!其信然矣!吾兄之盛德而夭其嗣矣!汝之纯明宜业其家者,不克蒙其泽矣!所谓天者诚难测,而神者诚难明矣!所谓理者不可推,而寿者不可知矣!

虽然,吾自今年来,苍苍者或化而为白矣,动摇者或脱而落矣。毛血日益衰,志气日益微,几何不从汝而死也。死而有知,其几何离;其无知,悲不几时,而不悲者无穷期矣。

汝之子始十岁,吾之子始五岁。少而强者不可保,如此孩提者,又可冀其成立邪?呜呼哀哉!呜呼哀哉!

汝去年书云:“比得软脚病,往往而剧。”吾曰:“是疾也,江南之人,常常有之。”未始以为忧也。呜呼!其竟以此而殒其生乎?抑别有疾而至斯极乎?

汝之书,六月十七日也。东野云,汝殁以六月二日;耿兰之报无月日。盖东野之使者,不知问家人以月日;如耿兰之报,不知当言月日。东野与吾书,乃问使者,使者妄称以应之乎。其然乎?其不然乎?

今吾使建中祭汝,吊汝之孤与汝之乳母。彼有食,可守以待终丧,则待终丧而取以来;如不能守以终丧,则遂取以来。其余奴婢,并令守汝丧。吾力能改葬,终葬汝于先人之兆,然后惟其所愿。

呜呼!汝病吾不知时,汝殁吾不知日,生不能相养以共居,殁不能抚汝以尽哀,敛不凭其棺,窆不临其穴。吾行负神明,而使汝夭;不孝不慈,而不能与汝相养以生,相守以死。一在天之涯,一在地之角,生而影不与吾形相依,死而魂不与吾梦相接。吾实为之,其又何尤!彼苍者天,曷其有极!自今已往,吾其无意于人世矣!当求数顷之田于伊颍之上,以待余年,教吾子与汝子,幸其成;长吾女与汝女,待其嫁,如此而已。

呜呼,言有穷而情不可终,汝其知也邪?其不知也邪?呜呼哀哉!尚飨!


\chapter*{送董邵南游河北序}
\addcontentsline{toc}{chapter}{送董邵南游河北序}
\begin{center}
	\textbf{[唐朝]韩愈}
\end{center}

燕赵古称多感慨悲歌之士。董生举进士,屡不得志于有司,怀抱利器,郁郁适兹土。吾知其必有合也。董生勉乎哉!

夫以子之不遇时,苟慕义强仁者皆爱惜焉。矧燕赵之士出乎其性者哉!然吾尝闻风俗与化移易,吾恶知其今不异于古所云邪?聊以吾子之行卜之也。董生勉乎哉!

吾因子有所感矣。为我吊望诸君之墓,而观于其市,复有昔时屠狗者乎?为我谢曰:“明天子在上,可以出而仕矣。”


\chapter*{百丈山记}
\addcontentsline{toc}{chapter}{百丈山记}
\begin{center}
	\textbf{[宋朝]朱熹}
\end{center}


登百丈山三里许,右俯绝壑,左控垂崖,垒石为磴,十余级乃得度。山之胜,盖自此始。


循磴而东,即得小涧。石梁跨于其上。皆苍藤古木,虽盛夏亭午无暑气。水皆清澈,自高淙下,其声溅溅然。度石梁,循两崖曲折而上,得山门。小屋三间,不能容十许人,然前瞰涧水,后临石池,风来两峡间,终日不绝。门内跨池又为石梁。度而北,蹑石梯,数级入庵。庵才老屋数间,卑庳迫隘,无足观。独其西阁为胜。水自西谷中循石罅奔射出阁下,南与东谷水并注池中。自池而出,乃为前所谓小涧者。阁据其上流,当水石峻激相搏处,最为可玩。乃壁其后,无所睹。独夜卧其上,则枕席之下,终夕潺潺。久而益悲,为可爱耳。


出山门而东十许步,得石台。下临峭岸,深昧险绝。于林薄间东南望,见瀑布自前岩穴瀵涌而出,投空下数十尺。其沫乃如散珠喷雾,目光烛之,璀璨夺目,不可正视。台当山西南缺,前揖芦山,一峰独秀出,而数百里间峰峦高下亦皆历历在眼。日薄西山,余光横照,紫翠重迭,不可殚数。旦起下视,白云满川,如海波起伏。而远近诸山出其中者,皆若飞浮来往。或涌或没,顷刻万变。台东径断,乡人凿石容磴以度,而作神祠于其东,水旱祷焉。畏险者或不敢度。然山之可观者,至是则亦穷矣。


余与刘充父、平父、吕叔敬、表弟徐周宾游之。既皆赋诗以纪其胜,余又叙次其详如此。而其最可观者,石磴、小涧、山门、石台、西阁、瀑布也。因各别为小诗以识其处,呈同游诸君。又以告夫欲往而未能者。



\chapter*{封燕然山铭}
\addcontentsline{toc}{chapter}{封燕然山铭}
\begin{center}
	\textbf{[汉朝]班固}
\end{center}


惟永元元年秋七月,有汉元舅曰车骑将军窦宪,寅亮圣明,登翼王室,纳于大麓,维清缉熙。乃与执金吾耿秉,述职巡御。理兵于朔方。鹰扬之校,螭虎之士,爰该六师,暨南单于、东胡乌桓、西戎氐羌,侯王君长之群,骁骑三万。元戎轻武,长毂四分,云辎蔽路,万有三千余乘。勒以八阵,莅以威神,玄甲耀目,朱旗绛天。遂陵高阙,下鸡鹿,经碛卤,绝大漠,斩温禺以衅鼓,血尸逐以染锷。然后四校横徂,星流彗扫,萧条万里,野无遗寇。于是域灭区殚,反旆而旋,考传验图,穷览其山川。遂逾涿邪,跨安侯,乘燕然,蹑冒顿之区落,焚老上之龙庭。上以摅高、文之宿愤,光祖宗之玄灵;下以安固后嗣,恢拓境宇,振大汉之天声。兹所谓一劳而久逸,暂费而永宁者也,乃遂封山刊石,昭铭盛德。其辞曰:

\begin{center}
	
	铄王师兮征荒裔,
	
	剿凶虐兮截海外。
	
	夐其邈兮亘地界,
	
	封神丘兮建隆嵑,
	
	熙帝载兮振万世!
	
\end{center}


\chapter*{五柳先生传}
\addcontentsline{toc}{chapter}{五柳先生传}
\begin{center}
	\textbf{[晋朝]陶渊明}
\end{center}


先生不知何许人也,亦不详其姓字,宅边有五柳树,因以为号焉。闲静少言,不慕荣利。好读书,不求甚解;每有会意,便欣然忘食。性嗜酒,家贫不能常得。亲旧知其如此,或置酒而招之;造饮辄尽,期在必醉。既醉而退,曾不吝情去留。环堵萧然,不蔽风日;短褐穿结,箪瓢屡空,晏如也。常著文章自娱,颇示己志。忘怀得失,以此自终。


赞曰:黔娄之妻有言:“不戚戚于贫贱,不汲汲于富贵。”其言兹若人之俦乎?衔觞赋诗,以乐其志,无怀氏之民欤?葛天氏之民欤?



\chapter*{为徐敬业讨武曌檄}
\addcontentsline{toc}{chapter}{为徐敬业讨武曌檄}
\begin{center}
	\textbf{[唐朝]骆宾王}
\end{center}

伪临朝武氏者,性非和顺,地实寒微。昔充太宗下陈,曾以更衣入侍。洎乎晚节,秽乱春宫。潜隐先帝之私,阴图后房之嬖。入门见嫉,蛾眉不肯让人;掩袖工谗,狐媚偏能惑主。践元后于翚翟,陷吾君于聚麀。加以虺蜴为心,豺狼成性,近狎邪僻,残害忠良,杀姊屠兄,弑君鸩母。人神之所同嫉,天地之所不容。犹复包藏祸心,窥窃神器。君之爱子,幽之于别宫;贼之宗盟,委之以重任。呜呼!霍子孟之不作,朱虚侯之已亡。燕啄皇孙,知汉祚之将尽;龙漦帝后,识夏庭之遽衰。

敬业皇唐旧臣,公侯冢子。奉先君之成业,荷本朝之厚恩。宋微子之兴悲,良有以也;袁君山之流涕,岂徒然哉!是用气愤风云,志安社稷。因天下之失望,顺宇内之推心,爰举义旗,以清妖孽。南连百越,北尽三河,铁骑成群,玉轴相接。海陵红粟,仓储之积靡穷;江浦黄旗,匡复之功何远?班声动而北风起,剑气冲而南斗平。喑呜则山岳崩颓,叱吒则风云变色。以此制敌,何敌不摧;以此图功,何功不克!

公等或家传汉爵,或地协周亲,或膺重寄于爪牙,或受顾命于宣室。言犹在耳,忠岂忘心?一抔之土未干,六尺之孤何托?倘能转祸为福,送往事居,共立勤王之勋,无废旧君之命,凡诸爵赏,同指山河。若其眷恋穷城,徘徊歧路,坐昧先几之兆,必贻后至之诛。请看今日之域中,竟是谁家之天下!移檄州郡,咸使知闻。


\chapter*{上书谏猎}
\addcontentsline{toc}{chapter}{上书谏猎}
\begin{center}
	\textbf{[汉朝]司马相如}
\end{center}

臣闻物有同类而殊能者,故力称乌获,捷言庆忌,勇期贲、育。臣之愚,窃以为人诚有之,兽亦宜然。今陛下好陵阻险,射猛兽,卒然遇逸材之兽,骇不存之地,犯属车之清尘,舆不及还辕,人不暇施巧,虽有乌获、逢蒙之技不能用,枯木朽枝尽为难矣。是胡越起于毂下,而羌夷接轸也,岂不殆哉!虽万全而无患,然本非天子之所宜近也。

且夫清道而后行,中路而驰,犹时有衔橛之变。况乎涉丰草,骋丘虚,前有利兽之乐,而内无存变之意,其为害也不难矣。夫轻万乘之重不以为安,乐出万有一危之途以为娱,臣窃为陛下不取。

盖明者远见于未萌,而知者避危于无形,祸固多藏于隐微而发于人之所忽者也。故鄙谚曰:“家累千金,坐不垂堂。”此言虽小,可以喻大。臣愿陛下留意幸察。


\chapter*{掩耳盗铃}
\addcontentsline{toc}{chapter}{掩耳盗铃}
\begin{center}
	\textbf{[秦朝]吕不韦}
\end{center}


范氏之亡也,百姓有得钟者,欲负而走,则钟大不可负;以锤毁之,钟况然有声。恐人闻之而夺己也,遽掩其耳。恶人闻之,可也;恶己自闻之,悖也!

\chapter*{子产论尹何为邑}
\addcontentsline{toc}{chapter}{子产论尹何为邑}
\begin{center}
	\textbf{[春秋战国]左丘明}
\end{center}


子皮欲使尹何为邑。子产曰:“少,未知可否。”子皮曰:“愿,吾爱之,不吾叛也。使夫往而学焉,夫亦愈知治矣。”子产曰;“不可。人之爱人,求利之也。今吾子爱人则以政。犹未能操刀而使割也,其伤实多。子之爱人,伤之而已,其谁敢求爱于子?子于郑国,栋也。栋折榱崩,侨将厌焉,敢不尽言?子有美锦,不使人学制焉。大官大邑,身之所庇也,而使学者制焉。其为美锦,不亦多乎?侨闻学而后入政,未闻以政学者也。若果行此,必有所害。譬如田猎,射御贯,则能获禽;若未尝登车射御,则败绩厌覆是惧,何暇思获?


子皮曰:“善哉!虎不敏。吾闻君子务知大者、远者,小人务知小者、近者。我,小人也。衣服附在吾身,我知而慎之;大官、大邑,所以庇身也,我远而慢之。微子之言,吾不知也。他日我曰:‘子为郑国,我为吾家,以庇焉,其可也。’今而后知不足。自今请虽吾家,听子而行。”子产曰:“人心之不同,如其面焉。吾岂敢谓子面如吾面乎?抑心所谓危,亦以告也。”子皮以为忠,故委政焉。子产是以能为郑国。



\chapter*{与陈给事书}
\addcontentsline{toc}{chapter}{与陈给事书}
\begin{center}
	\textbf{[唐朝]韩愈}
\end{center}

愈再拜:愈之获见于阁下有年矣。始者亦尝辱一言之誉。贫贱也,衣食于奔走,不得朝夕继见。其后,阁下位益尊,伺候于门墙者日益进。夫位益尊,则贱者日隔;伺候于门墙者日益进,则爱博而情不专。愈也道不加修,而文日益有名。夫道不加修,则贤者不与;文日益有名,则同进者忌。始之以日隔之疏,加之以不专之望,以不与者之心,而听忌者之说。由是阁下之庭,无愈之迹矣。

去年春,亦尝一进谒于左右矣。温乎其容,若加其新也;属乎其言,若闵其穷也。退而喜也,以告于人。其后,如东京取妻子,又不得朝夕继见。及其还也,亦尝一进谒于左右矣。邈乎其容,若不察其愚也;悄乎其言,若不接其情也。退而惧也,不敢复进。

今则释然悟,翻然悔曰:其邈也,乃所以怒其来之不继也;其悄也,乃所以示其意也。不敏之诛,无所逃避。不敢遂进,辄自疏其所以,并献近所为《复志赋》以下十首为一卷,卷有标轴。《送孟郊序》一首,生纸写,不加装饰。皆有揩字注字处,急于自解而谢,不能俟更写。阁下取其意而略其礼可也。

愈恐惧再拜。


\chapter*{梓人传}
\addcontentsline{toc}{chapter}{梓人传}
\begin{center}
	\textbf{[唐朝]柳宗元}
\end{center}


裴封叔之第,在光德里。有梓人款其门,愿佣隙宇而处焉。所职,寻、引、规、矩、绳、墨,家不居砻斫之器。问其能,曰:“吾善度材,视栋宇之制,高深圆方短长之宜,吾指使而群工役焉。舍我,众莫能就一宇。故食于官府,吾受禄三倍;作于私家,吾收其直太半焉。”他日,入其室,其床阙足而不能理,曰:“将求他工。”余甚笑之,谓其无能而贪禄嗜货者。

其后京兆尹将饰官署,余往过焉。委群材,会群工,或执斧斤,或执刀锯,皆环立。向之梓人左持引,右执杖,而中处焉。量栋宇之任,视木之能举,挥其杖,曰“斧!”彼执斧者奔而右;顾而指曰:“锯!”彼执锯者趋而左。俄而,斤者斫,刀者削,皆视其色,俟其言,莫敢自断者。其不胜任者,怒而退之,亦莫敢愠焉。画宫于堵,盈尺而曲尽其制,计其毫厘而构大厦,无进退焉。既成,书于上栋曰:“某年、某月、某日、某建”。则其姓字也。凡执用之工不在列。余圜视大骇,然后知其术之工大矣。

继而叹曰:彼将舍其手艺,专其心智,而能知体要者欤!吾闻劳心者役人,劳力者役于人。彼其劳心者欤!能者用而智者谋,彼其智者欤!是足为佐天子,相天下法矣。物莫近乎此也。彼为天下者本于人。其执役者为徒隶,为乡师、里胥;其上为下士;又其上为中士,为上士;又其上为大夫,为卿,为公。离而为六职,判而为百役。外薄四海,有方伯、连率。郡有守,邑有宰,皆有佐政;其下有胥吏,又其下皆有啬夫、版尹以就役焉,犹众工之各有执伎以食力也。

彼佐天子相天下者,举而加焉,指而使焉,条其纲纪而盈缩焉,齐其法制而整顿焉;犹梓人之有规、矩、绳、墨以定制也。择天下之士,使称其职;居天下之人,使安其业。视都知野,视野知国,视国知天下,其远迩细大,可手据其图而究焉,犹梓人画宫于堵,而绩于成也。能者进而由之,使无所德;不能者退而休之,亦莫敢愠。不炫能,不矜名,不亲小劳,不侵众官,日与天下之英才,讨论其大经,犹梓人之善运众工而不伐艺也。夫然后相道得而万国理矣。

相道既得,万国既理,天下举首而望曰:「吾相之功也!」后之人循迹而慕曰:「彼相之才也!」士或谈殷、周之理者,曰:「伊、傅、周、召。」其百执事之勤劳,而不得纪焉;犹梓人自名其功,而执用者不列也。大哉相乎!通是道者,所谓相而已矣。其不知体要者反此;以恪勤为公,以簿书为尊,炫能矜名,亲小劳,侵众官,窃取六职、百役之事,听听于府庭,而遗其大者远者焉,所谓不通是道者也。犹梓人而不知绳墨之曲直,规矩之方圆,寻引之短长,姑夺众工之斧斤刀锯以佐其艺,又不能备其工,以至败绩,用而无所成也,不亦谬欤!

或曰:「彼主为室者,傥或发其私智,牵制梓人之虑,夺其世守,而道谋是用。虽不能成功,岂其罪耶?亦在任之而已!」

余曰:「不然!夫绳墨诚陈,规矩诚设,高者不可抑而下也,狭者不可张而广也。由我则固,不由我则圮。彼将乐去固而就圮也,则卷其术,默其智,悠尔而去。不屈吾道,是诚良梓人耳!其或嗜其货利,忍而不能舍也,丧其制量,屈而不能守也,栋桡屋坏,则曰:『非我罪也』!可乎哉?可乎哉?」

余谓梓人之道类于相,故书而藏之。梓人,盖古之审曲面势者,今谓之「都料匠」云。余所遇者,杨氏,潜其名。

\chapter*{辨奸论}
\addcontentsline{toc}{chapter}{辨奸论}
\begin{center}
	\textbf{[宋朝]苏洵}
\end{center}


事有必至,理有固然。惟天下之静者,乃能见微而知著。月晕而风,础润而雨,人人知之。人事之推移,理势之相因,其疏阔而难知,变化而不可测者,孰与天地阴阳之事。而贤者有不知,其故何也?好恶乱其中,而利害夺其外也!


昔者,山巨源见王衍曰:“误天下苍生者,必此人也!”郭汾阳见卢杞曰:“此人得志。吾子孙无遗类矣!”自今而言之,其理固有可见者。以吾观之,王衍之为人,容貌言语,固有以欺世而盗名者。然不忮不求,与物浮沉。使晋无惠帝,仅得中主,虽衍百千,何从而乱天下乎?卢杞之奸,固足以败国。然而不学无文,容貌不足以动人,言语不足以眩世,非德宗之鄙暗,亦何从而用之?由是言之,二公之料二子,亦容有未必然也!


今有人,口诵孔、老之言,身履夷、齐之行,收召好名之士、不得志之人,相与造作言语,私立名字,以为颜渊、孟轲复出,而阴贼险狠,与人异趣。是王衍、卢杞合而为一人也。其祸岂可胜言哉?夫面垢不忘洗,衣垢不忘浣。此人之至情也。今也不然,衣臣虏之衣。食犬彘之食,囚首丧面,而谈诗书,此岂其情也哉?凡事之不近人情者,鲜不为大奸慝,竖刁、易牙、开方是也。以盖世之名,而济其未形之患。虽有愿治之主,好贤之相,犹将举而用之。则其为天下患,必然而无疑者,非特二子之比也。


孙子曰:“善用兵者,无赫赫之功。”使斯人而不用也,则吾言为过,而斯人有不遇之叹。孰知祸之至于此哉?不然。天下将被其祸,而吾获知言之名,悲夫!



\chapter*{父善游}
\addcontentsline{toc}{chapter}{父善游}
\begin{center}
	\textbf{[秦朝]吕不韦}
\end{center}


有过于江上者,见人方引婴儿而欲投之江中,婴儿啼。人问其故。曰:“此其父善游。”


其父虽善游,其子岂遽善游哉?以此任物,亦必悖矣。



\chapter*{相州昼锦堂记}
\addcontentsline{toc}{chapter}{相州昼锦堂记}
\begin{center}
	\textbf{[宋朝]欧阳修}
\end{center}


仕宦而至将相,富贵而归故乡。此人情之所荣,而今昔之所同也。


盖士方穷时,困厄闾里,庸人孺子,皆得易而侮之。若季子不礼于其嫂,买臣见弃于其妻。一旦高车驷马,旗旄导前,而骑卒拥后,夹道之人,相与骈肩累迹,瞻望咨嗟;而所谓庸夫愚妇者,奔走骇汗,羞愧俯伏,以自悔罪于车尘马足之间。此一介之士,得志于当时,而意气之盛,昔人比之衣锦之荣者也。


惟大丞相魏国公则不然:公,相人也,世有令德,为时名卿。自公少时,已擢高科,登显仕。海内之士,闻下风而望余光者,盖亦有年矣。所谓将相而富贵,皆公所宜素有;非如穷厄之人,侥幸得志于一时,出于庸夫愚妇之不意,以惊骇而夸耀之也。然则高牙大纛,不足为公荣;桓圭衮冕,不足为公贵。惟德被生民,而功施社稷,勒之金石,播之声诗,以耀后世而垂无穷,此公之志,而士亦以此望于公也。岂止夸一时而荣一乡哉!


公在至和中,尝以武康之节,来治于相,乃作“昼锦”之堂于后圃。既又刻诗于石,以遗相人。其言以快恩仇、矜名誉为可薄,盖不以昔人所夸者为荣,而以为戒。于此见公之视富贵为何如,而其志岂易量哉!故能出入将相,勤劳王家,而夷险一节。至于临大事,决大议,垂绅正笏,不动声色,而措天下于泰山之安:可谓社稷之臣矣!其丰功盛烈,所以铭彝鼎而被弦歌者,乃邦家之光,非闾里之荣也。


余虽不获登公之堂,幸尝窃诵公之诗,乐公之志有成,而喜为天下道也。于是乎书。


尚书吏部侍郎、参知政事欧阳修记。



\chapter*{越州赵公救灾记}
\addcontentsline{toc}{chapter}{越州赵公救灾记}
\begin{center}
	\textbf{[宋朝]曾巩}
\end{center}


熙宁八年夏,吴越大旱。九月,资政殿大学士知越州赵公,前民之未饥,为书问属县灾所被者几乡,民能自食者有几,当廪于官者几人,沟防构筑可僦民使治之者几所,库钱仓粟可发者几何,富人可募出粟者几家,僧道士食之羡粟书于籍者其几具存,使各书以对,而谨其备。


州县史录民之孤老疾弱不能自食者二万一千九百余人以告。故事,岁廪穷人,当给粟三千石而止。公敛富人所输,及僧道士食之羡者,得粟四万八千余石,佐其费。使自十月朔,人受粟日一升,幼小半之。忧其众相蹂也,使受粟者男女异日,而人受二日之食。忧其流亡也,于城市郊野为给粟之所凡五十有七,使各以便受之而告以去其家者勿给。计官为不足用也,取吏之不在职而寓于境者,给其食而任以事。不能自食者,有是具也。能自食者,为之告富人无得闭粜。又为之官粟,得五万二千余石,平其价予民。为粜粟之所凡十有八,使籴者自便如受粟。又僦民完成四千一百丈,为工三万八千,计其佣与钱,又与粟再倍之。民取息钱者,告富人纵予之而待熟,官为责其偿。弃男女者,使人得收养之。


明年春,大疫。为病坊,处疾病之无归者。募僧二人,属以视医药饮食,令无失所恃。凡死者,使在处随收瘗之。


法,廪穷人尽三月当止,是岁尽五月而止。事有非便文者,公一以自任,不以累其属。有上请者,或便宜多辄行。公于此时,蚤夜惫心力不少懈,事细巨必躬亲。给病者药食多出私钱。民不幸罹旱疫,得免于转死;虽死得无失敛埋,皆公力也。


是时旱疫被吴越,民饥馑疾疠,死者殆半,灾未有巨于此也。天子东向忧劳,州县推布上恩,人人尽其力。公所拊循,民尤以为得其依归。所以经营绥辑先后终始之际,委曲纤悉,无不备者。其施虽在越,其仁足以示天下;其事虽行于一时,其法足以传后。盖灾沴之行,治世不能使之无,而能为之备。民病而后图之,与夫先事而为计者,则有间矣;不习而有为,与夫素得之者,则有间矣。予故采于越,得公所推行,乐为之识其详,岂独以慰越人之思,半使吏之有志于民者不幸而遇岁之灾,推公之所已试,其科条可不待顷而具,则公之泽岂小且近乎!


公元丰二年以大学士加太子保致仕,家于衢。其直道正行在于朝廷,岂弟之实在于身者,此不著。著其荒政可师者,以为《越州赵公救灾记》云。



\chapter*{阅江楼记}
\addcontentsline{toc}{chapter}{阅江楼记}
\begin{center}
	\textbf{[明朝]宋濂}
\end{center}


金陵为帝王之州。自六朝迄于南唐,类皆偏据一方,无以应山川之王气。逮我皇帝,定鼎于兹,始足以当之。由是声教所暨,罔间朔南;存神穆清,与天同体。虽一豫一游,亦可为天下后世法。京城之西北有狮子山,自卢龙蜿蜒而来。长江如虹贯,蟠绕其下。上以其地雄胜,诏建楼于巅,与民同游观之乐。遂锡嘉名为“阅江”云。


登览之顷,万象森列,千载之秘,一旦轩露。岂非天造地设,以俟大一统之君,而开千万世之伟观者欤?当风日清美,法驾幸临,升其崇椒,凭阑遥瞩,必悠然而动遐思。见江汉之朝宗,诸侯之述职,城池之高深,关阨之严固,必曰:“此朕沐风栉雨、战胜攻取之所致也。”中夏之广,益思有以保之。见波涛之浩荡,风帆之上下,番舶接迹而来庭,蛮琛联肩而入贡,必曰:“此朕德绥威服,覃及外内之所及也。”四陲之远,益思所以柔之。见两岸之间、四郊之上,耕人有炙肤皲足之烦,农女有捋桑行馌之勤,必曰:“此朕拔诸水火、而登于衽席者也。”万方之民,益思有以安之。触类而思,不一而足。臣知斯楼之建,皇上所以发舒精神,因物兴感,无不寓其致治之思,奚此阅夫长江而已哉?彼临春、结绮,非弗华矣;齐云、落星,非不高矣。不过乐管弦之淫响,藏燕赵之艳姬。一旋踵间而感慨系之,臣不知其为何说也。


虽然,长江发源岷山,委蛇七千余里而始入海,白涌碧翻,六朝之时,往往倚之为天堑;今则南北一家,视为安流,无所事乎战争矣。然则,果谁之力欤?逢掖之士,有登斯楼而阅斯江者,当思帝德如天,荡荡难名,与神禹疏凿之功同一罔极。忠君报上之心,其有不油然而兴者耶?


臣不敏,奉旨撰记,欲上推宵旰图治之切者,勒诸贞珉。他若留连光景之辞,皆略而不陈,惧亵也。



\chapter*{阴饴甥对秦伯}
\addcontentsline{toc}{chapter}{阴饴甥对秦伯}
\begin{center}
	\textbf{[春秋战国]左丘明}
\end{center}


十月,晋阴饴甥会秦伯,盟于王城。


秦伯曰:“晋国和乎?”对曰:“不和。小人耻失其君而悼丧其亲,不惮征缮以立圉也。曰:‘必报仇,宁事戎狄。’君子爱其君而知其罪,不惮征缮以待秦命。曰:‘必报德,有死无二。’以此不和。”秦伯曰:“国谓君何?”对曰:“小人戚,谓之不免;君子恕,以为必归。小人曰:‘我毒秦,秦岂归君?’君子曰:‘我知罪矣,秦必归君。贰而执之,服而舍之,德莫厚焉,刑莫威焉。服者怀德,贰者畏刑,此一役也,秦可以霸。纳而不定,废而不立,以德为怨,秦不其然。’”秦伯曰:“是吾心也。”


改馆晋侯,馈七牢焉。



\chapter*{狱中杂记}
\addcontentsline{toc}{chapter}{狱中杂记}
\begin{center}
	\textbf{[清朝]方苞}
\end{center}

康熙五十一年三月,余在刑部狱,见死而由窦出者,日四三人。有洪洞令杜君者,作而言曰:“此疫作也。今天时顺正,死者尚稀,往岁多至日数十人。”余叩所以。杜君曰:“是疾易传染,遘者虽戚属不敢同卧起。而狱中为老监者四,监五室,禁卒居中央,牖其前以通明,屋极有窗以达气。旁四室则无之,而系囚常二百余。每薄暮下管键,矢溺皆闭其中,与饮食之气相薄,又隆冬,贫者席地而卧,春气动,鲜不疫矣。狱中成法,质明启钥,方夜中,生人与死者并踵顶而卧,无可旋避,此所以染者众也。又可怪者,大盗积贼,杀人重囚,气杰旺,染此者十不一二,或随有瘳,其骈死,皆轻系及牵连佐证法所不及者。”余曰:“京师有京兆狱,有五城御史司坊,何故刑部系囚之多至此?”杜君曰:“迩年狱讼,情稍重,京兆、五城即不敢专决;又九门提督所访缉纠诘,皆归刑部;而十四司正副郎好事者及书吏、狱官、禁卒,皆利系者之多,少有连,必多方钩致。苟入狱,不问罪之有无,必械手足,置老监,俾困苦不可忍,然后导以取保,出居于外,量其家之所有以为剂,而官与吏剖分焉。中家以上,皆竭资取保;其次‘求脱械居监外板屋,费亦数十金;惟极贫无依,则械系不稍宽,为标准以警其余。或同系,情罪重者,反出在外,而轻者、无罪者罹其毒。积忧愤,寝食违节,及病,又无医药,故往往至死。”余伏见圣上好生之德,同于往圣。每质狱词,必于死中求其生,而无辜者乃至此。傥仁人君子为上昌言:除死刑及发塞外重犯,其轻系及牵连未结正者,别置一所以羁之,手足毋械。所全活可数计哉?或曰:“狱旧有室五,名曰现监,讼而未结正者居之。傥举旧典,可小补也。杜君曰:“上推恩,凡职官居板屋。今贫者转系老监,而大盗有居板屋者。此中可细诘哉!不若别置一所,为拔本塞源之道也。”余同系朱翁、余生及在狱同官僧某,遘疫死,皆不应重罚。又某氏以不孝讼其子,左右邻械系入老监,号呼达旦。余感焉,以杜君言泛讯之,众言同,于是乎书。

凡死刑狱上,行刑者先俟于门外,使其党入索财物,名曰“斯罗”。富者就其戚属,贫则面语之。其极刑,曰:“顺我,即先刺心;否则,四肢解尽,心犹不死。”其绞缢,曰:“顺我,始缢即气绝;否则,三缢加别械,然后得死。”唯大辟无可要,然犹质其首。用此,富者赂数十百金,贫亦罄衣装;绝无有者,则治之如所言。主缚者亦然,不如所欲,缚时即先折筋骨。每岁大决,勾者十四三,留者十六七,皆缚至西市待命。其伤于缚者,即幸留,病数月乃瘳,或竟成痼疾。余尝就老胥而问焉:“彼于刑者、缚者,非相仇也,期有得耳;果无有,终亦稍宽之,非仁术乎?”曰:“是立法以警其余,且惩后也;不如此,则人有幸心。”主梏扑者亦然。余同逮以木讯者三人:一人予三十金,骨微伤,病间月;一人倍之,伤肤,兼旬愈;一人六倍,即夕行步如平常。或叩之曰:“罪人有无不均,既各有得,何必更以多寡为差?”曰:“无差,谁为多与者?”孟子曰:“术不可不慎。”信夫!

部中老胥,家藏伪章,文书下行直省,多潜易之,增减要语,奉行者莫辨也。其上闻及移关诸部,犹未敢然。功令:大盗未杀人及他犯同谋多人者,止主谋一二人立决;余经秋审皆减等发配。狱词上,中有立决者,行刑人先俟于门外。命下,遂缚以出,不羁晷刻。有某姓兄弟以把持公仓,法应立决,狱具矣,胥某谓曰:“予我千金,吾生若。”叩其术,曰:“是无难,别具本章,狱词无易,取案末独身无亲戚者二人易汝名,俟封奏时潜易之而已。”其同事者曰:“是可欺死者,而不能欺主谳者,倘复请之,吾辈无生理矣。”胥某笑曰:“复请之,吾辈无生理,而主谳者亦各罢去。彼不能以二人之命易其官,则吾辈终无死道也。”竟行之,案末二人立决。主者口呿舌挢,终不敢诘。余在狱,犹见某姓,狱中人群指曰:“是以某某易其首者。”胥某一夕暴卒,众皆以为冥谪云。

凡杀人,狱词无谋、故者,经秋审入矜疑,即免死。吏因以巧法。有郭四者,凡四杀人,复以矜疑减等,随遇赦。将出,日与其徒置酒酣歌达曙。或叩以往事,一一详述之,意色扬扬,若自矜诩。噫!渫恶吏忍于鬻狱,无责也;而道之不明,良吏亦多以脱人于死为功,而不求其情,其枉民也亦甚矣哉!

奸民久于狱,与胥卒表里,颇有奇羡。山阴李姓以杀人系狱,每岁致数百金。康熙四十八年,以赦出。居数月,漠然无所事。其乡人有杀人者,因代承之。盖以律非故杀,必久系,终无死法也。五十一年,复援赦减等谪戍,叹曰:“吾不得复入此矣!”故例:谪戍者移顺天府羁候。时方冬停遣,李具状求在狱候春发遣,至再三,不得所请,怅然而出。


\chapter*{齐桓下拜受胙}
\addcontentsline{toc}{chapter}{齐桓下拜受胙}
\begin{center}
	\textbf{[春秋战国]左丘明}
\end{center}


夏,会于葵丘,寻盟,且修好,礼也。


王使宰孔赐齐侯胙,曰:“天子有事于文武,使孔赐伯舅胙。”齐侯将下拜。孔曰:“且有后命。天子使孔曰:‘以伯舅耋老,加劳,赐一级,无下拜!”’对曰:“天威不违颜咫尺,小白余敢贪天子之命‘无下拜’!恐陨越于下,以遗天子羞,敢不下拜?”下,拜,登,受。



\chapter*{鹦鹉灭火}
\addcontentsline{toc}{chapter}{鹦鹉灭火}
\begin{center}
	\textbf{[南北朝]刘义庆}
\end{center}

有鹦鹉飞集他山,山中禽兽辄相爱。鹦鹉自念虽乐,此山虽乐,然非吾久居之地,遂去,禽兽依依不舍后数月,山中大火。鹦鹉遥见,心急如焚,遂入水沾羽,飞而洒之。

天神言:“汝虽有好意,然何足道也?”对曰:“虽知区区水滴不能救,然吾尝侨居是山,禽兽善待,皆为兄弟,吾不忍见其毁于火也!”

天神嘉其义,即为之灭火。


\chapter*{桐叶封弟辨}
\addcontentsline{toc}{chapter}{桐叶封弟辨}
\begin{center}
	\textbf{[唐朝]柳宗元}
\end{center}


古之传者有言:成王以桐叶与小弱弟戏,曰:“以封汝。”周公入贺。王曰:“戏也。”周公曰:“天子不可戏。”乃封小弱弟于唐。


吾意不然。王之弟当封邪,周公宜以时言于王,不待其戏而贺以成之也。不当封邪,周公乃成其不中之戏,以地以人与小弱者为之主,其得为圣乎?且周公以王之言不可苟焉而已,必从而成之邪?设有不幸,王以桐叶戏妇寺,亦将举而从之乎?凡王者之德,在行之何若。设未得其当,虽十易之不为病;要于其当,不可使易也,而况以其戏乎!若戏而必行之,是周公教王遂过也。


吾意周公辅成王,宜以道,从容优乐,要归之大中而已,必不逢其失而为之辞。又不当束缚之,驰骤之,使若牛马然,急则败矣。且家人父子尚不能以此自克,况号为君臣者邪!是直小丈夫缺缺者之事,非周公所宜用,故不可信。


或曰:封唐叔,史佚成之。


\chapter*{龙井题名记}
\addcontentsline{toc}{chapter}{龙井题名记}
\begin{center}
	\textbf{[宋朝]秦观}
\end{center}


元丰二年,中秋后一日,余自吴兴来杭,东还会稽。龙井有辨才大师,以书邀余入山。比出郭,日已夕,航湖至普宁,遇道人参寥,问龙井所遣篮舆,则曰:“以不时至,去矣。”


是夕,天宇开霁,林间月明,可数毫发。遂弃舟,从参寥策杖并湖而行。出雷峰,度南屏,濯足于惠因涧,入灵石坞,得支径上风篁岭,憩于龙井亭,酌泉据石而饮之。自普宁凡经佛寺十五,皆寂不闻人声。道旁庐舍,灯火隐显,草木深郁,流水激激悲鸣,殆非人间之境。行二鼓,始至寿圣院,谒辨才于朝音堂,明日乃还。



\chapter*{蔺相如完璧归赵论}
\addcontentsline{toc}{chapter}{蔺相如完璧归赵论}
\begin{center}
	\textbf{[明朝]王世贞}
\end{center}

蔺相如之完璧,人皆称之。予未敢以为信也。

夫秦以十五城之空名,诈赵而胁其璧。是时言取璧者,情也,非欲以窥赵也。赵得其情则弗予,不得其情则予;得其情而畏之则予,得其情而弗畏之则弗予。此两言决耳,奈之何既畏而复挑其怒也!

且夫秦欲璧,赵弗予璧,两无所曲直也。入璧而秦弗予城,曲在秦;秦出城而璧归,曲在赵。欲使曲在秦,则莫如弃璧;畏弃璧,则莫如弗予。夫秦王既按图以予城,又设九宾,斋而受璧,其势不得不予城。璧入而城弗予,相如则前请曰:“臣固知大王之弗予城也。夫璧非赵璧乎?而十五城秦宝也。今使大王以璧故,而亡其十五城,十五城之子弟,皆厚怨大王以弃我如草芥也。大王弗与城,而绐赵璧,以一璧故,而失信于天下,臣请就死于国,以明大王之失信!”秦王未必不返璧也。今奈何使舍人怀而逃之,而归直于秦?

是时秦意未欲与赵绝耳。令秦王怒而僇相如于市,武安君十万众压邯郸,而责璧与信,一胜而相如族,再胜而璧终入秦矣。

吾故曰:蔺相如之获全于璧也,天也。若其劲渑池,柔廉颇,则愈出而愈妙于用。所以能完赵者,天固曲全之哉!


\chapter*{泷冈阡表}
\addcontentsline{toc}{chapter}{泷冈阡表}
\begin{center}
	\textbf{[宋朝]欧阳修}
\end{center}


呜呼!惟我皇考崇公,卜吉于泷冈之六十年,其子修始克表于其阡。非敢缓也,盖有待也。


修不幸,生四岁而孤。太夫人守节自誓;居穷,自力于衣食,以长以教俾至于成人。太夫人告之曰:汝父为吏廉,而好施与,喜宾客;其俸禄虽薄,常不使有余。曰:“毋以是为我累。”故其亡也,无一瓦之覆,一垄之植,以庇而为生;吾何恃而能自守邪?吾于汝父,知其一二,以有待于汝也。自吾为汝家妇,不及事吾姑;然知汝父之能养也。汝孤而幼,吾不能知汝之必有立;然知汝父之必将有后也。吾之始归也,汝父免于母丧方逾年,岁时祭祀,则必涕泣,曰:“祭而丰,不如养之薄也。”间御酒食,则又涕泣,曰:“昔常不足,而今有余,其何及也!”吾始一二见之,以为新免于丧适然耳。既而其后常然,至其终身,未尝不然。吾虽不及事姑,而以此知汝父之能养也。汝父为吏,尝夜烛治官书,屡废而叹。吾问之,则曰:“此死狱也,我求其生不得尔。”吾曰:“生可求乎?”曰:“求其生而不得,则死者与我皆无恨也;矧求而有得邪,以其有得,则知不求而死者有恨也。夫常求其生,犹失之死,而世常求其死也。”回顾乳者剑汝而立于旁,因指而叹,曰:“术者谓我岁行在戌将死,使其言然,吾不及见儿之立也,后当以我语告之。”其平居教他子弟,常用此语,吾耳熟焉,故能详也。其施于外事,吾不能知;其居于家,无所矜饰,而所为如此,是真发于中者邪!呜呼!其心厚于仁者邪!此吾知汝父之必将有后也。汝其勉之!夫养不必丰,要于孝;利虽不得博于物,要其心之厚于仁。吾不能教汝,此汝父之志也。”修泣而志之,不敢忘。


先公少孤力学,咸平三年进士及第,为道州判官,泗绵二州推官;又为泰州判官。享年五十有九,葬沙溪之泷冈。


太夫人姓郑氏,考讳德仪,世为江南名族。太夫人恭俭仁爱而有礼;初封福昌县太君,进封乐安、安康、彭城三郡太君。自其家少微时,治其家以俭约,其后常不使过之,曰:“吾儿不能苟合于世,俭薄所以居患难也。”其后修贬夷陵,太夫人言笑自若,曰:“汝家故贫贱也,吾处之有素矣。汝能安之,吾亦安矣。”


自先公之亡二十年,修始得禄而养。又十有二年,烈官于朝,始得赠封其亲。又十年,修为龙图阁直学士,尚书吏部郎中,留守南京,太夫人以疾终于官舍,享年七十有二。又八年,修以非才入副枢密,遂参政事,又七年而罢。自登二府,天子推恩,褒其三世,盖自嘉祐以来,逢国大庆,必加宠锡。皇曾祖府君累赠金紫光禄大夫、太师、中书令;曾祖妣累封楚国太夫人。皇祖府君累赠金紫光禄大夫、太师、中书令兼尚书令,祖妣累封吴国太夫人。皇考崇公累赠金紫光禄大夫、太师、中书令兼尚书令。皇妣累封越国太夫人。今上初郊,皇考赐爵为崇国公,太夫人进号魏国。


于是小子修泣而言曰:“呜呼!为善无不报,而迟速有时,此理之常也。惟我祖考,积善成德,宜享其隆,虽不克有于其躬,而赐爵受封,显荣褒大,实有三朝之锡命,是足以表见于后世,而庇赖其子孙矣。”乃列其世谱,具刻于碑,既又载我皇考崇公之遗训,太夫人之所以教,而有待于修者,并揭于阡。俾知夫小子修之德薄能鲜,遭时窃位,而幸全大节,不辱其先者,其来有自。熙宁三年,岁次庚戌,四月辛酉朔,十有五日乙亥,男推诚、保德、崇仁、翊戴功臣,观文殿学士,特进,行兵部尚书,知青州军州事,兼管内劝农使,充京东路安抚使,上柱国,乐安郡开国公,食邑四千三百户,食实封一千二百户,修表。



\chapter*{原毁}
\addcontentsline{toc}{chapter}{原毁}
\begin{center}
	\textbf{[唐朝]韩愈}
\end{center}

古之君子,其责己也重以周,其待人也轻以约。重以周,故不怠;轻以约,故人乐为善。

闻古之人有舜者,其为人也,仁义人也。求其所以为舜者,责于己曰:“彼,人也;予,人也。彼能是,而我乃不能是!”早夜以思,去其不如舜者,就其如舜者。闻古之人有周公者,其为人也,多才与艺人也。求其所以为周公者,责于己曰:“彼,人也;予,人也。彼能是,而我乃不能是!”早夜以思,去其不如周公者,就其如周公者。舜,大圣人也,后世无及焉;周公,大圣人也,后世无及焉。是人也,乃曰:“不如舜,不如周公,吾之病也。”是不亦责于身者重以周乎!其于人也,曰:“彼人也,能有是,是足为良人矣;能善是,是足为艺人矣。”取其一,不责其二;即其新,不究其旧:恐恐然惟惧其人之不得为善之利。一善易修也,一艺易能也,其于人也,乃曰:“能有是,是亦足矣。”曰:“能善是,是亦足矣。”不亦待于人者轻以约乎?

今之君子则不然。其责人也详,其待己也廉。详,故人难于为善;廉,故自取也少。己未有善,曰:“我善是,是亦足矣。”己未有能,曰:“我能是,是亦足矣。”外以欺于人,内以欺于心,未少有得而止矣,不亦待其身者已廉乎?

其于人也,曰:“彼虽能是,其人不足称也;彼虽善是,其用不足称也。”举其一,不计其十;究其旧,不图其新:恐恐然惟惧其人之有闻也。是不亦责于人者已详乎?

夫是之谓不以众人待其身,而以圣人望于人,吾未见其尊己也。

虽然,为是者,有本有原,怠与忌之谓也。怠者不能修,而忌者畏人修。吾尝试之矣,尝试语于众曰:“某良士,某良士。”其应者,必其人之与也;不然,则其所疏远不与同其利者也;不然,则其畏也。不若是,强者必怒于言,懦者必怒于色矣。又尝语于众曰:“某非良士,某非良士。”其不应者,必其人之与也,不然,则其所疏远不与同其利者也,不然,则其畏也。不若是,强者必说于言,懦者必说于色矣。

是故事修而谤兴,德高而毁来。呜呼!士之处此世,而望名誉之光,道德之行,难已!

将有作于上者,得吾说而存之,其国家可几而理欤!


\chapter*{春王正月}
\addcontentsline{toc}{chapter}{春王正月}
\begin{center}
	\textbf{[春秋战国]公羊高}
\end{center}

元年者何?君之始年也。春者何?岁之始也。王者孰谓?谓文王也。曷为先言“王”而后言“正月?”王正月也。何言乎王正月?大一统也。

公何以不言即位?成公意也。何成乎公之意?公将平国而反之桓。曷为反之桓?桓幼而贵,隐长而卑。其为尊卑也微,国人莫知。隐长又贤,诸大夫扳隐而立之。隐于是焉而辞立,则未知桓之将必得立也;且如桓立,则恐诸大夫之不能相幼君也。故凡隐之立,为桓立也。隐长又贤,何以不宜立?立適以长不以贤,立子以贵不以长。桓何以贵?母贵也。母贵,则子何以贵?子以母贵,母以子贵。


\chapter*{游灵岩记}
\addcontentsline{toc}{chapter}{游灵岩记}
\begin{center}
	\textbf{[明朝]高启}
\end{center}


吴城东无山,唯西为有山,其峰联岭属,纷纷靡靡,或起或伏,而灵岩居其词,拔其挺秀,若不肯与众峰列。望之者,咸知其有异也。


山仰行而上,有亭焉,居其半,盖以节行者之力,至此而得少休也。由亭而稍上,有穴窈然,曰西施之洞;有泉泓然,曰浣花之池;皆吴王夫差宴游之遗处也。又其上则有草堂,可以容栖迟;有琴台,可以周眺览;有轩以直洞庭之峰,曰抱翠;有阁以瞰具区之波,曰涵空,虚明动荡,用号奇观。盖专此郡之美者,山;而专此山之美者,阁也。


启,吴人,游此虽甚亟,然山每匿幽閟胜,莫可搜剔,如鄙予之陋者。今年春,从淮南行省参知政事临川饶公与客十人复来游。升于高,则山之佳者悠然来。入于奥,则石之奇者突然出。氛岚为之蹇舒,杉桧为之拂舞。幽显巨细,争献厥状,披豁呈露,无有隐循。然后知于此山为始著于今而素昧于昔也。


夫山之异于众者,尚能待人而自见,而况人之异于众者哉!公顾瞻有得,因命客赋诗,而属启为之记。启谓:“天于诡奇之地不多设,人于登临之乐不常遇。有其地而非其人,有其人而非其地,皆不足以尽夫游观之乐也。今灵岩为名山,诸公为名士,盖必相须而适相值,夫岂偶然哉!宜其目领而心解,景会而理得也。若启之陋,而亦与其有得焉,顾非幸也欤?启为客最少,然敢执笔而不辞者,亦将有以私识其幸也!”十人者,淮海秦约、诸暨姜渐、河南陆仁、会稽张宪、天台詹参、豫章陈增、吴郡金起、金华王顺、嘉陵杨基、吴陵刘胜也。 



\chapter*{晏子不死君难}
\addcontentsline{toc}{chapter}{晏子不死君难}
\begin{center}
	\textbf{[春秋战国]左丘明}
\end{center}


崔武子见棠姜而美之,遂取之。庄公通焉。崔子弑之。


晏子立于崔氏之门外。其人曰:“死乎?”曰:“独吾君也乎哉,吾死也?”曰:“行乎?”曰:“吾罪也乎哉,吾亡也?”曰:“归乎?”曰:“君死,安归?君民者,岂以陵民?社稷是主。臣君者,岂为其口实?社稷是养。故君为社稷死,则死之;为社稷亡,则亡之。若为己死,而为己亡,非其私暱,谁敢任之?且人有君而弑之,吾焉得死之?而焉得亡之?将庸何归?”门启而入,枕尸股而哭。兴,三踊而出。人谓崔子:“必杀之。”崔子曰:“民之望也,舍之得民。”



\chapter*{子产不毁乡校颂}
\addcontentsline{toc}{chapter}{子产不毁乡校颂}
\begin{center}
	\textbf{[唐朝]韩愈}
\end{center}

我思古人,伊郑之侨。以礼相国,人未安其教;游于乡之校,众口嚣嚣。或谓子产:“毁乡校则止。”曰:“何患焉?可以成美。夫岂多言,亦各其志:善也吾行,不善吾避;维善维否,我于此视。川不可防,言不可弭。下塞上聋,邦其倾矣。”既乡校不毁,而郑国以理。

在周之兴,养老乞言;及其已衰,谤者使监。成败之迹,昭哉可观。

维是子产,执政之式。维其不遇,化止一国。诚率此道,相天下君;交畅旁达,施及无垠,於虖!四海所以不理,有君无臣。谁其嗣之?我思古人!


\chapter*{纵囚论}
\addcontentsline{toc}{chapter}{纵囚论}
\begin{center}
	\textbf{[宋朝]欧阳修}
\end{center}


信义行于君子,而刑戮施于小人。刑入于死者,乃罪大恶极,此又小人之尤甚者也。宁以义死,不苟幸生,而视死如归,此又君子之尤难者也。方唐太宗之六年,录大辟囚三百余人,纵使还家,约其自归以就死。是以君子之难能,期小人之尤者以必能也。其囚及期,而卒自归无后者。是君子之所难,而小人之所易也。此岂近于人情哉?


或曰:罪大恶极,诚小人矣;及施恩德以临之,可使变而为君子。盖恩德入人之深,而移人之速,有如是者矣。曰:太宗之为此,所以求此名也。然安知夫纵之去也,不意其必来以冀免,所以纵之乎?又安知夫被纵而去也,不意其自归而必获免,所以复来乎?夫意其必来而纵之,是上贼下之情也;意其必免而复来,是下贼上之心也。吾见上下交相贼以成此名也,乌有所谓施恩德与夫知信义者哉?不然,太宗施德于天下,于兹六年矣,不能使小人不为极恶大罪,而一日之恩,能使视死如归,而存信义。此又不通之论也!


然则何为而可?曰:纵而来归,杀之无赦。而又纵之,而又来,则可知为恩德之致尔。然此必无之事也。若夫纵而来归而赦之,可偶一为之尔。若屡为之,则杀人者皆不死。是可为天下之常法乎?不可为常者,其圣人之法乎?是以尧、舜、三王之治,必本于人情,不立异以为高,不逆情以干誉。



\chapter*{报刘一丈书}
\addcontentsline{toc}{chapter}{报刘一丈书}
\begin{center}
	\textbf{[明朝]宗臣}
\end{center}

数千里外,得长者时赐一书,以慰长想,即亦甚幸矣;何至更辱馈遗,则不才益将何以报焉?书中情意甚殷,即长者之不忘老父,知老父之念长者深也。

至以「上下相孚,才德称位」语不才,则不才有深感焉。夫才德不称,固自知之矣;至於不孚之病,则尤不才为甚。

且今之所谓孚者,何哉?日夕策马,候权者之门。门者故不入,则甘言媚词,作妇人状,袖金以私之。即门者持刺入,而主人又不即出见;立厩中仆马之间,恶气袭衣袖,即饥寒毒热不可忍,不去也。抵暮,则前所受赠金者,出报客曰:「相公倦,谢客矣!客请明日来!」即明日,又不敢不来。夜披衣坐,闻鸡鸣,即起盥栉,走马抵门;门者怒曰:「为谁?」则曰:「昨日之客来。」则又怒曰:「何客之勤也?岂有相公此时出见客乎?」客心耻之,强忍而与言曰:「亡奈何矣,姑容我入!」门者又得所赠金,则起而入之;又立向所立厩中。幸主者出,南面召见,则惊走匍匐阶下。主者曰:「进!」则再拜,故迟不起;起则上所上寿金。主者故不受,则固请。主者故固不受,则又固请,然後命吏纳之。则又再拜,又故迟不起;起则五六揖始出。出揖门者曰:「官人幸顾我,他日来,幸无阻我也!」门者答揖。大喜奔出,马上遇所交识,即扬鞭语曰:「适自相公家来,相公厚我,厚我!」且虚言状。即所交识,亦心畏相公厚之矣。相公又稍稍语人曰:「某也贤!某也贤!」闻者亦心许交赞之。

此世所谓上下相孚也,长者谓仆能之乎?前所谓权门者,自岁时伏腊,一刺之外,即经年不往也。闲道经其门,则亦掩耳闭目,跃马疾走过之,若有所追逐者,斯则仆之褊衷,以此长不见怡於长吏,仆则愈益不顾也。每大言曰:「人生有命,吾惟有命,吾惟守分而已。」长者闻之,得无厌其为迂乎?

乡园多故,不能不动客子之愁。至于长者之抱才而困,则又令我怆然有感。天之与先生者甚厚,亡论长者不欲轻弃之,即天意亦不欲长者之轻弃之也,幸宁心哉!


\chapter*{楚归晋知罃}
\addcontentsline{toc}{chapter}{楚归晋知罃}
\begin{center}
	\textbf{[春秋战国]左丘明}
\end{center}


晋人归楚公子谷臣,与连尹襄老之尸于楚,以求知罃。于是荀首佐中军矣,故楚人许之。


王送知罃,曰:“子其怨我乎?”对曰:“二国治戎,臣不才,不胜其任,以为俘馘。执事不以衅鼓,使归即戮,君之惠也。臣实不才,又谁敢怨?”


王曰:“然则德我乎?”对曰:“二国图其社稷,而求纾其民,各惩其忿,以相宥也,两释累囚,以成其好。二国有好,臣不与及,其谁敢德?”


王曰:“子归何以报我?”对曰:“臣不任受怨,君亦不任受德。无怨无德,不知所报。”


王曰:“虽然,必告不谷。”对曰:“以君之灵,累臣得归骨于晋,寡君之以为戮,死且不朽。若从君之惠而免之,以赐君之外臣首;首其请于寡君,而以戮于宗,亦死且不朽。若不获命,而使嗣宗职,次及于事,而帅偏师以脩封疆,虽遇执事,其弗敢违。其竭力致死,无有二心,以尽臣礼。所以报也!


王曰:“晋未可与争。”重为之礼而归之。



\chapter*{王孙满对楚子}
\addcontentsline{toc}{chapter}{王孙满对楚子}
\begin{center}
	\textbf{[春秋战国]左丘明}
\end{center}


楚子伐陆浑之戎,遂至于雒,观兵于周疆。定王使王孙满劳楚子。楚子问鼎之大小轻重焉。


对曰:“在德不在鼎。昔夏之方有德也,远方图物,贡金九牧,铸鼎象物,百物而为之备,使民知神奸。故民入川泽山林,不逢不若。螭魅罔两,莫能逢之。用能协于上下,以承天休。桀有昏德,鼎迁于商,载祀六百。商纣暴虐,鼎迁于周。德之休明,虽小,重也。其奸回昏乱,虽大,轻也。天祚明德,有所止。成王定鼎于郏鄏,卜世三十,卜年七百,天所命也。周德虽衰,天命未改。鼎之轻重,未可问也。”



\chapter*{唐太宗吞蝗}
\addcontentsline{toc}{chapter}{唐太宗吞蝗}
\begin{center}
	\textbf{[唐朝]吴兢}
\end{center}


观二年,京师旱,蝗虫大起。太宗入苑视禾,见蝗虫,掇数枚而曰:“人以谷为命,而汝食之,是害于百姓。百姓有过,在予一人。尔其有灵,但当蚀我心,无害百姓。”将吞之,左右遽谏曰:“恐诚疾,不可!”太宗:“所冀移灾朕躬,何疾之避!”遂吞之。

\chapter*{石碏谏宠州吁}
\addcontentsline{toc}{chapter}{石碏谏宠州吁}
\begin{center}
	\textbf{[春秋战国]左丘明}
\end{center}


卫庄公娶于齐东宫得臣之妹,曰庄姜。美而无子,卫人所为赋《硕人》也。又娶于陈,曰厉妫。生孝伯,蚤死。其娣戴妫生桓公,庄姜以为己子。


公子州吁,嬖人之子也。有宠而好兵,公弗禁,庄姜恶之。


石碏谏曰:“臣闻爱子,教之以义方,弗纳于邪。骄奢淫佚,所自邪也。四者之来,宠禄过也。将立州吁,乃定之矣;若犹未也,阶之为祸。夫宠而不骄,骄而能降,降而不憾,憾而能眕者,鲜矣。且夫贱妨贵,少陵长,远间亲,新间旧,小加大,淫破义,所谓六逆也。君义,臣行,父慈,子孝,兄爱,弟敬,所谓六顺也。去顺效逆,所以速祸也。君人者,将祸是[通“事”]务去,而速之,无乃不可乎?”弗听。


其子厚与州吁游,禁之,不可。桓公立,乃老。



\chapter*{送薛存义序}
\addcontentsline{toc}{chapter}{送薛存义序}
\begin{center}
	\textbf{[唐朝]柳宗元}
\end{center}


河东薛存义将行,柳子载肉于俎,崇酒於觞,追而送之江浒,饮食之。且告曰:“凡吏于土者,若知其职乎?盖民之役,非以役民而已也。凡民之食于土者,出其什一佣乎吏,使司平于我也。今我受其直,怠其事者,天下皆然。岂惟怠之,又从而盗之。向使佣一夫于家,受若值,怠若事,又盗若货器,则必甚怒而黜罚之矣。以今天下多类此,而民莫敢肆其怒与黜罚者,何哉?势不同也。势不同而理同,如吾民何?有达于理者,得不恐而畏乎!”


存义假令零陵二年矣。早作而夜思,勤力而劳心;讼者平,赋者均,老弱无怀诈暴憎。其为不虚取直也的矣,其知恐而畏也审矣。


吾贱且辱,不得与考绩幽明之说;于其往也,故赏以酒肉而重之以辞。



\chapter*{曹刿论战}
\addcontentsline{toc}{chapter}{曹刿论战}
\begin{center}
	\textbf{[春秋战国]左丘明}
\end{center}


十年春,齐师伐我。公将战。曹刿请见。其乡人曰:“肉食者谋之,又何间焉?”刿曰:“肉食者鄙,未能远谋。”乃入见。问:“何以战?”公曰:“衣食所安,弗敢专也,必以分人。”对曰:“小惠未徧,民弗从也。”公曰:“牺牲玉帛,弗敢加也,必以信。”对曰:“小信未孚,神弗福也。”公曰:“小大之狱,虽不能察,必以情。”对曰:“忠之属也。可以一战。战则请从。”(徧同:遍)


公与之乘,战于长勺。公将鼓之。刿曰:“未可。”齐人三鼓。刿曰:“可矣。”齐师败绩。公将驰之。刿曰:“未可。”下视其辙,登轼而望之,曰:“可矣。”遂逐齐师。


既克,公问其故。对曰:“夫战,勇气也。一鼓作气,再而衰,三而竭。彼竭我盈,故克之,夫大国,难测也,惧有伏焉。吾视其辙乱,望其旗靡,故逐之。”



\chapter*{臧哀伯谏纳郜鼎}
\addcontentsline{toc}{chapter}{臧哀伯谏纳郜鼎}
\begin{center}
	\textbf{[春秋战国]左丘明}
\end{center}


夏四月,取郜大鼎于宋,纳于大庙,非礼也。


臧哀伯谏曰:“君人者,将昭德塞违,以临照百官;犹惧或失之,故昭令德以示子孙。是以清庙茅屋,大路越席,大羹不致,粢食不凿,昭其俭也;衮冕黻珽,带裳幅舄,衡紞纮綖,昭其度也;藻率鞞鞛,鞶厉游缨,昭其数也;火龙黼黻,昭其文也;五色比象,昭其物也;钖鸾和铃,昭其声也;三辰旂旗,昭其明也。夫德,俭而有度,登降有数。文物以纪之,声明以发之,以临照百官,百官于是乎戒惧,而不敢易纪律。今灭德立违,而置其赂器于大庙,以明示百官。百官象之,其又何诛焉?国家之败,由官邪也;官之失德,宠赂章也。郜鼎在庙,章孰甚焉?武王克商,迁九鼎于雒邑,义士犹或非之,而况将昭违乱之赂器于大庙。其若之何?”公不听。



\chapter*{苏武传(节选)}
\addcontentsline{toc}{chapter}{苏武传(节选)}
\begin{center}
	\textbf{[汉朝]班固}
\end{center}


武字子卿,少以父任,兄弟并为郎,稍迁至栘中厩监。时汉连伐胡,数通使相窥观。匈奴留汉使郭吉、路充国等前后十余辈,匈奴使来,汉亦留之以相当。天汉元年,且鞮侯单于初立,恐汉袭之,乃曰:「汉天子我丈人行也。」尽归汉使路充国等。武帝嘉其义,乃遣武以中郎将使持节送匈奴使留在汉者,因厚赂单于,答其善意。


武与副中郎将张胜及假吏常惠等募士斥候百余人俱。既至匈奴,置币遗单于;单于益骄,非汉所望也。方欲发使送武等,会缑王与长水虞常等谋反匈奴中。缑王者,昆邪王姊子也,与昆邪王俱降汉,后随浞野侯没胡中,及卫律所将降者,阴相与谋,劫单于母阏氏归汉。会武等至匈奴。虞常在汉时,素与副张胜相知,私候胜曰:「闻汉天子甚怨卫律,常能为汉伏弩射杀之,吾母与弟在汉,幸蒙其赏赐。」张胜许之,以货物与常。后月余,单于出猎,独阏氏子弟在。虞常等七十余人欲发,其一人夜亡告之。单于子弟发兵与战,缑王等皆死,虞常生得。


单于使卫律治其事。张胜闻之,恐前语发,以状语武。武曰:「事如此,此必及我,见犯乃死,重负国!」欲自杀,胜惠共止之。虞常果引张胜。单于怒,召诸贵人议,欲杀汉使者。左伊秩訾曰:「即谋单于,何以复加?宜皆降之。」单于使卫律召武受辞。武谓惠等:「屈节辱命,虽生何面目以归汉?」引佩刀自刺。卫律惊,自抱持武。驰召医,凿地为坎,置煴火,覆武其上,蹈其背,以出血。武气绝,半日复息。惠等哭,舆归营。单于壮其节,朝夕遣人候问武,而收系张胜。


武益愈。单于使使晓武,会论虞常,欲因此时降武。剑斩虞常已,律曰:「汉使张胜谋杀单于近臣,当死;单于募降者,赦罪。」举剑欲击之,胜请降。律谓武曰:「副有罪,当相坐。」武曰:「本无谋,又非亲属,何谓相坐?」复举剑拟之,武不动。律曰:「苏君,律前负汉归匈奴,幸蒙大恩,赐号称王,拥众数万,马畜弥山,富贵如此。苏君今日降,明日复然。空以身膏草野,谁复知之?」武不应。律曰:「君因我降,与君为兄弟;今不听吾计,后虽复欲见我,尚可得乎?」武骂律曰:「汝为人臣子,不顾恩义,畔主背亲,为降虏于蛮夷,何以女为见?且单于信女,使决人死生,不平心持正,反欲斗两主观祸败。南越杀汉使者,屠为九郡;宛王杀汉使者,头县北阙;朝鲜杀汉使者,即时诛灭。独匈奴未耳。若知我不降明,欲令两国相攻,匈奴之祸,从我始矣!」律知武终不可胁,白单于。单于愈益欲降之。乃幽武置大窖中,绝不饮食。天雨雪。武卧,啮雪与旃毛并咽之,数日不死。匈奴以为神,乃徙武北海上无人处,使牧羝。羝乳,乃得归。别其官属常惠等,各置他所。


武既至海上,廪食不至,掘野鼠去草实而食之。杖汉节牧羊,卧起操持,节旄尽落。积五、六年,单于弟于靬王弋射海上。武能网纺缴,檠弓弩,于靬王爱之,给其衣食。三岁余,王病,赐武马畜、服匿、穹庐。王死后,人众徙去。其冬,丁令盗武牛羊,武复穷厄。


初,武与李陵俱为侍中。武使匈奴明年,陵降,不敢求武。久之,单于使陵至海上,为武置酒设乐。因谓武曰:「单于闻陵与子卿素厚,故使陵来说足下,虚心欲相待。终不得归汉,空自苦亡人之地,信义安所见乎?前长君为奉车,从至雍棫阳宫,扶辇下除,触柱,折辕,劾大不敬,伏剑自刎,赐钱二百万以葬。孺卿从祠河东後土,宦骑与黄门驸马争船,推堕驸马河中,溺死,宦骑亡。诏使孺卿逐捕。不得,惶恐饮药而死。来时太夫人已不幸,陵送葬至阳陵。子卿妇年少,闻已更嫁矣。独有女弟二人,两女一男,今复十余年,存亡不可知。人生如朝露,何久自苦如此?陵始降时,忽忽如狂,自痛负汉;加以老母系保宫。子卿不欲降,何以过陵?且陛下春秋高,法令亡常,大臣亡罪夷灭者数十家,安危不可知。子卿尚复谁为乎?愿听陵计,勿复有云!」


武曰:「武父子亡功德,皆为陛下所成就,位列将,爵通侯,兄弟亲近,常愿肝脑涂地。今得杀身自效,虽蒙斧钺汤镬,诚甘乐之。臣事君,犹子事父也。子为父死,亡所恨,愿无复再言。」陵与武饮数日,复曰:「子卿,壹听陵言。」武曰:「自分已死久矣!王必欲降武,请毕今日之欢,效死于前!」陵见其至诚,喟然叹曰:「嗟呼!义士!陵与卫律之罪上通于天!」因泣下沾衿,与武决去。


陵恶自赐武,使其妻赐武牛羊数十头。后陵复至北海上,语武:「区脱捕得云中生口,言太守以下吏民皆白服,曰:『上崩。』」武闻之,南乡号哭,欧血,旦夕临。数月,昭帝即位。数年,匈奴与汉和亲。汉求武等。匈奴诡言武死。后汉使复至匈奴。常惠请其守者与俱,得夜见汉使,具自陈道。教使者谓单于言:「天子射上林中,得雁足有系帛书,言武等在某泽中。」使者大喜,如惠语以让单于。单于视左右而惊,谢汉使曰:「武等实在。」于是李陵置酒贺武曰:「今足下还归,扬名于匈奴,功显于汉室,虽古竹帛所载,丹青所画,何以过子卿!陵虽驽怯,令汉且贳陵罪,全其老母,使得奋大辱之积志,庶几乎曹柯之盟。此陵宿昔之所不忘也!收族陵家,为世大戮,陵尚复何顾乎?已矣!令子卿知吾心耳!异域之人,壹别长绝!」陵起舞,歌曰:「径万里兮度沙幕,为君将兮奋匈奴。路穷绝兮矢刃摧,士众灭兮名已隤,老母已死,虽欲报恩将安归?」


陵泣下数行,因与武决。单于召会武官属,前以降及物故,凡随武还者九人。武以始元六年春至京师,诏武奉一太牢谒武帝园庙,拜为典属国,秩中二千石,赐钱二百万,公田二顷,宅一区。常惠徐圣赵终根皆拜为中郎,赐帛各二百匹。其余六人,老归家,赐钱人十万,复终身。常惠后至右将军,封列侯,自有传。武留匈奴凡十九岁,始以强壮出,及还,须发尽白。



\chapter*{箕子碑}
\addcontentsline{toc}{chapter}{箕子碑}
\begin{center}
	\textbf{[唐朝]柳宗元}
\end{center}


凡大人之道有三:一曰正蒙难,二曰法授圣,三曰化及民。殷有仁人曰箕子,实具兹道以立于世,故孔子述六经之旨,尤殷勤焉。


当纣之时,大道悖乱,天威之动不能戒,圣人之言无所用。进死以并命,诚仁矣,无益吾祀,故不为。委身以存祀,诚仁矣,与亡吾国,故不忍。具是二道,有行之者矣。是用保其明哲,与之俯仰;晦是谟范,辱于囚奴;昏而无邪,隤而不息;故在易曰“箕子之明夷”,正蒙难也。及天命既改,生人以正,乃出大法,用为圣师。周人得以序彝伦而立大典;故在书曰“以箕子归作《洪范》”,法授圣也。及封朝鲜,推道训俗,惟德无陋,惟人无远,用广殷祀,俾夷为华,化及民也。率是大道,丛于厥躬,天地变化,我得其正,其大人欤?


呜乎!当其周时未至,殷祀未殄,比干已死,微子已去,向使纣恶未稔而自毙,武庚念乱以图存,国无其人,谁与兴理?是固人事之或然者也。然则先生隐忍而为此,其有志于斯乎?


唐某年,作庙汲郡,岁时致祀,嘉先生独列于易象,作是颂云:


蒙难以正,授圣以谟。宗祀用繁,夷民其苏。宪宪大人,显晦不渝。圣人之仁,道合隆污。明哲在躬,不陋为奴。冲让居礼,不盈称孤。高而无危,卑不可逾。非死非去,有怀故都。时诎而伸,卒为世模。易象是列,文王为徒。大明宣昭,崇祀式孚。古阙颂辞,继在后儒。



\chapter*{祁奚请免叔向}
\addcontentsline{toc}{chapter}{祁奚请免叔向}
\begin{center}
	\textbf{[春秋战国]左丘明}
\end{center}


栾盈出奔楚。宣子杀羊舌虎,囚叔向。人谓叔向曰:“子离于罪,其为不知乎?”叔向曰:“与其死亡若何?诗曰:‘优哉游哉,聊以卒岁。’知也。”


乐王鲋见叔向曰:“吾为子请。”叔向弗应,出不拜。其人皆咎叔向。叔向曰:“必祁大夫。”室老闻之曰:“乐王鲋言于君无不行,求赦吾子,吾子不许;祁大夫所不能也,而曰必由之。何也?”叔向曰:“乐王鲋从君者也,何能行?祁大夫外举不弃仇,内举不失亲,其独遗我乎?诗曰:‘有觉德行,四国顺之。’夫子,觉者也。”


晋侯问叔向之罪于乐王鲋。对曰:“不弃其亲,其有焉。”


于是祁奚老矣,闻之,乘驲而见宣子,曰:“《诗》曰:‘惠我无疆,子孙保之。’《书》曰:‘圣有谟勋,明征定保。’夫谋而鲜过,惠训不倦者,叔向有焉,社稷之固也。犹将十世宥之,以劝能者。今壹不免其身,以弃社稷,不亦惑乎?鲧殛而禹兴;伊尹放大甲而相之,卒无怨色;管蔡为戮,周公右王。若之何其以虎也弃社稷?子为善,谁敢不勉,多杀何为?”宣子说,与之乘,以言诸公而免之。不见叔向而归,叔向亦不告免焉而朝。



\chapter*{郑子家告赵宣子}
\addcontentsline{toc}{chapter}{郑子家告赵宣子}
\begin{center}
	\textbf{[春秋战国]左丘明}
\end{center}


晋侯合诸侯于扈,平宋也。


于是晋侯不见郑伯,以为贰于楚也。郑子家使执讯而与之书,以告赵宣子曰:“寡君即位三年,召蔡侯而与之事君。九月,蔡侯入于敝邑以行,敝邑以侯宣多之难,寡君是以不得与蔡侯偕,十一月,克减侯宣多而随蔡侯以朝于执事。十二年六月,归生佐寡君之嫡夷,以请陈侯于楚而朝诸君。十四年七月寡君又朝,以蒇陈事。十五年五月,陈侯自敝邑往朝于君。往年正月,烛之武往朝夷也。八月,寡君又往朝。以陈蔡之密迩于楚,而不敢贰焉,则敝邑之故也。虽敝邑之事君,何以不免?在位之中,一朝于襄,而再见于君,夷与孤之二三臣,相及于绛。虽我小国,则蔑以过之矣。今大国曰:‘尔未逞吾志。’敝邑有亡,无以加焉。古人有言曰:‘畏首畏尾,身其余几?’又曰:‘鹿死不择音。’小国之事大国也,德,则其人也;不德,则其鹿也。铤而走险,急何能择?命之罔极,亦知亡矣。将悉敝赋以待于鯈,唯执事命之。文公二年,朝于齐;四年,为齐侵蔡,亦获成于楚。居大国之间而从于强令,岂有罪也?大国若弗图,无所逃命。”


晋巩朔行成于郑,赵穿、公婿池为质焉。



\chapter*{自祭文}
\addcontentsline{toc}{chapter}{自祭文}
\begin{center}
	\textbf{[晋朝]陶渊明}
\end{center}

岁惟丁卯,律中无射。天寒夜长,风气萧索,鸿雁于征,草木黄落。陶子将辞逆旅之馆,永归于本宅。故人凄其相悲,同祖行于今夕。羞以嘉蔬,荐以清酌。候颜已冥,聆音愈漠。呜呼哀哉!

茫茫大块,悠悠高旻,是生万物,余得为人。自余为人,逢运之贫,箪瓢屡罄,絺绤冬陈。含欢谷汲,行歌负薪,翳翳柴门,事我宵晨,春秋代谢,有务中园,载耘载籽,乃育乃繁。欣以素牍,和以七弦。冬曝其日,夏濯其泉。勤靡余劳,心有常闲。乐天委分,以至百年。

惟此百年,夫人爱之,惧彼无成,愒日惜时。存为世珍,殁亦见思。嗟我独迈,曾是异兹。宠非己荣,涅岂吾缁?捽兀穷庐,酣饮赋诗。识运知命,畴能罔眷。余今斯化,可以无恨。寿涉百龄,身慕肥遁,从老得终,奚所复恋!

寒暑愈迈,亡既异存,外姻晨来,良友宵奔,葬之中野,以安其魂。窅窅我行,萧萧墓门,奢耻宋臣,俭笑王孙,廓兮已灭,慨焉已遐,不封不树,日月遂过。匪贵前誉,孰重后歌?人生实难,死如之何?鸣呼哀哉!


\chapter*{霍光传(节选)}
\addcontentsline{toc}{chapter}{霍光传(节选)}
\begin{center}
	\textbf{[汉朝]班固}
\end{center}


霍光,字子孟,票骑将军去病弟也。父中孺,河东平阳人也,以县吏给事平阳侯家,与侍者卫少儿私通而生去病。中孺吏毕归家,娶妇生光,因绝不相闻。久之,少儿女弟子夫得幸于武帝,立为皇后,去病以皇后姊子贵幸。既壮大,乃自知父为霍中孺,未及求问,会为票骑将军击匈奴,道出河东,河东太守郊迎,负弩矢先驱至平阳传舍,遣吏迎霍中孺。中孺趋入拜谒,将军迎拜,因跪曰:“去病不早自知为大人遗体也。”中孺扶服叩头,曰:“老臣得托命将军,此天力也。”去病大为中孺买田宅奴婢而去。还,复过焉,乃将光西至长安,时年十余岁,任光为郎,稍迁诸曹侍中。去病死后,光为奉车都尉光禄大夫,出则奉车,入侍左右,出入禁闼二十余年,小心谨慎,未尝有过,甚见亲信。征和二年,卫太子为江充所败,而燕王旦、广陵王胥皆多过失。是时上年老,宠姬钩弋赵倢伃有男,上心欲以为嗣,命大臣辅之。察群臣唯光任大重,可属社稷。上乃使黄门画者画周公负成王朝诸侯以赐光。后元二年春,上游五柞宫,病笃,光涕泣问曰:“如有不讳,谁当嗣者?”上曰:“君未谕前画意邪?立少子,君行周公之事。”上以光为大司马大将军,日磾为车骑将军,及太仆上官桀为左将军,搜粟都尉桑弘羊为御史大,皆拜卧内床下,受遗诏辅少主。明日,武帝崩,太子枭尊号,是为孝昭皇帝。帝年八岁,政事一决于光。遗诏封光为博陆侯。


光为人沉静详审,长才七尺三寸,白皙,疏眉目,美须髯。每出入下殿门,止进有常处,郎仆射窃识视之,不失尺寸,其资性端正如此。初辅幼主,政自己出,天下想闻其风采。殿中尝有怪,一夜群臣相惊,光召尚符玺郎郎不肯授光。光欲夺之,郎按剑曰:“臣头可得,玺不可得也!”光甚谊之。明日,诏增此郎秩二等。众庶莫不多光。


光与左将军桀结婚相亲,光长女为桀子安妻,有女年与帝相配,桀因帝姊鄂邑盖主内安女后宫为倢伃,数月立为皇后。父安为票骑将军,封桑乐侯。光时休沐出,桀辄入代光决事。桀父子既尊盛,而德长公主。公主内行不修,近幸河间丁外人。桀、安欲为外人求封,幸依国家故事以列侯尚公主者,光不许。又为外人求光禄大夫,欲令得召见,又不许。长主大以是怨光。而桀、安数为外人求官爵弗能得,亦惭。自先帝时,桀已为九卿,位在光右。及父子并为将军,有椒房中宫之重,皇后亲安女,光乃其外祖,而顾专制朝事,由是与光争权。


燕王旦自以昭帝兄,常怀怨望。及御史大夫桑弘羊建造酒榷盐铁,为国兴利,伐其功,欲为子弟得官,亦怨恨光。于是盖主、上官桀、安及弘羊皆与燕王旦通谋,诈令人为燕王上书,言光出都肄羽林,道上称跸,太官先置;又引苏武前使匈奴,拘留二十年不降,还乃为典属国,而大将军长史敞亡功为搜粟都尉;又擅调益莫府校尉;光专权自恣,疑有非常,臣旦愿归符玺,入宿卫,察奸臣变。候司光出沐日奏之。桀欲从中下其事,桑弘羊当与诸大臣共执退光。书奏,帝不肯下。


明旦,光闻之,止画室中不入。上问:“大将军安在?”左将军桀对曰:“以燕王告其罪,故不敢入。”有诏召大将军。光入,免冠军顿首谢,上曰:“将军冠。朕知是书诈也,将军亡罪。”光曰:“陛下何以知之?”上曰:“将军之广明,都郎属耳。调校尉以来未能十日,燕王何以得知之?且将军为非,不须校尉。”是时帝年十四,尚书左右皆惊,而上书者果亡,捕之甚急。桀等惧,白上:“小事不足遂。”上不听。


后桀党与有谮光者,上辄怒曰:“大将军忠臣,先帝所属以辅朕身,敢有毁者坐之。”自是桀等不敢复言,乃谋令长公主置酒请光,伏兵格杀之,因废帝,迎立燕王为天子。事发觉,光尽诛桀、安、弘羊、外人宗族。燕王、盖主皆自杀。光威震海内。昭帝既冠,遂委任光,迄十三年,百姓充实,四夷宾服。


元平元年,昭帝崩,亡嗣。武帝六男独有广陵王胥在,群臣议所立,咸持广陵王。王本以行失道,先帝所不用。光内不自安。郎有上书言:“周太王废太伯立王季,文王舍伯邑考立武王,唯在所宜,虽废长立少可也。广陵王不可以承宗庙。”言合光意。光以其书视丞相敞等,擢郎为九江太守,即日承皇太后诏,遣行大鸿胪事少府乐成、宗正德、光禄大夫吉、中郎将利汉迎昌邑王贺。


贺者,武帝孙,昌邑哀王子也。既至,即位,行淫乱。光忧懑,独以问所亲故吏大司农田延年。延年曰:“将军为国柱石,审此人不可,何不建白太后,更选贤而立之?”光曰:“今欲如是,于古尝有此否?”延年曰:“伊尹相殷,废太甲以安宗庙,后世称其忠。将军若能行此,亦汉之伊尹也。”光乃引延年给事中,阴与车骑将军张安世图计,遂召丞相、御史、将军、列侯、中二千石、大夫、博士会议未央宫。光曰:“昌邑王行昏乱,恐危社稷,如何?”群臣皆惊鄂失色,莫敢发言,但唯唯而已。田延年前,离席按剑,曰:“先帝属将军以幼孤,寄将军以天下,以将军忠贤能安刘氏也。今群下鼎沸,社稷将倾,且汉之传谥常为孝者,以长有天下,令宗庙血食也。如令汉家绝祀,将军虽死,何面目见先帝于地下乎?今日之议,不得旋踵。群臣后应者,臣请剑斩之。”光谢曰:“九卿责光是也。天下匈匈不安,光当受难。”于是议者皆叩头,曰:“万姓之命在于将军,唯大将军令。”


光即与群臣俱见白太后,具陈昌邑王不可以承宗庙状。皇太后乃车驾幸未央承明殿,诏诸禁门毋内昌邑群臣。王入朝太后还,乘辇欲归温室,中黄门宦者各持门扇,王入,门闭,昌邑群臣不得入。王曰:“何为?”大将军跪曰:“有皇太后诏,毋内昌邑群臣。”王曰:“徐之,何乃惊人如是!”光使尽驱出昌邑群臣,置金马门外。车骑将军安世将羽林骑收缚二百余人,皆送廷尉诏狱。令故昭帝侍中中臣侍守王。光敕左右:“谨宿卫,卒有物故自裁,令我负天下,有杀主名。”王尚未自知当废,谓左右:“我故群臣从官安得罪,而大将军尽系之乎?”顷之,有太后诏召王。王闻召,意恐,乃曰:“我安得罪而召我哉!”太后被珠襦,盛服坐武帐中,侍御数百人皆持兵,期门武士陛戟,陈列殿下。群臣以次上殿,召昌邑王伏前听诏。光与群臣连名奏王,……荒淫迷惑,失帝王礼谊,乱汉制度,……当废。……皇太后诏曰:“可。”光令王起拜受诏,王曰:“闻天子有争臣七人,虽无道不失天下。”光曰:“皇太后诏废,安得天子!”乃即持其手,解脱其玺组,奉上太后,扶王下殿,出金马门,群臣随送。王西面拜,曰:“愚戆不任汉事。”起就乘舆副车。大将军光送至昌邑邸,光谢曰:“王行自绝于天,臣等驽怯,不能杀身报德。臣宁负王,不敢负社稷。愿王自爱,臣长不复见左右。”光涕泣而去。群臣奏言:“古者废放之人屏于远方,不及以政,请徙王贺汉中房陵县。”太后诏归贺昌邑,赐汤沐邑二千户。昌邑群臣坐亡辅导之谊,陷王于恶,光悉诛杀二百余人。出死,号呼市中曰:“当断不断,反受其乱。”


光坐庭中,会丞相以下议定所立。广陵王已前不用,及燕刺王反诛,其子不在议中。近亲唯有卫太子孙号皇曾孙在民间,咸称述焉。光遂与丞相敞等上奏曰:“《礼》曰:‘人道亲亲故尊祖,尊祖故敬宗。’大宗亡嗣,择支子孙贤者为嗣。孝武皇帝曾孙病已,武帝时有诏掖庭养视,至今年十八,师受《诗》、《论语》、《孝经》,躬行节俭,慈仁爱人,可以嗣孝昭皇帝后,奉承祖宗庙,子万姓。臣昧死以闻。”皇太后诏曰:“可。”光遣宗正刘德至曾孙家尚冠里,洗沐赐御衣,太仆以軨车迎曾孙就斋宗正府,入未央宫见皇太后,封为阳武侯。而光奉上皇帝玺绶,谒于高庙,是为孝宣皇帝。


明年,下诏曰:“夫褒有德,赏元功,古今通谊也。大司马大将军光宿卫忠正,宣德明恩,守节秉谊,以安宗庙。其以河北、东武阳益封光万七千户。”与故所食凡二万户。赏赐前后黄金七千斤,钱六千万,杂缯三万匹,奴婢百七十人,马二千匹,甲第一区。


自昭帝时,光子禹及兄孙云皆中郎将,云弟山奉车都尉侍中,领胡越兵。光两女婿为东西宫卫尉,昆弟、诸婿、外孙皆奉朝请,为诸曹大夫,骑都尉、给事中。党亲连体,根据于朝廷。光自后元秉持万机,及上即位,乃归政。上谦让不受,诸事皆先关白光,然后奏御天子。光每朝见,上虚己敛容,礼下之已甚。


光秉政前后二十年。地节二年春病笃,车驾自临问光病,上为之涕泣。光上书谢恩曰:“愿分国邑三千户,以封兄孙奉车都尉山为列侯,奉兄骠骑将军去病祀。”事下丞相御史,即日拜光子禹为右将军。


光薨,上及皇太后亲临光丧。太中大夫任宣与侍御史五人持节护丧事。中二千石治莫府冢上。赐金钱、缯絮、绣被百领,衣五十箧,璧珠玑玉衣,梓宫、便房、黄肠题凑各一具,枞木外臧椁十五具。东园温明,皆如乘舆制度。载光尸柩以辒辌车,黄屋在纛,发材官轻车北军五校士军陈至茂陵,以送其葬。谥曰宣成侯。发三河卒穿复士,起冢祠堂。置园邑三百家,长丞奉守如旧法。


初,霍氏指西汉权臣霍光子孙奢侈,茂陵徐生曰:“霍氏必亡。夫奢则不逊,不逊必侮上;侮上者,逆道也。在人之右,众必害之。霍氏秉权日久,害之者多矣。天下害之,而又行以逆道,不亡何待!”乃上疏,言:“霍氏泰盛;陛下即爱厚之,宜以时抑制,无使至亡。”书三上,辄报闻。


其后,霍氏诛灭,而告霍氏者皆封。人为徐生上书曰:“臣闻客有过主人者,见其灶直突注:突,烟囱,傍有积薪。客谓主人:‘更为曲突,远徙其薪;不者,且有火患。’主人嘿然不应。俄而家果失火,邻里共救之,幸而得息。于是杀牛置酒,谢其邻人。灼烂者在于上行,余各以功次座,而不录言曲突者。人谓主人曰:‘乡使听客之言,不费牛酒,终亡火患。今论功而请宾,曲突徙薪无恩泽,焦头烂额为上客耶?’主人乃寤而请之。今茂陵徐福数上书言霍氏且有变,宜防绝之。乡使福说得行,则国亡裂土出爵之费,臣亡逆乱诛灭之败。往事既已,而福独不蒙其功。唯陛下察之——贵徙薪曲突之策,使居焦发灼烂之右。”上乃赐福帛十匹,后以为郎。


宣帝始立,谒见高庙,大将军霍光从骖乘,上内严惮之,若有芒刺在背。后车骑将军张安世代光骖乘,天子从容肆体,甚安近焉。及光身死。而宗族竟诛。故俗传之曰:“威震主者不畜。霍氏之祸,萌于骖乘。”


赞曰:霍光以结发内侍,起于阶闼之间,确然秉志,谊形于主。受襁褓之托,任汉室之寄,当庙堂,拥幼君,摧燕王,仆上官,因权制敌,以成其忠。处废置之际,临大节而不可夺,遂匡国家,安社稷。拥昭立宣,光为师保,虽周公、阿衡,何以加此!然光不学亡术,暗于大理,阴妻邪谋,立女为后,湛溺盈溢之欲,以增颠覆之祸,死财三年,宗族诛夷,哀哉!昔霍叔封于晋,晋即河东,光岂其苗裔乎?金日磾夷狄亡国,羁虏汉庭,而以笃敬寤主,忠信自著,勒功上将,传国后嗣,世名忠孝,七世内侍,何其盛也!本以休屠作金人为祭天主,故因赐姓金氏云。



\chapter*{愚溪诗序}
\addcontentsline{toc}{chapter}{愚溪诗序}
\begin{center}
	\textbf{[唐朝]柳宗元}
\end{center}

灌水之阳有溪焉,东流入于潇水。或曰:冉氏尝居也,故姓是溪为冉溪。或曰:可以染也,名之以其能,故谓之染溪。予以愚触罪,谪潇水上。爱是溪,入二三里,得其尤绝者家焉。古有愚公谷,今予家是溪,而名莫能定,士之居者,犹龂龂然,不可以不更也,故更之为愚溪。

愚溪之上,买小丘,为愚丘。自愚丘东北行六十步,得泉焉,又买居之,为愚泉。愚泉凡六穴,皆出山下平地,盖上出也。合流屈曲而南,为愚沟。遂负土累石,塞其隘,为愚池。愚池之东为愚堂。其南为愚亭。池之中为愚岛。嘉木异石错置,皆山水之奇者,以予故,咸以愚辱焉。

夫水,智者乐也。今是溪独见辱于愚,何哉?盖其流甚下,不可以溉灌。又峻急多坻石,大舟不可入也。幽邃浅狭,蛟龙不屑,不能兴云雨,无以利世,而适类于予,然则虽辱而愚之,可也。

宁武子“邦无道则愚”,智而为愚者也;颜子“终日不违如愚”,睿而为愚者也。皆不得为真愚。今予遭有道而违于理,悖于事,故凡为愚者,莫我若也。夫然,则天下莫能争是溪,予得专而名焉。

溪虽莫利于世,而善鉴万类,清莹秀澈,锵鸣金石,能使愚者喜笑眷慕,乐而不能去也。予虽不合于俗,亦颇以文墨自慰,漱涤万物,牢笼百态,而无所避之。以愚辞歌愚溪,则茫然而不违,昏然而同归,超鸿蒙,混希夷,寂寥而莫我知也。于是作《八愚诗》,纪于溪石上。


\chapter*{待漏院记}
\addcontentsline{toc}{chapter}{待漏院记}
\begin{center}
	\textbf{[宋朝]王禹偁}
\end{center}

天道不言,而品物亨、岁功成者,何谓也?四时之吏,五行之佐,宣其气矣。圣人不言而百姓亲、万邦宁者,何谓也?三公论道,六卿分职,张其教矣。是知君逸于上,臣劳于下,法乎天也。古之善相天下者,自咎、夔至房、魏,可数也,是不独有其德,亦皆务于勤耳,况夙兴夜寐,以事一人。卿大夫犹然,况宰相乎!朝廷自国初因旧制,设宰臣待漏院于丹凤门之右,示勤政也。至若北阙向曙,东方未明,相君启行,煌煌火城;相君至止,哕哕銮声。金门未辟,玉漏犹滴,彻盖下车,于焉以息。待漏之际,相君其有思乎?

其或兆民未安,思所泰之;四夷未附,思所来之。兵革未息,何以弭之;田畴多芜,何以辟之。贤人在野,我将进之;佞臣立朝,我将斥之。六气不和,灾眚荐至,愿避位以禳之;五刑未措,欺诈日生,请修德以厘之。忧心忡忡,待旦而入,九门既启,四聪甚迩。相君言焉,时君纳焉。皇风于是乎清夷,苍生以之而富庶。若然,总百官、食万钱,非幸也,宜也。

其或私仇未复,思所逐之;旧恩未报,思所荣之。子女玉帛,何以致之;车马器玩,何以取之。奸人附势,我将陟之;直士抗言,我将黜之。三时告灾,上有忧也,构巧词以悦之;群吏弄法,君闻怨言,进谄容以媚之。私心慆慆,假寐而坐,九门既开,重瞳屡回。相君言焉,时君惑焉。政柄于是乎隳哉,帝位以之而危矣。若然,则下死狱、投远方,非不幸也,亦宜也。

是知一国之政,万人之命,悬于宰相,可不慎欤?复有无毁无誉,旅进旅退,窃位而苟禄,备员而全身者,亦无所取焉。

棘寺小吏王某为文,请志院壁,用规于执政者。


\chapter*{蹇叔哭师}
\addcontentsline{toc}{chapter}{蹇叔哭师}
\begin{center}
	\textbf{[春秋战国]左丘明}
\end{center}


冬,晋文公卒。庚辰,将殡于曲沃。出绛,柩有声如牛。卜偃使大夫拜,曰:“君命大事将有西师过轶我,击之,必大捷焉。”


杞子自郑使告于秦曰:“郑人使我掌其北门之管,若潜师以来,国可得也。”穆公访诸蹇叔。蹇叔曰:“劳师以袭远,非所闻也。师劳力竭,远主备之,无乃不可乎?师之所为,郑必知之。勤而无所,必有悖心。且行千里,其谁不知?”公辞焉。召孟明、西乞、白乙使出师于东门之外。蹇叔哭之曰:“孟子!吾见师之出而不见其入也。”公使谓之曰:“尔何知!中寿,尔墓之木拱矣!”


蹇叔之子与师,哭而送之,曰:“晋人御师必于崤,有二陵焉。其南陵,夏后皋之墓地;其北陵,文王之所辟风雨也,必死是间,余收尔骨焉?秦师遂东。



\chapter*{介之推不言禄}
\addcontentsline{toc}{chapter}{介之推不言禄}
\begin{center}
	\textbf{[春秋战国]左丘明}
\end{center}


晋侯赏从亡者,介之推不言禄,禄亦弗及。


推曰:“献公之子九人,唯君在矣。惠、怀无亲,外内弃之。天未绝晋,必将有主。主晋祀者,非君而谁?天实置之,而二三子以为己力,不亦诬乎?窃人之财,犹谓之盗。况贪天之功,以为己力乎?下义其罪,上赏其奸。上下相蒙,难与处矣。”


其母曰:“盍亦求之?以死谁怼?”


对曰:“尤而效之,罪又甚焉!且出怨言,不食其食。”


其母曰:“亦使知之,若何?”


对曰:“言,身之文也。身将隐,焉用文之?是求显也。”


其母曰:“能如是乎?与汝偕隐。”遂隐而死。


晋侯求之,不获,以绵上(地名)为之田。曰:“以志吾过,且旌善人。”



\chapter*{河中石兽}
\addcontentsline{toc}{chapter}{河中石兽}
\begin{center}
	\textbf{[清朝]纪昀}
\end{center}


沧州南一寺临河干,山门圮于河,二石兽并沉焉。阅十余岁,僧募金重修,求石兽于水中,竟不可得。以为顺流下矣,棹数小舟,曳铁钯,寻十余里无迹。


一讲学家设帐寺中,闻之笑曰:“尔辈不能究物理,是非木杮,岂能为暴涨携之去?乃石性坚重,沙性松浮,湮于沙上,渐沉渐深耳。沿河求之,不亦颠乎?”众服为确论。


一老河兵闻之,又笑曰:“凡河中失石,当求之于上流。盖石性坚重,沙性松浮,水不能冲石,其反激之力,必于石下迎水处啮沙为坎穴,渐激渐深,至石之半,石必倒掷坎穴中。如是再啮,石又再转。转转不已,遂反溯流逆上矣。求之下流,固颠;求之地中,不更颠乎?”如其言,果得于数里外。然则天下之事,但知其一,不知其二者多矣,可据理臆断欤?(转转一作:再转)


\chapter*{永州韦使君新堂记}
\addcontentsline{toc}{chapter}{永州韦使君新堂记}
\begin{center}
	\textbf{[唐朝]柳宗元}
\end{center}


将为穹谷嵁岩渊池于郊邑之中,则必辇山石,沟涧壑,陵绝险阻,疲极人力,乃可以有为也。然而求天作地生之状,咸无得焉。逸其人,因其地,全其天,昔之所难,今于是乎在。


永州实惟九疑之麓。其始度土者,环山为城。有石焉,翳于奥草;有泉焉,伏于土涂。蛇虺之所蟠,狸鼠之所游。茂树恶木,嘉葩毒卉,乱杂而争植,号为秽墟。


韦公之来,既逾月,理甚无事。望其地,且异之。始命芟其芜,行其涂。积之丘如,蠲之浏如。既焚既酾,奇势迭出。清浊辨质,美恶异位。视其植,则清秀敷舒;视其蓄,则溶漾纡余。怪石森然,周于四隅。或列或跪,或立或仆,窍穴逶邃,堆阜突怒。乃作栋宇,以为观游。凡其物类,无不合形辅势,效伎于堂庑之下。外之连山高原,林麓之崖,间厕隐显。迩延野绿,远混天碧,咸会于谯门之内。


已乃延客入观,继以宴娱。或赞且贺曰:“见公之作,知公之志。公之因土而得胜,岂不欲因俗以成化?公之择恶而取美,岂不欲除残而佑仁?公之蠲浊而流清,岂不欲废贪而立廉?公之居高以望远,岂不欲家抚而户晓?夫然,则是堂也,岂独草木土石水泉之适欤?山原林麓之观欤?将使继公之理者,视其细知其大也。”宗元请志诸石,措诸壁,编以为二千石楷法。



\chapter*{宋玉对楚王问}
\addcontentsline{toc}{chapter}{宋玉对楚王问}
\begin{center}
	\textbf{[春秋战国]宋玉}
\end{center}


楚襄王问于宋玉曰:“先生其有遗行与?何士民众庶不誉之甚也!”


宋玉对曰:“唯,然,有之!愿大王宽其罪,使得毕其辞。客有歌于郢中者,其始曰《下里》、《巴人》,国中属而和者数千人。其为《阳阿》、《薤露》,国中属而和者数百人。其为《阳春》、《白雪》,国中有属而和者,不过数十人。引商刻羽,杂以流徵,国中属而和者,不过数人而已。是其曲弥高,其和弥寡。


故鸟有凤而鱼有鲲。凤皇上击九千里,绝云霓,负苍天,足乱浮云,翱翔乎杳冥之上。夫蕃篱之鷃,岂能与之料天地之高哉?鲲鱼朝发昆仑之墟,暴鬐于碣石,暮宿于孟诸。夫尺泽之鲵,岂能与之量江海之大哉?故非独鸟有凤而鱼有鲲,士亦有之。夫圣人瑰意琦行,超然独处,世俗之民,又安知臣之所为哉?”



\chapter*{送石处士序}
\addcontentsline{toc}{chapter}{送石处士序}
\begin{center}
	\textbf{[唐朝]韩愈}
\end{center}

河阳军节度、御史大夫乌公,为节度之三月,求士于从事之贤者。有荐石先生者。公曰:“先生何如?”曰:“先生居嵩、邙、瀍、谷之间,冬一裘,夏一葛,食朝夕,饭一盂,蔬一盘。人与之钱,则辞;请与出游,未尝以事免;劝之仕,不应。坐一室,左右图书。与之语道理,辨古今事当否,论人高下,事后当成败,若河决下流而东注;若驷马驾轻车就熟路,而王良、造父为之先后也;若烛照、数计而龟卜也。”大夫曰:“先生有以自老,无求于人,其肯为某来邪?”从事曰:“大夫文武忠孝,求士为国,不私于家。方今寇聚于恒,师还其疆,农不耕收,财粟殚亡。吾所处地,归输之涂,治法征谋,宜有所出。先生仁且勇。若以义请而强委重焉,其何说之辞?”于是撰书词,具马币,卜日以受使者,求先生之庐而请焉。

先生不告于妻子,不谋于朋友,冠带出见客,拜受书礼于门内。宵则沫浴,戒行李,载书册,问道所由,告行于常所来往。晨则毕至,张上东门外。酒三行,且起,有执爵而言者曰:“大夫真能以义取人,先生真能以道自任,决去就。为先生别。”又酌而祝曰:“凡去就出处何常,惟义之归。遂以为先生寿。”又酌而祝曰:“使大夫恒无变其初,无务富其家而饥其师,无甘受佞人而外敬正士,无昧于谄言,惟先生是听,以能有成功,保天子之宠命。”又祝曰:“使先生无图利于大夫而私便其身。”先生起拜祝辞曰:“敢不敬蚤夜以求从祝规。”于是东都之人士咸知大夫与先生果能相与以有成也。遂各为歌诗六韵,遣愈为之序云。


\chapter*{后廿九日复上宰相书}
\addcontentsline{toc}{chapter}{后廿九日复上宰相书}
\begin{center}
	\textbf{[唐朝]韩愈}
\end{center}

三月十六日,前乡贡进士韩愈,谨再拜言相公阁下。

愈闻周公之为辅相,其急于见贤也,方一食三吐其哺,方一沐三握其发。天下之贤才皆已举用,奸邪谗佞欺负之徒皆已除去,四海皆已无虞,九夷八蛮之在荒服之外者皆已宾贡,天灾时变、昆虫草木之妖皆已销息,天下之所谓礼、乐、刑、政教化之具皆已修理,风俗皆已敦厚,动植之物、风雨霜露之所沾被者皆已得宜,休征嘉瑞、麟凤龟龙之属皆已备至,而周公以圣人之才,凭叔父之亲,其所辅理承化之功又尽章章如是。其所求进见之士,岂复有贤于周公者哉?不惟不贤于周公而已,岂复有贤于时百执事者哉?岂复有所计议、能补于周公之化者哉?然而周公求之如此其急,惟恐耳目有所不闻见,思虑有所未及,以负成王托周公之意,不得于天下之心。如周公之心,设使其时辅理承化之功未尽章章如是,而非圣人之才,而无叔父之亲,则将不暇食与沐矣,岂特吐哺握发为勤而止哉?维其如是,故于今颂成王之德,而称周公之功不衰。

今阁下为辅相亦近耳。天下之贤才岂尽举用?奸邪谗佞欺负之徒岂尽除去?四海岂尽无虞?九夷、八蛮之在荒服之外者岂尽宾贡?天灾时变、昆虫草木之妖岂尽销息?天下之所谓礼、乐、刑、政教化之具岂尽修理?风俗岂尽敦厚?动植之物、风雨霜露之所沾被者岂尽得宜?休征嘉瑞、麟凤龟龙之属岂尽备至?其所求进见之士,虽不足以希望盛德,至比于百执事,岂尽出其下哉?其所称说,岂尽无所补哉?今虽不能如周公吐哺握发,亦宜引而进之,察其所以而去就之,不宜默默而已也。

愈之待命,四十馀日矣。书再上,而志不得通。足三及门,而阍人辞焉。惟其昏愚,不知逃遁,故复有周公之说焉。阁下其亦察之。古之士三月不仕则相吊,故出疆必载质。然所以重于自进者,以其于周不可则去之鲁,于鲁不可则去之齐,于齐不可则去之宋,之郑,之秦,之楚也。今天下一君,四海一国,舍乎此则夷狄矣,去父母之邦矣。故士之行道者,不得于朝,则山林而已矣。山林者,士之所独善自养,而不忧天下者之所能安也。如有忧天下之心,则不能矣。故愈每自进而不知愧焉,书亟上,足数及门,而不知止焉。宁独如此而已,惴惴焉惟,不得出大贤之门下是惧。亦惟少垂察焉。渎冒威尊,惶恐无已。愈再拜。


\chapter*{送区册序}
\addcontentsline{toc}{chapter}{送区册序}
\begin{center}
	\textbf{[唐朝]韩愈}
\end{center}

阳山,天下之穷处也。陆有丘陵之险,虎豹之虞。江流悍急,横波之石,廉利侔剑戟,舟上下失势,破碎沦溺者,往往有之。县廓无居民,官无丞尉,夹江荒茅篁竹之间,小吏十余家,皆鸟言夷面。始至,言语不通,画地为字,然后可告以出租赋,奉期约。是以宾客游从之士,无所为而至。愈待罪于斯,且半岁矣。

有区生者,誓言相好,自南海挐舟而来。升自宾阶,仪观甚伟,坐与之语,文义卓然。庄周云:“逃空虚者,闻人足音跫然而喜矣!”况如斯人者,岂易得哉!入吾室,闻《诗》、《书》仁义之说,欣然喜,若有志于其间也。与之翳嘉林,坐石矶,投竿而渔,陶然以乐,若能遗外声利,而不厌乎贫贱也。岁之初吉,归拜其亲,酒壶既倾,序以识别。


\chapter*{触龙说赵太后}
\addcontentsline{toc}{chapter}{触龙说赵太后}
\begin{center}
	\textbf{[汉朝]刘向}
\end{center}


赵太后新用事,秦急攻之。赵氏求救于齐,齐曰:“必以长安君为质,兵乃出。”太后不肯,大臣强谏。太后明谓左右:“有复言令长安君为质者,老妇必唾其面。”


左师触龙言愿见太后。太后盛气而揖之。入而徐趋,至而自谢,曰:“老臣病足,曾不能疾走,不得见久矣。窃自恕,而恐太后玉体之有所郄也,故愿望见太后。”太后曰:“老妇恃辇而行。”曰:“日食饮得无衰乎?”曰:“恃粥耳。”曰:“老臣今者殊不欲食,乃自强步,日三四里,少益耆食,和于身。”太后曰:“老妇不能。”太后之色少解。


左师公曰:“老臣贱息舒祺,最少,不肖;而臣衰,窃爱怜之。愿令得补黑衣之数,以卫王宫。没死以闻。”太后曰:“敬诺。年几何矣?”对曰:“十五岁矣。虽少,愿及未填沟壑而托之。”太后曰:“丈夫亦爱怜其少子乎?”对曰:“甚于妇人。”太后笑曰:“妇人异甚。”对曰:“老臣窃以为媪之爱燕后贤于长安君。”曰:“君过矣!不若长安君之甚。”左师公曰:“父母之爱子,则为之计深远。媪之送燕后也,持其踵,为之泣,念悲其远也,亦哀之矣。已行,非弗思也,祭祀必祝之,祝曰:‘必勿使反。’岂非计久长,有子孙相继为王也哉?”太后曰:“然。”


左师公曰:“今三世以前,至于赵之为赵,赵王之子孙侯者,其继有在者乎?”曰:“无有。”曰:“微独赵,诸侯有在者乎?”曰:“老妇不闻也。”“此其近者祸及身,远者及其子孙。岂人主之子孙则必不善哉?位尊而无功,奉厚而无劳,而挟重器多也。今媪尊长安君之位,而封之以膏腴之地,多予之重器,而不及今令有功于国,—旦山陵崩,长安君何以自托于赵?老臣以媪为长安君计短也,故以为其爱不若燕后。”太后曰:“诺,恣君之所使之。”


于是为长安君约车百乘,质于齐,齐兵乃出。


子义闻之曰:“人主之子也、骨肉之亲也,犹不能恃无功之尊、无劳之奉,已守金玉之重也,而况人臣乎。”



\chapter*{乐羊子妻}
\addcontentsline{toc}{chapter}{乐羊子妻}
\begin{center}
	\textbf{[南北朝]范晔}
\end{center}


河南乐羊子之妻者,不知何氏之女也。


羊子尝行路,得遗金一饼,还以与妻。妻曰:“妾闻志士不饮‘盗泉’之水,廉者不受嗟来之食,况拾遗求利,以污其行乎!”羊子大惭,乃捐金于野,而远寻师学。


一年来归,妻跪问其故,羊子曰:“久行怀思,无它异也。”妻乃引刀趋机而言曰:“此织生自蚕茧,成于机杼。一丝而累,以至于寸,累寸不已,遂成丈匹。今若断斯织也,则捐失成功,稽废时日。夫子积学,当‘日知其所亡’,以就懿德;若中道而归,何异断斯织乎?”羊子感其言,复还终业,遂七年不返。


尝有它舍鸡谬入园中,姑盗杀而食之,妻对鸡不餐而泣。姑怪问其故。妻曰:“自伤居贫,使食有它肉。”姑竟弃之。后盗欲有犯妻者,乃先劫其姑。妻闻,操刀而出。盗人曰:“释汝刀从我者可全,不从我者,则杀汝姑。”妻仰天而叹,举刀刎颈而死。盗亦不杀其姑。太守闻之,即捕杀贼盗,而赐妻缣帛,以礼葬之,号曰“贞义”。



\chapter*{过秦论}
\addcontentsline{toc}{chapter}{过秦论}
\begin{center}
	\textbf{[汉朝]贾谊}
\end{center}


上篇

秦孝公据崤函之固,拥雍州之地,君臣固守以窥周室,有席卷天下,包举宇内,囊括四海之意,并吞八荒之心。当是时也,商君佐之,内立法度,务耕织,修守战之具;外连衡而斗诸侯。于是秦人拱手而取西河之外。

孝公既没,惠文、武、昭襄蒙故业,因遗策,南取汉中,西举巴、蜀,东割膏腴之地,北收要害之郡。诸侯恐惧,会盟而谋弱秦,不爱珍器重宝肥饶之地,以致天下之士,合从缔交,相与为一。当此之时,齐有孟尝,赵有平原,楚有春申,魏有信陵。此四君者,皆明智而忠信,宽厚而爱人,尊贤而重士,约从离衡,兼韩、魏、燕、楚、齐、赵、宋、卫、中山之众。于是六国之士,有宁越、徐尚、苏秦、杜赫之属为之谋,齐明、周最、陈轸、召滑、楼缓、翟景、苏厉、乐毅之徒通其意,吴起、孙膑、带佗、倪良、王廖、田忌、廉颇、赵奢之伦制其兵。尝以十倍之地,百万之众,叩关而攻秦。秦人开关延敌,九国之师,逡巡而不敢进。秦无亡矢遗镞之费,而天下诸侯已困矣。于是从散约败,争割地而赂秦。秦有余力而制其弊,追亡逐北,伏尸百万,流血漂橹。因利乘便,宰割天下,分裂山河。强国请服,弱国入朝。延及孝文王、庄襄王,享国之日浅,国家无事。

及至始皇,奋六世之余烈,振长策而御宇内,吞二周而亡诸侯,履至尊而制六合,执敲扑而鞭笞天下,威振四海。南取百越之地,以为桂林、象郡;百越之君,俯首系颈,委命下吏。乃使蒙恬北筑长城而守藩篱,却匈奴七百余里。胡人不敢南下而牧马,士不敢弯弓而报怨。于是废先王之道,焚百家之言,以愚黔首;隳名城,杀豪杰,收天下之兵,聚之咸阳,销锋镝,铸以为金人十二,以弱天下之民。然后践华为城,因河为池,据亿丈之城,临不测之渊,以为固。良将劲弩守要害之处,信臣精卒陈利兵而谁何。天下已定,始皇之心,自以为关中之固,金城千里,子孙帝王万世之业也。

始皇既没,余威震于殊俗。然陈涉瓮牖绳枢之子,氓隶之人,而迁徙之徒也;才能不及中人,非有仲尼、墨翟之贤,陶朱、猗顿之富;蹑足行伍之间,而倔起阡陌之中,率疲弊之卒,将数百之众,转而攻秦,斩木为兵,揭竿为旗,天下云集响应,赢粮而景从。山东豪俊遂并起而亡秦族矣。

且夫天下非小弱也,雍州之地,崤函之固,自若也。陈涉之位,非尊于齐、楚、燕、赵、韩、魏、宋、卫、中山之君也;锄耰棘矜,非铦于钩戟长铩也;谪戍之众,非抗于九国之师也;深谋远虑,行军用兵之道,非及向时之士也。然而成败异变,功业相反,何也?试使山东之国与陈涉度长絜大,比权量力,则不可同年而语矣。然秦以区区之地,致万乘之势,序八州而朝同列,百有余年矣;然后以六合为家,崤函为宫;一夫作难而七庙隳,身死人手,为天下笑者,何也?仁义不施而攻守之势异也。


中篇

秦灭周祀,并海内,兼诸侯,南面称帝,以养四海。天下之士,斐然向风。若是,何也?曰:近古之无王者久矣。周室卑微,五霸既灭,令不行于天下。是以诸侯力政,强凌弱,众暴寡,兵革不休,士民罢弊。今秦南面而王天下,是上有天子也。既元元之民冀得安其性命,莫不虚心而仰上。当此之时,专威定功,安危之本,在于此矣。

秦王怀贪鄙之心,行自奋之智,不信功臣,不亲士民,废王道而立私爱,焚文书而酷刑法,先诈力而后仁义,以暴虐为天下始。夫兼并者高诈力,安危者贵顺权,此言取与守不同术也。秦离战国而王天下,其道不易,其政不改,是其所以取之守之者无异也。孤独而有之,故其亡可立而待也。借使秦王论上世之事,并殷、周之迹,以制御其政,后虽有淫骄之主,犹未有倾危之患也。故三王之建天下,名号显美,功业长久。

今秦二世立,天下莫不引领而观其政。夫寒者利裋褐,而饥者甘糟糠。天下嚣嚣,新主之资也。此言劳民之易为仁也。向使二世有庸主之行而任忠贤,臣主一心而忧海内之患,缟素而正先帝之过;裂地分民以封功臣之后,建国立君以礼天下;虚囹圄而免刑戮,去收孥污秽之罪,使各反其乡里;发仓廪,散财币,以振孤独穷困之士;轻赋少事,以佐百姓之急;约法省刑,以持其后,使天下之人皆得自新,更节修行,各慎其身;塞万民之望,而以盛德与天下,天下息矣。即四海之内皆欢然各自安乐其处,惟恐有变。虽有狡害之民,无离上之心,则不轨之臣无以饰其智,而暴乱之奸弭矣。

二世不行此术,而重以无道:坏宗庙与民,更始作阿房之宫;繁刑严诛,吏治刻深;赏罚不当,赋敛无度。天下多事,吏不能纪;百姓困穷,而主不收恤。然后奸伪并起,而上下相遁;蒙罪者众,刑戮相望于道,而天下苦之。自群卿以下至于众庶,人怀自危之心,亲处穷苦之实,咸不安其位,故易动也。是以陈涉不用汤、武之贤,不借公侯之尊,奋臂于大泽,而天下响应者,其民危也。

故先王者,见终始不变,知存亡之由。是以牧民之道,务在安之而已矣。下虽有逆行之臣,必无响应之助。故曰:“安民可与为义,而危民易与为非”,此之谓也。贵为天子,富有四海,身在于戮者,正之非也。是二世之过也。


下篇

秦兼诸侯山东三十余郡,脩津关,据险塞,缮甲兵而守之。然陈涉率散乱之众数百,奋臂大呼,不用弓戟之兵,鉏耰白梃,望屋而食,横行天下。秦人阻险不守,关梁不闭,长戟不刺,强弩不射。楚师深入,战于鸿门,曾无藩篱之难。于是山东诸侯并起,豪俊相立。秦使章邯将而东征,章邯因其三军之众,要市于外,以谋其上。群臣之不相信,可见于此矣。子婴立,遂不悟。借使子婴有庸主之材而仅得中佐,山东虽乱,三秦之地可全而有,宗庙之祀宜未绝也。

秦地被山带河以为固,四塞之国也。自缪公以来至于秦王二十余君,常为诸侯雄。此岂世贤哉?其势居然也。且天下尝同心并力攻秦矣,然困于险阻而不能进者,岂勇力智慧不足哉?形不利、势不便也。秦虽小邑,伐并大城,得阨塞而守之。诸侯起于匹夫,以利会,非有素王之行也。其交未亲,其民未附,名曰亡秦,其实利之也。彼见秦阻之难犯,必退师。案土息民以待其弊,收弱扶罢以令大国之君,不患不得意于海内。贵为天子,富有四海,而身为禽者,救败非也。

秦王足己而不问,遂过而不变。二世受之,因而不改,暴虐以重祸。子婴孤立无亲,危弱无辅。三主之惑,终身不悟,亡不亦宜乎?当此时也,也非无深谋远虑知化之士也,然所以不敢尽忠指过者,秦俗多忌讳之禁也,——忠言未卒于口而身糜没矣。故使天下之士倾耳而听,重足而立,阖口而不言。是以三主失道,而忠臣不谏,智士不谋也。天下已乱,奸不上闻,岂不悲哉!先王知壅蔽之伤国也,故置公卿、大夫、士,以饰法设刑而天下治。其强也,禁暴诛乱而天下服;其弱也,王霸征而诸侯从;其削也,内守外附而社稷存。故秦之盛也,繁法严刑而天下震;及其衰也,百姓怨而海内叛矣。故周王序得其道,千余载不绝;秦本末并失,故不能长。由是观之,安危之统相去远矣。

鄙谚曰:“前事之不忘,后事之师也。”是以君子为国,观之上古,验之当世,参之人事,察盛衰之理,审权势之宜,去就有序,变化因时,故旷日长久而社稷安矣。



\chapter*{大铁椎传}
\addcontentsline{toc}{chapter}{大铁椎传}
\begin{center}
	\textbf{[明朝]魏禧}
\end{center}

庚戌十一月,予自广陵归,与陈子灿同舟。子灿年二十八,好武事,予授以左氏兵谋兵法,因问:“数游南北,逢异人乎?”子灿为述大铁椎,作《大铁椎传》。

大铁椎,不知何许人,北平陈子灿省兄河南,与遇宋将军家。宋,怀庆青华镇人,工技击,七省好事者皆来学,人以其雄健,呼宋将军云。宋弟子高信之,亦怀庆人,多力善射,长子灿七岁,少同学,故尝与过宋将军。

时座上有健啖客,貌甚寝,右胁夹大铁椎,重四五十斤,饮食拱揖不暂去。柄铁折叠环复,如锁上练,引之长丈许。与人罕言语,语类楚声。扣其乡及姓字,皆不答。

既同寝,夜半,客曰:“吾去矣!”言讫不见。子灿见窗户皆闭,惊问信之。信之曰:“客初至,不冠不袜,以蓝手巾裹头,足缠白布,大铁椎外,一物无所持,而腰多白金。吾与将军俱不敢问也。”子灿寐而醒,客则鼾睡炕上矣。

一日,辞宋将军曰:“吾始闻汝名,以为豪,然皆不足用。吾去矣!”将军强留之,乃曰:“吾数击杀响马贼,夺其物,故仇我。久居,祸且及汝。今夜半,方期我决斗某所。”宋将军欣然曰:“吾骑马挟矢以助战。”客曰:“止!贼能且众,吾欲护汝,则不快吾意。”宋将军故自负,且欲观客所为,力请客。客不得已,与偕行。将至斗处,送将军登空堡上,曰:“但观之,慎弗声,令贼知也。”

时鸡鸣月落,星光照旷野,百步见人。客驰下,吹觱篥数声。顷之,贼二十余骑四面集,步行负弓矢从者百许人。一贼提刀突奔客,客大呼挥椎,贼应声落马,马首裂。众贼环而进,客奋椎左右击,人马仆地,杀三十许人。宋将军屏息观之,股栗欲堕。忽闻客大呼曰:“吾去矣。”尘滚滚东向驰去。后遂不复至。

魏禧论曰:子房得力士,椎秦皇帝博浪沙中。大铁椎其人欤?天生异人,必有所用之。予读陈同甫《中兴遗传》,豪俊、侠烈、魁奇之士,泯泯然不见功名于世者,又何多也!岂天之生才不必为人用欤?抑用之自有时欤?子灿遇大铁椎为壬寅岁,视其貌当年三十,然大铁椎今年四十耳。子灿又尝见其写市物帖子,甚工楷书也。


\chapter*{黄鹤楼记}
\addcontentsline{toc}{chapter}{黄鹤楼记}
\begin{center}
	\textbf{[唐朝]阎伯理}
\end{center}

州城西南隅,有黄鹤楼者。《图经》云:“费祎登仙,尝驾黄鹤返憩于此,遂以名楼。”事列《神仙》之传,迹存《述异》之志。观其耸构巍峨,高标巃嵸,上倚河汉,下临江流;重檐翼馆,四闼霞敞;坐窥井邑,俯拍云烟:亦荆吴形胜之最也。何必濑乡九柱、东阳八咏,乃可赏观时物、会集灵仙者哉。

刺使兼侍御史、淮西租庸使、荆岳沔等州都团练使,河南穆公名宁,下车而乱绳皆理,发号而庶政其凝。或逶迤退公,或登车送远,游必于是,宴必于是。极长川之浩浩,见众山之累累。王室载怀,思仲宣之能赋;仙踪可揖,嘉叔伟之芳尘。乃喟然曰:“黄鹤来时,歌城郭之并是;浮云一去,惜人世之俱非。”有命抽毫,纪兹贞石。

时皇唐永泰元年,岁次大荒落,月孟夏,日庚寅也。


\chapter*{六国论}
\addcontentsline{toc}{chapter}{六国论}
\begin{center}
	\textbf{[宋朝]苏洵}
\end{center}


六国破灭,非兵不利,战不善,弊在赂秦。赂秦而力亏,破灭之道也。或曰:六国互丧,率赂秦耶?曰:不赂者以赂者丧,盖失强援,不能独完。故曰:弊在赂秦也。


秦以攻取之外,小则获邑,大则得城。较秦之所得,与战胜而得者,其实百倍;诸侯之所亡,与战败而亡者,其实亦百倍。则秦之所大欲,诸侯之所大患,固不在战矣。思厥先祖父,暴霜露,斩荆棘,以有尺寸之地。子孙视之不甚惜,举以予人,如弃草芥。今日割五城,明日割十城,然后得一夕安寝。起视四境,而秦兵又至矣。然则诸侯之地有限,暴秦之欲无厌,奉之弥繁,侵之愈急。故不战而强弱胜负已判矣。至于颠覆,理固宜然。古人云:“以地事秦,犹抱薪救火,薪不尽,火不灭。”此言得之。


齐人未尝赂秦,终继五国迁灭,何哉?与嬴而不助五国也。五国既丧,齐亦不免矣。燕赵之君,始有远略,能守其土,义不赂秦。是故燕虽小国而后亡,斯用兵之效也。至丹以荆卿为计,始速祸焉。赵尝五战于秦,二败而三胜。后秦击赵者再,李牧连却之。洎牧以谗诛,邯郸为郡,惜其用武而不终也。且燕赵处秦革灭殆尽之际,可谓智力孤危,战败而亡,诚不得已。向使三国各爱其地,齐人勿附于秦,刺客不行,良将犹在,则胜负之数,存亡之理,当与秦相较,或未易量。


呜呼!以赂秦之地,封天下之谋臣,以事秦之心,礼天下之奇才,并力西向,则吾恐秦人食之不得下咽也。悲夫!有如此之势,而为秦人积威之所劫,日削月割,以趋于亡。为国者无使为积威之所劫哉!


夫六国与秦皆诸侯,其势弱于秦,而犹有可以不赂而胜之之势。苟以天下之大,而从六国破亡之故事,是又在六国下矣。



\chapter*{随园记}
\addcontentsline{toc}{chapter}{随园记}
\begin{center}
	\textbf{[清朝]袁枚}
\end{center}


金陵自北门桥西行二里,得小仓山,山自清凉胚胎,分两岭而下,尽桥而止。蜿蜒狭长,中有清池水田,俗号干河沿。河未干时,清凉山为南唐避暑所,盛可想也。凡称金陵之胜者,南曰雨花台,西南曰莫愁湖,北曰钟山,东曰冶城,东北曰孝陵,曰鸡鸣寺。登小仓山,诸景隆然上浮。凡江湖之大,云烟之变,非山之所有者,皆山之所有也。


康熙时,织造隋公当山之北巅,构堂皇,缭垣牖,树之荻千章,桂千畦,都人游者,翕然盛一时,号曰随园。因其姓也。后三十年,余宰江宁,园倾且颓弛,其室为酒肆,舆台嚾呶,禽鸟厌之不肯妪伏,百卉芜谢,春风不能花。余恻然而悲,问其值,曰三百金,购以月俸。茨墙剪园,易檐改途。随其高,为置江楼;随其下,为置溪亭;随其夹涧,为之桥;随其湍流,为之舟;随其地之隆中而欹侧也,为缀峰岫;随其蓊郁而旷也,为设宧窔。或扶而起之,或挤而止之,皆随其丰杀繁瘠,就势取景,而莫之夭阏者,故仍名曰随园,同其音,易其义。


落成叹曰:“使吾官于此,则月一至焉;使吾居于此,则日日至焉。二者不可得兼,舍官而取园者也。”遂乞病,率弟香亭、甥湄君移书史居随园。闻之苏子曰:“君子不必仕,不必不仕。”然则余之仕与不仕,与居兹园之久与不久,亦随之而已。夫两物之能相易者,其一物之足以胜之也。余竟以一官易此园,园之奇,可以见矣。


己巳三月记。 



\chapter*{子产告范宣子轻币}
\addcontentsline{toc}{chapter}{子产告范宣子轻币}
\begin{center}
	\textbf{[春秋战国]左丘明}
\end{center}


范宣子为政,诸侯之币重,郑人病之。


二月,郑伯如晋。子产寓书于子西,以告宣子,曰:“子为晋国,四邻诸侯,不闻令德而闻重币。侨也惑之。侨闻君子长国家者,非无贿之患,而无令名之难,夫诸侯之贿,聚于公室,则诸侯贰;若吾子赖之,则晋国贰。诸侯贰则晋国坏,晋国贰则子之家坏。何没没也?将焉用贿?


夫令名,德之舆也。德,国家之基也。有基无坏,无亦是务乎?有德则乐,乐则能久。诗云:‘乐只君子,邦家之基。’有令德也夫!‘上帝临女,无贰尔心。’有令名也夫!恕思以明德,则令名载而行之,是以远至迩安。毋宁使人谓子,子实生我,而谓子浚我以生乎?象有齿以焚其身,贿也。”


宣子说,乃轻币。



\chapter*{酒箴}
\addcontentsline{toc}{chapter}{酒箴}
\begin{center}
	\textbf{[汉朝]扬雄}
\end{center}

子犹瓶矣。观瓶之居,居井之眉。处高临深,动而近危。酒醪不入口,臧水满怀。不得左右,牵于纆徽。一旦叀礙,为瓽所轠。身提黄泉,骨肉为泥。自用如此,不如鸱夷。

鸱夷滑稽,腹大如壶。尽日盛酒,人复借酤。常为国器,讬于属车。出入两宫,经营公家。由是言之,酒何过乎? 


\chapter*{七发}
\addcontentsline{toc}{chapter}{七发}
\begin{center}
	\textbf{[汉朝]枚乘}
\end{center}


楚太子有疾,而吴客往问之,曰:“伏闻太子玉体不安,亦少间乎?”太子曰:“惫!谨谢客。”客因称曰:“今时天下安宁,四宇和平,太子方富于年。意者久耽安乐,日夜无极,邪气袭逆,中若结轖。纷屯澹淡,嘘唏烦酲,惕惕怵怵,卧不得瞑。虚中重听,恶闻人声,精神越渫,百病咸生。聪明眩曜,悦怒不平。久执不废,大命乃倾。太子岂有是乎?”太子曰:“谨谢客。赖君之力,时时有之,然未至于是也”。”客曰:“今夫贵人之子,必宫居而闺处,内有保母,外有傅父,欲交无所。饮食则温淳甘膬,脭醲肥厚;衣裳则杂遝曼暖,燂烁热暑。虽有金石之坚,犹将销铄而挺解也,况其在筋骨之间乎哉?故曰:纵耳目之欲,恣支体之安者,伤血脉之和。且夫出舆入辇,命曰蹶痿之机;洞房清官,命曰寒热之媒;皓齿蛾眉,命曰伐性之斧;甘脆肥脓,命曰腐肠之药。今太子肤色靡曼,四支委随,筋骨挺解,血脉淫濯,手足堕窳;越女侍前,齐姬奉后;往来游醼,纵恣于曲房隐间之中。此甘餐毒药,戏猛兽之爪牙也。所从来者至深远,淹滞永久而不废,虽令扁鹊治内,巫咸治外,尚何及哉!今如太子之病者,独宜世之君子,博见强识,承间语事,变度易意,常无离侧,以为羽翼。淹沈之乐,浩唐之心,遁佚之志,其奚由至哉!’’太子曰:“诺。病已,请事此言。”


客曰:“今太子之病,可无药石针刺灸疗而已,可以要言妙道说而去也。不欲闻之乎?”太子曰:“仆愿闻之。”


客曰:“龙门之桐,高百尺而无枝。中郁结之轮菌,根扶疏以分离。上有千仞之峰,下临百丈之溪。湍流遡波,又澹淡之。其根半死半生。冬则烈风漂霰、飞雪之所激也,夏则霄霆、霹雳之所感也。朝则鹂黄、鳱鴠鸣焉,暮则羁雌、迷鸟宿焉。独鹄晨号乎其上,鹍鸡哀鸣翔乎其下。于是背秋涉冬,使琴挚斫斩以为琴,野茧之丝以为弦,孤子之钩以为隐,九寡之珥以为约。使师堂操《畅》,伯子牙为之歌。歌曰:‘麦秀蔪兮雉朝飞,向虚壑兮背槁槐,依绝区兮临回溪。’飞鸟闻之,翕翼而不能去;野兽闻之,垂耳而不能行;蚑、蟜、蝼、蚁闻之,柱喙而不能前。此亦天下之至悲也,太子能强起听之乎?”太子曰:“仆病未能也。”


客曰:“犓牛之腴,菜以笋蒲。肥狗之和,冒以山肤。楚苗之食,安胡之飰,抟之不解,一啜而散。于是使伊尹煎熬,易牙调和。熊蹯之胹,芍药之酱。薄耆之炙,鲜鲤之鱠。秋黄之苏,白露之茹。兰英之酒,酌以涤口。山梁之餐,豢豹之胎。小飰大歠,如汤沃雪。此亦天下之至美也,太子能强起尝之乎?”太子曰:“仆病未能也。”


客曰:“钟、岱之牡,齿至之车,前似飞鸟,后类距虚。穱麦服处,躁中烦外。羁坚辔,附易路。于是伯乐相其前后,王良、造父为之御,秦缺、楼季为之右。此两人者,马佚能止之,车覆能起之。于是使射千镒之重,争千里之逐。此亦天下之至骏也,太子能强起乘之乎?”太子曰:“仆病未能也。”


客曰:“既登景夷之台,南望荆山,北望汝海,左江右湖,其乐无有。于是使博辩之士,原本山川,极命草木,比物属事,离辞连类。浮游览观,乃下置酒于虞杯之宫。连廊四注,台城层构,纷纭玄绿。辇道邪交,黄池纡曲。溷章、白鹭,孔鸟、鶤鹄,鵷雏、鵁鶄,翠鬣紫缨。螭龙、德牧,邕邕群鸣。阳鱼腾跃,奋翼振鳞。漃漻薵蓼,蔓草芳苓。女桑、河柳,素叶紫茎。苗松、豫章,条上造天。梧桐、并阊,极望成林。众芳芬郁,乱于五风。从容猗靡,消息阳阴。列坐纵酒,荡乐娱心。景春佐酒,杜连理音。滋味杂陈,肴糅错该。练色娱目,流声悦耳。于是乃发《激楚》之结风,扬郑、卫之皓乐。使先施、徵舒、阳文、段干、吴娃、闾娵、傅予之徒,杂裾垂髾,目窕心与。揄流波,杂杜若,蒙清尘,被兰泽,嬿服而御。此亦天下之靡丽、皓侈、广博之乐也,太子能强起游乎?太子曰:“仆病未能也。”


客曰:“将为太子驯骐骥之马,驾飞軨之舆,乘牡骏之乘。右夏服之劲箭,左乌号之雕弓。游涉乎云林,周驰乎兰泽,弭节乎江浔。掩青苹,游清风。陶阳气,荡春心。逐狡兽,集轻禽。于是极犬马之才,困野兽之足,穷相御之智巧,恐虎豹,慴鸷鸟。逐马鸣镳,鱼跨麋角。履游麕兔,蹈践麖鹿,汗流沫坠,寃伏陵窘。无创而死者,固足充后乘矣。此校猎之至壮也,太子能强起游乎?”太子曰:“卜病未能也。”然阳气见于眉宇之间,侵淫而上,几满大宅。


客见太子有悦色,遂推而进之曰:“冥火薄天,兵车雷运,旌旗偃蹇,羽毛肃纷。驰骋角逐,慕味争先。徼墨广博,观望之有圻;纯粹全牺,献之公门。太子曰:“善!愿复闻之。”


客曰:“未既。于是榛林深泽,烟云闇莫,兕虎并作。毅武孔猛,袒裼身薄。白刃磑磑,矛戟交错。收获掌功,赏赐金帛。掩苹肆若,为牧人席。旨酒嘉肴,羞炰脍灸,以御宾客。涌觞并起,动心惊耳。诚不必悔,决绝以诺;贞信之色,形于金石。高歌陈唱,万岁无斁。此真太子之所喜也,能强起而游乎?”太子曰:“仆甚愿从,直恐为诸大夫累耳。”然而有起色矣。


客曰:“将以八月之望,与诸侯远方交游兄弟,并往观涛乎广陵之曲江。至则未见涛之形也,徒观水力之所到,则恤然足以骇矣。观其所驾轶者,所擢拔者,所扬汩者,所温汾者,所涤汔者,虽有心略辞给,固未能缕形其所由然也。怳兮忽兮,聊兮栗兮,混汩汩兮,忽兮慌兮,俶兮傥兮,浩瀇瀁兮,慌旷旷兮。秉意乎南山,通望乎东海。虹洞兮苍天,极虑乎崖涘。流揽无穷,归神日母。汩乘流而下降兮,或不知其所止。或纷纭其流折兮,忽缪往而不来。临朱汜而远逝兮,中虚烦而益怠。莫离散而发曙兮,内存心而自持。于是澡概胸中,洒练五藏,澹澉手足,頮濯发齿。揄弃恬怠,输写淟浊,分决狐疑,发皇耳目。当是之时,虽有淹病滞疾,犹将伸伛起躄,发瞽披聋而观望之也,况直眇小烦懑,酲醲病酒之徒哉!故曰:发蒙解惑,不足以言也。”太子曰:“善,然则涛何气哉?”


客曰:“不记也。然闻于师曰,似神而非者三:疾雷闻百里;江水逆流,海水上潮;山出内云,日夜不止。衍溢漂疾,波涌而涛起。其始起也,洪淋淋焉,若白鹭之下翔。其少进也,浩浩溰溰,如素车白马帷盖之张。其波涌而云乱,扰扰焉如三军之腾装。其旁作而奔起也,飘飘焉如轻车之勒兵。六驾蛟龙,附从太白。纯驰皓蜺,前后络绎。顒顒卬卬,椐椐强强,莘莘将将。壁垒重坚,沓杂似军行。訇隐匈磕,轧盘涌裔,原不可当。观其两旁,则滂渤怫郁,闇漠感突,上击下律,有似勇壮之卒,突怒而无畏。蹈壁冲津,穷曲随隈,逾岸出追。遇者死,当者坏。初发乎或围之津涯,荄轸谷分。回翔青篾,衔枚檀桓。弭节伍子之山,通厉骨母之场,凌赤岸,篲扶桑,横奔似雷行,诚奋厥武,如振如怒,沌沌浑浑,状如奔马。混混庉庉,声如雷鼓。发怒庢沓,清升逾跇,侯波奋振,合战于藉藉之口。鸟不及飞,鱼不及回,兽不及走。纷纷翼翼,波涌云乱,荡取南山,背击北岸。覆亏丘陵,平夷西畔。险险戏戏,崩坏陂池,决胜乃罢。瀄汩潺湲,披扬流洒。横暴之极,鱼鳖失势,颠倒偃侧,沋沋湲湲,蒲伏连延。神物恠疑,不可胜言。直使人踣焉,洄闇凄怆焉。此天下恠异诡观也,太子能强起观之乎?”太子曰:“仆病未能也。”


客曰:“将为太子奏方术之士有资略者,若庄周、魏牟、杨朱、墨翟、便蜎、詹何之伦,使之论天下之精微,理万物之是非。孔、老览观,孟子筹之,万不失一。此亦天下要言妙道也,太子岂欲闻之乎?”


于是太子据几而起,曰:“涣乎若一听圣人辩士之言。”涊然汗出,霍然病已。



\chapter*{杨氏之子}
\addcontentsline{toc}{chapter}{杨氏之子}
\begin{center}
	\textbf{[南北朝]刘义庆}
\end{center}

梁国杨氏子九岁,甚聪惠。孔君平诣其父,父不在,乃呼儿出。为设果,果有杨梅。孔指以示儿曰:“此是君家果。”儿应声答曰:“未闻孔雀是夫子家禽。”


\chapter*{圆圆曲}
\addcontentsline{toc}{chapter}{圆圆曲}
\begin{center}
	\textbf{[清朝]吴伟业}
\end{center}

鼎湖当日弃人间,破敌收京下玉关。恸哭六军俱缟素,冲冠一怒为红颜。红颜流落非吾恋,逆贼天亡自荒宴。电扫黄巾定黑山,哭罢君亲再相见。

相见初经田窦家,侯门歌舞出如花。许将戚里箜篌伎,等取将军油壁车。家本姑苏浣花里,圆圆小字娇罗绮。梦向夫差苑里游,宫娥拥入君王起。前身合是采莲人,门前一片横塘水。横塘双桨去如飞,何处豪家强载归。此际岂知非薄命,此时唯有泪沾衣。薰天意气连宫掖,明眸皓齿无人惜。夺归永巷闭良家,教就新声倾坐客。坐客飞觞红日暮,一曲哀弦向谁诉?白晳通侯最少年,拣取花枝屡回顾。早携娇鸟出樊笼,待得银河几时渡?恨杀军书抵死催,苦留后约将人误。相约恩深相见难,一朝蚁贼满长安。可怜思妇楼头柳,认作天边粉絮看。遍索绿珠围内第,强呼绛树出雕阑。若非壮士全师胜,争得蛾眉匹马还?

蛾眉马上传呼进,云鬟不整惊魂定。蜡炬迎来在战场,啼妆满面残红印。专征萧鼓向秦川,金牛道上车千乘。斜谷云深起画楼,散关月落开妆镜。传来消息满江乡,乌桕红经十度霜。教曲伎师怜尚在,浣纱女伴忆同行。旧巢共是衔泥燕,飞上枝头变凤凰。长向尊前悲老大,有人夫婿擅侯王。当时只受声名累,贵戚名豪竞延致。一斛明珠万斛愁,关山漂泊腰肢细。错怨狂风飏落花,无边春色来天地。

尝闻倾国与倾城,翻使周郎受重名。妻子岂应关大计,英雄无奈是多情。全家白骨成灰土,一代红妆照汗青。君不见,馆娃初起鸳鸯宿,越女如花看不足。香径尘生乌自啼,屧廊人去苔空绿。换羽移宫万里愁,珠歌翠舞古梁州。为君别唱吴宫曲,汉水东南日夜流!


\chapter*{陈万年教子}
\addcontentsline{toc}{chapter}{陈万年教子}
\begin{center}
	\textbf{[汉朝]班固}
\end{center}


陈万年乃朝中重臣也,尝病,召子咸教戒于床下。语至三更,咸睡,头触屏风。万年大怒,欲杖之,曰:“乃公戒汝,汝反睡,不听吾言,何也?”咸叩头谢曰:“具晓所言,大要教咸谄也。”万年乃不复言。

\chapter*{吴许越成}
\addcontentsline{toc}{chapter}{吴许越成}
\begin{center}
	\textbf{[春秋战国]左丘明}
\end{center}


吴王夫差败越于夫椒,报槜李也。遂入越。越子以甲楯五千保于会稽,使大夫种因吴太宰嚭以行成。


吴子将许之。伍员曰:“不可。臣闻之:‘树德莫如滋,去疾莫如尽。’昔有过浇杀斟灌以伐斟鄩,灭夏后相。后缗方娠,逃出自窦,归于有仍,生少康焉,为仍牧正。惎浇能戒之。浇使椒求之,逃奔有虞,为之庖正,以除其害。虞思于是妻之以二姚,而邑诸纶,有田一成,有众一旅。能布其德,而兆其谋,以收夏众,抚其官职;使女艾谍浇,使季杼诱豷,遂灭过、戈,复禹之绩。祀夏配天,不失旧物。今吴不如过,而越大于少康,或将丰之,不亦难乎?勾践能亲而务施,施不失人,亲不弃劳,与我同壤而世为仇雠。于是乎克而弗取,将又存之,违天而长寇雠,后虽悔之,不可食已。姬之衰也,日可俟也。介在蛮夷,而长寇雠,以是求伯,必不行矣。”


弗听。退而告人曰:“越十年生聚,而十年教训,二十年之外,吴其为沼乎!”



\chapter*{祭鳄鱼文}
\addcontentsline{toc}{chapter}{祭鳄鱼文}
\begin{center}
	\textbf{[唐朝]韩愈}
\end{center}

维年月日,潮州刺史韩愈使军事衙推秦济,以羊一、猪一,投恶溪之潭水,以与鳄鱼食,而告之曰:

昔先王既有天下,列山泽,罔绳擉刃,以除虫蛇恶物为民害者,驱而出之四海之外。及后王德薄,不能远有,则江汉之间,尚皆弃之以与蛮、夷、楚、越;况潮岭海之间,去京师万里哉!鳄鱼之涵淹卵育于此,亦固其所。今天子嗣唐位,神圣慈武,四海之外,六合之内,皆抚而有之;况禹迹所揜,扬州之近地,刺史、县令之所治,出贡赋以供天地宗庙百神之祀之壤者哉?鳄鱼其不可与刺史杂处此土也。

刺史受天子命,守此土,治此民,而鳄鱼睅然不安溪潭,据处食民畜、熊、豕、鹿、獐,以肥其身,以种其子孙;与刺史亢拒,争为长雄;刺史虽驽弱,亦安肯为鳄鱼低首下心,伈伈睍睍,为民吏羞,以偷活于此邪!且承天子命以来为吏,固其势不得不与鳄鱼辨。

鳄鱼有知,其听刺史言:潮之州,大海在其南,鲸、鹏之大,虾、蟹之细,无不归容,以生以食,鳄鱼朝发而夕至也。今与鳄鱼约:尽三日,其率丑类南徙于海,以避天子之命吏;三日不能,至五日;五日不能,至七日;七日不能,是终不肯徙也。是不有刺史、听从其言也;不然,则是鳄鱼冥顽不灵,刺史虽有言,不闻不知也。夫傲天子之命吏,不听其言,不徙以避之,与冥顽不灵而为民物害者,皆可杀。刺史则选材技吏民,操强弓毒矢,以与鳄鱼从事,必尽杀乃止。其无悔!


\chapter*{烛之武退秦师}
\addcontentsline{toc}{chapter}{烛之武退秦师}
\begin{center}
	\textbf{[春秋战国]左丘明}
\end{center}


晋侯、秦伯围郑,以其无礼于晋,且贰于楚也。晋军函陵,秦军氾南。


佚之狐言于郑伯曰:“国危矣,若使烛之武见秦君,师必退。”公从之。辞曰:“臣之壮也,犹不如人;今老矣,无能为也已。”公曰:“吾不能早用子,今急而求子,是寡人之过也。然郑亡,子亦有不利焉!”许之。


夜缒而出,见秦伯,曰:“秦、晋围郑,郑既知亡矣。若亡郑而有益于君,敢以烦执事。越国以鄙远,君知其难也,焉用亡郑以陪邻?邻之厚,君之薄也。若舍郑以为东道主,行李之往来,共其乏困,君亦无所害。且君尝为晋君赐矣,许君焦、瑕,朝济而夕设版焉,君之所知也。夫晋,何厌之有?既东封郑,又欲肆其西封,若不阙秦,将焉取之?阙秦以利晋,唯君图之。”秦伯说,与郑人盟。使杞子、逢孙、杨孙戍之,乃还。


子犯请击之。公曰:“不可。微夫人之力不及此。因人之力而敝之,不仁;失其所与,不知;以乱易整,不武。吾其还也。”亦去之。


(选自《左传》)



\chapter*{子产坏晋馆垣}
\addcontentsline{toc}{chapter}{子产坏晋馆垣}
\begin{center}
	\textbf{[春秋战国]左丘明}
\end{center}


公薨之月,子产相郑伯以如晋,晋侯以我丧故,未之见也。子产使尽坏其馆之垣,而纳车马焉。


士文伯让之,曰:“敝邑以政刑之不修,寇盗充斥,无若诸侯之属辱在寡君者何,是以令吏人完客所馆,高其闬闳,厚其墙垣,以无忧客使。今吾子坏之,虽从者能戒,其若异客何?以敝邑之为盟主,缮完葺墙,以待宾客。若皆毁之,其何以共命?寡君使匄请命。


对曰:“以敝邑褊小,介于大国,诛求无时,是以不敢宁居,悉索敝赋,以来会时事。逢执事之不闲,而未得见;又不获闻命,未知见时。不敢输币,亦不敢暴露。其输之,则君之府实也,非荐陈之,不敢输也。其暴露之,则恐燥湿之不时而朽蠹,以重敝邑之罪。侨闻文公之为盟主也,宫室卑庳,无观台榭,以崇大诸侯之馆,馆如公寝;库厩缮修,司空以时平易道路,圬人以时塓馆宫室;诸侯宾至,甸设庭燎,仆人巡宫,车马有所,宾从有代,巾车脂辖,隶人、牧、圉,各瞻其事;百官之属各展其物;公不留宾,而亦无废事;忧乐同之,事则巡之,教其不知,而恤其不足。宾至如归,无宁灾患;不畏寇盗,而亦不患燥湿。今铜鞮之宫数里,而诸侯舍于隶人,门不容车,而不可逾越;盗贼公行。而天疠不戒。宾见无时,命不可知。若又勿坏,是无所藏币以重罪也。敢请执事,将何所命之?虽君之有鲁丧,亦敝邑之忧也。若获荐币,修垣而行,君之惠也,敢惮勤劳?”


文伯复命。赵文子曰:“信。我实不德,而以隶人之垣以赢诸侯,是吾罪也。”使士文伯谢不敏焉。


晋侯见郑伯,有加礼,厚其宴好而归之。乃筑诸侯之馆。


叔向曰:“辞之不可以已也如是夫!子产有辞,诸侯赖之,若之何其释辞也?《诗》曰:‘辞之辑矣,民之协矣;辞之怿矣,民之莫矣。’其知之矣。”



\chapter*{大鹏赋 并序}
\addcontentsline{toc}{chapter}{大鹏赋 并序}
\begin{center}
	\textbf{[唐朝]李白}
\end{center}

余昔于江陵,见天台司马子微,谓余有仙风道骨,可与神游八极之表。因著大鹏遇希有鸟赋以自广。此赋已传于世,往往人间见之。悔其少作,未穷宏达之旨,中年弃之。及读晋书,睹阮宣子大鹏赞,鄙心陋之。遂更记忆,多将旧本不同。今复存手集,岂敢传诸作者?庶可示之子弟而已。其辞曰:

南华老仙,发天机于漆园。吐峥嵘之高论,开浩荡之奇言。徵至怪于齐谐,谈北溟之有鱼。吾不知其几千里,其名曰鲲。化成大鹏,质凝胚浑。脱鬐鬣于海岛,张羽毛于天门。刷渤澥之春流,晞扶桑之朝暾。燀赫乎宇宙,凭陵乎昆仑。一鼓一舞,烟朦沙昏。五岳为之震荡,百川为之崩奔。

乃蹶厚地,揭太清。亘层霄,突重溟。激三千以崛起,向九万而迅征。背嶪太山之崔嵬,翼举长云之纵横。左回右旋,倏阴忽明。历汗漫以夭矫,羾阊阖之峥嵘。簸鸿蒙,扇雷霆。斗转而天动,山摇而海倾。怒无所搏,雄无所争。固可想象其势,仿佛其形。

若乃足萦虹蜺,目耀日月。连轩沓拖,挥霍翕忽。喷气则六合生云,洒毛则千里飞雪。邈彼北荒,将穷南图。运逸翰以傍击,鼓奔飙而长驱。烛龙衔光以照物,列缺施鞭而启途。块视三山,杯观五湖。其动也神应,其行也道俱。任公见之而罢钓,有穷不敢以弯弧。莫不投竿失镞,仰之长吁。

尔其雄姿壮观,坱轧河汉。上摩苍苍,下覆漫漫。盘古开天而直视,羲和倚日以旁叹。缤纷乎八荒之间,掩映乎四海之半。当胸臆之掩昼,若混茫之未判。忽腾覆以回转,则霞廓而雾散。

然后六月一息,至于海湄。欻翳景以横翥,逆高天而下垂。憩乎泱漭之野,入乎汪湟之池。猛势所射,馀风所吹。溟涨沸渭,岩峦纷披。天吴为之怵栗,海若为之躨跜。巨鳌冠山而却走,长鲸腾海而下驰。缩壳挫鬣,莫之敢窥。吾亦不测其神怪之若此,盖乃造化之所为。

岂比夫蓬莱之黄鹄,夸金衣与菊裳?耻苍梧之玄凤,耀彩质与锦章。既服御于灵仙,久驯扰于池隍。精卫殷勤于衔木,鶢鶋悲愁乎荐觞。天鸡警晓于蟠桃,踆乌晰耀于太阳。不旷荡而纵适,何拘挛而守常?未若兹鹏之逍遥,无厥类乎比方。不矜大而暴猛,每顺时而行藏。参玄根以比寿,饮元气以充肠。戏旸谷而徘徊,冯炎洲而抑扬。

俄而希有鸟见谓之曰:伟哉鹏乎,此之乐也。吾右翼掩乎西极,左翼蔽乎东荒。跨蹑地络,周旋天纲。以恍惚为巢,以虚无为场。我呼尔游,尔同我翔。于是乎大鹏许之,欣然相随。此二禽已登于寥廓,而斥鷃之辈,空见笑于藩篱。


\chapter*{邹忌讽齐王纳谏}
\addcontentsline{toc}{chapter}{邹忌讽齐王纳谏}
\begin{center}
	\textbf{[汉朝]刘向}
\end{center}


邹忌修八尺有余,而形貌昳丽。朝服衣冠,窥镜,谓其妻曰:“我孰与城北徐公美?”其妻曰:“君美甚,徐公何能及君也!”城北徐公,齐国之美丽者也。忌不自信,而复问其妾曰:“吾孰与徐公美?”妾曰:“徐公何能及君也!”旦日,客从外来,与坐谈,问之客曰:“吾与徐公孰美?”客曰:“徐公不若君之美也。”明日徐公来,孰视之,自以为不如;窥镜而自视,又弗如远甚。暮寝而思之,曰:“吾妻之美我者,私我也;妾之美我者,畏我也;客之美我者,欲有求于我也。”


于是入朝见威王,曰:“臣诚知不如徐公美。臣之妻私臣,臣之妾畏臣,臣之客欲有求于臣,皆以美于徐公。今齐地方千里,百二十城,宫妇左右莫不私王,朝廷之臣莫不畏王,四境之内莫不有求于王:由此观之,王之蔽甚矣。”


王曰:“善。”乃下令:“群臣吏民能面刺寡人之过者,受上赏;上书谏寡人者,受中赏;能谤讥于市朝,闻寡人之耳者,受下赏。”令初下,群臣进谏,门庭若市;数月之后,时时而间进;期年之后,虽欲言,无可进者。燕、赵、韩、魏闻之,皆朝于齐.此所谓战胜于朝廷。(谤讥一作:谤议)



\chapter*{司马错论伐蜀}
\addcontentsline{toc}{chapter}{司马错论伐蜀}
\begin{center}
	\textbf{[汉朝]刘向}
\end{center}


司马错与张仪争论于秦惠王前,司马错欲伐蜀,张仪曰:“不如伐韩。”王曰:“请闻其说。”


对曰:“亲魏善楚,下兵三川,塞轘辕、缑氏之口,当屯留之道,魏绝南阳,楚临南郑,秦攻新城宜阳,以临二周之郊,诛周主之罪,侵楚魏之地。周自知不救,九鼎宝器必出。据九鼎,按图籍,挟天子以令天下,天下莫敢不听,此王业也。今夫蜀,西僻之国也,而戎狄之长也,敝兵劳众不足以成名,得其地不足以为利。臣闻:‘争名者于朝,争利者于市。’今三川、周室,天下之市朝也,而王不争焉,顾争于戎狄,去王业远矣。”


司马错曰:“不然。臣闻之:‘欲富国者,务广其地;欲强兵者,务富其民;欲王者,务博其德。三资者备,而王随之矣。’今王之地小民贫,故臣愿从事于易。夫蜀,西僻之国也,而戎狄之长也,而有桀纣之乱。以秦攻之,譬如使豺狼逐群羊也。取其地足以广国也,得其财足以富民,缮兵不伤众,而彼已服矣。故拔一国,而天下不以为暴;利尽西海,诸侯不以为贪。是我一举而名实两附,而又有禁暴止乱之名。今攻韩劫天子,劫天子,恶名也,而未必利也,又有不义之名。而攻天下之所不欲,危!臣请谒其故:周,天下之宗室也;韩,周之与国也。周自知失九鼎,韩自知亡三川,则必将二国并力合谋,以因于齐、赵而求解乎楚、魏。以鼎与楚,以地与魏,王不能禁。此臣所谓危,不如伐蜀之完也。”


惠王曰:“善!寡人听子。”卒起兵伐蜀,十月取之,遂定蜀,蜀主更号为侯,而使陈庄相蜀。蜀既属,秦益强富厚,轻诸侯。



\chapter*{高帝求贤诏}
\addcontentsline{toc}{chapter}{高帝求贤诏}
\begin{center}
	\textbf{[汉朝]班固}
\end{center}


盖闻王者莫高于周文,伯者莫高于齐桓,皆待贤人而成名。今天下贤者智能,岂特古之人乎?患在人主不交故也,士奚由进?今吾以天之灵,贤士大夫,定有天下,以为一家。欲其长久,世世奉宗庙亡绝也。贤人已与我共平之矣,而不与吾共安利之,可乎?贤士大夫有肯从我游者,吾能尊显之。布告天下,使明知朕意。


御史大夫昌下相国,相国酂侯下诸侯王,御史中执法下郡守,其有意称明德者,必身劝,为之驾,遣诣相国府,署行义年,有而弗言,觉免。年老癃病,勿遣。



\chapter*{治安策}
\addcontentsline{toc}{chapter}{治安策}
\begin{center}
	\textbf{[汉朝]贾谊}
\end{center}


臣窃惟事势,可为痛哭者一,可为流涕者二,可为长太息者六,若其它背理而伤道者,难遍以疏举。进言者皆曰天下已安已治矣,臣独以为未也。曰安且治者,非愚则谀,皆非事实知治乱之体者也。夫抱火厝之积薪之下而寝其上,火未及燃,因谓之安,方今之势,何以异此!本末舛逆,首尾衡决,国制抢攘,非甚有纪,胡可谓治!陛下何不一令臣得熟数之于前,因陈治安之策,试详择焉!


夫射猎之娱,与安危之机孰急?使为治劳智虑,苦身体,乏钟鼓之乐,勿为可也。乐与今同,而加之诸侯轨道,兵革不动,民保首领,匈叙宾服,四荒乡风,百姓素朴,狱讼衰息。大数既得,则天下顺治,海内之气,清和咸理,生为明帝,没为明神,名誉之美,垂于无穷。《礼》祖有功而宗有德,使顾成之庙称为太宗,上配太祖,与汉亡极。建久安之势,成长治之业,以承祖庙,以奉六亲,至孝也;以幸天下,以育群生,至仁也;立经陈纪,轻重同得,后可以为万世法程,虽有愚幼不肖之嗣,犹得蒙业而安,至明也。以陛下之明达,因使少知治体者得佐下风,致此非难也。其具可素陈于前,愿幸无忽。臣谨稽之天地,验之往古,按之当今之务,日夜念此至孰也,虽使禹舜复生,为陛下计,亡以易此。


夫树国固,必相疑之势也,下数被其殃,上数爽其忧,甚非所以安上而全下也。今或亲弟谋为东帝,亲兄之子西乡而击,今吴又见告矣。天子春秋鼎盛,行义未过,德泽有加焉,犹尚如是,况莫大诸侯权力且十此者乎!


然而天下少安,何也?大国之王幼弱未壮,汉之所置傅相方握其事。数年之后,诸侯之王大抵皆冠,血气方刚,汉之傅相称病而赐罢,彼自丞尉以上徧置私人,如此,有异淮南、济北之为邪?此时而欲为治安,虽尧舜不治。


黄帝曰:“日中必熭,操刀必割。”今令此道顺,而全安甚易;不肯早为,已乃堕骨肉之属而抗刭之,岂有异秦之季世乎!夫以天子之位,乘今之时,因天之助,尚惮以危为安,以乱为治,假设陛下居齐桓之处,将不合诸侯而匡天下乎?臣又以知陛下有所必不能矣。假设天下如曩时,淮阴侯尚王楚,黥布王淮南,彭越王梁,韩信王韩,张敖王赵,贯高为相,卢绾王燕,陈狶在代,令此六七公者皆亡恙,当是时而陛下即天子位,能自安乎?臣有以知陛下之不能也。天下肴乱,高皇帝与诸公倂起,非有仄室之势以豫席之也。诸公幸者乃为中涓,其次仅得舍人,材之不逮至远也。高皇帝以明圣威武即天子位,割膏腴之地以王诸公,多者百余城,少者乃三四十县,德至渥也,然其后十年之间,反者九起。陛下之与诸公,非亲角材而臣之也,又非身封王之也,自高皇帝不能以是一岁为安,故臣知陛下之不能也。


然尚有可诿者,曰疏。臣请试言其亲者。假令悼惠王王齐,元王王楚,中子王赵,幽王王淮阳,共王王梁,灵王王燕,厉王王淮南,六七贵人皆亡恙,当是时陛下即位,能为治乎?臣又知陛下之不能也。若此诸王,虽名为臣,实皆有布衣昆弟之心,虑无不帝制而天子自为者。擅爵人,赦死罪,甚者或戴黄屋,汉法令非行也。虽行不轨如厉王者,令之不肯听,召之安可致乎!幸而来至,法安可得加!动一亲戚,天下圜视而起,陛下之臣虽有悍如冯敬者,适启其口,匕首已陷其胸矣。陛下虽贤,谁与领此?


故疏者必危,亲者必乱,已然之效也。其异姓负强而动者,汉已幸胜之矣,又不易其所以然。同姓袭是迹而动,既有徵矣,其势尽又复然。殃祸之变未知所移,明帝处之尚不能以安,后世将如之何!


屠牛坦一朝解十二牛,而芒刃不顿者,所排击剥割,皆众理解也。至于髋髀之所,非斤则斧。夫仁义恩厚,人主之芒刃也;权势法制,人主之斤斧也。今诸侯王皆众髋髀也,释斤斧之用,而欲婴以芒刃,臣以为不缺则折。胡不用之淮南、济北?势不可也。


臣窃迹前事,大抵强者先反,淮阴王楚最强,则最先反;韩信倚胡,则又反;贯高因赵资,则又反;陈狶兵精,则又反;彭越用梁,则又反;黥布用淮南,则又反;卢绾最弱,最后反。长沙乃在二万五千户耳,功少而最完,势疏而最忠,非独性异人也,亦形势然也。曩令樊、郦、绛、灌据数十城而王,今虽以残亡可也;令信、越之伦列为彻侯而居,虽至今存可也。


然则天下之大计可知已。欲诸王之皆忠附,则莫若令如长沙王,欲臣子之勿菹醢,则莫若令如樊郦等;欲天下之治安,莫若众建诸侯而少其力。力少则易使以义,国小则亡邪心。令海内之势如身之使臂,臂之使指,莫不制从。诸侯之君不敢有异心,辐凑并进而归命天子,虽在细民,且知其安,故天下咸知陛下之明。割地定制,令齐、赵、楚各为若干国,使悼惠王、幽王、元王之子孙毕以次各受祖之分地,地尽而止,及燕、梁它国皆然。其分地众而子孙少者,建以为国,空而置之,须其子孙生者,举使君之。诸侯之地其削颇入汉者,为徙其侯国,及封其子孙也,所以数偿之;一寸之地,一人之众,天子亡所利焉,诚以定治而已,故天下咸知陛下之廉。地制壹定,宗室子孙莫虑不王,下无倍畔之心,上无诛伐之志,故天下咸知陛下之仁。法立而不犯,令行而不逆,贯高、利几之谋不生,柴奇、开章不计不萌,细民乡善,大臣致顺,故天下咸知陛下之义。卧赤子天下之上而安,植遗腹,朝委裘,而天下不乱。当时大治,后世诵圣。壹动而五业附,陛下谁惮而久不为此?


天下之势方病大瘇。一胫之大几如要,一指之大几如股,平居不可屈信,一二指搐,身虑亡聊。失今不治,必为锢疾,后虽有扁鹊,不能为已。病非徒瘇也,又苦蹠戾。元王之子,帝之从弟也,今之王者,从弟之子也。惠王之子,亲兄子也;今之王者,兄子之子也。亲者或亡分地以安天下,疏者或制大权以逼天子,臣故曰非徒病瘇也,又苦蹠戾。可痛哭者,此病是也。


天下之势方倒县。凡天子者,天下之首,何也?上也。蛮夷者,天下之足,何也?下也。今匈奴嫚娒侵掠,至不敬也,为天下患,至亡已也,而汉岁金絮采缯以奉之。夷狄征令,是主上之操也;天子共贡,是臣下之礼也。足反居上,首顾居下,倒县如此,莫之能解,犹为国有人乎?非亶倒县而已,又类辟,且病痱。夫辟者一面病,痱者一方痛。今西边北边之郡,虽有长爵不轻得复,五尺以上不轻得息,斥候望烽燧不得卧,将吏被介胄而睡,臣故曰一方病矣。医能治之,而上不使,可为流涕者此也。


陛下何忍以帝皇之号为戎人诸侯,势既卑辱,而祸不息,长此安穷!进谋者率以为是,固不可解也,亡具甚矣。臣窃料匈奴之众不过汉一大县,以天下之大困于一县之众,甚为执事者羞之。陛下何不试以臣为属国之官以主匈奴?行臣之计,请必系单于之颈而制其命,伏中行说而笞其背,举匈奴之众唯上之令。今不猎猛敌而猎田彘,不搏反寇而搏畜菟,玩细娱而不图大患,非所以为安也。德可远施,威可远加,而直数百里外威令不信,可为流涕者此也。


今民卖僮者,为之绣衣丝履偏诸缘,内之闲中,是古天子后服,所以庙而不宴者也,而庶人得以衣婢妾。白縠之表,薄纨之里,以偏诸,美者黼绣,是古天子之服,今富人大贾嘉会召客者以被墙。古者以奉一帝一后而节适,今庶人屋壁得为帝服,倡优下贱得为后饰,然而天下不屈者,殆未有也。且帝之身自衣皁绨,而富民墙屋被文绣;天子之后以缘其领,庶人孽妾缘其履:此臣所谓舛也。夫百人作之不能衣一人,欲天下亡寒,胡可得也?一人耕之,十人聚而食之,欲天下亡饥,不可得也。饥寒切于民之肌肤,欲其亡为奸邪,不可得也。国已屈矣,盗贼直须时耳,然而献计者曰“毋动”,为大耳。夫俗至大不敬也,至亡等也,至冒上也,进计者犹曰“毋为”,可为长太息者此也。


商君遗礼义,弃仁恩,并心于进取。行之二岁,秦俗日败。故秦人家富子壮则出分,家贫子壮则出赘。借父耰鉏,虑有德色;母取箕帚,立而谇语。抱哺其于,与公并倨;妇姑不相说,则反唇而相稽。其慈子耆利,不同禽兽者亡几耳。然并心而赴时犹曰蹶六国,兼天下。功成求得矣,终不知反廉愧之节,仁义之厚。信并兼之法,遂进取之业,天下大败,众掩寡,智欺愚,勇威怯,壮陵衰,其乱至矣,是以大贤起之,威震海内,德从天下。曩之为秦者,今转而为汉矣。然其遗风余俗,犹尚未改。今世以侈靡相竞,而上亡制度,弃礼谊,捐廉耻日甚,可谓月异而岁不同矣。逐利不耳,虑非顾行也,今其甚者杀父兄矣。盗者剟寝户之帘,搴两庙之器,白昼大都之中剽吏而夺之金。矫伪者出几十万石粟,赋六百余万钱,乘传而行郡国,此其亡行义之尤至者也。而大臣特以簿书不报,期会之间,以为大故。至于俗流失,世坏败,因恬而不知怪,虑不动于耳目,以为是适然耳。夫移风易俗,使天下回心而乡道,类非俗吏之所能为也。俗吏之所务,在于刀笔筐箧,而不知大体。陛下又不自忧,窃为陛下惜之。


夫立君臣,等上下,使父子有礼,六亲有纪,此非天之所为,人之所设也。夫人之所设,不为不立,不植则僵,不修则坏。《管子》曰:“礼义廉耻,是谓四维;四维不张,国乃灭亡。”使管子愚人也则可,管子而少知治体,则是岂可不为寒心哉!秦灭四维而不张,故君臣乖乱,六亲殃戮,奸人并起,万民离叛,凡十三岁,而社稷为虚。今四维犹未备也,故奸人几幸,而众心疑惑。岂如今定经制,令君君臣臣,上下有差,父子六亲各得其宜,奸人亡所几幸,而群臣众信,是不疑惑!此业一定,世世常安,而后有所持循矣。若夫经制不定,是犹度江河亡维楫,中流而遇风波,舩必覆矣。可为长太息者此也。


夏为天子,十有余世,而殷受之。殷为天子,二十余世,而周受之。周为天子,三十余世,而秦受之。秦为天子,二世而亡。人性不甚相远也,何三代之君有道之长,而秦无道之暴也?其故可知也。古之王者,太子乃生,固举以礼,使士负之,有司齐肃端冕,见之南郊,见于天也。过阙则下,过庙则趋,孝子之道也。故自为赤子而教固已行矣。昔者成王幼在襁抱之中,召公为太保,周公为太傅,太公为太师。保,保其身体;傅,传之德义;师,道之教训:此三公之职也。于是为置三少,皆上大夫也,曰少保、少傅、少师,是与太子宴者也。故乃孩子提有识,三公、三少固明孝仁礼义以道习之,逐去邪人,不使见恶行。于是皆选天下之端士孝悌博闻有道术者以卫翼之,使与太子居处出入。故太子乃生而见正事,闻正言,行正道,左右前后皆正人也。夫习与正人居之,不能毋正,犹生长于齐不能不齐言也;习与不正人居之,不能毋不正,犹生长于楚之地不能不楚言也。故择其所耆,必先受业,乃得尝之;择其所乐,必先有习,乃得为之。孔子曰:“少成若天性,习贯如自然。”及太子少长,知妃色,则入于学。学者,所学之官也。《学礼》曰:“帝入东学,上亲而贵仁,则亲疏有序而恩相及矣;帝入南学,上齿而贵信,则长幼有差而民不诬矣;帝入西学,上贤而贵德,则圣智在位而功不遗矣;帝入北学,上贵而尊爵,则贵贱有等而下不矣;帝入太学,承师问道,退习而考于太傅,太傅罚其不则而匡其不及,则德智长而治道得矣。此五学者既成于上,则百姓黎民化辑于下矣。”及太于既冠成人,免于保傅之严,则有记过之史,彻膳之宰,进善之旌,诽谤之木,敢谏之鼓。瞽史诵诗,工诵箴谏,大夫进谋,士传民语。习与智长,故切而不媿;化与心成,故中道若性。三代之礼:春朝朝日,秋暮夕月,所以明有敬也;春秋入学,坐国老,执酱而亲馈之,所以明有孝也;行以鸾和,步中《采齐》,趣中《肆夏》,所以明有度也;其于禽兽,见其生不食其死,闻其声不食其肉,故远庖厨,所以长恩,且明有仁也。


夫三代之所以长久者,以其辅翼太子有此具也。及秦而不然。其俗固非贵辞让也,所上者告讦也;固非贵礼义也,所上者刑罚也。使赵高傅胡亥而教之狱,所习者非斩劓人,则夷人之三族也。故胡亥今日即位而明日射人,忠谏者谓之诽谤,深计者谓之妖言,其视杀人若艾草菅然。岂惟胡亥之性恶哉?彼其所以道之者非其理故也。


鄙谚曰:“不习为吏,视已成事。”又曰:“前车覆,后车诫。”夫三代之所以长久者,其已事可知也;然而不能从者,是不法圣智也。秦世之所以亟绝者,其辙迹可见也;然而不避,是后车又将覆也。夫存亡之变,治乱之机,其要在是矣。天下之命,县于太子;太子之善,在于早谕教与选左右。夫心未滥而先谕教,则化易成也;开于道术智谊之指,则教之力也。若其服习积贯,则左右而已。夫胡、粤之人,生而同声,耆欲不异,及其长而成俗,累数译而不能相通,行者有虽死而不相为者,则教习然也。臣故曰选左右早谕教最急。夫教得而左右正,则太子正矣,太子正而天下定矣。《书》曰:“一人有庆,兆民赖之。”此时务也。


凡人之智,能见已然,不能见将然。夫礼者禁于将然之前,而法者禁于己然之后,是故法之所用易见,而礼之所为生难知也。若夫庆赏以劝善,刑罚以惩恶,先王执此之政,坚如金石,行此之令,信如四时,据此之公,无私如天地耳,岂顾不用哉?然而曰礼云礼云者,贵绝恶于未萌,而起教于微眇,使民日迁善远罪而不自知也。孔于曰:“听讼,吾犹人也,必也使毋讼乎!”为人主计者,莫如先审取舍,取舍之极定于内,而安危之萌应于外矣。安者非一日而安也,危者非一日而危也,皆以积渐然,不可不察也。人主之所积,在其取舍,以礼义治之者,积礼义;以刑罚治之者,积刑罚。刑罚积而民怨背,札义积而民和亲。故世主欲民之善同,而所以使民善者或异。或道之以德教,或殴之以法令。道之以德教者,德教洽而民气乐;殴之以法令者,法令极而民风哀。哀乐之感,祸福之应也。秦王之欲尊宗庙而安子孙,与汤武同,然而汤武广大其德行,六七百岁而弗失,秦王治天下,十余岁则大败。此亡它故矣,汤武之定取舍审而秦王之定取舍不审矣。夫天下,大器也。今人之置器,置诸安处则安,置诸危处则危。天下之情与器亡以异,在天子之所置之。汤武置天下于仁义礼乐,而德泽洽,禽兽草木广裕,德被蛮貊四夷,累子孙数十世,此天下所共闻也。秦王置天下于法令刑罚,德泽亡一有,而怨毒盈于世,下憎恶之如仇,祸几及身,子孙诛绝,此天下之所共见也。是非其明效大验邪!人之言曰:“听言之道,必以其事观之,则言者莫敢妄言。”今或言礼谊之不如法令,教化之不如刑罚,人主胡不引殷、周、秦事以观之也?


人主之尊譬如堂,群臣如陛,众庶如地。故陛九级上,廉远地,则堂高;陛亡级,廉近地,则堂卑。高者难攀,卑者易陵,理势然也。故古者圣王制为等列,内有公卿大夫士,外有公侯伯子男,然后有官师小吏,延及庶人,等级分明,而天子加焉,故其尊不可及也。里谚曰:“欲投鼠而忌器。”此善谕也。鼠近于器,尚惮不投,恐伤其器,况于贵臣之近主乎!廉耻节礼以治君子,故有赐死而亡戮辱。是以黥劓之罪不及太夫,以其离主上不远也,礼不敢齿君之路马,蹴其刍者有罚;见君之几杖则起,遭君之乘车则下,入正门则趋;君之宠臣虽或有过,刑戮之罪不加其身者,尊君之故也。此所以为主上豫远不敬也,所以体貌大臣而厉其节也。今自王侯三公之贵,皆天子之所改容而礼之也,古天子之所谓伯父、伯舅也,而令与众庶同黥劓刖笞弃市之法,然则堂不亡陛乎?被戮辱者不泰迫乎?廉耻不行,大臣无乃握重权,大官而有徒隶亡耻之心乎?夫望夷之事,二世见当以重法者,投鼠而不忌器之习也。


臣闻之,履虽鲜不加于枕,冠虽敝不以苴履。夫尝已在贵宠之位,天子改容而体貌之矣,吏民尝俯伏以敬畏之矣,今而有过,帝令废之可也,退之可也,赐之死可也,灭之可也;若夫束缚之,系緤之,输之司寇,编之徒官,司寇小吏詈骂而榜笞之,殆非所以令众庶见也。夫卑贱者习知尊贵者之一旦,吾亦乃可以加此也,非所以习天下也,非尊尊贵贵之化也。夫天子之所尝敬,众庶之所尝宠,死而死耳,贱人安宜得如此而顿辱之哉!


豫让事中行之君,智伯伐而灭之,移事智伯。及赵灭智伯,豫让衅面吞炭,必报襄子,五起而不中。人问豫子,豫子曰:“中行众人畜我,我故众人事之;智伯国士遇我,我故国士报之。”故此一豫让也,反君事仇,行若狗彘,已而抗节致忠,行出乎列士,人主使然也。故主上遇其大臣如遇犬马,彼将犬马自为也;如遇官徒,彼将官徒自为也。顽顿亡耻,诟亡节,廉耻不立,且不自好,苟若而可,故见利则逝,见便则夺。主上有败,则因而挺之矣;主上有患,则吾苟免而已,立而观之耳;有便吾身者,则欺卖而利之耳。人主将何便于此?群下至众,而主上至少也,所托财器职业者粹于群下也。俱亡耻,俱苟妄,则主上最病。故古者礼不及庶人,刑不至大夫,所以厉宠臣之节也。古者大臣有坐不廉而废者,不谓不廉,曰“簠簋不饰”;坐污秽淫乱男女亡别者,不曰污秽,曰“帷薄不修”,坐罢软不胜任者,不谓罢软,曰“下官不职”。故贵大臣定有其罪矣,犹未斥然正以呼之也,尚迁就而为之讳也。故其在大谴大何之域者,闻谴何则白冠缨,盘水加剑,造请室而请罪耳,上不执缚系引而行也。其有中罪者,闻命而自弛,上不使人颈而加也。其有大罪者,闻命则北面再拜,跌而自裁,上不使捽抑而刑之也,曰:“子大夫自有过耳!吾遇子有礼矣。”遇之有礼,故群臣自憙;婴以廉耻,故人矜节行。上设廉礼义以遇其臣,而臣不以节行报其上者,则非人类也。故化成俗定,则为人臣者主耳忘身,国耳忘家,公耳忘私,利不苟就,害不苟去,唯义所在。上之化也,故父兄之臣诚死宗庙,法度之臣诚死社稷,辅翼之臣诚死君上,守圄扞敌之臣诚死城郭封疆。故曰圣人有金城者,比物此志也。彼且为我死,故吾得与之俱生;彼且为我亡,故吾得与之俱存;夫将为我危,故吾得与之皆安。顾行而忘利,守节而仗义,故可以托不御之权,可以寄六尺之孤。此厉廉耻行礼谊之所致也,主上何丧焉!此之不为,而顾彼之久行,故曰可为长太息者此也。



\chapter*{季札观周乐}
\addcontentsline{toc}{chapter}{季札观周乐}
\begin{center}
	\textbf{[春秋战国]左丘明}
\end{center}


吴公子札来聘。……请观于周乐。使工为之歌《周南》、《召南》,曰:“美哉!始基之矣,犹未也,然勤而不怨矣。为之歌《邶》、《鄘》、《卫》,曰:“美哉,渊乎!忧而不困者也。吾闻卫康叔、武公之德如是,是其《卫风》乎?”为之歌《王》曰:“美哉!思而不惧,其周之东乎!”为之歌《郑》,曰:“美哉!其细已甚,民弗堪也。是其先亡乎!”为之歌《齐》,曰:“美哉,泱泱乎!大风也哉!表东海者,其大公乎?国未可量也。”为之歌《豳》,曰:“美哉,荡乎!乐而不淫,其周公之东乎?”为之歌《秦》,曰:“此之谓夏声。夫能夏则大,大之至也,其周之旧乎!”为之歌《魏》,曰:“美哉,渢渢乎!大而婉,险而易行,以德辅此,则明主也!”为之歌《唐》,曰:“思深哉!其有陶唐氏之遗民乎?不然,何忧之远也?非令德之后,谁能若是?”为.之歌《陈》,曰:“国无主,其能久乎!”自《郐》以下无讥焉!


为之歌《小雅》,曰。“美哉!思而不贰,怨而不言,其周德之衰乎?犹有先王之遗民焉!”为之歌《大雅》,曰:“广哉!熙熙乎!曲而有直体,其文王之德乎?”


为之歌《颂》,曰:“至矣哉!直而不倨,曲而不屈;迩而不逼,远而不携;迁而不淫,复而不厌;哀而不愁,乐而不荒;用而不匮,广而不宣;施而不费,取而不贪;处而不底,行而不流。五声和,八风平;节有度,守有序。盛德之所同也!”


见舞《象箾》、《南龠》者,曰:“美哉,犹有憾!”见舞《大武》者,曰:“美哉,周之盛也,其若此乎?”见舞《韶濩》者,曰:“圣人之弘也,而犹有惭德,圣人之难也!”见舞《大夏》者,曰:“美哉!勤而不德。非禹,其谁能修之!”见舞《韶箾》者“,曰:“德至矣哉!大矣,如天之无不帱也,如地之无不载也!虽甚盛德,其蔑以加于此矣。观止矣!若有他乐,吾不敢请已!”



\chapter*{剑阁铭}
\addcontentsline{toc}{chapter}{剑阁铭}
\begin{center}
	\textbf{[晋朝]张载}
\end{center}

岩岩梁山,积石峨峨。远属荆衡,近缀岷嶓。南通邛僰,北达褒斜。狭过彭碣,高逾嵩华。

惟蜀之门,作固作镇。是曰剑阁,壁立千仞。穷地之险,极路之峻。世浊则逆,道清斯顺。闭由往汉,开自有晋。

秦得百二,并吞诸侯。齐得十二,田生献筹。矧兹狭隘,土之外区。一人荷戟,万夫趑趄。形胜之地,匪亲勿居。

昔在武侯,中流而喜。山河之固,见屈吴起。兴实在德,险亦难恃。洞庭孟门,二国不祀。自古迄今,天命匪易。凭阻作昏,鲜不败绩。公孙既灭,刘氏衔璧。覆车之轨,无或重迹。勒铭山阿,敢告梁益。


\chapter*{咏雪}
\addcontentsline{toc}{chapter}{咏雪}
\begin{center}
	\textbf{[南北朝]刘义庆}
\end{center}

谢太傅寒雪日内集,与儿女讲论文义。俄而雪骤,公欣然曰:“白雪纷纷何所似?”兄子胡儿曰:“撒盐空中差可拟。”兄女曰:“未若柳絮因风起。”公大笑乐。即公大兄无奕女,左将军王凝之妻也。


\chapter*{驳复仇议}
\addcontentsline{toc}{chapter}{驳复仇议}
\begin{center}
	\textbf{[唐朝]柳宗元}
\end{center}


臣伏见天后时,有同州下邽人徐元庆者,父爽为县吏赵师韫所杀,卒能手刃父仇,束身归罪。当时谏臣陈子昂建议诛之而旌其闾;且请“编之于令,永为国典”。臣窃独过之。


臣闻礼之大本,以防乱也。若曰无为贼虐,凡为子者杀无赦。刑之大本,亦以防乱也。若曰无为贼虐,凡为理者杀无赦。其本则合,其用则异,旌与诛莫得而并焉。诛其可旌,兹谓滥;黩刑甚矣。旌其可诛,兹谓僭;坏礼甚矣。果以是示于天下,传于后代,趋义者不知所向,违害者不知所立,以是为典可乎?盖圣人之制,穷理以定赏罚,本情以正褒贬,统于一而已矣。


向使刺谳其诚伪,考正其曲直,原始而求其端,则刑礼之用,判然离矣。何者?若元庆之父,不陷于公罪,师韫之诛,独以其私怨,奋其吏气,虐于非辜,州牧不知罪,刑官不知问,上下蒙冒,吁号不闻;而元庆能以戴天为大耻,枕戈为得礼,处心积虑,以冲仇人之胸,介然自克,即死无憾,是守礼而行义也。执事者宜有惭色,将谢之不暇,而又何诛焉?


其或元庆之父,不免于罪,师韫之诛,不愆于法,是非死于吏也,是死于法也。法其可仇乎?仇天子之法,而戕奉法之吏,是悖骜而凌上也。执而诛之,所以正邦典,而又何旌焉?


且其议曰:“人必有子,子必有亲,亲亲相仇,其乱谁救?”是惑于礼也甚矣。礼之所谓仇者,盖其冤抑沉痛而号无告也;非谓抵罪触法,陷于大戮。而曰“彼杀之,我乃杀之”。不议曲直,暴寡胁弱而已。其非经背圣,不亦甚哉!


《周礼》:“调人,掌司万人之仇。凡杀人而义者,令勿仇;仇之则死。有反杀者,邦国交仇之。”又安得亲亲相仇也?《春秋公羊传》曰:“父不受诛,子复仇可也。父受诛,子复仇,此推刃之道,复仇不除害。”今若取此以断两下相杀,则合于礼矣。且夫不忘仇,孝也;不爱死,义也。元庆能不越于礼,服孝死义,是必达理而闻道者也。夫达理闻道之人,岂其以王法为敌仇者哉?议者反以为戮,黩刑坏礼,其不可以为典,明矣。


请下臣议附于令。有断斯狱者,不宜以前议从事。谨议。



\chapter*{寄欧阳舍人书}
\addcontentsline{toc}{chapter}{寄欧阳舍人书}
\begin{center}
	\textbf{[宋朝]曾巩}
\end{center}


巩顿首再拜,舍人先生:


去秋人还,蒙赐书及所撰先大父墓碑铭。反复观诵,感与惭并。夫铭志之著于世,义近于史,而亦有与史异者。盖史之于善恶,无所不书,而铭者,盖古之人有功德材行志义之美者,惧后世之不知,则必铭而见之。或纳于庙,或存于墓,一也。苟其人之恶,则于铭乎何有?此其所以与史异也。其辞之作,所以使死者无有所憾,生者得致其严。而善人喜于见传,则勇于自立;恶人无有所纪,则以愧而惧。至于通材达识,义烈节士,嘉言善状,皆见于篇,则足为后法。警劝之道,非近乎史,其将安近?


及世之衰,为人之子孙者,一欲褒扬其亲而不本乎理。故虽恶人,皆务勒铭,以夸后世。立言者既莫之拒而不为,又以其子孙之所请也,书其恶焉,则人情之所不得,于是乎铭始不实。后之作铭者,常观其人。苟托之非人,则书之非公与是,则不足以行世而传后。故千百年来,公卿大夫至于里巷之士,莫不有铭,而传者盖少。其故非他,托之非人,书之非公与是故也。


然则孰为其人而能尽公与是欤?非畜道德而能文章者,无以为也。盖有道德者之于恶人,则不受而铭之,于众人则能辨焉。而人之行,有情善而迹非,有意奸而外淑,有善恶相悬而不可以实指,有实大于名,有名侈于实。犹之用人,非畜道德者,恶能辨之不惑,议之不徇?不惑不徇,则公且是矣。而其辞之不工,则世犹不传,于是又在其文章兼胜焉。故曰,非畜道德而能文章者无以为也,岂非然哉!


然畜道德而能文章者,虽或并世而有,亦或数十年或一二百年而有之。其传之难如此,其遇之难又如此。若先生之道德文章,固所谓数百年而有者也。先祖之言行卓卓,幸遇而得铭,其公与是,其传世行后无疑也。而世之学者,每观传记所书古人之事,至其所可感,则往往衋然不知涕之流落也,况其子孙也哉?况巩也哉?其追睎祖德而思所以传之之繇,则知先生推一赐于巩而及其三世。其感与报,宜若何而图之?


抑又思若巩之浅薄滞拙,而先生进之,先祖之屯蹶否塞以死,而先生显之,则世之魁闳豪杰不世出之士,其谁不愿进于门?潜遁幽抑之士,其谁不有望于世?善谁不为,而恶谁不愧以惧?为人之父祖者,孰不欲教其子孙?为人之子孙者,孰不欲宠荣其父祖?此数美者,一归于先生。既拜赐之辱,且敢进其所以然。所谕世族之次,敢不承教而加详焉?愧甚,不宣。巩再拜。



\chapter*{人有亡斧者}
\addcontentsline{toc}{chapter}{人有亡斧者}
\begin{center}
	\textbf{[秦朝]吕不韦}
\end{center}


人有亡斧者,意其邻人之子。视其行步,窃斧也;视其颜色,窃斧也;听其言语,窃斧也;动作态度,无为而不窃斧者也。俄而掘其沟而得其斧,他日,复见其邻之子,其行动、颜色、动作皆无似窃斧者也。

\chapter*{臧僖伯谏观鱼}
\addcontentsline{toc}{chapter}{臧僖伯谏观鱼}
\begin{center}
	\textbf{[春秋战国]左丘明}
\end{center}


春,公将如棠观鱼者。臧僖伯谏曰:“凡物不足以讲大事,其材不足以备器用,则君不举焉。君将纳民于轨物者也。故讲事以度轨量,谓之‘轨’;取材以章物采,谓之‘物’。不轨不物,谓之乱政。乱政亟行,所以败也。故春蒐、夏苗、秋狝、冬狩,皆于农隙以讲事也。三年而治兵,入而振旅,归而饮至,以数军实。昭文章,明贵贱,辨等列,顺少长,习威仪也。鸟兽之肉不登于俎,皮革、齿牙、骨角、毛羽不登于器,则君不射,古之制也。若夫山林川泽之实,器用之资,皂隶之事,官司之守,非君所及也。”


公曰:“吾将略地焉。”遂往,陈鱼而观之。僖伯称疾不从。


书曰:“公矢鱼与棠。”非礼也,且言远地也。



\chapter*{驹支不屈于晋}
\addcontentsline{toc}{chapter}{驹支不屈于晋}
\begin{center}
	\textbf{[春秋战国]左丘明}
\end{center}


会于向,将执戎子驹支。范宣子亲数诸朝。曰:“来,姜戎氏。昔秦人迫逐乃祖吾离于瓜州,乃祖吾离被苫盖,蒙荆棘,以来归我先君。我先君惠公有不腆之田,与女剖分而食之。今诸侯之事我寡君不如昔者,盖言语漏泄,则职女之由。诘朝之事,尔无与焉!与,将执女。”


对曰:“昔秦人负恃其众,贪于土地,逐我诸戎。惠公蠲其大德,谓我诸戎是四岳之裔胄也,毋是翦弃。赐我南鄙之田,狐狸所居,豺狼所嗥。我诸戎除翦其荆棘,驱其狐狸豺狼,以为先君不侵不叛之臣,至于今不贰。昔文公与秦伐郑,秦人窃与郑盟而舍戍焉,于是乎有肴之师。晋御其上,戎亢其下,秦师不复,我诸戎实然。譬如捕鹿,晋人角之,诸戎掎之,与晋踣之,戎何以不免?自是以来,晋之百役,与我诸戎相继于时,以从执政,犹肴志也,岂敢离逷?令官之师旅,无乃实有所阙,以携诸侯,而罪我诸戎。我诸戎饮食衣服不与华同,贽币不通,言语不达,何恶之能为?不与于会,亦无瞢焉。”赋《青蝇》而退。


宣子辞焉,使即事于会,成恺悌也。



\chapter*{砚眼}
\addcontentsline{toc}{chapter}{砚眼}
\begin{center}
	\textbf{[明朝]冯梦龙}
\end{center}

明有陆庐峰者,于京城待用。尝于市遇一佳砚,议价未定。既还邸,使仆往,以一金易归。仆持砚归,公讶其不类。门人坚称其是。公曰:“前观砚有鸲鹆眼,今何无之?”答曰:“吾嫌其微凸,路值石工,幸有余银,令磨而平之。”公大惋惜。盖此砚佳处即在鸲鹆眼也。


\chapter*{难蜀父老}
\addcontentsline{toc}{chapter}{难蜀父老}
\begin{center}
	\textbf{[汉朝]司马相如}
\end{center}

汉兴七十有八载,德茂存乎六世,威武纷纭,湛恩汪濊(huì),群生澍濡,洋溢乎方外。于是乃命使西征,随流而攘(rǎng),风之所被,罔不披靡。因朝冉从駹(máng),定筰(zuó)存邛,略斯榆,举苞满,结轨还辕,东乡(xiǎng)将报,至于成都。


耆(qí)老大夫荐(jìn)绅先生之徒二十有七人,俨然造焉。辞毕,因进曰:“盖闻天子之于夷狄也,其义羁縻勿绝而已。今罢三郡之士,通夜郎之途,三年于兹而功不竟,士卒劳倦,万民不赡;今又接以西夷,百姓力屈,恐不能卒业,此亦使者之累也,窃为左右患之。且夫邛、筰、西僰(bó)之与中国并也,历年兹多不可记已。仁者不以德来,强者不以力并,意者其殆不可乎!今割齐民以附夷狄,弊所恃以事无用。鄙人固陋,不识所谓。”


使者曰:“乌谓此邪!必若所云,则是蜀不变服而巴不化俗也。余尚恶闻若说。然斯事体大,固非观者之所觏(gòu)也。余之行急,其详不可闻已。请为大夫粗陈其略:


“盖世必有非常之人,然后有非常之事;有非常之事,然后有非常之功。非常者,固常人之所异也。故曰非常之原,黎民惧焉;及臻厥成,天下晏如也。昔者洪水沸出,泛滥衍溢,人民登降移徙,崎岖而不安。夏后氏戚之,及堙洪水,决江疏河,洒沉赡菑(zāi),东归之于海,而天下永宁。当斯之勤,岂唯民哉?心烦于虑而身亲其劳,躬胝(zhī)无胈(bá),肤不生毛,故休烈显乎无穷,声称浃乎于兹。


“且夫贤君之践位也,岂特委琐握龊(chuò),拘文牵俗,循诵习传,当世取说云尔哉!必将崇论闳(hóng)议,创业垂统,为万世规。故驰鹜乎兼容并包,而勤思乎参天贰地。且《诗》不云乎,‘普天之下,莫非王土;率土之滨,莫非王臣。’是以六合之内,八方之外,浸浔衍溢,怀生之物有不浸润于泽者,贤君耻之。今封疆之内,冠带之伦,咸获嘉祉,靡有阙遗矣。而夷狄殊俗之国,辽接异党之地,舟舆不通,人迹罕至,政教未加,流风犹微。内之则犯义侵礼于边境,外之则邪行横作:放弑其上,君臣易位,尊卑失序,父兄不辜,幼孤为奴,系累号泣,内向而怨,曰:‘盖闻中国有至仁焉,德洋而恩普,物靡不得其所,今独曷为遗己!’举踵恩慕,若枯旱之望雨。盭(lì)夫为之垂涕,况乎上圣,又恶能已?故北出师以讨强胡,南驰使以诮劲越。四面风德,二方之君鳞集仰流,愿得受号者以亿计。故乃关沫若,徼(jiào)牂(zāng)牁(kē),镂灵山,梁孙原。创道德之途,垂仁义之统。将博恩广施,远抚长驾,使疏逖(tì)不闭,阻深暗昧,得耀乎光明,以偃甲兵于此,而息诛伐于彼。遐迩一体,中外褆(tí)福,不亦康乎?夫拯民于沉溺,奉至尊之休德,反衰世之陵迟,继周氏之绝业,斯乃天子之急务也。百姓虽劳,又恶可以已哉?


“且夫王事固未有不始于忧勤,而终于佚乐者也。然则受命之符合在于此矣。方将增泰山之封,加梁父之事,鸣和鸾,扬乐颂,上咸五,下登三。观者未睹指,闻者未闻音,犹鹪明已翔乎寥廓,而罗者犹视乎薮泽。悲夫!”


于是诸大夫芒然其所怀来,而失阙所以进,喟然并称曰:“允哉汉德,此鄙人之所愿闻也。百姓虽怠,请以身先之。”敞罔靡徙,因迁延而辞避。



\chapter*{信陵君救赵论}
\addcontentsline{toc}{chapter}{信陵君救赵论}
\begin{center}
	\textbf{[明朝]唐顺之}
\end{center}

论者以窃符为信陵君之罪,余以为此未足以罪信陵也。夫强秦之暴亟矣,今悉兵以临赵,赵必亡。赵,魏之障也。赵亡,则魏且为之后。赵、魏,又楚、燕、齐诸国之障也,赵、魏亡,则楚、燕、齐诸国为之后。天下之势,未有岌岌于此者也。故救赵者,亦以救魏;救一国者,亦以救六国也。窃魏之符以纾魏之患,借一国之师以分六国之灾,夫奚不可者?

然则信陵果无罪乎?曰:又不然也。余所诛者,信陵君之心也。

信陵一公子耳,魏固有王也。赵不请救于王,而谆谆焉请救于信陵,是赵知有信陵,不知有王也。平原君以婚姻激信陵,而信陵亦自以婚姻之故,欲急救赵,是信陵知有婚姻,不知有王也。其窃符也,非为魏也,非为六国也,为赵焉耳。非为赵也,为一平原君耳。使祸不在赵,而在他国,则虽撤魏之障,撤六国之障,信陵亦必不救。使赵无平原,而平原亦非信陵之姻戚,虽赵亡,信陵亦必不救。则是赵王与社稷之轻重,不能当一平原公子,而魏之兵甲所恃以固其社稷者,只以供信陵君一姻戚之用。幸而战胜,可也,不幸战不胜,为虏于秦,是倾魏国数百年社稷以殉姻戚,吾不知信陵何以谢魏王也。

夫窃符之计,盖出于侯生,而如姬成之也。侯生教公子以窃符,如姬为公子窃符于王之卧内,是二人亦知有信陵,不知有王也。余以为信陵之自为计,曷若以唇齿之势激谏于王,不听,则以其欲死秦师者而死于魏王之前,王必悟矣。侯生为信陵计,曷若见魏王而说之救赵,不听,则以其欲死信陵君者而死于魏王之前,王亦必悟矣。如姬有意于报信陵,曷若乘王之隙而日夜劝之救,不听,则以其欲为公子死者而死于魏王之前,王亦必悟矣。如此,则信陵君不负魏,亦不负赵;二人不负王,亦不负信陵君。何为计不出此?信陵知有婚姻之赵,不知有王。内则幸姬,外则邻国,贱则夷门野人,又皆知有公子,不知有王。则是魏仅有一孤王耳。

呜呼!自世之衰,人皆习于背公死党之行而忘守节奉公之道,有重相而无威君,有私仇而无义愤,如秦人知有穰侯,不知有秦王,虞卿知有布衣之交,不知有赵王,盖君若赘旒久矣。由此言之,信陵之罪,固不专系乎符之窃不窃也。其为魏也,为六国也,纵窃符犹可。其为赵也,为一亲戚也,纵求符于王,而公然得之,亦罪也。

虽然,魏王亦不得无罪也。兵符藏于卧内,信陵亦安得窃之?信陵不忌魏王,而径请之如姬,其素窥魏王之疏也;如姬不忌魏王,而敢于窃符,其素恃魏王之宠也。木朽而蛀生之矣。古者人君持权于上,而内外莫敢不肃。则信陵安得树私交于赵?赵安得私请救于信陵?如姬安得衔信陵之恩?信陵安得卖恩于如姬?履霜之渐,岂一朝一夕也哉!由此言之,不特众人不知有王,王亦自为赘旒也。

故信陵君可以为人臣植党之戒,魏王可以为人君失权之戒。《春秋》书葬原仲、翚帅师。嗟夫!圣人之为虑深矣!


\chapter*{西铭}
\addcontentsline{toc}{chapter}{西铭}
\begin{center}
	\textbf{[宋朝]张载}
\end{center}

乾称父,坤称母;予兹藐焉,乃混然中处。故天地之塞,吾其体;天地之帅,吾其性。民,吾同胞;物,吾与也。

大君者,吾父母宗子;其大臣,宗子之家相也。尊高年,所以长其长;慈孤弱,所以幼其幼;圣,其合德;贤,其秀也。凡天下疲癃、残疾、惸独、鳏寡,皆吾兄弟之颠连而无告者也。

于时保之,子之翼也;乐且不忧,纯乎孝者也。违曰悖德,害仁曰贼,济恶者不才,其践形,惟肖者也。

知化则善述其事,穷神则善继其志。不愧屋漏为无忝,存心养性为匪懈。恶旨酒,崇伯子之顾养;育英才,颍封人之锡类。不弛劳而厎豫,舜其功也;无所逃而待烹,申生其恭也。体其受而归全者,参乎!勇于从而顺令者,伯奇也。

富贵福泽,将厚吾之生也;贫贱忧戚,庸玉汝于成也。存,吾顺事;没,吾宁也。


\chapter*{言兵事疏}
\addcontentsline{toc}{chapter}{言兵事疏}
\begin{center}
	\textbf{[汉朝]晁错}
\end{center}

晁错,颍川人也。学申商刑名於轵张恢生所,与雒阳宋孟及刘带同师。以文学为太常掌故。

错为人峭直刻深。孝文时,天下亡治尚书者,独闻齐有伏生,故秦博士,治尚书,年九十余,老不可徵。乃诏太常,使人受之。太常遣错受尚书伏生所,还,因上书称说。诏以为太子舍人,门大夫,迁博士。又上书言:“人主所以尊显功名扬於万世之後者,以知术数也。故人主知所以临制臣下而治其众,则群臣畏服矣;知所以听言受事,则不欺蔽矣;知所以安利万民,则海内必从矣;知所以忠孝事上,则臣子之行备矣:此四者,臣窃为皇太子急之。人臣之议或曰皇太子亡以知事为也,臣之愚,诚以为不然。窃观上世之君,不能奉其宗庙而劫杀於其臣者,皆不知术数者也。(皇太子所读书多矣,而未深知术数者也。)皇太子所读书多矣,而未深知术数者,不问书说也。夫多诵而不知其说,所谓劳苦而不为功。臣窃观皇太子材智高奇,驭射伎艺过人绝远,然於术数未有所守者,以陛下为心也。窃愿陛下幸择圣人之术可用今世者,以赐皇太子,因时使太子陈明於前。唯陛下裁察。”上善之,於是拜错为太子家令。以其辩得幸太子,太子家号曰“智囊”。

是时匈奴强,数寇边,上发兵以御之。错上言兵事,曰:臣闻汉兴以来,胡虏数入边地,小入则小利,大入则大利;高后时再入陇西,攻城屠邑,敺略畜产;其後复入陇西,杀吏卒,大寇盗。窃闻战胜之威,民气百倍;败兵之卒,没世不复。自高后以来,陇西三困於匈奴矣,民气破伤,亡有胜意。今兹陇西之吏,赖社稷之神灵,奉陛下之明诏,和辑士卒,底厉其节,起破伤之民以当乘胜之匈奴,用少击众,杀一王,败其众而(法曰)大有利。非陇西之民有勇怯,乃将吏之制巧拙异也。故兵法曰:“有必胜之将,无必胜之民。”繇此观之,安边境,立功名,在於良将,不可不择也。

臣又闻用兵,临战合刃之急者三:一曰得地形,二曰卒服习,三曰器用利。兵法曰:丈五之沟,渐车之水,山林积石,经川丘阜,屮木所在,此步兵之地也,车骑二不当一。土山丘陵,曼衍相属,平原广野,此车骑之地,步兵十不当一。平陵相远,川谷居间,仰高临下,此弓弩之地也,短兵百不当一。两陈相近,平地浅(草)〔屮〕,可前可後,此长戟之地也,剑楯三不当一。(雚)〔萑〕苇竹萧,屮木蒙茏,支叶茂接,此矛鋋之地也,长戟二不当一。曲道相伏,险厄相薄,此剑楯之地也,弓弩三不当一。士不选练,卒不服习,起居不精,动静不集,趋利弗及,避难不毕,前击後解,与金鼓之(音)〔指〕相失,此不习勒卒之过也,百不当十。兵不完利,与空手同;甲不坚密,与袒裼同;弩不可以及远,与短兵同;射不能中,与亡矢同;中不能入,与亡镞同:此将不省兵之祸也,五不当一。故兵法曰:器械不利,以其卒予敌也;卒不可用,以其将予敌也;将不知兵,以其主予敌也;君不择将,以其国予敌也。四者,(国)〔兵〕之至要也。

臣又闻小大异形,强弱异势,险易异备。夫卑身以事强,小国之形也;合小以攻大,敌国之形也;以蛮夷攻蛮夷,中国之形也。今匈奴地形技艺与中国异。上下山阪,出入溪涧,中国之马弗与也;险道倾仄,且驰且射,中国之骑弗与也;风雨罢劳,饥渴不困,中国之人弗与也:此匈奴之长技也。若夫平原易地,轻车突骑,则匈奴之众易挠乱也;劲弩长戟,射疏及远,则匈奴之弓弗能格也;坚甲利刃,长短相杂,游弩往来,什伍俱前,则匈奴之兵弗能当也;材官驺发,矢道同的,则匈奴之革笥木荐弗能支也;下马地斗,剑戟相接,去就相薄,则匈奴之足弗能给也:此中国之长技也。以此观之,匈奴之长技三,中国之长技五。陛下又兴数十万之众,以诛数万之匈奴,众寡之计,以一击十之术也。

虽然,兵,凶器;战,危事也。以大为小,以强为弱,在俛卬之间耳。夫以人之死争胜,跌而不振,则悔之亡及也。帝王之道,出於万全。今降胡义渠蛮夷之属来归谊者,其众数千,饮食长技与匈奴同,可赐之坚甲絮衣,劲弓利矢,益以边郡之良骑。令明将能知其习俗和辑其心者,以陛下之明约将之。即有险阻,以此当之;平地通道,则以轻车材官制之。两军相为表里,各用其长技,衡加之以众,此万全之术也。

传曰:“狂夫之言,而明主择焉。”臣错愚陋,昧死上狂言,唯陛下财择。文帝嘉之,乃赐错玺书宠答焉,曰:“皇帝问太子家令:上书言兵体三章,闻之。书言”狂夫之言,而明主择焉“。今则不然。言者不狂,而择者不明,国之大患,故在於此。使夫不明择於不狂,是以万听而万不当也。”错复言守边备塞,劝农力本,当世急务二事,曰:臣闻秦时北攻胡貉,筑塞河上,南攻杨粤,置戍卒焉。其起兵而攻胡、粤者,非以卫边地而救民死也,贪戾而欲广大也,故功未立而天下乱。且夫起兵而不知其势,战则为人禽,屯则卒积死。夫胡貉之地,积阴之处也,木皮三寸,冰厚六尺,食肉而饮酪,其人密理,鸟兽毳毛,其性能寒。杨粤之地少阴多阳,其人疏理,鸟兽希毛,其性能暑。秦之戍卒不能其水土,戍者死於边,输者偾於道。秦民见行,如往弃市,因以谪发之,名曰“谪戍”。先发吏有谪及赘婿、贾人,後以尝有市籍者,又後以大父母、父母尝有市籍者,後入闾,取其左。发之不顺,行者深怨,有背畔之心。凡民守战至死而不降北者,以计为之也。故战胜守固则有拜爵之赏,攻城屠邑则得其财卤以富家室,故能使其众蒙矢石,赴汤火,视死如生。今秦之发卒也,有万死之害,而亡铢两之报,死事之後不得一算之复,天下明知祸烈及己也。陈胜行戍,至於大泽,为天下先倡,天下从之如流水者,秦以威劫而行之之敝也。

胡人衣食之业不着於地,其势易以扰乱边竟。何以明之?胡人食肉饮酪,衣皮毛,非有城郭田宅之归居,如飞鸟走兽於广野,美草甘水则止,草尽水竭则移。以是观之,往来转徙,时至时去,此胡人之生业,而中国之所以离南亩也。今使胡人数处转牧行猎於塞下,或当燕代,或当上郡、北地、陇西,以候备塞之卒,卒少则入。陛下不救,则边民绝望而有降敌之心;救之,少发则不足,多发,远县才至,则胡又已去。聚而不罢,为费甚大;罢之,则胡复入。如此连年,则中国贫苦而民不安矣。

陛下幸忧边境,遣将吏发卒以治塞,甚大惠也。然令远方之卒守塞,一岁而更,不知胡人之能,不如选常居者,家室田作,且以备之。以便为之高城深堑,具蔺石,布渠答,复为一城其内,城间百五十步。要害之处,通川之道,调立城邑,毋下千家,为中周虎落。先为室屋,具田器,乃募罪人及免徒复作令居之;不足,募以丁奴婢赎罪及输奴婢欲以拜爵者;不足,乃募民之欲往者。皆赐高爵,复其家。予冬夏衣,廪食,能自给而止。郡县之民得买其爵,以自增至卿。其亡夫若妻者,县官买予之。人情非有匹敌,不能久安其处。塞下之民,禄利不厚,不可使久居危难之地。胡人入驱而能止其所驱者,以其半予之,县官为赎其民。如是,则邑里相救助,赴胡不避死。非以德上也,欲全亲戚而利其财也。此与东方之(戎)〔戍〕卒不习地势而心畏胡者,功相万也。以陛下之时,徙民实边,使远方无屯戍之事,塞下之民父子相保,亡系虏之患,利施後世,名称圣明,其与秦之行怨民,相去远矣。

上从其言,募民徙塞下。

错复言:陛下幸募民相徙以实塞下,使屯戍之事益省,输将之费益寡,甚大惠也。下吏诚能称厚惠,奉明法,存恤所徙之老弱,善遇其壮士,和辑其心而勿侵刻,使先至者安乐而不思故乡,则贫民相募而劝往矣。臣闻古之徙远方以实广虚也,相其阴阳之和,尝其水泉之味,审其土地之宜,观其□木之饶,然後营邑立城,制里割宅,通田作之道,正阡陌之界,先为筑室,家有一堂二内,门户之闭,置器物焉,民至有所居,作有所用,此民所以轻去故乡而劝之新(色)〔邑〕也。为置医巫,以救疾病,以修祭祀,男女有昏,生死相恤,坟墓相从,种树畜长,室屋完安,此所以使民乐其处而有长居之心也。

臣又闻古之制边县以备敌也,使五家为伍,伍有长;十长一里,里有假士;四里一连,连有假五百;十连一邑,邑有假候:皆择其邑之贤材有护,习地形知民心者,居则习民於射法,出则教民於应敌。故卒伍成於内,则军正定於外。服习以成,勿令迁徙,幼则同游,长则共事。夜战声相知,则足以相救;昼战目相见,则足以相识;驩爱之心,足以相死。如此而劝以厚赏,威以重罚,则前死不还踵矣。所徙之民非壮有材力,但费衣粮,不可用也;虽有材力,不得良吏,犹亡功也。

陛下绝匈奴不与和亲,臣窃意其冬来南也,壹大治,则终身创矣。欲立威者,始於折胶,来而不能困,使得气去,後未易服也。愚臣亡识,唯陛下财察。

後诏有司举贤良文学士,错在选中。上亲策诏之,曰:惟十有五年九月壬子,皇帝曰:昔者大禹勤求贤士,施及方外,四极之内,舟车所至,人迹所及,靡不闻命,以辅其不逮;近者献其明,远者通厥聪,比善戮力,以翼天子。是以大禹能亡失德,夏以长楙。高皇帝亲除大害,去乱从,并建豪英,以为官师,为谏争,辅天子之阙,而翼戴汉宗也。赖天之灵,宗庙之福,方内以安,泽及四夷。今朕获执天子之正,以承宗庙之祀,朕既不德,又不敏,明弗能烛,而智不能治,此大夫之所着闻也。故诏有司、诸侯王、三公、九卿及主郡吏,各帅其志,以选贤良明於国家之大体,通於人事之终始,及能直言极谏者,各有人数,将以匡朕之不逮。二三大夫之行当此三道,朕甚嘉之,故登大夫于朝,亲谕朕志。大夫其上三道之要,及永惟朕之不德,吏之不平,政之不宣,民之不宁,四者之阙,悉陈其志,毋有所隐。上以荐先帝之宗庙,下以兴愚民之休利,着之于篇,朕亲览焉,观大夫所以佐朕,至与不至。书之,周之密之,重之闭之。兴自朕躬,大夫其正论,毋枉执事。乌虖,戒之!二三大夫其帅志毋怠!

错对曰:平阳侯臣窋、汝阴侯臣灶、颍阴侯臣何、廷尉臣宜昌、陇西太守臣昆邪所选贤良太子家令臣错昧死再拜言:臣窃闻古之贤主莫不求贤以为辅翼,故黄帝得力牧而为五帝〔先〕,大禹得咎繇而为三王祖,齐桓得筦子而为五伯长。今陛下讲于大禹及高皇帝之建豪英也,退托於不明,以求贤良,让之至也。臣窃观上世之传,若高皇帝之建功业,陛下之德厚而得贤佐,皆有司之所览,刻於玉版,藏於金匮,历之春秋,纪之後世,为帝者祖宗,与天地相终。今臣窋等乃以臣错充赋,甚不称明诏求贤之意。臣错□茅臣,亡识知,昧死上愚对,曰:诏策曰“明於国家大体”,愚臣窃以古之五帝明之。臣闻五帝神圣,其臣莫能及,故自亲事,处于法宫之中,明堂之上;动静上配天,下顺地,中得人。故众生之类亡不覆也,根着之徒亡不载也;烛以光明,亡偏异也;德上及飞鸟,下至水虫草木诸产,皆被其泽。然後阴阳调,四时节,日月光,风雨时,膏露降,五谷孰,祆孽灭,贼气息,民不疾疫,河出图,洛出书,神龙至,凤鸟翔,德泽满天下,灵光施四海。此谓配天地,治国大体之功也。

诏策曰“通於人事终始”,愚臣窃以古之三王明之。臣闻三王臣主俱贤,故合谋相辅,计安天下,莫不本於人情。人情莫不欲寿,三王生而不伤也;人情莫不欲富,三王厚而不困也;人情莫不欲安,三王扶而不危也;人情莫不欲逸,三王节其力而不尽也。其为法令也,合於人情而後行之;其动众使民也,本於人事然後为之。取人以己,内恕及人。情之所恶,不以强人;情之所欲,不以禁民。是以天下乐其政,归其德,望之若父母,从之若流水;百姓和亲,国家安宁,名位不失,施及後世。此明於人情终始之功也。

诏策曰“直言极谏”,愚臣窃以五伯之臣明之。臣闻五伯不及其臣,故属之以国,任之以事。五伯之佐之为人臣也,察身而不敢诬,奉法令不容私,尽心力不敢矜,遭患难不避死,见贤不居其上,受禄不过其量,不以亡能居尊显之位。自行若此,可谓方正之士矣。其立法也,非以苦民伤众而为之机陷也,以之兴利除害,尊主安民而救暴乱也。其行赏也,非虚取民财妄予人也,以劝天下之忠孝而明其功也。故功多者赏厚,功少者赏薄。如此,敛民财以顾其功,而民不恨者,知与而安己也。其行罚也,非以忿怒妄诛而从暴心也,以禁天下不忠不孝而害国者也。故罪大者罚重,罪小者罚轻。如此,民虽伏罪至死而不怨者,知罪罚之至,自取之也。立法若此,可谓平正之吏矣。法之逆者,请而更之,不以伤民;主行之暴者,逆而复之,不以伤国。救主之失,补主之过,扬主之美,明主之功,使主内亡邪辟之行,外亡骞污之名。事君若此,可谓直言极谏之士矣。此五伯之所以德匡天下,威正诸侯,功业甚美,名声章明。举天下之贤主,五伯与焉,此身不及其臣而使得直言极谏补其不逮之功也。今陛下人民之众,威武之重,德惠之厚,令行禁止之势,万万於五伯,而赐愚臣策曰“匡朕之不逮”,愚臣何足以识陛下之高明而奉承之!

诏策曰“吏之不平,政之不宣,民之不宁”,愚臣窃以秦事明之。臣闻秦始并天下之时,其主不及三王,而臣不及其佐,然功力不迟者,何也?地形便,山川利,财用足,民利战。其所与并者六国,六国者,臣主皆不肖,谋不辑,民不用,故当此之时,秦最富强。夫国富强而邻国乱者,帝王之资也,故秦能兼六国,立为天子。当此之时,三王之功不能进焉。及其末涂之衰也,任不肖而信谗贼;宫室过度,耆慾亡极,民力罢尽,赋敛不节;矜奋自贤,群臣恐谀,骄溢纵恣,不顾患祸;妄赏以随(善)〔喜〕意,妄诛以快怒心,法令烦憯,刑罚暴酷,轻绝人命,身自射杀;天下寒心,莫安其处。奸邪之吏,乘其乱法,以成其威,狱官主断,生杀自恣。上下瓦解,各自为制。秦始乱之时,吏之所先侵者,贫人贱民也;至其中节,所侵者富人吏家也;及其末涂,所侵者宗室大臣也。是故亲疏皆危,外内咸怨,离散逋逃,人有走心。陈胜先倡,天下大溃,绝祀亡世,为异姓福。此吏不平,政不宣,民不宁之祸也。今陛下配天象地,覆露万民,绝秦之迹,除其乱法;躬亲本事,废去淫末;除苛解娆,宽大爱人;肉刑不用,罪人亡帑;非谤不治,铸钱者除;通关去塞,不孽诸侯;宾礼长老,爱恤少孤;罪人有期,後宫出嫁;尊赐孝悌,农民不租;明诏军师,爱士大夫;求进方正,废退奸邪;除去阴刑,害民者诛;忧劳百姓,列侯就都;亲耕节用,视民不奢。所为天下兴利除害,变法易故,以安海内者,大功数十,皆上世之所难及,陛下行之,道纯德厚,元元之民幸矣。

诏策曰“永惟朕之不德”,愚臣不足以当之。

诏策曰“悉陈其志,毋有所隐”,愚臣窃以五帝之贤臣明之。臣闻五帝其臣莫能及,则自亲之;三王臣主俱贤,则共忧之;五伯不及其臣,则任使之。此所以神明不遗,而圣贤不废也,故各当其世而立功德焉。传曰“往者不可及,来者犹可待,能明其世者谓之天子”,此之谓也。窃闻战不胜者易其地,民贫穷者变其业。今以陛下神明德厚,资财不下五帝,临制天下,至今十有六年,民不益富,盗贼不衰,边竟未安,其所以然,意者陛下未之躬亲,而待群臣也。今执事之臣皆天下之选已,然莫能望陛下清光,譬之犹五帝之佐也。陛下不自躬亲,而待不望清光之臣,臣窃恐神明之遗也。日损一日,岁亡一岁,日月益暮,盛德不及究於天下,以传万世,愚臣不自度量,窃为陛下惜之。昧死上狂惑□茅之愚,臣言唯陛下财择。

时贾谊已死,对策者百余人,唯错为高第,繇是迁中大夫。

错又言宜削诸侯事,及法令可更定者,书凡三十篇。孝文虽不尽听,然奇其材。当是时,太子善错计策,爰盎诸大功臣多不好错。

景帝即位,以错为内史。错数请间言事,辄听,幸倾九卿,法令多所更定。丞相申屠嘉心弗便,力未有以伤。内史府居太上庙堧中,门东出,不便,错乃穿门南出,凿庙堧垣。丞相大怒,欲因此过为奏请诛错。错闻之,即请间为上言之。丞相奏事,因言错擅凿庙垣为门,请下廷尉诛。上曰:“此非庙垣,乃堧中垣,不致於法。”丞相谢。罢朝,因怒谓长史曰:“吾当先斩以闻,乃先请,固误。”丞相遂发病死。错以此愈贵。

迁为御史大夫,请诸侯之罪过,削其支郡。奏上,上〔令〕公卿列侯宗室〔杂议〕,莫敢难,独窦婴争之,繇此与错有隙。错所更令三十章,诸侯讙譁。错父闻之,从颍川来,谓错曰:“上初即位,公为政用事,侵削诸侯,疏人骨肉,口让多怨,公何为也!”错曰:“固也。不如此,天子不尊,宗庙不安。”父曰:“刘氏安矣,而晁氏危,吾去公归矣!”遂饮药死,曰:“吾不忍见祸逮身。”後十余日,吴楚七国俱反,以诛错为名。上与错议出军事,错欲令上自将兵,而身居守。会窦婴言爰盎,诏召入见,上方与错调兵食。上问盎曰:“君尝为吴相,知吴臣田禄伯为人虖?今吴楚反,於公意何如?”对曰:“不足忧也,今破矣。”上曰:“吴王即山铸钱,煮海为盐,诱天下豪桀,白头举事,此其计不百全,岂发虖?何以言其无能为也?”盎对曰:“吴铜盐之利则有之,安得豪桀而诱之!诚令吴得豪桀,亦且辅而为谊,不反矣。吴所诱,皆亡赖子弟,亡命铸钱奸人,故相诱以乱。”错曰:“盎策之善。”上问曰:“计安出?”盎对曰:“愿屏左右。”上屏人,独错在。盎曰:“臣所言,人臣不得知。”乃屏错。错趋避东箱,甚恨。上卒问盎,对曰:“吴楚相遗书,言高皇帝王子弟各有分地,今贼臣晁错擅适诸侯,削夺之地,以故反名为西共诛错,复故地而罢。方今计,独有斩错,发使赦吴楚七国,复其故地,则兵可毋血刃而俱罢。”於是上默然,良久曰:“顾诚何如,吾不爱一人谢天下。”盎曰:“愚计出此,唯上孰计之。”乃拜盎为太常,密装治行。

後十余日,丞相青翟、中尉嘉、廷尉□劾奏错曰:“吴王反逆亡道,欲危宗庙,天下所当共诛。今御史大夫错议曰:”兵数百万,独属群臣,不可信,陛下不如自出临兵,使错居守。徐、僮之旁吴所未下者可以予吴。“错不称陛下德信,欲疏群臣百姓,又欲以城邑予吴,亡臣子礼,大逆无道。错当要斩,父母妻子同产无少长皆弃巿。臣请论如法。”制曰:“可。”错殊不知。乃使中尉召错,绐载行巿。错衣朝衣斩东巿。

错已死,谒者仆射邓公为校尉,击吴楚为将。还,上书言军事,见上。上问曰:“道军所来,闻晁错死,吴楚罢不?”邓公曰:“吴为反数十岁矣,发怒削地,以诛错为名,其意不在错也。且臣恐天下之士拑口不敢复言矣。”上曰:“何哉?”邓公曰:“夫晁错患诸侯强大不可制,故请削之,以尊京师,万世之利也。计画始行,卒受大戮,内杜忠臣之口,外为诸侯报仇,臣窃为陛下不取也。”於景帝喟然长息,曰:“公言善,吾亦恨之。”乃拜邓公为城阳中尉。

邓公,成固人也,多奇计。建元年中,上招贤良,公卿言邓先。邓先时免,起家为九卿。一年,复谢病免归。其子章,以修黄老言显诸公间。

赞曰:爰盎虽不好学,亦善傅会,仁心为质,引义慷慨。遭孝文初立,资适逢世。时已变易,及吴壹说,果於用辩,身亦不遂。晁错锐於为国远虑,而不见身害。其父睹之,经於沟渎,亡益救败,不如赵母指括,以全其宗。悲夫!错虽不终,世哀其忠。故论其施行之语着于篇。


\chapter*{书洛阳名园记后}
\addcontentsline{toc}{chapter}{书洛阳名园记后}
\begin{center}
	\textbf{[宋朝]李格非}
\end{center}

洛阳处天下之中,挟崤渑之阻,当秦陇之襟喉,而赵魏之走集,盖四方必争之地也。天下当无事则已,有事,则洛阳先受兵。予故尝曰:“洛阳之盛衰,天下治乱之候也。”

方唐贞观、开元之间,公卿贵戚开馆列第于东都者,号千有余邸。及其乱离,继以五季之酷,其池塘竹树,兵车蹂践,废而为丘墟。高亭大榭,烟火焚燎,化而为灰烬,与唐俱灭而共亡,无馀处矣。予故尝曰:“园圃之废兴,洛阳盛衰之候也。”

且天下之治乱,候于洛阳之盛衰而知;洛阳之盛衰,候于园圃之废兴而得。则《名园记》之作,予岂徒然哉?

呜呼!公卿大夫方进于朝,放乎一己之私以自为,而忘天下之治忽,欲退享此乐,得乎?唐之末路是已。(唐之末路是已一作:矣)


\chapter*{亲政篇}
\addcontentsline{toc}{chapter}{亲政篇}
\begin{center}
	\textbf{[明朝]王鏊}
\end{center}

《易》之《泰》:“上下交而其志同。”其《否》曰:“上下不交而天下无邦。”盖上之情达于下,下之情达于上,上下一体,所以为“泰”。下之情壅阏而不得上闻,上下间隔,虽有国而无国矣,所以为“否”也。

交则泰,不交则否,自古皆然,而不交之弊,未有如近世之甚者。君臣相见,止于视朝数刻;上下之间,章奏批答相关接,刑名法度相维持而已。非独沿袭故事,亦其地势使然。何也?国家常朝于奉天门,未尝一日废,可谓勤矣。然堂陛悬绝,威仪赫奕,御史纠仪,鸿胪举不如法,通政司引奏,上特视之,谢恩见辞,湍湍而退,上何尝治一事,下何尝进一言哉?此无他,地势悬绝,所谓堂上远于万里,虽欲言无由言也。

愚以为欲上下之交,莫若复古内朝之法。盖周之时有三朝:库门之外为正朝,询谋大臣在焉;路门之外为治朝,日视朝在焉;路门之内为内朝,亦曰燕朝。《玉藻》云:“君日出而视朝,退视路寝听政。”盖视朝而见群臣,所以正上下之分;听政而视路寝,所以通远近之情。汉制:大司马、左右前后将军、侍中、散骑诸吏为中朝,丞相以下至六百石为外朝。唐皇城之北南三门曰承天,元正、冬至受万国之朝贡,则御焉,盖古之外朝也。其北曰太极门,其西曰太极殿,朔、望则坐而视朝,盖古之正朝也。又北曰两仪殿,常日听朝而视事,盖古之内朝也。宋时常朝则文德殿,五日一起居则垂拱殿,正旦、冬至、圣节称贺则大庆殿,赐宴则紫宸殿或集英殿,试进士则崇政殿。侍从以下,五日一员上殿,谓之轮对,则必入陈时政利害。内殿引见,亦或赐坐,或免穿靴,盖亦有三朝之遗意焉。盖天有三垣,天子象之。正朝,象太极也;外朝,象天市也;内朝,象紫微也。自古然矣。

国朝圣节、冬至、正旦大朝则会奉天殿,即古之正朝也。常日则奉天门,即古之外朝也。而内朝独缺。然非缺也,华盖、谨身、武英等殿,岂非内朝之遗制乎?洪武中如宋濂、刘基,永乐以来如杨士奇、杨荣等,日侍左右,大臣蹇义、夏元吉等,常奏对便殿。于斯时也,岂有壅隔之患哉?今内朝未复,临御常朝之后,人臣无复进见,三殿高閟,鲜或窥焉。故上下之情,壅而不通;天下之弊,由是而积。孝宗晚年,深感有慨于斯,屡召大臣于便殿,讲论天下事。方将有为,而民之无禄,不及睹至治之美,天下至今以为恨矣。

惟陛下远法圣祖,近法孝宗,尽铲近世壅隔之弊。常朝之外,即文华、武英二殿,仿古内朝之意,大臣三日或五日一次起居,侍从、台谏各一员上殿轮对;诸司有事咨决,上据所见决之,有难决者,与大臣面议之;不时引见群臣,凡谢恩辞见之类,皆得上殿陈奏。虚心而问之,和颜色而道之,如此,人人得以自尽。陛下虽身居九重,而天下之事灿然毕陈于前。外朝所以正上下之分,内朝所以通远近之情。如此,岂有近时壅隔之弊哉?唐、虞之时,明目达聪,嘉言罔伏,野无遗贤,亦不过是而已。


\chapter*{青霞先生文集序}
\addcontentsline{toc}{chapter}{青霞先生文集序}
\begin{center}
	\textbf{[明朝]茅坤}
\end{center}

青霞沈君,由锦衣经历上书诋宰执,宰执深疾之。方力构其罪,赖明天子仁圣,特薄其谴,徙之塞上。当是时,君之直谏之名满天下。已而,君纍然携妻子,出家塞上。会北敌数内犯,而帅府以下,束手闭垒,以恣敌之出没,不及飞一镞以相抗。甚且及敌之退,则割中土之战没者与野行者之馘以为功。而父之哭其子,妻之哭其夫,兄之哭其弟者,往往而是,无所控吁。君既上愤疆埸之日弛,而又下痛诸将士之日菅刈我人民以蒙国家也,数呜咽欷歔;,而以其所忧郁发之于诗歌文章,以泄其怀,即集中所载诸什是也。

君故以直谏为重于时,而其所著为诗歌文章,又多所讥刺,稍稍传播,上下震恐。始出死力相煽构,而君之祸作矣。君既没,而中朝之士虽不敢讼其事,而一时阃寄所相与谗君者,寻且坐罪罢去。又未几,故宰执之仇君者亦报罢。而君之故人俞君,于是裒辑其生平所著若干卷,刻而传之。而其子襄,来请予序之首简。

茅子受读而题之曰:若君者,非古之志士之遗乎哉?孔子删《诗》,自《小弁》之怨亲,《巷伯》之刺谗而下,其间忠臣、寡妇、幽人、怼士之什,并列之为“风”,疏之为“雅”,不可胜数。岂皆古之中声也哉?然孔子不遽遗之者,特悯其人,矜其志。犹曰“发乎情,止乎礼义”,“言之者无罪,闻之者足以为戒”焉耳。予尝按次春秋以来,屈原之《骚》疑于怨,伍胥之谏疑于胁,贾谊之《疏》疑于激,叔夜之诗疑于愤,刘蕡之对疑于亢。然推孔子删《诗》之旨而裒次之,当亦未必无录之者。君既没,而海内之荐绅大夫,至今言及君,无不酸鼻而流涕。呜呼!集中所载《鸣剑》、《筹边》诸什,试令后之人读之,其足以寒贼臣之胆,而跃塞垣战士之马,而作之忾也,固矣!他日国家采风者之使出而览观焉,其能遗之也乎?予谨识之。

至于文词之工不工,及当古作者之旨与否,非所以论君之大者也,予故不著。嘉靖癸亥孟春望日归安茅坤拜手序。


\chapter*{陈太丘与友期}
\addcontentsline{toc}{chapter}{陈太丘与友期}
\begin{center}
	\textbf{[南北朝]刘义庆}
\end{center}

陈太丘与友期行,期日中,过中不至,太丘舍去,去后乃至。元方时年七岁,门外戏。客问元方:“尊君在不?”答曰:“待君久不至,已去。”友人便怒:“非人哉!与人期行,相委而去。”元方曰:“君与家君期日中。日中不至,则是无信;对子骂父,则是无礼。”友人惭,下车引之,元方入门不顾。


\chapter*{刑赏忠厚之至论}
\addcontentsline{toc}{chapter}{刑赏忠厚之至论}
\begin{center}
	\textbf{[宋朝]苏轼}
\end{center}

尧、舜、禹、汤、文、武、成、康之际,何其爱民之深,忧民之切,而待天下以君子长者之道也。有一善,从而赏之,又从而咏歌嗟叹之,所以乐其始而勉其终。有一不善,从而罚之,又从而哀矜惩创之,所以弃其旧而开其新。故其吁俞之声,欢休惨戚,见于虞、夏、商、周之书。成、康既没,穆王立,而周道始衰,然犹命其臣吕侯,而告之以祥刑。其言忧而不伤,威而不怒,慈爱而能断,恻然有哀怜无辜之心,故孔子犹有取焉。《传》曰:“赏疑从与,所以广恩也;罚疑从去,所以慎刑也。”


当尧之时,皋陶为士。将杀人,皋陶曰“杀之”三,尧曰“宥之”三。故天下畏皋陶执法之坚,而乐尧用刑之宽。四岳曰“鲧可用”,尧曰“不可,鲧方命圮族”,既而曰:“试之”。何尧之不听皋陶之杀人,而从四岳之用鲧也?然则圣人之意,盖亦可见矣。《书》曰:“罪疑惟轻,功疑惟重。与其杀不辜,宁失不经。”呜呼,尽之矣。


可以赏,可以无赏,赏之过乎仁;可以罚,可以无罚,罚之过乎义。过乎仁,不失为君子;过乎义,则流而入于忍人。故仁可过也,义不可过也。古者赏不以爵禄,刑不以刀锯。赏之以爵禄,是赏之道行于爵禄之所加,而不行于爵禄之所不加也。刑之以刀锯,是刑之威施于刀锯之所及,而不施于刀锯之所不及也。先王知天下之善不胜赏,而爵禄不足以劝也;知天下之恶不胜刑,而刀锯不足以裁也。是故疑则举而归之于仁,以君子长者之道待天下,使天下相率而归于君子长者之道。故曰:忠厚之至也。


《诗》曰:“君子如祉,乱庶遄已。君子如怒,乱庶遄沮。”夫君子之已乱,岂有异术哉?时其喜怒,而无失乎仁而已矣。《春秋》之义,立法贵严,而责人贵宽。因其褒贬之义,以制赏罚,亦忠厚之至也。



\chapter*{荀巨伯探病友}
\addcontentsline{toc}{chapter}{荀巨伯探病友}
\begin{center}
	\textbf{[南北朝]刘义庆}
\end{center}

荀巨伯远看友人疾值胡贼攻郡,友人语巨伯曰:“吾今死矣,子可去。”巨伯曰:“远来相视,子令吾去,败义以求生,岂荀巨伯所行邪?”贼既至,谓巨伯曰:“大军至,一郡尽空,汝何男子,而敢独止?“巨伯曰:“友人有疾,不忍委之,宁以我身代友人命。”贼相谓曰:“我辈无义之人,而入有义之国。”遂班军而还,一郡并获全。


\chapter*{义田记}
\addcontentsline{toc}{chapter}{义田记}
\begin{center}
	\textbf{[宋朝]钱公辅}
\end{center}

范文正公,苏人也,平生好施与,择其亲而贫,疏而贤者,咸施之。

方贵显时,置负郭常稔之田千亩,号曰义田,以养济群族之人。日有食,岁有衣,嫁娶凶葬,皆有赡。择族之长而贤者主其计,而时共出纳焉。日食人一升,岁衣人一缣,嫁女者五十千,再嫁者三十千,娶妇者三十千,再娶者十五千,葬者如再嫁之数,葬幼者十千。族之聚者九十口,岁入给稻八百斛。以其所入,给其所聚,沛然有余而无穷。屏而家居俟代者与焉;仕而居官者罢其给。此其大较也。

初,公之未贵显也,尝有志于是矣,而力未逮者二十年。既而为西帅,及参大政,于是始有禄赐之入,而终其志。公既殁,后世子孙修其业,承其志,如公之存也。公虽位充禄厚,而贫终其身。殁之日,身无以为敛,子无以为丧,唯以施贫活族之义,遗其子而已。

昔晏平仲敝车羸马,桓子曰:「是隐君之赐也。」晏子曰:「自臣之贵,父之族,无不乘车者;母之族,无不足于衣食者;妻之族,无冻馁者;齐国之士,待臣而举火者,三百余人。以此而为隐君之赐乎?彰君之赐乎?」于是齐侯以晏子之觞而觞桓子。予尝爱晏子好仁,齐侯知贤,而桓子服义也。又爱晏子之仁有等级,而言有次也;先父族,次母族,次妻族,而后及其疏远之贤。孟子曰:「亲亲而仁民,仁民而爱物。」晏子为近之。观文正之义,贤于平仲,其规模远举又疑过之。

呜呼!世之都三公位,享万锺禄,其邸第之雄,车舆之饰,声色之多,妻孥之富,止乎一己而已,而族之人不得其门而入者,岂少也哉!况于施贤乎!其下为卿,为大夫,为士,廪稍之充,奉养之厚,止乎一己而已;而族之人操瓢囊为沟中瘠者,又岂少哉?况于他人乎!是皆公之罪人也。

公之忠义满朝廷,事业满边隅,功名满天下,后必有史官书之者,予可无录也。独高其义,因以遗于世云。


\chapter*{周郑交质}
\addcontentsline{toc}{chapter}{周郑交质}
\begin{center}
	\textbf{[春秋战国]左丘明}
\end{center}


郑武公、庄公为平王卿士。王贰于虢,郑伯怨王。王曰:“无之。”故周郑交质。王子狐为质于郑,郑公子忽为质于周。


王崩,周人将畀虢公政。四月,郑祭足帅师取温之麦。秋,又取成周之禾。周郑交恶。


君子曰:“信不由中,质无益也。明恕而行,要之以礼,虽无有质,谁能间之?苟有明信,涧溪沼沚之毛,苹蘩蕴藻之菜,筐筥錡釜之器,潢污行潦之水,可荐於鬼神,可羞於王公,而况君子结二国之信,行之以礼,又焉用质?《风》有《采蘩》、《采苹》,《雅》有《行苇》、《泂酌》,昭忠信也。”



\chapter*{答客难}
\addcontentsline{toc}{chapter}{答客难}
\begin{center}
	\textbf{[汉朝]东方朔}
\end{center}


客难东方朔曰:“苏秦、张仪一当万乘之主,而身都卿相之位,泽及后世。今子大夫修先王之术,慕圣人之义,讽诵诗书百家之言,不可胜记,著于竹帛;唇腐齿落,服膺而不可释,好学乐道之效,明白甚矣;自以为智能海内无双,则可谓博闻辩智矣。然悉力尽忠,以事圣帝,旷日持久,积数十年,官不过侍郎,位不过执戟。意者尚有遗行邪?同胞之徒,无所容居,其故何也?”


东方先生喟然长息,仰而应之曰:“是故非子之所能备。彼一时也,此一时也,岂可同哉?夫苏秦、张仪之时,周室大坏,诸侯不朝,力政争权,相擒以兵,并为十二国,未有雌雄。得士者强,失士者亡,故说得行焉。身处尊位,珍宝充内,外有仓麋,泽及后世,子孙长享。今则不然:圣帝德流,天下震慑,诸侯宾服,连四海之外以为带,安于覆盂;天下平均,合为一家,动发举事,犹运之掌,贤与不肖何以异哉?遵天之道,顺地之理,物无不得其所;故绥之则安,动之则苦;尊之则为将,卑之则为虏;抗之则在青云之上,抑之则在深渊之下;用之则为虎,不用则为鼠;虽欲尽节效情,安知前后?夫天地之大,士民之众,竭精驰说,并进辐凑者,不可胜数;悉力慕之,困于衣食,或失门户。使苏秦、张仪与仆并生于今之世,曾不得掌故,安敢望侍郎乎!传曰:‘天下无害,虽有圣人,无所施才;上下和同,虽有贤者,无所立功。’故曰:时异事异。


“虽然,安可以不务修身乎哉!《诗》曰:‘鼓钟于宫,声闻于外。’‘鹤鸣九皋,声闻于天’。苟能修身,何患不荣!太公体行仁义,七十有二,乃设用于文武,得信厥说。封于齐,七百岁而不绝。此士所以日夜孳孳,修学敏行,而不敢怠也。譬若鹡鸰,飞且鸣矣。传曰:‘天不为人之恶寒而辍其冬,地不为人之恶险而辍其广,君子不为小人之匈匈而易其行。’‘天有常度,地有常形,君子有常行;君子道其常,小人计其功。”诗云:‘礼义之不愆,何恤人之言?’水至清则无鱼,人至察则无徒;冕而前旒,所以蔽明;黈纩充耳,所以塞聪。明有所不见,聪有所不闻,举大德,赦小过,无求备于一人之义也。枉而直之,使自得之;优而柔之,使自求之;揆而度之,使自索之。盖圣人之教化如此,欲其自得之;自得之,则敏且广矣。


“今世之处士,时虽不用,块然无徒,廓然独居;上观许山,下察接舆;计同范蠡,忠合子胥;天下和平,与义相扶,寡偶少徒,固其宜也。子何疑于予哉?若大燕之用乐毅,秦之任李斯,郦食其之下齐,说行如流,曲从如环;所欲必得,功若丘山;海内定,国家安;是遇其时者也,子又何怪之邪?语曰:‘以管窥天,以蠡测海,以筵撞钟,’岂能通其条贯,考其文理,发其音声哉?犹是观之,譬由鼱鼩之袭狗,孤豚之咋虎,至则靡耳,何功之有?今以下愚而非处士,虽欲勿困,固不得已,此适足以明其不知权变,而终惑于大道也。”



\chapter*{卖柑者言}
\addcontentsline{toc}{chapter}{卖柑者言}
\begin{center}
	\textbf{[明朝]刘基}
\end{center}


杭有卖果者,善藏柑,涉寒暑不溃。出之烨然,玉质而金色。置于市,贾十倍,人争鬻之。


予贸得其一,剖之,如有烟扑口鼻,视其中,则干若败絮。予怪而问之曰:“若所市于人者,将以实笾豆,奉祭祀,供宾客乎?将炫外以惑愚瞽也?甚矣哉,为欺也!”


卖者笑曰:“吾业是有年矣,吾赖是以食吾躯。吾售之,人取之,未尝有言,而独不足子所乎?世之为欺者不寡矣,而独我也乎?吾子未之思也。


今夫佩虎符、坐皋比者,洸洸乎干城之具也,果能授孙、吴之略耶?峨大冠、拖长绅者,昂昂乎庙堂之器也,果能建伊、皋之业耶?盗起而不知御,民困而不知救,吏奸而不知禁,法斁而不知理,坐糜廪粟而不知耻。观其坐高堂,骑大马,醉醇醴而饫肥鲜者,孰不巍巍乎可畏,赫赫乎可象也?又何往而不金玉其外,败絮其中也哉?今子是之不察,而以察吾柑!”


予默默无以应。退而思其言,类东方生滑稽之流。岂其愤世疾邪者耶?而托于柑以讽耶?



\chapter*{宫之奇谏假道}
\addcontentsline{toc}{chapter}{宫之奇谏假道}
\begin{center}
	\textbf{[春秋战国]左丘明}
\end{center}


晋侯复假道于虞以伐虢。


宫之奇谏曰:“虢,虞之表也。虢亡,虞必从之。晋不可启,寇不可翫。一之谓甚,其可再乎?谚所谓‘辅车相依,唇亡齿寒’者,其虞、虢之谓也。”


公曰:“晋,吾宗也,岂害我哉?”


对曰:“大伯、虞仲,大王之昭也。大伯不从,是以不嗣。虢仲、虢叔,王季之穆也,为文王卿士,勋在王室,藏于盟府。将虢是灭,何爱于虞!且虞能亲于桓、庄乎?其爱之也,桓、庄之族何罪?而以为戮,不唯逼乎?亲以宠逼,犹尚害之,况以国乎?”


公曰:“吾享祀丰洁,神必据我。”


对曰:“臣闻之,鬼神非人实亲,惟德是依。故《周书》曰:‘皇天无亲,惟德是辅。’又曰:‘黍稷非馨,明德惟馨。’又曰:‘民不易物,惟德繄物。’如是,则非德,民不和,神不享矣。神所冯依,将在德矣。若晋取虞,而明德以荐馨香,神其吐之乎?”


弗听,许晋使。


宫之奇以其族行,曰:“虞不腊矣。在此行也,晋不更举矣。”


八月甲午,晋侯围上阳,问于卜偃曰:“吾其济乎?”


对曰:“克之。”


公曰:“何时?”


对曰:“童谣曰:‘丙之晨,龙尾伏辰,均服振振,取虢之旂。鹑之贲贲,天策炖炖,火中成军,虢公其奔。’其九月、十月之交乎!丙子旦,日在尾,月在策,鹑火中,必是时也。”


冬,十二月丙子朔,晋灭虢,虢公丑奔京师。师还,馆于虞,遂袭虞,灭之,执虞公.及其大夫井伯,从媵秦穆姬。而修虞祀,且归其职贡于王,故书曰:“晋人执虞公。”罪虞公,言易也。



\chapter*{答李翊书}
\addcontentsline{toc}{chapter}{答李翊书}
\begin{center}
	\textbf{[唐朝]韩愈}
\end{center}

六月二十六日,愈白。李生足下:生之书辞甚高,而其问何下而恭也。能如是,谁不欲告生以其道?道德之归也有日矣,况其外之文乎?抑愈所谓望孔子之门墙而不入于其宫者,焉足以知是且非邪?虽然,不可不为生言之。

生所谓“立言”者,是也;生所为者与所期者,甚似而几矣。抑不知生之志:蕲胜于人而取于人邪?将蕲至于古之立言者邪?蕲胜于人而取于人,则固胜于人而可取于人矣!将蕲至于古之立言者,则无望其速成,无诱于势利,养其根而俟其实,加其膏而希其光。根之茂者其实遂,膏之沃者其光晔。仁义之人,其言蔼如也。

抑又有难者。愈之所为,不自知其至犹未也;虽然,学之二十余年矣。始者,非三代两汉之书不敢观,非圣人之志不敢存。处若忘,行若遗,俨乎其若思,茫乎其若迷。当其取于心而注于手也,惟陈言之务去,戛戛乎其难哉!其观于人,不知其非笑之为非笑也。如是者亦有年,犹不改。然后识古书之正伪,与虽正而不至焉者,昭昭然白黑分矣,而务去之,乃徐有得也。

当其取于心而注于手也,汩汩然来矣。其观于人也,笑之则以为喜,誉之则以为忧,以其犹有人之说者存也。如是者亦有年,然后浩乎其沛然矣。吾又惧其杂也,迎而距之,平心而察之,其皆醇也,然后肆焉。虽然,不可以不养也,行之乎仁义之途,游之乎诗书之源,无迷其途,无绝其源,终吾身而已矣。

气,水也;言,浮物也。水大而物之浮者大小毕浮。气之与言犹是也,气盛则言之短长与声之高下者皆宜。虽如是,其敢自谓几于成乎?虽几于成,其用于人也奚取焉?虽然,待用于人者,其肖于器邪?用与舍属诸人。君子则不然。处心有道,行己有方,用则施诸人,舍则传诸其徒,垂诸文而为后世法。如是者,其亦足乐乎?其无足乐也?

有志乎古者希矣,志乎古必遗乎今。吾诚乐而悲之。亟称其人,所以劝之,非敢褒其可褒而贬其可贬也。问于愈者多矣,念生之言不志乎利,聊相为言之。愈白。


\chapter*{晏子使楚}
\addcontentsline{toc}{chapter}{晏子使楚}
\begin{center}
	\textbf{[汉朝]刘向}
\end{center}


一

晏子使楚。楚人以晏子短,楚人为小门于大门之侧而延晏子。晏子不入,曰:“使狗国者从狗门入,今臣使楚,不当从此门入。”傧者更道,从大门入。见楚王。王曰:“齐无人耶?”晏子对曰:“齐之临淄三百闾,张袂成阴,挥汗成雨,比肩继踵而在,何为无人?”王曰:“然则何为使予?”晏子对曰:“齐命使,各有所主:其贤者使使贤主,不肖者使使不肖主。婴最不肖,故宜使楚矣!”(张袂成阴一作:张袂成帷)


二

晏子将使楚。楚王闻之,谓左右曰:“晏婴,齐之习辞者也。今方来,吾欲辱之,何以也?”左右对曰:“为其来也,臣请缚一人,过王而行,王曰:‘何为者也?’对曰:‘齐人也。’王曰:‘何坐?’曰:‘坐盗。’


三

晏子至,楚王赐晏子酒,酒酣,吏二缚一人诣王。王曰:“缚者曷为者也?”对曰:“齐人也,坐盗。”王视晏子曰:“齐人固善盗乎?”晏子避席对曰:“婴闻之,橘生淮南则为橘,生于淮北则为枳,叶徒相似,其实味不同。所以然者何?水土异也。今民生长于齐不盗,入楚则盗,得无楚之水土使民善盗耶?”王笑曰:“圣人非所与熙也,寡人反取病焉。”



\chapter*{送天台陈庭学序}
\addcontentsline{toc}{chapter}{送天台陈庭学序}
\begin{center}
	\textbf{[明朝]宋濂}
\end{center}


西南山水,惟川蜀最奇。然去中州万里,陆有剑阁栈道之险,水有瞿塘、滟滪之虞。跨马行,则篁竹间山高者,累旬日不见其巅际。临上而俯视,绝壑万仞,杳莫测其所穷,肝胆为之悼栗。水行,则江石悍利,波恶涡诡,舟一失势尺寸,辄糜碎土沉,下饱鱼鳖。其难至如此。故非仕有力者,不可以游;非材有文者,纵游无所得;非壮强者,多老死于其地。嗜奇之士恨焉。


天台陈君庭学,能为诗,由中书左司掾,屡从大将北征,有劳,擢四川都指挥司照磨,由水道至成都。成都,川蜀之要地,扬子云、司马相如、诸葛武侯之所居,英雄俊杰战攻驻守之迹,诗人文士游眺饮射赋咏歌呼之所,庭学无不历览。既览必发为诗,以纪其景物时世之变,于是其诗益工。越三年,以例自免归,会予于京师;其气愈充,其语愈壮,其志意愈高;盖得于山水之助者侈矣。


予甚自愧,方予少时,尝有志于出游天下,顾以学未成而不暇。及年壮方可出,而四方兵起,无所投足。逮今圣主兴而宇内定,极海之际,合为一家,而予齿益加耄矣。欲如庭学之游,尚可得乎?


然吾闻古之贤士,若颜回、原宪,皆坐守陋室,蓬蒿没户,而志意常充然,有若囊括于天地者。此其故何也?得无有出于山水之外者乎?庭学其试归而求焉?苟有所得,则以告予,予将不一愧而已也!



\chapter*{送石昌言使北引}
\addcontentsline{toc}{chapter}{送石昌言使北引}
\begin{center}
	\textbf{[宋朝]苏洵}
\end{center}


昌言举进士时,吾始数岁,未学也。忆与群儿戏先府君侧,昌言从旁取枣栗啖我;家居相近,又以亲戚故,甚狎。昌言举进士,日有名。吾后渐长,亦稍知读书,学句读、属对、声律,未成而废。昌言闻吾废学,虽不言,察其意,甚恨。后十余年,昌言及第第四人,守官四方,不相闻。吾日益壮大,乃能感悔,摧折复学。又数年,游京师,见昌言长安,相与劳问,如平生欢。出文十数首,昌言甚喜称善。吾晚学无师,虽日当文,中甚自惭;及闻昌言说,乃颇自喜。今十余年,又来京师,而昌言官两制,乃为天子出使万里外强悍不屈之虏庭,建大旆,从骑数百,送车千乘,出都门,意气慨然。自思为儿时,见昌言先府君旁,安知其至此?富贵不足怪,吾于昌言独有感也!大丈夫生不为将,得为使,折冲口舌之间足矣。


往年彭任从富公使还,为我言曰:“既出境,宿驿亭。闻介马数万骑驰过,剑槊相摩,终夜有声,从者怛然失色。及明,视道上马迹,尚心掉不自禁。”凡虏所以夸耀中国者,多此类。中国之人不测也,故或至于震惧而失辞,以为夷狄笑。呜呼!何其不思之甚也!昔者奉春君使冒顿,壮士健马皆匿不见,是以有平城之役。今之匈奴,吾知其无能为也。孟子曰:“说大人则藐之。”况与夷狄!请以为赠。



\chapter*{师旷撞晋平公}
\addcontentsline{toc}{chapter}{师旷撞晋平公}
\begin{center}
	\textbf{[春秋战国]韩非}
\end{center}

晋平公与群臣饮,饮酣,乃喟然叹曰:“莫乐为人君!惟其言而莫之违。”师旷侍坐于前,援琴撞之。公被衽而避,琴坏于壁。公曰:“太师谁撞?”师旷曰:“今者有小人言于侧者,故撞之。”公曰:“寡人也。”师旷曰:“哑!是非君人者之言也。”左右请除之。公曰:“释之,以为寡人戒。”


\chapter*{陈元方候袁公}
\addcontentsline{toc}{chapter}{陈元方候袁公}
\begin{center}
	\textbf{[南北朝]刘义庆}
\end{center}

陈元方年十一时,候袁公。袁公问曰:“贤家君在太丘,远近称之,何所履行?”元方曰:“老父在太丘,强者绥之以德,弱者抚之以仁,恣其所安,久而益敬。”袁公曰:“孤往者尝为邺令,正行此事。不知卿家君法孤,孤法卿父?”元方曰:“周公、孔子异世而出,周旋动静,万里如一。周公不师孔子,孔子亦不师周公。”


\chapter*{送温处士赴河阳军序}
\addcontentsline{toc}{chapter}{送温处士赴河阳军序}
\begin{center}
	\textbf{[唐朝]韩愈}
\end{center}

伯乐一过冀北之野,而马群遂空。夫冀北马多天下。伯乐虽善知马,安能空其郡邪?解之者曰:“吾所谓空,非无马也,无良马也。伯乐知马,遇其良,辄取之,群无留良焉。苟无良,虽谓无马,不为虚语矣。”

东都,固士大夫之冀北也。恃才能深藏而不市者,洛之北涯曰石生,其南涯曰温生。大夫乌公,以鈇钺镇河阳之三月,以石生为才,以礼为罗,罗而致之幕下。未数月也,以温生为才,于是以石生为媒,以礼为罗,又罗而致之幕下。东都虽信多才士,朝取一人焉,拔其尤;暮取一人焉,拔其尤。自居守河南尹,以及百司之执事,与吾辈二县之大夫,政有所不通,事有所可疑,奚所咨而处焉?士大夫之去位而巷处者,谁与嬉游?小子后生,于何考德而问业焉?缙绅之东西行过是都者,无所礼于其庐。若是而称曰:“大夫乌公一镇河阳,而东都处士之庐无人焉。”岂不可也?

夫南面而听天下,其所托重而恃力者,惟相与将耳。相为天子得人于朝廷,将为天子得文武士于幕下,求内外无治,不可得也。愈縻于兹,不能自引去,资二生以待老。今皆为有力者夺之,其何能无介然于怀邪?生既至,拜公于军门,其为吾以前所称,为天下贺;以后所称,为吾致私怨于尽取也。留守相公首为四韵诗歌其事,愈因推其意而序之。


\chapter*{吕相绝秦}
\addcontentsline{toc}{chapter}{吕相绝秦}
\begin{center}
	\textbf{[春秋战国]左丘明}
\end{center}


夏四月戊午,晋侯使吕相绝秦,曰:“昔逮我献公及穆公相好,戮力同心,申之以盟誓,重之以昏姻。天祸晋国,文公如齐,惠公如秦。无禄,献公即世。穆公不忘旧德,俾我惠公用能奉祀于晋。又不能成大勋,而为韩之师。亦悔于厥心,用集我文公。是穆之成也。


“文公躬擐甲胄,跋履山川,逾越险阻,征东之诸侯,虞、夏、商、周之胤,而朝诸秦,则亦既报旧德矣。郑人怒君之疆埸,我文公帅诸侯及秦围郑。秦大夫不询于我寡君,擅及郑盟。诸侯疾之,将致命于秦。文公恐惧,绥静诸侯,秦师克还无害,则是我有大造于西也。


“无禄,文公即世;穆为不吊,蔑死我君,寡我襄公,迭我肴地,奸绝我好,伐我保城。殄灭我费滑,散离我兄弟,挠乱我同盟,倾覆我国家。我襄公未忘君之旧勋,而惧社稷之陨,是以有淆之师。犹愿赦罪于穆公,穆公弗听,而即楚谋我。天诱其衷,成王陨命,穆公是以不克逞志于我。


“穆、襄即世,康、灵即位。康公,我之自出,又欲阙翦我公室,倾覆我社稷,帅我蝥贼,以来荡摇我边疆,我是以有令狐之役。康犹不悛,入我河曲,伐我涑川,俘我王官,翦我羁马,我是以有河曲之战。东道之不通,则是康公绝我好也。


“及君之嗣也,我君景公引领西望曰:‘庶抚我乎!’君亦不惠称盟,利吾有狄难,入我河县,焚我箕、郜,芟夷我农功,虔刘我边垂,我是以有辅氏之聚。君亦悔祸之延,而欲徼福于先君献、穆,使伯车来命我景公曰:‘吾与女同好弃恶,复脩旧德,以追念前勋。’言誓未就,景公即世,我寡君是以有令狐之会。君又不祥,背弃盟誓。白狄及君同州,君之仇雠,而我昏姻也。君来赐命曰:‘吾与女伐狄。’寡君不敢顾昏姻。畏君之威,而受命于吏。君有二心于狄,曰:‘晋将伐女。’狄应且憎,是用告我。楚人恶君之二三其德也,亦来告我曰:‘秦背令狐之盟,而来求盟于我:“昭告昊天上帝、秦三公、楚三王曰:‘余虽与晋出入,余唯利是视。’”不榖恶其无成德,是用宣之,以惩不壹。’诸侯备闻此言,斯是用痛心疾首,暱就寡人。寡人帅以听命,唯好是求。君若惠顾诸侯,矜哀寡人,而赐之盟,则寡人之愿也,其承宁诸侯以退,岂敢徼乱?君若不施大惠,寡人不佞,其不能以诸侯退矣。敢尽布之执事,俾执事实图利之。”



\chapter*{蝜蝂传}
\addcontentsline{toc}{chapter}{蝜蝂传}
\begin{center}
	\textbf{[唐朝]柳宗元}
\end{center}

蝜蝂者,善负小虫也。行遇物,辄持取,卬其首负之。背愈重,虽困剧不止也。其背甚涩,物积因不散,卒踬仆不能起。人或怜之,为去其负。苟能行,又持取如故。又好上高,极其力不已,至坠地死。

今世之嗜取者,遇货不避,以厚其室,不知为己累也,唯恐其不积。及其怠而踬也,黜弃之,迁徙之,亦以病矣。苟能起,又不艾。日思高其位,大其禄,而贪取滋甚,以近于危坠,观前之死亡,不知戒。虽其形魁然大者也,其名人也,而智则小虫也。亦足哀夫!


\chapter*{游兰溪}
\addcontentsline{toc}{chapter}{游兰溪}
\begin{center}
	\textbf{[宋朝]苏轼}
\end{center}


黄州东南三十里为沙湖,亦曰螺师店。予买田其间,因往相田得疾。闻麻桥人庞安常善医而聋。遂往求疗。安常虽聋,而颖悟绝人,以纸画字,书不数字,辄深了人意。余戏之曰:“余以手为口,君以眼为耳,皆一时异人也。”疾愈,与之同游清泉寺。寺在蕲水郭门外二里许。有王逸少洗笔泉,水极甘,下临兰溪,溪水西流。余作歌云:“山下兰芽短浸溪,松间沙路净无泥,萧萧暮雨子规啼。谁道人生无再少?君看流水尚能西,休将白发唱黄鸡。”是日剧饮而归。



\chapter*{争臣论}
\addcontentsline{toc}{chapter}{争臣论}
\begin{center}
	\textbf{[唐朝]韩愈}
\end{center}

或问谏议大夫阳城于愈,可以为有道之士乎哉?学广而闻多,不求闻于人也。行古人之道,居于晋之鄙。晋之鄙人,熏其德而善良者几千人。大臣闻而荐之,天子以为谏议大夫。人皆以为华,阳子不色喜。居于位五年矣,视其德,如在野,彼岂以富贵移易其心哉?

愈应之曰:是《易》所谓恒其德贞,而夫子凶者也。恶得为有道之士乎哉?在《易·蛊》之“上九”云:“不事王侯,高尚其事。”《蹇》之“六二”则曰:“王臣蹇蹇,匪躬之故。”夫亦以所居之时不一,而所蹈之德不同也。若《蛊》之“上九”,居无用之地,而致匪躬之节;以《蹇》之“六二”,在王臣之位,而高不事之心,则冒进之患生,旷官之刺兴。志不可则,而尤不终无也。今阳子在位,不为不久矣;闻天下之得失,不为不熟矣;天子待之,不为不加矣。而未尝一言及于政。视政之得失,若越人视秦人之肥瘠,忽焉不加喜戚于其心。问其官,则曰谏议也;问其禄,则曰下大夫之秩秩也;问其政,则曰我不知也。有道之士,固如是乎哉?且吾闻之:有官守者,不得其职则去;有言责者,不得其言则去。今阳子以为得其言乎哉?得其言而不言,与不得其言而不去,无一可者也。阳子将为禄仕乎?古之人有云:“仕不为贫,而有时乎为贫。”谓禄仕者也。宜乎辞尊而居卑,辞富而居贫,若抱关击柝者可也。盖孔子尝为委吏矣,尝为乘田矣,亦不敢旷其职,必曰“会计当而已矣”,必曰“牛羊遂而已矣”。若阳子之秩禄,不为卑且贫,章章明矣,而如此,其可乎哉?

或曰:否,非若此也。夫阳子恶讪上者,恶为人臣招其君之过而以为名者。故虽谏且议,使人不得而知焉。《书》曰:“尔有嘉谟嘉猷,则人告尔后于内,尔乃顺之于外,曰:斯谟斯猷,惟我后之德”若阳子之用心,亦若此者。愈应之曰:若阳子之用心如此,滋所谓惑者矣。入则谏其君,出不使人知者,大臣宰相者之事,非阳子之所宜行也。夫阳子,本以布衣隐于蓬蒿之下,主上嘉其行谊,擢在此位,官以谏为名,诚宜有以奉其职,使四方后代,知朝廷有直言骨鲠之臣,天子有不僭赏、从谏如流之美。庶岩穴之士,闻而慕之,束带结发,愿进于阙下,而伸其辞说,致吾君于尧舜,熙鸿号于无穷也。若《书》所谓,则大臣宰相之事,非阳子之所宜行也。且阳子之心,将使君人者恶闻其过乎?是启之也。

或曰:阳子之不求闻而人闻之,不求用而君用之。不得已而起。守其道而不变,何子过之深也?愈曰:自古圣人贤士,皆非有求于闻用也。闵其时之不平,人之不义,得其道。不敢独善其身,而必以兼济天下也。孜孜矻矻,死而后已。故禹过家门不入,孔席不暇暖,而墨突不得黔。彼二圣一贤者,岂不知自安佚之为乐哉诚畏天命而悲人穷也。夫天授人以贤圣才能,岂使自有余而已,诚欲以补其不足者也。耳目之于身也,耳司闻而目司见,听其是非,视其险易,然后身得安焉。圣贤者,时人之耳目也;时人者,圣贤之身也。且阳子之不贤,则将役于贤以奉其上矣;若果贤,则固畏天命而闵人穷也。恶得以自暇逸乎哉?

或曰:吾闻君子不欲加诸人,而恶讦以为直者。若吾子之论,直则直矣,无乃伤于德而费于辞乎?好尽言以招人过,国武子之所以见杀于齐也,吾子其亦闻乎?愈曰:君子居其位,则思死其官。未得位,则思修其辞以明其道。我将以明道也,非以为直而加入也。且国武子不能得善人,而好尽言于乱国,是以见杀。《传》曰:“惟善人能受尽言。”谓其闻而能改之也。子告我曰:“阳子可以为有之士也。”今虽不能及已,阳子将不得为善人乎哉?


\chapter*{获麟解}
\addcontentsline{toc}{chapter}{获麟解}
\begin{center}
	\textbf{[唐朝]韩愈}
\end{center}

麟之为灵,昭昭也。咏于《诗》,书于《春秋》,杂出于传记百家之书,虽妇人小子皆知其为祥也。

然麟之为物,不畜于家,不恒有于天下。其为形也不类,非若马牛犬豕豺狼麋鹿然。然则虽有麟,不可知其为麟也。

角者吾知其为牛,鬣者吾知其为马,犬豕豺狼麋鹿,吾知其为犬豕豺狼麋鹿。惟麟也,不可知。不可知,则其谓之不祥也亦宜。虽然,麟之出,必有圣人在乎位。麟为圣人出也。圣人者,必知麟,麟之果不为不祥也。

又曰:“麟之所以为麟者,以德不以形。”若麟之出不待圣人,则谓之不祥也亦宜。


\chapter*{赠黎安二生序}
\addcontentsline{toc}{chapter}{赠黎安二生序}
\begin{center}
	\textbf{[宋朝]曾巩}
\end{center}


赵郡苏轼,余之同年友也。自蜀以书至京师遗余,称蜀之士,曰黎生、安生者。既而黎生携其文数十万言,安生携其文亦数千言,辱以顾余。读其文,诚闳壮隽伟,善反复驰骋,穷尽事理;而其材力之放纵,若不可极者也。二生固可谓魁奇特起之士,而苏君固可谓善知人者也。


顷之,黎生补江陵府司法参军。将行,请予言以为赠。余曰:「余之知生,既得之于心矣,乃将以言相求于外邪?」黎生曰:「生与安生之学于斯文,里之人皆笑以为迂阔。今求子之言,盖将解惑于里人。」余闻之,自顾而笑。


夫世之迂阔,孰有甚于予乎?知信乎古,而不知合乎世;知志乎道,而不知同乎俗。此余所以困于今而不自知也。世之迂阔,孰有甚于予乎?今生之迂,特以文不近俗,迂之小者耳,患为笑于里之人。若余之迂大矣,使生持吾言而归,且重得罪,庸讵止于笑乎?


然则若余之于生,将何言哉?谓余之迂为善,则其患若此;谓为不善,则有以合乎世,必违乎古,有以同乎俗,必离乎道矣。生其无急于解里人之惑,则于是焉,必能择而取之。


遂书以赠二生,并示苏君,以为何如也?



\chapter*{沧浪亭记}
\addcontentsline{toc}{chapter}{沧浪亭记}
\begin{center}
	\textbf{[明朝]归有光}
\end{center}


浮图文瑛居大云庵,环水,即苏子美沧浪亭之地也。亟求余作《沧浪亭记》,曰:“昔子美之记,记亭之胜也。请子记吾所以为亭者。”


余曰:昔吴越有国时,广陵王镇吴中,治南园于子城之西南;其外戚孙承祐,亦治园于其偏。迨淮海纳土,此园不废。苏子美始建沧浪亭,最后禅者居之:此沧浪亭为大云庵也。有庵以来二百年,文瑛寻古遗事,复子美之构于荒残灭没之余:此大云庵为沧浪亭也。


夫古今之变,朝市改易。尝登姑苏之台,望五湖之渺茫,群山之苍翠,太伯、虞仲之所建,阖闾、夫差之所争,子胥、种、蠡之所经营,今皆无有矣。庵与亭何为者哉?虽然,钱镠因乱攘窃,保有吴越,国富兵强,垂及四世。诸子姻戚,乘时奢僭,宫馆苑囿,极一时之盛。而子美之亭,乃为释子所钦重如此。可以见士之欲垂名于千载,不与其澌然而俱尽者,则有在矣。


文瑛读书喜诗,与吾徒游,呼之为沧浪僧云。



\chapter*{原道}
\addcontentsline{toc}{chapter}{原道}
\begin{center}
	\textbf{[唐朝]韩愈}
\end{center}

博爱之谓仁,行而宜之之谓义,由是而之焉之谓道,足乎己无待于外之谓德。仁与义为定名,道与德为虚位。故道有君子小人,而德有凶有吉。老子之小仁义,非毁之也,其见者小也。坐井而观天,曰天小者,非天小也。彼以煦煦为仁,孑孑为义,其小之也则宜。其所谓道,道其所道,非吾所谓道也。其所谓德,德其所德,非吾所谓德也。凡吾所谓道德云者,合仁与义言之也,天下之公言也。老子之所谓道德云者,去仁与义言之也,一人之私言也。

周道衰,孔子没,火于秦,黄老于汉,佛于晋、魏、梁、隋之间。其言道德仁义者,不入于杨,则归于墨;不入于老,则归于佛。入于彼,必出于此。入者主之,出者奴之;入者附之,出者污之。噫!后之人其欲闻仁义道德之说,孰从而听之?老者曰:“孔子,吾师之弟子也。”佛者曰:“孔子,吾师之弟子也。”为孔子者,习闻其说,乐其诞而自小也,亦曰“吾师亦尝师之”云尔。不惟举之于口,而又笔之于其书。噫!后之人虽欲闻仁义道德之说,其孰从而求之?

甚矣,人之好怪也,不求其端,不讯其末,惟怪之欲闻。古之为民者四,今之为民者六。古之教者处其一,今之教者处其三。农之家一,而食粟之家六。工之家一,而用器之家六。贾之家一,而资焉之家六。奈之何民不穷且盗也?

古之时,人之害多矣。有圣人者立,然后教之以相生相养之道。为之君,为之师。驱其虫蛇禽兽,而处之中土。寒然后为之衣,饥然后为之食。木处而颠,土处而病也,然后为之宫室。为之工以赡其器用,为之贾以通其有无,为之医药以济其夭死,为之葬埋祭祀以长其恩爱,为之礼以次其先后,为之乐以宣其湮郁,为之政以率其怠倦,为之刑以锄其强梗。相欺也,为之符、玺、斗斛、权衡以信之。相夺也,为之城郭甲兵以守之。害至而为之备,患生而为之防。今其言曰:“圣人不死,大盗不止。剖斗折衡,而民不争。”呜呼!其亦不思而已矣。如古之无圣人,人之类灭久矣。何也?无羽毛鳞介以居寒热也,无爪牙以争食也。

是故君者,出令者也;臣者,行君之令而致之民者也;民者,出粟米麻丝,作器皿,通货财,以事其上者也。君不出令,则失其所以为君;臣不行君之令而致之民,则失其所以为臣;民不出粟米麻丝,作器皿,通货财,以事其上,则诛。今其法曰,必弃而君臣,去而父子,禁而相生相养之道,以求其所谓清净寂灭者。呜呼!其亦幸而出于三代之后,不见黜于禹、汤、文、武、周公、孔子也。其亦不幸而不出于三代之前,不见正于禹、汤、文、武、周公、孔子也。

帝之与王,其号虽殊,其所以为圣一也。夏葛而冬裘,渴饮而饥食,其事虽殊,其所以为智一也。今其言曰:“曷不为太古之无事”?”是亦责冬之裘者曰:“曷不为葛之之易也?”责饥之食者曰:“曷不为饮之之易也?”传曰:“古之欲明明德于天下者,先治其国;欲治其国者,先齐其家;欲齐其家者,先修其身;欲修其身者,先正其心;欲正其心者,先诚其意。”然则古之所谓正心而诚意者,将以有为也。今也欲治其心而外天下国家,灭其天常,子焉而不父其父,臣焉而不君其君,民焉而不事其事。孔子之作《春秋》也,诸侯用夷礼则夷之,进于中国则中国之。经曰:“夷狄之有君,不如诸夏之亡。”《诗》曰:戎狄是膺,荆舒是惩”今也举夷狄之法,而加之先王之教之上,几何其不胥而为夷也?

夫所谓先王之教者,何也?博爱之谓仁,行而宜之之谓义。由是而之焉之谓道。足乎己无待于外之谓德。其文:《诗》、《书》、《易》、《春秋》;其法:礼、乐、刑、政;其民:士、农、工、贾;其位:君臣、父子、师友、宾主、昆弟、夫妇;其服:麻、丝;其居:宫、室;其食:粟米、果蔬、鱼肉。其为道易明,而其为教易行也。是故以之为己,则顺而祥;以之为人,则爱而公;以之为心,则和而平;以之为天下国家,无所处而不当。是故生则得其情,死则尽其常。效焉而天神假,庙焉而人鬼飨。曰:“斯道也,何道也?”曰:“斯吾所谓道也,非向所谓老与佛之道也。尧以是传之舜,舜以是传之禹,禹以是传之汤,汤以是传之文、武、周公,文、武、周公传之孔子,孔子传之孟轲,轲之死,不得其传焉。荀与扬也,择焉而不精,语焉而不详。由周公而上,上而为君,故其事行。由周公而下,下而为臣,故其说长。然则如之何而可也?曰:“不塞不流,不止不行。人其人,火其书,庐其居。明先王之道以道之,鳏寡孤独废疾者有养也。其亦庶乎其可也!”


\chapter*{地震}
\addcontentsline{toc}{chapter}{地震}
\begin{center}
	\textbf{[清朝]蒲松龄}
\end{center}


康熙七年六月十七日戌刻,地大震。余适客稷下,方与表兄李笃之对烛饮。忽闻有声如雷,自东南来,向西北去。众骇异,不解其故。俄而几案摆簸,酒杯倾覆;屋梁椽柱,错折有声。相顾失色。久之,方知地震,各疾趋出。见楼阁房舍,仆而复起;墙倾屋塌之声,与儿啼女号,喧如鼎沸。


人眩晕不能立,坐地上,随地转侧。河水倾泼丈余,鸡鸣犬吠满城中。逾一时许,始稍定。视街上,则男女裸聚,竞相告语,并忘其未衣也。后闻某处井倾仄,不可汲;某家楼台南北易向;栖霞山裂;沂水陷穴,广数亩。此真非常之奇变也。



\chapter*{虎丘记}
\addcontentsline{toc}{chapter}{虎丘记}
\begin{center}
	\textbf{[明朝]袁宏道}
\end{center}


虎丘去城可七八里,其山无高岩邃壑,独以近城,故箫鼓楼船,无日无之。凡月之夜,花之晨,雪之夕,游人往来,纷错如织,而中秋为尤胜。


每至是日,倾城阖户,连臂而至。衣冠士女,下迨蔀屋,莫不靓妆丽服,重茵累席,置酒交衢间。从千人石上至山门,栉比如鳞,檀板丘积,樽罍云泻,远而望之,如雁落平沙,霞铺江上,雷辊电霍,无得而状。


布席之初,唱者千百,声若聚蚊,不可辨识。分曹部署,竟以歌喉相斗,雅俗既陈,妍媸自别。未几而摇手顿足者,得数十人而已;已而明月浮空,石光如练,一切瓦釜,寂然停声,属而和者,才三四辈;一箫,一寸管,一人缓板而歌,竹肉相发,清声亮彻,听者魂销。比至夜深,月影横斜,荇藻凌乱,则箫板亦不复用;一夫登场,四座屏息,音若细发,响彻云际,每度一字,几尽一刻,飞鸟为之徘徊,壮士听而下泪矣。


剑泉深不可测,飞岩如削。千顷云得天池诸山作案,峦壑竞秀,最可觞客。但过午则日光射人,不堪久坐耳。文昌阁亦佳,晚树尤可观。而北为平远堂旧址,空旷无际,仅虞山一点在望,堂废已久,余与江进之谋所以复之,欲祠韦苏州、白乐天诸公于其中;而病寻作,余既乞归,恐进之之兴亦阑矣。山川兴废,信有时哉!


吏吴两载,登虎丘者六。最后与江进之、方子公同登,迟月生公石上。歌者闻令来,皆避匿去。余因谓进之曰:“甚矣,乌纱之横,皂隶之俗哉!他日去官,有不听曲此石上者,如月!”今余幸得解官称吴客矣。虎丘之月,不知尚识余言否耶? 



\chapter*{鲁人锯竿入城}
\addcontentsline{toc}{chapter}{鲁人锯竿入城}
\begin{center}
	\textbf{[晋朝]邯郸淳}
\end{center}


鲁有执长竿入城门者,初竖执之,不可入。横执之,亦不可入。计无所出。俄有老父至,曰:“吾非圣人,但见事多矣,何不以锯中截而入。”遂依而截之。世之愚,莫之及也。

\chapter*{送李愿归盘谷序}
\addcontentsline{toc}{chapter}{送李愿归盘谷序}
\begin{center}
	\textbf{[唐朝]韩愈}
\end{center}

太行之阳有盘谷。盘谷之间,泉甘而土肥,草木丛茂,居民鲜少。或曰:“谓其环两山之间,故曰‘盘’。”或曰:“是谷也,宅幽而势阻,隐者之所盘旋。”友人李愿居之。

愿之言曰:“人之称大丈夫者,我知之矣:利泽施于人,名声昭于时,坐于庙朝,进退百官,而佐天子出令;其在外,则树旗旄,罗弓矢,武夫前呵,从者塞途,供给之人,各执其物,夹道而疾驰。喜有赏,怒有刑。才畯满前,道古今而誉盛德,入耳而不烦。曲眉丰颊,清声而便体,秀外而惠中,飘轻裾,翳长袖,粉白黛绿者,列屋而闲居,妒宠而负恃,争妍而取怜。大丈夫之遇知于天子、用力于当世者之所为也。吾非恶此而逃之,是有命焉,不可幸而致也。

穷居而野处,升高而望远,坐茂树以终日,濯清泉以自洁。采于山,美可茹;钓于水,鲜可食。起居无时,惟适之安。与其有誉于前,孰若无毁于其后;与其有乐于身,孰若无忧于其心。车服不维,刀锯不加,理乱不知,黜陟不闻。大丈夫不遇于时者之所为也,我则行之。

伺候于公卿之门,奔走于形势之途,足将进而趑趄,口将言而嗫嚅,处污秽而不羞,触刑辟而诛戮,侥幸于万一,老死而后止者,其于为人,贤不肖何如也?”

昌黎韩愈闻其言而壮之,与之酒而为之歌曰:“盘之中,维子之宫;盘之土,维子之稼;盘之泉,可濯可沿;盘之阻,谁争子所?窈而深,廓其有容;缭而曲,如往而复。嗟盘之乐兮,乐且无央;虎豹远迹兮,蛟龙遁藏;鬼神守护兮,呵禁不祥。饮且食兮寿而康,无不足兮奚所望!膏吾车兮秣吾马,从子于盘兮,终吾生以徜徉!”


\chapter*{山市}
\addcontentsline{toc}{chapter}{山市}
\begin{center}
	\textbf{[清朝]蒲松龄}
\end{center}


奂山山市,邑八景之一也,然数年恒不一见。孙公子禹年与同人饮楼上,忽见山头有孤塔耸起,高插青冥,相顾惊疑,念近中无此禅院。无何,见宫殿数十所,碧瓦飞甍,始悟为山市。未几,高垣睥睨,连亘六七里,居然城郭矣。中有楼若者,堂若者,坊若者,历历在目,以亿万计。忽大风起,尘气莽莽然,城市依稀而已。既而风定天清,一切乌有,惟危楼一座,直接霄汉。楼五架,窗扉皆洞开;一行有五点明处,楼外天也。


层层指数,楼愈高,则明渐少。数至八层,裁如星点。又其上,则黯然缥缈,不可计其层次矣。而楼上人往来屑屑,或凭或立,不一状。逾时,楼渐低,可见其顶;又渐如常楼;又渐如高舍;倏忽如拳如豆,遂不可见。


又闻有早行者,见山上人烟市肆,与世无别,故又名“鬼市”云。



\chapter*{与山巨源绝交书}
\addcontentsline{toc}{chapter}{与山巨源绝交书}
\begin{center}
	\textbf{[三国]嵇康}
\end{center}


康白:足下昔称吾于颍川,吾常谓之知言。然经怪此意尚未熟悉于足下,何从便得之也?前年从河东还,显宗、阿都说足下议以吾自代,事虽不行,知足下故不知之。足下傍通,多可而少怪;吾直性狭中,多所不堪,偶与足下相知耳。闲闻足下迁,惕然不喜,恐足下羞庖人之独割,引尸祝以自助,手荐鸾刀,漫之膻腥,故具为足下陈其可否。


吾昔读书,得并介之人,或谓无之,今乃信其真有耳。性有所不堪,真不可强。今空语同知有达人无所不堪,外不殊俗,而内不失正,与一世同其波流,而悔吝不生耳。老子、庄周,吾之师也,亲居贱职;柳下惠、东方朔,达人也,安乎卑位,吾岂敢短之哉!又仲尼兼爱,不羞执鞭;子文无欲卿相,而三登令尹,是乃君子思济物之意也。所谓达能兼善而不渝,穷则自得而无闷。以此观之,故尧、舜之君世,许由之岩栖,子房之佐汉,接舆之行歌,其揆一也。仰瞻数君,可谓能遂其志者也。故君子百行,殊途而同致,循性而动,各附所安。故有处朝廷而不出,入山林而不返之论。且延陵高子臧之风,长卿慕相如之节,志气所托,不可夺也。吾每读尚子平、台孝威传,慨然慕之,想其为人。少加孤露,母兄见骄,不涉经学。性复疏懒,筋驽肉缓,头面常一月十五日不洗,不大闷痒,不能沐也。每常小便而忍不起,令胞中略转乃起耳。又纵逸来久,情意傲散,简与礼相背,懒与慢相成,而为侪类见宽,不攻其过。又读《庄》、《老》,重增其放,故使荣进之心日颓,任实之情转笃。此犹禽鹿,少见驯育,则服从教制;长而见羁,则狂顾顿缨,赴蹈汤火;虽饰以金镳,飨以嘉肴,愈思长林而志在丰草也。


阮嗣宗口不论人过,吾每师之而未能及;至性过人,与物无伤,唯饮酒过差耳。至为礼法之士所绳,疾之如仇,幸赖大将军保持之耳。吾不如嗣宗之资,而有慢弛之阙;又不识人情,暗于机宜;无万石之慎,而有好尽之累。久与事接,疵衅日兴,虽欲无患,其可得乎?又人伦有礼,朝廷有法,自惟至熟,有必不堪者七,甚不可者二:卧喜晚起,而当关呼之不置,一不堪也。抱琴行吟,弋钓草野,而吏卒守之,不得妄动,二不堪也。危坐一时,痹不得摇,性复多虱,把搔无已,而当裹以章服,揖拜上官,三不堪也。素不便书,又不喜作书,而人间多事,堆案盈机,不相酬答,则犯教伤义,欲自勉强,则不能久,四不堪也。不喜吊丧,而人道以此为重,已为未见恕者所怨,至欲见中伤者;虽瞿然自责,然性不可化,欲降心顺俗,则诡故不情,亦终不能获无咎无誉如此,五不堪也。不喜俗人,而当与之共事,或宾客盈坐,鸣声聒耳,嚣尘臭处,千变百伎,在人目前,六不堪也。心不耐烦,而官事鞅掌,机务缠其心,世故烦其虑,七不堪也。又每非汤、武而薄周、孔,在人间不止,此事会显,世教所不容,此甚不可一也。刚肠疾恶,轻肆直言,遇事便发,此甚不可二也。以促中小心之性,统此九患,不有外难,当有内病,宁可久处人间邪?又闻道士遗言,饵术黄精,令人久寿,意甚信之;游山泽,观鱼鸟,心甚乐之;一行作吏,此事便废,安能舍其所乐而从其所惧哉!


夫人之相知,贵识其天性,因而济之。禹不逼伯成子高,全其节也;仲尼不假盖于子夏,护其短也;近诸葛孔明不逼元直以入蜀,华子鱼不强幼安以卿相,此可谓能相终始,真相知者也。足下见直木不可以为轮,曲木不可以为桷,盖不欲枉其天才,令得其所也。故四民有业,各以得志为乐,唯达者为能通之,此足下度内耳。不可自见好章甫,强越人以文冕也;己嗜臭腐,养鸳雏以死鼠也。吾顷学养生之术,方外荣华,去滋味,游心于寂寞,以无为为贵。纵无九患,尚不顾足下所好者。又有心闷疾,顷转增笃,私意自试,不能堪其所不乐。自卜已审,若道尽途穷则已耳。足下无事冤之,令转于沟壑也。


吾新失母兄之欢,意常凄切。女年十三,男年八岁,未及成人,况复多病。顾此悢悢,如何可言!今但愿守陋巷,教养子孙,时与亲旧叙离阔,陈说平生,浊酒一杯,弹琴一曲,志愿毕矣。足下若嬲之不置,不过欲为官得人,以益时用耳。足下旧知吾潦倒粗疏,不切事情,自惟亦皆不如今日之贤能也。若以俗人皆喜荣华,独能离之,以此为快;此最近之,可得言耳。然使长才广度,无所不淹,而能不营,乃可贵耳。若吾多病困,欲离事自全,以保余年,此真所乏耳,岂可见黄门而称贞哉!若趣欲共登王途,期于相致,时为欢益,一旦迫之,必发狂疾。自非重怨,不至于此也。


野人有快炙背而美芹子者,欲献之至尊,虽有区区之意,亦已疏矣。愿足下勿似之。其意如此,既以解足下,并以为别。嵇康白。



\chapter*{一毛不拔}
\addcontentsline{toc}{chapter}{一毛不拔}
\begin{center}
	\textbf{[晋朝]邯郸淳}
\end{center}


一猴死,见冥王,求转人身。王曰:“既欲做人,须将毛尽拔去。”即唤夜叉拔之。方拔一根,猴不胜痛叫。王笑曰:“看你一毛不拔,如何做人?”

\chapter*{子鱼论战}
\addcontentsline{toc}{chapter}{子鱼论战}
\begin{center}
	\textbf{[春秋战国]左丘明}
\end{center}


二十有二年春,公伐邾,取须句。夏,宋公、卫侯、许男、滕子伐郑。秋,八月丁未,及邾人战于升陉。冬,十有一月己巳朔,宋公及楚人战于泓,宋师败绩。


楚人伐宋以救郑。宋公将战。大司马固谏曰:“天之弃商久矣,君将兴之,弗可赦也已。”弗听。冬十一月己巳朔,宋公及楚人战于泓。宋人既成列,楚人未既济。司马曰:“彼众我寡,及其未既济也,请击之。”公曰:“不可。”既济而未成列,又以告。公曰:“未可。”既陈而后击之,宋师败绩。公伤股,门官歼焉。


国人皆咎公。公曰:“君子不重伤,不禽二毛。古之为军也,不以阻隘也。寡人虽亡国之余,不鼓不成列。”子鱼曰:“君未知战。勍敌之人,隘而不列,天赞我也。阻而鼓之,不亦可乎?犹有惧焉!且今之勍者,皆我敌也。虽及胡耇,获则取之,何有于二毛?明耻教战,求杀敌也。伤未及死,如何勿重?若爱重伤,则如勿伤;爱其二毛,则如服焉。三军以利用也,金鼓以声气也。利而用之,阻隘可也;声盛致志,鼓儳可也。”



\chapter*{答苏武书}
\addcontentsline{toc}{chapter}{答苏武书}
\begin{center}
	\textbf{[汉朝]李陵}
\end{center}

子卿足下:

勤宣令德,策名清时,荣问休畅,幸甚幸甚。远托异国,昔人所悲,望风怀想,能不依依?昔者不遗,远辱还答,慰诲勤勤,有逾骨肉,陵虽不敏,能不慨然?

自从初降,以至今日,身之穷困,独坐愁苦。终日无睹,但见异类。韦韝毳幕,以御风雨;羶肉酪浆,以充饥渴。举目言笑,谁与为欢?胡地玄冰,边土惨裂,但闻悲风萧条之声。凉秋九月,塞外草衰。夜不能寐,侧耳远听,胡笳互动,牧马悲鸣,吟啸成群,边声四起。晨坐听之,不觉泪下。嗟乎子卿,陵独何心,能不悲哉!

与子别后,益复无聊,上念老母,临年被戮;妻子无辜,并为鲸鲵;身负国恩,为世所悲。子归受荣,我留受辱,命也如何?身出礼义之乡,而入无知之俗;违弃君亲之恩,长为蛮夷之域,伤已!令先君之嗣,更成戎狄之族,又自悲矣。功大罪小,不蒙明察,孤负陵心区区之意。每一念至,忽然忘生。陵不难刺心以自明,刎颈以见志,顾国家于我已矣,杀身无益,适足增羞,故每攘臂忍辱,辙复苟活。左右之人,见陵如此,以为不入耳之欢,来相劝勉。异方之乐,只令人悲,增忉怛耳。

嗟乎子卿,人之相知,贵相知心,前书仓卒,未尽所怀,故复略而言之。

昔先帝授陵步卒五千,出征绝域。五将失道,陵独遇战,而裹万里之粮,帅徒步之师;出天汉之外,入强胡之域;以五千之众,对十万之军;策疲乏之兵,当新羁之马。然犹斩将搴旗,追奔逐北,灭迹扫尘,斩其枭帅,使三军之士,视死如归。陵也不才,希当大任,意谓此时,功难堪矣。匈奴既败,举国兴师。更练精兵,强逾十万。单于临阵,亲自合围。客主之形,既不相如;步马之势,又甚悬绝。疲兵再战,一以当千,然犹扶乘创痛,决命争首。死伤积野,余不满百,而皆扶病,不任干戈,然陵振臂一呼,创病皆起,举刃指虏,胡马奔走。兵尽矢穷,人无尺铁,犹复徒首奋呼,争为先登。当此时也,天地为陵震怒,战士为陵饮血。单于谓陵不可复得,便欲引还,而贼臣教之,遂使复战,故陵不免耳。

昔高皇帝以三十万众,困于平城。当此之时,猛将如云,谋臣如雨,然犹七日不食,仅乃得免。况当陵者,岂易为力哉?而执事者云云,苟怨陵以不死。然陵不死,罪也;子卿视陵,岂偷生之士而惜死之人哉?宁有背君亲,捐妻子而反为利者乎?然陵不死,有所为也,故欲如前书之言,报恩于国主耳,诚以虚死不如立节,灭名不如报德也。昔范蠡不殉会稽之耻,曹沬不死三败之辱,卒复勾践之仇,报鲁国之羞,区区之心,窃慕此耳。何图志未立而怨已成,计未从而骨肉受刑,此陵所以仰天椎心而泣血也。

足下又云:“汉与功臣不薄。”子为汉臣,安得不云尔乎?昔萧樊囚絷,韩彭葅醢,晁错受戮,周魏见辜。其余佐命立功之士,贾谊亚夫之徒,皆信命世之才,抱将相之具,而受小人之谗,并受祸败之辱,卒使怀才受谤,能不得展。彼二子之遐举,谁不为之痛心哉?陵先将军,功略盖天地,义勇冠三军,徒失贵臣之意,刭身绝域之表。此功臣义士所以负戟而长叹者也。何谓不薄哉?且足下昔以单车之使,适万乘之虏。遭时不遇,至于伏剑不顾;流离辛苦,几死朔北之野。丁年奉使,皓首而归;老母终堂,生妻去帷。此天下所希闻,古今所未有也。蛮貊之人,尚犹嘉子之节,况为天下之主乎?陵谓足下当享茅土之荐,受千乘之赏。闻子之归,赐不过二百万,位不过典属国,无尺土之封,加子之勤。而妨功害能之臣,尽为万户侯;亲戚贪佞之类,悉为廊庙宰。子尚如此,陵复何望哉?且汉厚诛陵以不死,薄赏子以守节,欲使远听之臣望风驰命,此实难矣,所以每顾而不悔者也。陵虽孤恩,汉亦负德。昔人有言:“虽忠不烈,视死如归。”陵诚能安,而主岂复能眷眷乎?男儿生以不成名,死则葬蛮夷中,谁复能屈身稽颡,还向北阙,使刀笔之吏弄其文墨邪?愿足下勿复望陵。

嗟乎子卿,夫复何言?相去万里,人绝路殊。生为别世之人,死为异域之鬼。长与足下生死辞矣。幸谢故人,勉事圣君。足下胤子无恙,勿以为念。努力自爱,时因北风,复惠德音。李陵顿首。


\chapter*{报孙会宗书}
\addcontentsline{toc}{chapter}{报孙会宗书}
\begin{center}
	\textbf{[汉朝]杨恽}
\end{center}

恽材朽行秽,文质无所底,幸赖先人余业,得备宿卫。遭遇时变,以获爵位。终非其任,卒与祸会。足下哀其愚,蒙赐书教督以所不及,殷勤甚厚。然窃恨足下不深推其终始,而猥随俗之毁誉也。言鄙陋之愚心,若逆指而文过;默而息乎,恐违孔氏各言尔志之义。故敢略陈其愚,惟君子察焉。

恽家方隆盛时,乘朱轮者十人,位在列卿,爵为通侯,总领从官,与闻政事。曾不能以此时有所建明,以宣德化,又不能与群僚同心并力,陪辅朝庭之遗忘,已负窃位素餐之责久矣。怀禄贪势,不能自退,遂遭变故,横被口语,身幽北阙,妻子满狱。当此之时,自以夷灭不足以塞责,岂意得全首领,复奉先人之丘墓乎?伏惟圣主之恩不可胜量。君子游道,乐以忘忧;小人全躯,说以忘罪。窃自念过已大矣,行已亏矣,长为农夫以末世矣。是故身率妻子,戮力耕桑,灌园治产,以给公上,不意当复用此为讥议也。

夫人情所不能止者,圣人弗禁。故君父至尊亲,送其终也,有时而既。臣之得罪,已三年矣。田家作苦。岁时伏腊,烹羊炰羔,斗酒自劳。家本秦也,能为秦声。妇赵女也,雅善鼓瑟。奴婢歌者数人,酒后耳热,仰天抚缶而呼乌乌。其诗曰:“田彼南山,芜秽不治。种一顷豆,落而为萁。人生行乐耳,须富贵何时!”是日也,奋袖低昂,顿足起舞;诚滛荒无度,不知其不可也。恽幸有余禄,方籴贱贩贵,逐什一之利。此贾竖之事,污辱之处,恽亲行之。下流之人,众毁所归,不寒而栗。虽雅知恽者,犹随风而靡,尚何称誉之有?董生不云乎:“明明求仁义,常恐不能化民者,卿大夫之意也。明明求财利,常恐困乏者,庶人之事也。”故道不同,不相为谋,今子尚安得以卿大夫之制而责仆哉!

夫西河魏土,文侯所兴,有段干木、田子方之遗风,漂然皆有节概,知去就之分。顷者足下离旧土,临安定,安定山谷之间,昆戎旧壤,子弟贪鄙,岂习俗之移人哉?于今乃睹子之志矣!方当盛汉之隆,愿勉旃,毋多谈。


\chapter*{柳子厚墓志铭}
\addcontentsline{toc}{chapter}{柳子厚墓志铭}
\begin{center}
	\textbf{[唐朝]韩愈}
\end{center}

子厚,讳宗元。七世祖庆,为拓跋魏侍中,封济阴公。曾伯祖奭,为唐宰相,与褚遂良、韩瑗俱得罪武后,死高宗朝。皇考讳镇,以事母弃太常博士,求为县令江南。其后以不能媚权贵,失御史。权贵人死,乃复拜侍御史。号为刚直,所与游皆当世名人。

子厚少精敏,无不通达。逮其父时,虽少年,已自成人,能取进士第,崭然见头角。众谓柳氏有子矣。其后以博学宏词,授集贤殿正字。俊杰廉悍,议论证据今古,出入经史百子,踔厉风发,率常屈其座人。名声大振,一时皆慕与之交。诸公要人,争欲令出我门下,交口荐誉之。

贞元十九年,由蓝田尉拜监察御史。顺宗即位,拜礼部员外郎。遇用事者得罪,例出为刺史。未至,又例贬永州司马。居闲,益自刻苦,务记览,为词章,泛滥停蓄,为深博无涯涘。而自肆于山水间。

元和中,尝例召至京师;又偕出为刺史,而子厚得柳州。既至,叹曰:“是岂不足为政邪?”因其土俗,为设教禁,州人顺赖。其俗以男女质钱,约不时赎,子本相侔,则没为奴婢。子厚与设方计,悉令赎归。其尤贫力不能者,令书其佣,足相当,则使归其质。观察使下其法于他州,比一岁,免而归者且千人。衡湘以南为进士者,皆以子厚为师,其经承子厚口讲指画为文词者,悉有法度可观。

其召至京师而复为刺史也,中山刘梦得禹锡亦在遣中,当诣播州。子厚泣曰:“播州非人所居,而梦得亲在堂,吾不忍梦得之穷,无辞以白其大人;且万无母子俱往理。”请于朝,将拜疏,愿以柳易播,虽重得罪,死不恨。遇有以梦得事白上者,梦得于是改刺连州。呜呼!士穷乃见节义。今夫平居里巷相慕悦,酒食游戏相徵逐,诩诩强笑语以相取下,握手出肺肝相示,指天日涕泣,誓生死不相背负,真若可信;一旦临小利害,仅如毛发比,反眼若不相识。落陷穽,不一引手救,反挤之,又下石焉者,皆是也。此宜禽兽夷狄所不忍为,而其人自视以为得计。闻子厚之风,亦可以少愧矣。

子厚前时少年,勇于为人,不自贵重顾籍,谓功业可立就,故坐废退。既退,又无相知有气力得位者推挽,故卒死于穷裔。材不为世用,道不行于时也。使子厚在台省时,自持其身,已能如司马刺史时,亦自不斥;斥时,有人力能举之,且必复用不穷。然子厚斥不久,穷不极,虽有出于人,其文学辞章,必不能自力,以致必传于后如今,无疑也。虽使子厚得所愿,为将相于一时,以彼易此,孰得孰失,必有能辨之者。

子厚以元和十四年十一月八日卒,年四十七。以十五年七月十日,归葬万年先人墓侧。子厚有子男二人:长曰周六,始四岁;季曰周七,子厚卒乃生。女子二人,皆幼。其得归葬也,费皆出观察使河东裴君行立。行立有节概,重然诺,与子厚结交,子厚亦为之尽,竟赖其力。葬子厚于万年之墓者,舅弟卢遵。遵,涿人,性谨慎,学问不厌。自子厚之斥,遵从而家焉,逮其死不去。既往葬子厚,又将经纪其家,庶几有始终者。

铭曰:“是惟子厚之室,既固既安,以利其嗣人。”


\chapter*{狼三则}
\addcontentsline{toc}{chapter}{狼三则}
\begin{center}
	\textbf{[清朝]蒲松龄}
\end{center}


其一

有屠人货肉归,日已暮,欻一狼来,瞰担上肉,似甚垂涎,随尾行数里。屠惧,示之以刃,少却;及走,又从之。屠无计,思狼所欲者肉,不如姑悬诸树而早取之。遂钩肉,翘足挂树间,示以空担。狼乃止。屠归。昧爽,往取肉,遥望树上悬巨物,似人缢死状。大骇,逡巡近视之,则死狼也。仰首细审,见狼口中含肉,钩刺狼腭,如鱼吞饵。时狼皮价昂,直十余金,屠小裕焉。缘木求鱼,狼则罹之,是可笑也。


其二

一屠晚归,担中肉尽,止有剩骨。途中两狼,缀行甚远。

屠惧,投以骨。一狼得骨止,一狼仍从。复投之,后狼止而前狼又至。骨已尽矣,而两狼之并驱如故。

屠大窘,恐前后受其敌。顾野有麦场,场主积薪其中,苫蔽成丘。屠乃奔倚其下,弛担持刀。狼不敢前,眈眈相向。

少时,一狼径去,其一犬坐于前。久之,目似瞑,意暇甚。屠暴起,以刀劈狼首,又数刀毙之。方欲行,转视积薪后,一狼洞其中,意将隧入以攻其后也。身已半入,止露尻尾。屠自后断其股,亦毙之。乃悟前狼假寐,盖以诱敌。

狼亦黠矣,而顷刻两毙,禽兽之变诈几何哉?止增笑耳。


其三

一屠暮行,为狼所逼。道旁有夜耕所遗行室,奔入伏焉。狼自苫中探爪入。屠急捉之,令不可去。但思无计可以死之。惟有小刀不盈寸,遂割破狼爪下皮,以吹豕之法吹之。极力吹移时,觉狼不甚动,方缚以带。出视,则狼胀如牛,股直不能屈,口张不得合。遂负之以归。

非屠,乌能作此谋也!

三事皆出于屠;则屠人之残爆,杀狼亦可用也。



\chapter*{后十九日复上宰相书}
\addcontentsline{toc}{chapter}{后十九日复上宰相书}
\begin{center}
	\textbf{[唐朝]韩愈}
\end{center}

二月十六日,前乡贡进士韩愈,谨再拜言相公阁下:

向上书及所著文后,待命凡十有九日,不得命。恐惧不敢逃遁,不知所为,乃复敢自纳于不测之诛,以求毕其说,而请命于左右。

愈闻之:蹈水火者之求免于人也,不惟其父兄子弟之慈爱,然后呼而望之也。将有介于其侧者,虽其所憎怨,苟不至乎欲其死者,则将大其声疾呼而望其仁之也。彼介于其侧者,闻其声而见其事,不惟其父兄子弟之慈爱,然后往而全之也。虽有所憎怨,苟不至乎欲其死者,则将狂奔尽气,濡手足,焦毛发,救之而不辞也。若是者何哉?其势诚急而其情诚可悲也。

愈之强学力行有年矣。愚不惟道之险夷,行且不息,以蹈于穷饿之水火,其既危且亟矣,大其声而疾呼矣。阁下其亦闻而见之矣,其将往而全之欤?抑将安而不救欤?有来言于阁下者曰:“有观溺于水而爇于火者,有可救之道,而终莫之救也。”阁下且以为仁人乎哉?不然,若愈者,亦君子之所宜动心者也。

或谓愈:“子言则然矣,宰相则知子矣,如时不可何?”愈窃谓之不知言者。诚其材能不足当吾贤相之举耳;若所谓时者,固在上位者之为耳,非天之所为也。前五六年时,宰相荐闻,尚有自布衣蒙抽擢者,与今岂异时哉?且今节度、观察使及防御营田诸小使等,尚得自举判官,无间于已仕未仕者;况在宰相,吾君所尊敬者,而曰不可乎?古之进人者,或取于盗,或举于管库。今布衣虽贱,犹足以方乎此。情隘辞蹙,不知所裁,亦惟少垂怜焉。

愈再拜。


\chapter*{吴子使札来聘}
\addcontentsline{toc}{chapter}{吴子使札来聘}
\begin{center}
	\textbf{[春秋战国]公羊高}
\end{center}

吴无君,无大夫,此何以有君,有大夫?贤季子也。何贤乎季子?让国也。其让国奈何?谒也,馀祭也,夷昧也,与季子同母者四。季子弱而才,兄弟皆爱之,同欲立之以为君。谒曰:“今若是迮而与季子国,季子犹不受也。请无与子而与弟,弟兄迭为君,而致国乎季子。”皆曰诺。故诸为君者皆轻死为勇,饮食必祝,曰:“天苟有吴国,尚速有悔于予身。”故谒也死,馀祭也立。馀祭也死,夷昧也立。夷昧也死,则国宜之季子者也,季子使而亡焉。僚者长庶也,即之。季之使而反,至而君之尔。阖庐曰:“先君之所以不与子国,而与弟者,凡为季子故也。将从先君之命与,则国宜之季子者也;如不从先君之命与子,我宜当立者也。僚恶得为君?”于是使专诸刺僚,而致国乎季子。季子不受,曰:“尔杀吾君,吾受尔国,是吾与尔为篡也。尔杀吾兄,吾又杀尔,是父子兄弟相杀,终身无已也。”去之延陵,终身不入吴国。故君子以其不受为义,以其不杀为仁,贤季子。则吴何以有君,有大夫?以季子为臣,则宜有君者也。札者何?吴季子之名也。春秋贤者不名,此何以名?许夷狄者,不一而足也。季子者,所贤也,曷为不足乎季子?许人臣者必使臣,许人子者必使子也。


\chapter*{哀时命}
\addcontentsline{toc}{chapter}{哀时命}
\begin{center}
	\textbf{[汉朝]庄忌}
\end{center}


\begin{center}
    
    哀时命之不及古人兮,夫何予生之不遘时!
    
    往者不可扳援兮,徠者不可与期。
    
    志憾恨而不逞兮,杼中情而属诗。
    
    夜炯炯而不寐兮,怀隐忧而历兹。
    
    心郁郁而无告兮,众孰可与深谋!
    
    欿愁悴而委惰兮,老冉冉而逮之。
    
    居处愁以隐约兮,志沉抑而不扬。
    
    道壅塞而不通兮,江河广而无梁。
    
    愿至昆仑之悬圃兮,采锺山之玉英。
    
    揽瑶木之橝枝兮,望阆风之板桐。
    
    弱水汩其为难兮,路中断而不通。
    
    势不能凌波以径度兮,又无羽翼而高翔。
    
    然隐悯而不达兮,独徙倚而彷徉。
    
    怅惝罔以永思兮,心纡轸而增伤。
    
    倚踌躇以淹留兮,日饥馑而绝粮。
    
    廓抱景而独倚兮,超永思乎故乡。
    
    廓落寂而无友兮,谁可与玩此遗芳?
    
    白日晼晼其將入兮,哀余寿之弗将。
    
    车既弊而马罢兮,蹇邅徊而不能行。
    
    身既不容于浊世兮,不知进退之宜当。
    
    冠崔嵬而切云兮,剑淋离而从横。
    
    衣摄叶以储与兮,左袪挂于榑桑;
    
    右衽拂于不周兮,六合不足以肆行。
    
    上同凿枘于伏戏兮,下合矩矱于虞唐。
    
    原尊节而式高兮,志犹卑夫禹汤。
    
    虽知困其不改操兮,终不以邪枉害方。
    
    世并举而好朋兮,壹斗斛而相量。
    
    众比周以肩迫兮,贤者远而隐藏。
    
    为凤皇作鹑笼兮,虽翕翅其不容。
    
    灵皇其不寤知兮,焉陈词而效忠。
    
    俗嫉妒而蔽贤兮,孰知余之从容?
    
    愿舒志而抽冯兮,庸讵知其吉凶?
    
    璋珪杂于甑窐兮,陇廉与孟娵同宫。
    
    举世以为恆俗兮,固将愁苦而终穷。
    
    幽独转而不寐兮,惟烦懑而盈匈。
    
    魂眇眇而驰骋兮,心烦冤之忡忡。
    
    志欿憾而不憺兮,路幽昧而甚难。
    
    塊独守此曲隅兮,然欿切而永叹。
    
    愁修夜而宛转兮,气涫沸其若波。
    
    握剞劂而不用兮,操规矩而无所施。
    
    骋骐骥于中庭兮,焉能极夫远道?
    
    置援狖于棂槛兮,夫何以责其捷巧?
    
    驷跛鳖而上山兮,吾固知其不能陞。
    
    释管晏而任臧获兮,何权衡之能称?
    
    箟簬杂于黀蒸兮,机蓬矢以射革。
    
    负檐荷以丈尺兮,欲伸要而不可得。
    
    外迫胁于机臂兮,上牵联于矰隿。
    
    肩倾侧而不容兮,固陿腹而不得息。
    
    务光自投于深渊兮,不获世之尘垢。
    
    孰魁摧之可久兮,愿退身而穷处。
    
    凿山楹而为室兮,下被衣于水渚。
    
    雾露濛濛其晨降兮,云依斐而承宇。
    
    虹霓纷其朝霞兮,夕淫淫而淋雨。
    
    怊茫茫而无归兮,怅远望此旷野。
    
    下垂钓于溪谷兮,上要求于仙者。
    
    与赤松而结友兮,比王侨而为耦。
    
    使枭杨先导兮,白虎为之前後。
    
    浮云雾而入冥兮,骑白鹿而容与。
    
    魂眐眐以寄独兮,汨徂往而不归。
    
    处卓卓而日远兮,志浩荡而伤怀。
    
    鸾凤翔于苍云兮,故矰缴而不能加。
    
    蛟龙潜于旋渊兮,身不挂于罔罗。
    
    知贪饵而近死兮,不如下游乎清波。
    
    宁幽隐以远祸兮,孰侵辱之可为。
    
    子胥死而成义兮,屈原沉于汨罗。
    
    虽体解其不变兮,岂忠信之可化。
    
    志怦怦而内直兮,履绳墨而不颇。
    
    执权衡而无私兮,称轻重而不差。
    
    摡尘垢之枉攘兮,除秽累而反真。
    
    形体白而质素兮,中皎洁而淑清。
    
    时猒饫而不用兮,且隐伏而远身。
    
    聊窜端而匿迹兮,嗼寂默而无声。
    
    独便悁而烦毒兮,焉发愤而筊抒。
    
    时暧暧其将罢兮,遂闷叹而无名。
    
    伯夷死于首阳兮,卒夭隐而不荣。
    
    太公不遇文王兮,身至死而不得逞。
    
    怀瑶象而佩琼兮,愿陈列而无正。
    
    生天坠之若过兮,忽烂漫而无成。
    
    邪气袭余之形体兮,疾憯怛而萌生。
    
    原壹见阳春之白日兮,恐不终乎永年。
\end{center}



\chapter*{与陈伯之书}
\addcontentsline{toc}{chapter}{与陈伯之书}
\begin{center}
	\textbf{[南北朝]丘迟}
\end{center}

迟顿首陈将军足下:无恙,幸甚,幸甚!将军勇冠三军,才为世出,弃燕雀之小志,慕鸿鹄以高翔!昔因机变化,遭遇明主,立功立事,开国称孤。朱轮华毂,拥旄万里,何其壮也!如何一旦为奔亡之虏,闻鸣镝而股战,对穹庐以屈膝,又何劣邪!

寻君去就之际,非有他故,直以不能内审诸己,外受流言,沈迷猖蹶,以至于此。圣朝赦罪责功,弃瑕录用,推赤心于天下,安反侧于万物。将军之所知,不假仆一二谈也。朱鲔涉血于友于,张绣剚刃於爱子,汉主不以为疑,魏君待之若旧。况将军无昔人之罪,而勋重於当世!夫迷途知返,往哲是与,不远而复,先典攸高。主上屈法申恩,吞舟是漏;将军松柏不剪,亲戚安居,高台未倾,爱妾尚在;悠悠尔心,亦何可言!今功臣名将,雁行有序,佩紫怀黄,赞帷幄之谋,乘轺建节,奉疆埸之任,并刑马作誓,传之子孙。将军独靦颜借命,驱驰毡裘之长,宁不哀哉!

夫以慕容超之强,身送东市;姚泓之盛,面缚西都。故知霜露所均,不育异类;姬汉旧邦,无取杂种。北虏僭盗中原,多历年所,恶积祸盈,理至燋烂。况伪孽昏狡,自相夷戮,部落携离,酋豪猜贰。方当系颈蛮邸,悬首藁街,而将军鱼游於沸鼎之中,燕巢於飞幕之上,不亦惑乎?

暮春三月,江南草长,杂花生树,群莺乱飞。见故国之旗鼓,感平生于畴日,抚弦登陴,岂不怆悢!

所以廉公之思赵将,吴子之泣西河,人之情也,将军独无情哉?想早励良规,自求多福。

当今皇帝盛明,天下安乐。白环西献,楛矢东来;夜郎滇池,解辫请职;朝鲜昌海,蹶角受化。唯北狄野心,掘强沙塞之间,欲延岁月之命耳!中军临川殿下,明德茂亲,揔兹戎重,吊民洛汭,伐罪秦中,若遂不改,方思仆言。聊布往怀,君其详之。丘迟顿首。


\chapter*{工之侨献琴}
\addcontentsline{toc}{chapter}{工之侨献琴}
\begin{center}
	\textbf{[明朝]刘基}
\end{center}


工之侨得良桐焉,斫而为琴,弦而鼓之,金声而玉应。自以为天下之美也,献之太常。使国工视之,曰:“弗古。”还之。


工之侨以归,谋诸漆工,作断纹焉;又谋诸篆工,作古窾焉。匣而埋诸土,期年出之,抱以适市。贵人过而见之,易之以百金,献诸朝。乐官传视,皆曰:“希世之珍也。”


工之侨闻之,叹曰:“悲哉世也!岂独一琴哉?莫不然矣!而不早图之,其与亡矣。”遂去,入于宕之山,不知其所终。



\chapter*{徐文长传}
\addcontentsline{toc}{chapter}{徐文长传}
\begin{center}
	\textbf{[明朝]袁宏道}
\end{center}


余少时过里肆中,见北杂剧有《四声猿》,意气豪达,与近时书生所演传奇绝异,题曰“天池生”,疑为元人作。后适越,见人家单幅上有署“田水月”者,强心铁骨,与夫一种磊块不平之气,字画之中,宛宛可见。意甚骇之,而不知田水月为何人。


一夕,坐陶编修楼,随意抽架上书,得《阙编》诗一帙。恶楮毛书,烟煤败黑,微有字形。稍就灯间读之,读未数首,不觉惊跃,忽呼石篑:“《阙编》何人作者?今耶?古耶?”石篑曰:“此余乡先辈徐天池先生书也。先生名渭,字文长,嘉、隆间人,前五六年方卒。今卷轴题额上有田水月者,即其人也。”余始悟前后所疑,皆即文长一人。又当诗道荒秽之时,获此奇秘,如魇得醒。两人跃起,灯影下,读复叫,叫复读,僮仆睡者皆惊起。余自是或向人,或作书,皆首称文长先生。有来看余者,即出诗与之读。一时名公巨匠,浸浸知向慕云。


文长为山阴秀才,大试辄不利,豪荡不羁。总督胡梅林公知之,聘为幕客。文长与胡公约:“若欲客某者,当具宾礼,非时辄得出入。”胡公皆许之。文长乃葛衣乌巾,长揖就坐,纵谈天下事,旁若无人。胡公大喜。是时公督数边兵,威振东南,介胄之士,膝语蛇行,不敢举头;而文长以部下一诸生傲之,信心而行,恣臆谈谑,了无忌惮。会得白鹿,属文长代作表。表上,永陵喜甚。公以是益重之,一切疏记,皆出其手。


文长自负才略,好奇计,谈兵多中。凡公所以饵汪、徐诸虏者,皆密相议然后行。尝饮一酒楼,有数健儿亦饮其下,不肯留钱。文长密以数字驰公,公立命缚健儿至麾下,皆斩之,一军股栗。有沙门负资而秽,酒间偶言于公,公后以他事杖杀之。其信任多此类。


胡公既怜文长之才,哀其数困,时方省试,凡入帘者,公密属曰:“徐子,天下才,若在本房,幸勿脱失。”皆曰:“如命。”一知县以他羁后至,至期方谒公,偶忘属,卷适在其房,遂不偶。


文长既已不得志于有司,遂乃放浪曲糵,恣情山水,走齐、鲁、燕、赵之地,穷览朔漠。其所见山奔海立,沙起云行,风鸣树偃,幽谷大都,人物鱼鸟,一切可惊可愕之状,一一皆达之于诗。其胸中又有一段不可磨灭之气,英雄失路、托足无门之悲,故其为诗,如嗔如笑,如水鸣峡,如种出土,如寡妇之夜哭,羁人之寒起。当其放意,平畴千里;偶尔幽峭,鬼语秋坟。文长眼空千古,独立一时。当时所谓达官贵人、骚士墨客,文长皆叱而奴之,耻不与交,故其名不出于越。悲夫!


一日,饮其乡大夫家。乡大夫指筵上一小物求赋,阴令童仆续纸丈余进,欲以苦之。文长援笔立成,竟满其纸,气韵遒逸,物无遁情,一座大惊。


文长喜作书,笔意奔放如其诗,苍劲中姿媚跃出。余不能书,而谬谓文长书决当在王雅宜、文征仲之上。不论书法,而论书神:先生者,诚八法之散圣,字林之侠客也。间以其余,旁溢为花草竹石,皆超逸有致。


卒以疑杀其继室,下狱论死。张阳和力解,乃得出。既出,倔强如初。晚年愤益深,佯狂益甚。显者至门,皆拒不纳。当道官至,求一字不可得。时携钱至酒肆,呼下隶与饮。或自持斧击破其头,血流被面,头骨皆折,揉之有声。或槌其囊,或以利锥锥其两耳,深入寸余,竟不得死。


石篑言:晚岁诗文益奇,无刻本,集藏于家。予所见者,《徐文长集》、《阙编》二种而已。然文长竟以不得志于时,抱愤而卒。


石公曰:先生数奇不已,遂为狂疾;狂疾不已,遂为囹圄。古今文人,牢骚困苦,未有若先生者也。虽然,胡公间世豪杰,永陵英主,幕中礼数异等,是胡公知有先生矣;表上,人主悦,是人主知有先生矣。独身未贵耳。先生诗文崛起,一扫近代芜秽之习,百世而下,自有定论,胡为不遇哉?梅客生尝寄余书曰:“文长吾老友,病奇于人,人奇于诗,诗奇于字,字奇于文,文奇于画。”余谓文长无之而不奇者也。无之而不奇,斯无之而不奇也哉!悲夫!



\chapter*{司马季主论卜}
\addcontentsline{toc}{chapter}{司马季主论卜}
\begin{center}
	\textbf{[明朝]刘基}
\end{center}


东陵侯既废,过司马季主而卜焉。季主曰:“君侯何卜也?”东陵侯曰:“久卧者思起,久蛰者思启,久懑者思嚏。吾闻之蓄极则泄,閟极则达。热极则风,壅极则通。一冬一春,靡屈不伸,一起一伏,无往不复。仆窃有疑,愿受教焉。”季主曰:“若是,则君侯已喻之矣,又何卜为?”东陵侯曰:“仆未究其奥也,愿先生卒教之。”季主乃言曰:“呜呼!天道何亲?惟德之亲;鬼神何灵?因人而灵。夫蓍,枯草也;龟,枯骨也,物也。人,灵于物者也,何不自听而听于物乎?且君侯何不思昔者也?有昔者必有今日,是故碎瓦颓垣,昔日之歌楼舞馆也;荒榛断梗,昔日之琼蕤玉树也;露蛬风蝉,昔日之凤笙龙笛也;鬼燐萤火,昔日之金釭华烛也;秋荼春荠,昔日之象白驼峰也;丹枫白荻,昔日之蜀锦齐纨也。昔日之所无,今日有之不为过;昔日之所有,今日无之不为不足。是故一昼一夜,华开者谢;一秋一春,物故者新。激湍之下,必有深潭;高丘之下,必有浚谷。君侯亦知之矣,何以卜为?”

\chapter*{寺人披见文公}
\addcontentsline{toc}{chapter}{寺人披见文公}
\begin{center}
	\textbf{[春秋战国]左丘明}
\end{center}


吕、郤畏逼,将焚公宫而弑晋侯。寺人披请见。公使让之,且辞焉,曰:“蒲城之役,君命一宿,女即至。其后余从狄君以田渭滨,女为惠公来求杀余,命女三宿,女中宿至。虽有君命何其速也?夫袪犹在,女其行乎!”对曰:“臣谓君之入也,其知之矣。若犹未也,又将及难。君命无二,古之制也。除君之恶,唯力是视。蒲人、狄人、余何有焉?即位,其无蒲、狄乎!齐桓公置射钩,而使管仲相。君若易之,何辱命焉?行者甚众,岂唯刑臣?”公见之,以难告。晋侯潜会秦伯于王城。己丑晦,公宫火。瑕甥、郤芮不获公,乃如河上,秦伯诱而杀之。

\chapter*{昔齐攻鲁,求其岑鼎}
\addcontentsline{toc}{chapter}{昔齐攻鲁,求其岑鼎}
\begin{center}
	\textbf{[春秋战国]左丘明}
\end{center}


昔齐攻鲁,求其岑鼎.鲁侯伪献他鼎而请盟焉。齐侯不信,曰:“若柳季云是,则请受之。”鲁欲使柳季。柳季曰:“君以鼎为国,信者亦臣之国,今欲破臣之国,全君之国,臣所难”鲁侯乃献岑鼎。

\chapter*{吴山图记}
\addcontentsline{toc}{chapter}{吴山图记}
\begin{center}
	\textbf{[明朝]归有光}
\end{center}


吴、长洲二县,在郡治所,分境而治。而郡西诸山,皆在吴县。其最高者,穹窿、阳山、邓尉、西脊、铜井。而灵岩,吴之故宫在焉,尚有西子之遗迹。若虎丘、剑池及天平、尚方、支硎,皆胜地也。而太湖汪洋三万六千顷,七十二峰沉浸其间,则海内之奇观矣。


余同年友魏君用晦为吴县,未及三年,以高第召入为给事中。君之为县,有惠爱,百姓扳留之,不能得,而君亦不忍于其民。由是好事者绘《吴山图》以为赠。


夫令之于民,诚重矣。令诚贤也,其地之山川草木,亦被其泽而有荣也;令诚不贤也,其地之山川草木,亦被其殃而有辱也。君于吴之山川,盖增重矣。异时吾民将择胜于岩峦之间,尸祝于浮屠、老子之宫也,固宜。而君则亦既去矣,何复惓惓于此山哉?昔苏子瞻称韩魏公去黄州四十馀年而思之不忘,至以为《思黄州》诗,子瞻为黄人刻之于石。然后知贤者于其所至,不独使其人之不忍忘而已,亦不能自忘于其人也。


君今去县已三年矣。一日,与余同在内庭,出示此图,展玩太息,因命余记之,噫!君之于吾吴有情如此,如之何而使吾民能忘之也!


\chapter*{张益州画像记}
\addcontentsline{toc}{chapter}{张益州画像记}
\begin{center}
	\textbf{[宋朝]苏洵}
\end{center}


至和元年秋,蜀人传言有寇至,边军夜呼,野无居人,谣言流闻,京师震惊。方命择帅,天子曰:“毋养乱,毋助变。众言朋兴,朕志自定。外乱不作,变且中起,不可以文令,又不可以武竞,惟朕一二大吏。孰为能处兹文武之间,其命往抚朕师?”乃推曰:张公方平其人。天子曰:“然。”公以亲辞,不可,遂行。


冬十一月至蜀,至之日,归屯军,撤守备,使谓郡县:“寇来在吾,无尔劳苦。”明年正月朔旦,蜀人相庆如他日,遂以无事。又明年正月,相告留公像于净众寺,公不能禁。


眉阳苏洵言于众曰:“未乱,易治也;既乱,易治也;有乱之萌,无乱之形,是谓将乱,将乱难治,不可以有乱急,亦不可以无乱弛。惟是元年之秋,如器之欹,未坠于地。惟尔张公,安坐于其旁,颜色不变,徐起而正之。既正,油然而退,无矜容。为天子牧小民不倦,惟尔张公。尔繄以生,惟尔父母。且公尝为我言‘民无常性,惟上所待。人皆曰蜀人多变,于是待之以待盗贼之意,而绳之以绳盗贼之法。重足屏息之民,而以斧令。于是民始忍以其父母妻子之所仰赖之身,而弃之于盗贼,故每每大乱。夫约之以礼,驱之以法,惟蜀人为易。至于急之而生变,虽齐、鲁亦然。吾以齐、鲁待蜀人,而蜀人亦自以齐、鲁之人待其身。若夫肆意于法律之外,以威劫齐民,吾不忍为也。’呜呼!爱蜀人之深,待蜀人之厚,自公而前,吾未始见也。”皆再拜稽首曰:“然。”


苏洵又曰:“公之恩在尔心,尔死在尔子孙,其功业在史官,无以像为也。且公意不欲,如何?”皆曰:“公则何事于斯?虽然,于我心有不释焉。今夫平居闻一善,必问其人之姓名与其乡里之所在,以至于其长短大小美恶之状,甚者或诘其平生所嗜好,以想见其为人。而史官亦书之于其传,意使天下之人,思之于心,则存之于目;存之于目,故其思之于心也固。由此观之,像亦不为无助。”苏洵无以诘,遂为之记。


公,南京人,为人慷慨有大节,以度量雄天下。天下有大事,公可属。系之以诗曰:天子在祚,岁在甲午。西人传言,有寇在垣。庭有武臣,谋夫如云。天子曰嘻,命我张公。公来自东,旗纛舒舒。西人聚观,于巷于涂。谓公暨暨,公来于于。公谓西人“安尔室家,无敢或讹。讹言不祥,往即尔常。春而条桑,秋尔涤场。”西人稽首,公我父兄。公在西囿,草木骈骈。公宴其僚,伐鼓渊渊。西人来观,祝公万年。有女娟娟,闺闼闲闲。有童哇哇,亦既能言。昔公未来,期汝弃捐。禾麻芃芃,仓庾崇崇。嗟我妇子,乐此岁丰。公在朝廷,天子股肱。天子曰归,公敢不承?作堂严严,有庑有庭。公像在中,朝服冠缨。西人相告,无敢逸荒。公归京师,公像在堂。



\chapter*{送虚白上人序}
\addcontentsline{toc}{chapter}{送虚白上人序}
\begin{center}
	\textbf{[明朝]高启}
\end{center}


余始不欲与佛者游,尝读东坡所作《勤上人诗序》,见其称勤之贤曰:“使勤得列于士大夫之间,必不负欧阳公。”余于是悲士大夫之风坏已久,而喜佛者之有可与游者。


去年春,余客居城西,读书之暇,因往云岩诸峰间,求所谓可与游者,而得虚白上人焉。


虚白形癯而神清,居众中不妄言笑。余始识于剑池之上,固心已贤之矣。入其室,无一物,弊箦折铛,尘埃萧然。寒不暖,衣一衲,饥不饱,粥一盂,而逍遥徜徉,若有余乐者。间出所为诗,则又纡徐怡愉,无急迫穷苦之态,正与其人类。


方春二三月时,云岩之游者盛,巨官要人,车马相属。主者撞钟集众,送迎唯谨,虚白方闭户寂坐如不闻;及余至,则曳败履起从,指幽导胜于长林绝壁之下,日入而后已。余益贤虚白,为之太息而有感焉。近世之士大夫,趋于途者骈然,议于庐者欢然,莫不恶约而愿盈,迭夸而交诋,使虚白袭冠带以齿其列,有肯为之者乎?或以虚白佛者也,佛之道贵静而无私,其能是亦宜耳!余曰:今之佛者无呶呶焉肆荒唐之言者乎?无逐逐焉从造请之役者乎?无高屋广厦以居美女丰食以养者乎?然则虚白之贤不惟过吾徒,又能过其徒矣。余是以乐与之游而不知厌也。


今年秋,虚白将东游,来请一言以为赠。余以虚白非有求于世者,岂欲余张之哉?故书所感者如此,一以风乎人,一以省于己,使无或有愧于虚白者而已。



\chapter*{郑庄公戒饬守臣}
\addcontentsline{toc}{chapter}{郑庄公戒饬守臣}
\begin{center}
	\textbf{[春秋战国]左丘明}
\end{center}


秋七月,公会齐侯、郑伯伐许。庚辰,傅于许。颍考叔取郑伯之旗蝥弧以先登,子都自下射之,颠。瑕叔盈又以蝥弧登,周麾而呼曰:“君登矣!”郑师毕登。壬午,遂入许。许庄公奔卫。齐侯以许让公。公曰:“君谓许不共,故从君讨之。许既伏其罪矣。虽君有命,寡人弗敢与闻。”乃与郑人。


郑伯使许大夫百里奉许叔以居许东偏,曰:“天祸许国,鬼神实不逞于许君,而假手于我寡人,寡人唯是一二父兄不能共亿,其敢以许自为功乎?寡人有弟,不能和协,而使糊其口于四方,其况能久有许乎?吾子其奉许叔以抚柔此民也,吾将使获也佐吾子。若寡人得没于地,天其以礼悔祸于许,无宁兹许公复奉其社稷,唯我郑国之有请谒焉,如旧昏媾,其能降以相从也。无滋他族实逼处此,以与我郑国争此土也。吾子孙其覆亡之不暇,而况能禋祀许乎?寡人之使吾子处此,不惟许国之为,亦聊以固吾圉也。”乃使公孙获处许西偏,曰:“凡而器用财贿,无置于许。我死,乃亟去之!吾先君新邑于此,王室而既卑矣,周之子孙日失其序。夫许,大岳之胤也。天而既厌周德矣,吾其能与许争乎?”


君子谓郑庄公“于是乎有礼。礼,经国家,定社稷,序民人,利后嗣者也。许,无刑而伐之,服而舍之,度德而处之,量力而行之,相时而动,无累后人,可谓知礼矣。”(序民人一作:序人民)



\chapter*{唐雎不辱使命}
\addcontentsline{toc}{chapter}{唐雎不辱使命}
\begin{center}
	\textbf{[汉朝]刘向}
\end{center}


秦王使人谓安陵君曰:“寡人欲以五百里之地易安陵,安陵君其许寡人!”安陵君曰:“大王加惠,以大易小,甚善;虽然,受地于先王,愿终守之,弗敢易!”秦王不说。安陵君因使唐雎使于秦。(说通:悦)


秦王谓唐雎曰:“寡人以五百里之地易安陵,安陵君不听寡人,何也?且秦灭韩亡魏,而君以五十里之地存者,以君为长者,故不错意也。今吾以十倍之地,请广于君,而君逆寡人者,轻寡人与?”唐雎对曰:“否,非若是也。安陵君受地于先王而守之,虽千里不敢易也,岂直五百里哉?”


秦王怫然怒,谓唐雎曰:“公亦尝闻天子之怒乎?”唐雎对曰:“臣未尝闻也。”秦王曰:“天子之怒,伏尸百万,流血千里。”唐雎曰:“大王尝闻布衣之怒乎?”秦王曰:“布衣之怒,亦免冠徒跣,以头抢地耳。”唐雎曰:“此庸夫之怒也,非士之怒也。夫专诸之刺王僚也,彗星袭月;聂政之刺韩傀也,白虹贯日;要离之刺庆忌也,仓鹰击于殿上。此三子者,皆布衣之士也,怀怒未发,休祲降于天,与臣而将四矣。若士必怒,伏尸二人,流血五步,天下缟素,今日是也。”挺剑而起。


秦王色挠,长跪而谢之曰:“先生坐!何至于此!寡人谕矣:夫韩、魏灭亡,而安陵以五十里之地存者,徒以有先生也。”



\chapter*{管仲论}
\addcontentsline{toc}{chapter}{管仲论}
\begin{center}
	\textbf{[宋朝]苏洵}
\end{center}


管仲相桓公,霸诸侯,攘夷狄,终其身齐国富强,诸侯不敢叛。管仲死,竖刁、易牙、开方用,威公薨于乱,五公子争立,其祸蔓延,讫简公,齐无宁岁。夫功之成,非成于成之日,盖必有所由起;祸之作,不作于作之日,亦必有所由兆。故齐之治也,吾不曰管仲,而曰鲍叔。及其乱也,吾不曰竖刁、易牙、开方,而曰管仲。何则?竖刁、易牙、开方三子,彼固乱人国者,顾其用之者,威公也。夫有舜而后知放四凶,有仲尼而后知去少正卯。彼威公何人也?顾其使威公得用三子者,管仲也。仲之疾也,公问之相。当是时也,吾意以仲且举天下之贤者以对。而其言乃不过曰:竖刁、易牙、开方三子,非人情,不可近而已。


呜呼!仲以为威公果能不用三子矣乎?仲与威公处几年矣,亦知威公之为人矣乎?威公声不绝于耳,色不绝于目,而非三子者则无以遂其欲。彼其初之所以不用者,徒以有仲焉耳。一日无仲,则三子者可以弹冠而相庆矣。仲以为将死之言可以絷威公之手足耶?夫齐国不患有三子,而患无仲。有仲,则三子者,三匹夫耳。不然,天下岂少三子之徒哉?虽威公幸而听仲,诛此三人,而其余者,仲能悉数而去之耶?呜呼!仲可谓不知本者矣。因威公之问,举天下之贤者以自代,则仲虽死,而齐国未为无仲也。夫何患三子者?不言可也。五伯莫盛于威、文,文公之才,不过威公,其臣又皆不及仲;灵公之虐,不如孝公之宽厚。文公死,诸侯不敢叛晋,晋习文公之余威,犹得为诸侯之盟主百余年。何者?其君虽不肖,而尚有老成人焉。威公之薨也,一乱涂地,无惑也,彼独恃一管仲,而仲则死矣。


夫天下未尝无贤者,盖有有臣而无君者矣。威公在焉,而曰天下不复有管仲者,吾不信也。仲之书,有记其将死论鲍叔、宾胥无之为人,且各疏其短。是其心以为数子者皆不足以托国。而又逆知其将死,则其书诞谩不足信也。吾观史鰌,以不能进蘧伯玉,而退弥子瑕,故有身后之谏。萧何且死,举曹参以自代。大臣之用心,固宜如此也。夫国以一人兴,以一人亡。贤者不悲其身之死,而忧其国之衰,故必复有贤者,而后可以死。彼管仲者,何以死哉?



\chapter*{齐国佐不辱命}
\addcontentsline{toc}{chapter}{齐国佐不辱命}
\begin{center}
	\textbf{[春秋战国]左丘明}
\end{center}


晋师从齐师,入自丘舆,击马陉。


齐侯使宾媚人赂以纪甗、玉磬与地。“不可,则听客之所为。”


宾媚人致赂,晋人不可,曰:“必以肖同叔子为质,而使齐之封内尽东其亩。”对曰:“肖同叔子非他,寡君之母也;若以匹敌,则亦晋君之母也。吾子布大命于诸侯,而曰必质其母以为信,其若王命何?且是以不孝令也。诗曰:‘孝子不匮,永锡尔类。’若以不孝令于诸侯,其无乃非德类也乎?先王疆理天下,物土之宜,而布其利。故诗曰:‘我疆我理,南东其亩。’今吾子疆理诸侯,而曰‘尽东其亩’而已;唯吾子戎车是利,无顾土宜,其无乃非先王之命也乎?反先王则不义,何以为盟主?其晋实有阙。四王之王也,树德而济同欲焉;五伯之霸也,勤而抚之,以役王命;今吾子求合诸侯,以逞无疆之欲。诗曰:‘布政优优,百禄是遒。’子实不优,而弃百禄,诸侯何害焉?不然,寡君之命使臣,则有辞矣。曰‘子以君师辱于敝邑,不腆敝赋,以犒从者;畏君之震,师徒桡败。吾子惠徼齐国之福,不泯其社稷,使继旧好,唯是先君之敝器、土地不敢爱。子又不许,请收合馀烬,背城借一。敝邑之幸,亦云从也;况其不幸,敢不唯命是听?’”



\chapter*{宋定伯捉鬼}
\addcontentsline{toc}{chapter}{宋定伯捉鬼}
\begin{center}
	\textbf{[晋朝]干宝}
\end{center}


南阳宋定伯,年少时,夜行逢鬼。问之,鬼言:“我是鬼。”鬼问:“汝复谁?”定伯诳之,言:“我亦鬼。”鬼问:“欲至何所?”答曰:“欲至宛市。”鬼言:“我亦欲至宛市。”遂行。


数里,鬼言:“步行太亟,可共递相担,何如?”定伯曰:“大善。”鬼便先担定伯数里。鬼言:“卿太重,将非鬼也?”定伯言:“我新鬼,故身重耳。”定伯因复担鬼,鬼略无重。如是再三。定伯复言:“我新鬼,不知有何所畏忌?”鬼答言:“惟不喜人唾。”于是共行。道遇水,定伯令鬼先渡,听之,了然无声音。定伯自渡,漕漼作声。鬼复言:“何以作声?”定伯曰:“新死,不习渡水故耳,勿怪吾也。”


行欲至宛市,定伯便担鬼著肩上,急持之。鬼大呼,声咋咋然,索下,不复听之。径至宛市中下著地,化为一羊,便卖之恐其变化,唾之。得钱千五百,乃去。于时石崇言:“定伯卖鬼,得钱千五百文。”



\chapter*{病梅馆记}
\addcontentsline{toc}{chapter}{病梅馆记}
\begin{center}
	\textbf{[清朝]龚自珍}
\end{center}

江宁之龙蟠,苏州之邓尉,杭州之西溪,皆产梅。或曰:“梅以曲为美,直则无姿;以欹为美,正则无景;以疏为美,密则无态。”固也。此文人画士,心知其意,未可明诏大号以绳天下之梅也;又不可以使天下之民斫直,删密,锄正,以夭梅病梅为业以求钱也。梅之欹之疏之曲,又非蠢蠢求钱之民能以其智力为也。有以文人画士孤癖之隐明告鬻梅者,斫其正,养其旁条,删其密,夭其稚枝,锄其直,遏其生气,以求重价,而江浙之梅皆病。文人画士之祸之烈至此哉!

予购三百盆,皆病者,无一完者。既泣之三日,乃誓疗之:纵之顺之,毁其盆,悉埋于地,解其棕缚;以五年为期,必复之全之。予本非文人画士,甘受诟厉,辟病梅之馆以贮之。

呜呼!安得使予多暇日,又多闲田,以广贮江宁、杭州、苏州之病梅,穷予生之光阴以疗梅也哉!


\chapter*{游虞山记}
\addcontentsline{toc}{chapter}{游虞山记}
\begin{center}
	\textbf{[清朝]沈德潜}
\end{center}

虞山去吴城才百里,屡欲游,未果。辛丑秋,将之江阴,舟行山下,望剑门入云际,未及登。丙午春,复如江阴,泊舟山麓,入吾谷,榜人诡云:“距剑门二十里。”仍未及登。

壬子正月八日,偕张子少弋、叶生中理往游,宿陶氏。明晨,天欲雨,客无意往,余已治筇屐,不能阻。自城北沿缘六七里,入破山寺,唐常建咏诗处,今潭名空心,取诗中意也。遂从破龙涧而上,山脉怒坼,赭石纵横,神物爪角痕,时隐时露。相传龙与神斗,龙不胜,破其山而去。说近荒惑,然有迹象,似可信。行四五里,层折而度,越峦岭,跻蹬道,遂陟椒极。有土坯磈礧,疑古时冢,然无碑碣志谁某。升望海墩,东向凝睇。是时云光黯甚,迷漫一色,莫辨瀛海。顷之,雨至,山有古寺可驻足,得少休憩。雨歇,取径而南,益露奇境:龈腭摩天,崭绝中断,两崖相嵌,如关斯劈,如刃斯立,是为剑门。以剑州、大剑、小剑拟之,肖其形也。侧足延,不忍舍去。遇山僧,更问名胜处。僧指南为太公石室;南而西为招真宫,为读书台;西北为拂水岩,水下奔如虹,颓风逆施,倒跃而上,上拂数十丈,又西有三杳石、石城、石门,山后有石洞通海,时潜海物,人莫能名。余识其言,欲问道往游,而云之飞浮浮,风之来冽冽,时雨飘洒,沾衣湿裘,而余与客难暂留矣。少霁,自山之面下,困惫而归。自是春阴连旬,不能更游。

噫嘻!虞山近在百里,两经其下,为践游屐。今之其地矣,又稍识面目,而幽邃窈窕,俱未探历。心甚怏怏。然天下之境,涉而即得,得而辄尽者,始焉欣欣,继焉索索,欲求余味,而了不可得,而得之甚艰,且得半而止者,转使人有无穷之思也。呜呼!岂独寻山也哉!


\chapter*{张衡传}
\addcontentsline{toc}{chapter}{张衡传}
\begin{center}
	\textbf{[南北朝]范晔}
\end{center}

张衡字平子,南阳西鄂人也。衡少善属文,游于三辅,因入京师,观太学,遂通五经,贯六艺。虽才高于世,而无骄尚之情。常从容淡静,不好交接俗人。永元中,举孝廉不行,连辟公府不就。时天下承平日久,自王侯以下,莫不逾侈。衡乃拟班固《两都》作《二京赋》,因以讽谏。精思傅会,十年乃成。大将军邓骘奇其才,累召不应。

衡善机巧,尤致思于天文、阴阳、历算。安帝雅闻衡善术学,公车特征拜郎中,再迁为太史令。遂乃研核阴阳,妙尽璇玑之正,作浑天仪,著《灵宪》、《算罔论》,言甚详明。

顺帝初,再转,复为太史令。衡不慕当世,所居之官辄积年不徙。自去史职,五载复还。

阳嘉元年,复造候风地动仪。以精铜铸成,员径八尺,合盖隆起,形似酒尊,饰以篆文山龟鸟兽之形。中有都柱,傍行八道,施关发机。外有八龙,首衔铜丸,下有蟾蜍,张口承之。其牙机巧制,皆隐在尊中,覆盖周密无际。如有地动,尊则振龙,机发吐丸,而蟾蜍衔之。振声激扬,伺者因此觉知。虽一龙发机,而七首不动,寻其方面,乃知震之所在。验之以事,合契若神。自书典所记,未之有也。尝一龙机发而地不觉动,京师学者咸怪其无征。后数日驿至,果地震陇西,于是皆服其妙。自此以后,乃令史官记地动所从方起。

时政事渐损,权移于下,衡因上疏陈事。后迁侍中,帝引在帷幄,讽议左右。尝问天下所疾恶者。宦官惧其毁己,皆共目之,衡乃诡对而出。阉竖恐终为其患,遂共谗之。衡常思图身之事,以为吉凶倚仗,幽微难明。乃作《思玄赋》以宣寄情志。

永和初,出为河间相。时国王骄奢,不遵典宪;又多豪右,共为不轨。衡下车,治威严,整法度,阴知奸党名姓,一时收禽,上下肃然,称为政理。视事三年,上书乞骸骨,征拜尚书。年六十二,永和四年卒。


\chapter*{讳辩}
\addcontentsline{toc}{chapter}{讳辩}
\begin{center}
	\textbf{[唐朝]韩愈}
\end{center}

愈与李贺书,劝贺举进士。贺举进士有名,与贺争名者毁之,曰贺父名晋肃,贺不举进士为是,劝之举者为非。听者不察也,和而唱之,同然一辞。皇甫湜曰:“若不明白,子与贺且得罪。”愈曰:“然。”

律曰:“二名不偏讳。”释之者曰:“谓若言‘征’不称‘在’,言‘在’不称‘征’是也。”律曰:“不讳嫌名。”释之者曰:“谓若‘禹’与‘雨’、‘丘’与‘蓲’之类是也。”今贺父名晋肃,贺举进士,为犯二名律乎?为犯嫌名律乎?父名晋肃,子不得举进士,若父名仁,子不得为人乎?夫讳始于何时?作法制以教天下者,非周公孔子欤?周公作诗不讳,孔子不偏讳二名,《春秋》不讥不讳嫌名,康王钊之孙,实为昭王。曾参之父名晳,曾子不讳昔。周之时有骐期,汉之时有杜度,此其子宜如何讳?将讳其嫌遂讳其姓乎?将不讳其嫌者乎?汉讳武帝名彻为通,不闻又讳车辙之辙为某字也;讳吕后名雉为野鸡,不闻又讳治天下之治为某字也。今上章及诏,不闻讳浒、势、秉、机也。惟宦官宫妾,乃不敢言谕及机,以为触犯。士君子言语行事,宜何所法守也?今考之于经,质之于律,稽之以国家之典,贺举进士为可邪?为不可邪?

凡事父母,得如曾参,可以无讥矣;作人得如周公孔子,亦可以止矣。今世之士,不务行曾参周公孔子之行,而讳亲之名,则务胜于曾参周公孔子,亦见其惑也。夫周公孔子曾参卒不可胜,胜周公孔子曾参,乃比于宦者宫妾,则是宦者宫妾之孝于其亲,贤于周公孔子曾参者邪?


\chapter*{送孟东野序}
\addcontentsline{toc}{chapter}{送孟东野序}
\begin{center}
	\textbf{[唐朝]韩愈}
\end{center}

大凡物不得其平则鸣:草木之无声,风挠之鸣。水之无声,风荡之鸣。其跃也,或激之;其趋也,或梗之;其沸也,或炙之。金石之无声,或击之鸣。人之于言也亦然,有不得已者而后言。其歌也有思,其哭也有怀,凡出乎口而为声者,其皆有弗平者乎!

乐也者,郁于中而泄于外者也,择其善鸣者而假之鸣。金、石、丝、竹、匏、土、革、木八者,物之善鸣者也。维天之于时也亦然,择其善鸣者而假之鸣。是故以鸟鸣春,以雷鸣夏,以虫鸣秋,以风鸣冬。四时之相推敚,其必有不得其平者乎?

其于人也亦然。人声之精者为言,文辞之于言,又其精也,尤择其善鸣者而假之鸣。其在唐、虞,咎陶、禹,其善鸣者也,而假以鸣,夔弗能以文辞鸣,又自假于《韶》以鸣。夏之时,五子以其歌鸣。伊尹鸣殷,周公鸣周。凡载于《诗》、《书》六艺,皆鸣之善者也。周之衰,孔子之徒鸣之,其声大而远。传曰:“天将以夫子为木铎。”其弗信矣乎!其末也,庄周以其荒唐之辞鸣。楚,大国也,其亡也以屈原鸣。臧孙辰、孟轲、荀卿,以道鸣者也。杨朱、墨翟、管夷吾、晏婴、老聃、申不害、韩非、慎到、田骈、邹衍、尸佼、孙武、张仪、苏秦之属,皆以其术鸣。秦之兴,李斯鸣之。汉之时,司马迁、相如、扬雄,最其善鸣者也。其下魏晋氏,鸣者不及于古,然亦未尝绝也。就其善者,其声清以浮,其节数以急,其辞淫以哀,其志弛以肆;其为言也,乱杂而无章。将天丑其德莫之顾邪?何为乎不鸣其善鸣者也!

唐之有天下,陈子昂、苏源明、元结、李白、杜甫、李观,皆以其所能鸣。其存而在下者,孟郊东野始以其诗鸣。其高出魏晋,不懈而及于古,其他浸淫乎汉氏矣。从吾游者,李翱、张籍其尤也。三子者之鸣信善矣。抑不知天将和其声,而使鸣国家之盛邪,抑将穷饿其身,思愁其心肠,而使自鸣其不幸邪?三子者之命,则悬乎天矣。其在上也奚以喜,其在下也奚以悲!东野之役于江南也,有若不释然者,故吾道其于天者以解之。


\chapter*{子产论政宽猛}
\addcontentsline{toc}{chapter}{子产论政宽猛}
\begin{center}
	\textbf{[春秋战国]左丘明}
\end{center}


郑子产有疾。谓子大叔曰:“我死,子必为政。唯有德者能以宽服民,其次莫如猛。夫火烈,民望而畏之,故鲜死焉。水懦弱,民狎而玩之,则多死焉,故宽难。”疾数月而卒。


大叔为政,不忍猛而宽。郑国多盗,取人于萑苻之泽。大叔悔之,曰:“吾早从夫子,不及此。”兴徒兵以攻萑苻之盗,尽杀之,盗少止。


仲尼曰:“善哉!政宽则民慢,慢则纠之以猛。猛则民残,残则施之以宽。宽以济猛;猛以济宽,政是以和。”《诗》曰:‘民亦劳止,汔可小康;惠此中国,以绥四方。’施之以宽也。‘毋从诡随,以谨无良;式遏寇虐,惨不畏明。’纠之以猛也。‘柔远能迩,以定我王。’平之以和也。又曰:‘不竞不絿,不刚不柔,布政优优,百禄是遒。’和之至也。”


及子产卒,仲尼闻之,出涕曰:“古之遗爱也。”



\chapter*{论积贮疏}
\addcontentsline{toc}{chapter}{论积贮疏}
\begin{center}
	\textbf{[汉朝]贾谊}
\end{center}


管子曰:“仓廪实而知礼节。”民不足而可治者,自古及今,未之尝闻。古之人曰:“一夫不耕,或受之饥;一女不织,或受之寒。”生之有时,而用之亡度,则物力必屈。古之治天下,至孅至悉也,,故其畜积足恃。今背本而趋末,食者甚众,是天下之大残也;淫侈之俗,日日以长,是天下之大贼也。残贼公行,莫之或止;大命将泛,莫之振救。生之者甚少,而靡之者甚多,天下财产何得不蹶!


汉之为汉,几四十年矣,公私之积,犹可哀痛!失时不雨,民且狼顾;岁恶不入,请卖爵子,既闻耳矣。安有为天下阽危者若是而上不惊者?世之有饥穰,天之行也,禹、汤被之矣。即不幸有方二三千里之旱,国胡以相恤?卒然边境有急,数千百万之众,国胡以馈之?兵旱相乘,天下大屈,有勇力者聚徒而衡击;罢夫羸老易子而咬其骨。政治未毕通也,远方之能疑者,并举而争起矣。乃骇而图之,岂将有及乎?


夫积贮者,天下之大命也。苟粟多而财有余,何为而不成?以攻则取,以守则固,以战则胜。怀敌附远,何招而不至!今殴民而归之农,皆著于本;使天下各食其力,末技游食之民,转而缘南亩,则畜积足而人乐其所矣。可以为富安天下,而直为此廪廪也,窃为陛下惜之。


节自《汉书·食货志》



\chapter*{赠白马王彪 并序}
\addcontentsline{toc}{chapter}{赠白马王彪 并序}
\begin{center}
	\textbf{[三国]曹植}
\end{center}

黄初四年五月,白马王、任城王与余俱朝京师、会节气。到洛阳,任城王薨。至七月,与白马王还国。后有司以二王归藩,道路宜异宿止,意毒恨之。盖以大别在数日,是用自剖,与王辞焉,愤而成篇。

谒帝承明庐,逝将归旧疆。清晨发皇邑,日夕过首阳。伊洛广且深,欲济川无梁。泛舟越洪涛,怨彼东路长。顾瞻恋城阙,引领情内伤。

太谷何寥廓,山树郁苍苍。霖雨泥我涂,流潦浩纵横。中逵绝无轨,改辙登高岗。修坂造云日,我马玄以黄。

玄黄犹能进,我思郁以纡。郁纡将何念,亲爱在离居。本图相与偕,中更不克俱。鸱枭鸣衡轭,豺狼当路衢。苍蝇间白黑,谗巧令亲疏。欲还绝无蹊,揽辔止踟蹰。(衡轭通:衡扼)

踟蹰亦何留?相思无终极。秋风发微凉,寒蝉鸣我侧。原野何萧条,白日忽西匿。归鸟赴乔林,翩翩厉羽翼。孤兽走索群,衔草不遑食。感物伤我怀,抚心长太息。

太息将何为,天命与我违。奈何念同生,一往形不归。孤魂翔故域,灵柩寄京师。存者忽复过,亡殁身自衰。人生处一世,去若朝露晞。年在桑榆间,影响不能追。自顾非金石,咄唶令心悲。

心悲动我神,弃置莫复陈。丈夫志四海,万里犹比邻。恩爱苟不亏,在远分日亲。何必同衾帱,然后展慇懃。忧思成疾疢,无乃儿女仁。仓卒骨肉情,能不怀苦辛?

苦辛何虑思,天命信可疑。虚无求列仙,松子久吾欺。变故在斯须,百年谁能持?离别永无会,执手将何时?王其爱玉体,俱享黄髪期。收泪即长路,援笔从此辞。


\chapter*{送穷文}
\addcontentsline{toc}{chapter}{送穷文}
\begin{center}
	\textbf{[唐朝]韩愈}
\end{center}

元和六年正月乙丑晦,主人使奴星结柳作车,缚草为船,载糗舆粮,牛繫轭下,引帆上樯。三揖穷鬼而告之曰:“闻子行有日矣,鄙人不敢问所涂,窃具船与车,备载糗粻,日吉时良,利行四方,子饭一盂,子啜一觞,携朋挚俦,去故就新,驾尘风,与电争先,子无底滞之尤,我有资送之恩,子等有意于行乎?”

屏息潜听,如闻音声,若啸若啼,砉敥嘎嘤,毛发尽竖,竦肩缩颈,疑有而无,久乃可明,若有言者曰:“吾与子居,四十年余,子在孩提,吾不子愚,子学子耕,求官与名,惟子是从,不变于初。门神户灵,我叱我呵,包羞诡随,志不在他。子迁南荒,热烁湿蒸,我非其乡,百鬼欺陵。太学四年,朝韮暮盐,唯我保汝,人皆汝嫌。自初及终,未始背汝,心无异谋,口绝行语,於何听闻,云我当去?是必夫子信谗,有间于予也。我鬼非人,安用车船,鼻齅臭香,糗粻可捐。单独一身,谁为朋俦,子苟备知,可数已不?子能尽言,可谓圣智,情状既露,敢不回避。”

主人应之曰:“予以吾为真不知也耶!子之朋俦,非六非四,在十去五,满七除二,各有主张,私立名字,捩手覆羹,转喉触讳,凡所以使吾面目可憎、语言无味者,皆子之志也。——其名曰智穷:矫矫亢亢,恶园喜方,羞为奸欺,不忍伤害;其次名曰学穷:傲数与名,摘抉杳微,高挹群言,执神之机;又其次曰文穷:不专一能,怪怪奇奇,不可时施,祗以自嬉;又其次曰命穷:影与行殊,而丑心妍,利居众后,责在人先;又其次曰交穷:磨肌戛骨,吐出心肝,企足以待,寘我仇怨。凡此五鬼,为吾五患,饥我寒我,兴讹造讪,能使我迷,人莫能间,朝悔其行,暮已复然,蝇营狗苟,驱去复还。”

言未毕,五鬼相与张眼吐舌,跳踉偃仆,抵掌顿脚,失笑相顾。徐谓主人曰:“子知我名,凡我所为,驱我令去,小黠大痴。人生一世,其久几何,吾立子名,百世不磨。小人君子,其心不同,惟乖於时,乃与天通。携持琬琰,易一羊皮,饫于肥甘,慕彼糠糜。天下知子,谁过于予。虽遭斥逐,不忍于疏,谓予不信,请质诗书。”

主人于是垂头丧气,上手称谢,烧车与船,延之上座。 


\chapter*{上枢密韩太尉书}
\addcontentsline{toc}{chapter}{上枢密韩太尉书}
\begin{center}
	\textbf{[宋朝]苏辙}
\end{center}

太尉执事:辙生好为文,思之至深。以为文者气之所形,然文不可以学而能,气可以养而致。孟子曰:“我善养吾浩然之气。”今观其文章,宽厚宏博,充乎天地之间,称其气之小大。太史公行天下,周览四海名山大川,与燕、赵间豪俊交游,故其文疏荡,颇有奇气。此二子者,岂尝执笔学为如此之文哉?其气充乎其中而溢乎其貌,动乎其言而见乎其文,而不自知也。

辙生十有九年矣。其居家所与游者,不过其邻里乡党之人;所见不过数百里之间,无高山大野可登览以自广;百氏之书,虽无所不读,然皆古人之陈迹,不足以激发其志气。恐遂汩没,故决然舍去,求天下奇闻壮观,以知天地之广大。过秦、汉之故都,恣观终南、嵩、华之高,北顾黄河之奔流,慨然想见古之豪杰。至京师,仰观天子宫阙之壮,与仓廪、府库、城池、苑囿之富且大也,而后知天下之巨丽。见翰林欧阳公,听其议论之宏辩,观其容貌之秀伟,与其门人贤士大夫游,而后知天下之文章聚乎此也。太尉以才略冠天下,天下之所恃以无忧,四夷之所惮以不敢发,入则周公、召公,出则方叔、召虎。而辙也未之见焉。

且夫人之学也,不志其大,虽多而何为?辙之来也,于山见终南、嵩、华之高,于水见黄河之大且深,于人见欧阳公,而犹以为未见太尉也。故愿得观贤人之光耀,闻一言以自壮,然后可以尽天下之大观而无憾者矣。

辙年少,未能通习吏事。向之来,非有取于斗升之禄,偶然得之,非其所乐。然幸得赐归待选,使得优游数年之间,将以益治其文,且学为政。太尉苟以为可教而辱教之,又幸矣!


\chapter*{子革对灵王}
\addcontentsline{toc}{chapter}{子革对灵王}
\begin{center}
	\textbf{[春秋战国]左丘明}
\end{center}


楚子狩于州来,次于颍尾,使荡侯、潘子、司马督、嚣尹午、陵尹喜帅师围徐以惧吴。楚子次于乾溪,以为之援。


雨雪,王皮冠,秦复陶,翠被,豹舄,执鞭以出,仆析父从。右尹子革夕,王见之。去冠被,舍鞭,与之语曰:“昔我先王熊绎与吕伋、王孙牟、燮父、禽父,并事康王,四国皆有分,我独无有。今吾使人于周,求鼎以为分,王其与我乎?”


对曰:“与君王哉!昔我先王熊绎,辟在荆山,筚路蓝缕,以处草莽,跋涉山林,以事天子,唯是桃弧、棘矢,以共御王事。齐,王舅也;晋及鲁、卫,王母弟也。楚是以无分,而彼皆有。今周与四国服事君王,将唯命是从,岂其爱鼎?”王曰:“昔我皇祖伯父昆吾,旧许是宅。今郑人贪赖其田,而不我与。我若求之,其与我乎?”


对曰:“与君王哉!周不爱鼎,郑敢爱田?”王曰:“昔诸侯远我而畏晋,今我大城陈、蔡、不羹,赋皆千乘,子与有劳焉。诸侯其畏我乎?”对曰:“畏君王哉!是四国者,专足畏也,又加之以楚,敢不畏君王哉?”


工尹路请曰:“君王命剥圭以为鏚柲,敢请命。”王入视之。析父谓子革:“吾子,楚国之望也!今与王言如响,国其若之何?”子革曰:“摩厉以须,王出,吾刃将斩矣。”


王出,复语。左史倚相趋过。王曰:“是良史也,子善视之。是能读《三坟》、《五典》、《八索》、《九丘》。”对曰:“臣尝问焉,昔穆王欲肆其心,周行天下,将皆必有车辙马迹焉。祭公谋父作《祈招》之诗,以止王心,王是以获没于祗宫。臣问其诗而不知也;若问远焉,其焉能知之?”


王曰:“子能乎?”对曰:“能。其《诗》曰:‘祈招之愔愔,式昭德音。思我王度,式如玉,式如金。形民之力,而无醉饱之心。’”


王揖而入,馈不食,寝不寐,数日。不能自克,以及于难。


仲尼曰:“古也有志:‘克己复礼,仁也。’信善哉!楚灵王若能如是,岂其辱于乾溪?”



\chapter*{送豆卢膺秀才南游序}
\addcontentsline{toc}{chapter}{送豆卢膺秀才南游序}
\begin{center}
	\textbf{[唐朝]柳宗元}
\end{center}


君子病无乎内而饰乎外,有乎内而不饰乎外者。无乎内而饰乎外,则是设覆为阱也,祸孰大焉;有乎内而不饰乎外,则是焚梓毁璞也,诟孰甚焉!于是有切磋琢磨、镞砺栝羽之道,圣人以为重。豆卢生,内之有者也,余是以好之,而欲其遂焉。而恒以幼孤羸馁为惧,恤恤焉游诸侯求给乎是,是固所以有乎内者也。然而不克专志于学,饰乎外者未大,吾愿子以《诗》、《礼》为冠屦,以《春秋》为襟带,以图史为佩服,琅乎璆璜冲牙之响发焉,煌乎山龙华虫之采列焉,则揖让周旋乎宗庙朝廷斯可也。惜乎余无禄食于世,不克称其欲,成其志,而姑欲其速反也,故诗而序云。

\chapter*{应科目时与人书}
\addcontentsline{toc}{chapter}{应科目时与人书}
\begin{center}
	\textbf{[唐朝]韩愈}
\end{center}

月日,愈再拜:天地之滨,大江之坟,有怪物焉,盖非常鳞凡介之品匹俦也。其得水,变化风雨,上下于天不难也。

其不及水,盖寻常尺寸之间耳,无高山大陵旷途绝险为之关隔也,然其穷涸,不能自致乎水,为獱獭之笑者,盖十八九矣。如有力者,哀其穷而运转之,盖一举手一投足之劳也。然是物也,负其异於众也,且曰:“烂死于沙泥,吾宁乐之;若俯首贴耳,摇尾而乞怜者,非我之志也。”是以有力者遇之,熟视之若无睹也。其死其生,固不可知也。

今又有有力者当其前矣,聊试仰首一鸣号焉,庸讵知有力者不哀其穷而忘一举手,一投足之劳,而转之清波乎?其哀之,命也;其不哀之,命也;知其在命,而且鸣号之者,亦命也。

愈今者,实有类于是,是以忘其疏愚之罪,而有是说焉。阁下其亦怜察之。


\chapter*{论贵粟疏}
\addcontentsline{toc}{chapter}{论贵粟疏}
\begin{center}
	\textbf{[汉朝]晁错}
\end{center}

圣王在上,而民不冻饥者,非能耕而食之,织而衣之也,为开其资财之道也。故尧、禹有九年之水,汤有七年之旱,而国亡捐瘠者,以畜积多而备先具也。今海内为一,土地人民之众不避汤、禹,加以亡天灾数年之水旱,而畜积未及者,何也?地有遗利,民有余力,生谷之土未尽垦,山泽之利未尽出也,游食之民未尽归农也。

民贫,则奸邪生。贫生于不足,不足生于不农,不农则不地著,不地著则离乡轻家,民如鸟兽。虽有高城深池,严法重刑,犹不能禁也。夫寒之于衣,不待轻暖;饥之于食,不待甘旨;饥寒至身,不顾廉耻。人情一日不再食则饥,终岁不制衣则寒。夫腹饥不得食,肤寒不得衣,虽慈母不能保其子,君安能以有其民哉?明主知其然也,故务民于农桑,薄赋敛,广畜积,以实仓廪,备水旱,故民可得而有也。

民者,在上所以牧之,趋利如水走下,四方无择也。夫珠玉金银,饥不可食,寒不可衣,然而众贵之者,以上用之故也。其为物轻微易藏,在于把握,可以周海内而无饥寒之患。此令臣轻背其主,而民易去其乡,盗贼有所劝,亡逃者得轻资也。粟米布帛生于地,长于时,聚于力,非可一日成也。数石之重,中人弗胜,不为奸邪所利;一日弗得而饥寒至。是故明君贵五谷而贱金玉。

今农夫五口之家,其服役者不下二人,其能耕者不过百亩,百亩之收不过百石。春耕,夏耘,秋获,冬藏,伐薪樵,治官府,给徭役;春不得避风尘,夏不得避署热,秋不得避阴雨,冬不得避寒冻,四时之间,无日休息。又私自送往迎来,吊死问疾,养孤长幼在其中。勤苦如此,尚复被水旱之灾,急政暴虐,赋敛不时,朝令而暮改。当具有者半贾而卖,无者取倍称之息;于是有卖田宅、鬻子孙以偿债者矣。而商贾大者积贮倍息,小者坐列贩卖,操其奇赢,日游都市,乘上之急,所卖必倍。故其男不耕耘,女不蚕织,衣必文采,食必粱肉;无农夫之苦,有阡陌之得。因其富厚,交通王侯,力过吏势,以利相倾;千里游遨,冠盖相望,乘坚策肥,履丝曳缟。此商人所以兼并农人,农人所以流亡者也。今法律贱商人,商人已富贵矣;尊农夫,农夫已贫贱矣。故俗之所贵,主之所贱也;吏之所卑,法之所尊也。上下相反,好恶乖迕,而欲国富法立,不可得也。

方今之务,莫若使民务农而已矣。欲民务农,在于贵粟;贵粟之道,在于使民以粟为赏罚。今募天下入粟县官,得以拜爵,得以除罪。如此,富人有爵,农民有钱,粟有所渫。夫能入粟以受爵,皆有余者也。取于有余,以供上用,则贫民之赋可损,所谓损有余、补不足,令出而民利者也。顺于民心,所补者三:一曰主用足,二曰民赋少,三曰劝农功。今令民有车骑马一匹者,复卒三人。车骑者,天下武备也,故为复卒。神农之教曰:“有石城十仞,汤池百步,带甲百万,而无粟,弗能守也。”以是观之,粟者,王者大用,政之本务。令民入粟受爵,至五大夫以上,乃复一人耳,此其与骑马之功相去远矣。爵者,上之所擅,出于口而无穷;粟者,民之所种,生于地而不乏。夫得高爵也免罪,人之所甚欲也。使天下人入粟于边,以受爵免罪,不过三岁,塞下之粟必多矣。

陛下幸使天下入粟塞下以拜爵,甚大惠也。窃窃恐塞卒之食不足用大渫天下粟。边食足以支五岁,可令入粟郡县矣;足支一岁以上,可时赦,勿收农民租。如此,德泽加于万民,民俞勤农。时有军役,若遭水旱,民不困乏,天下安宁;岁孰且美,则民大富乐矣。


\chapter*{不怕鬼 / 曹司农竹虚言}
\addcontentsline{toc}{chapter}{不怕鬼 / 曹司农竹虚言}
\begin{center}
	\textbf{[清朝]纪昀}
\end{center}


曹司农竹虚言,其族兄自歙往扬州,途经友人家。时盛夏,延坐书屋,甚轩爽,暮欲下榻其中。友人曰:“是有魅,夜不可居。”曹强居之。夜半,有物自门隙蠕蠕入,薄如夹纸。入室后,渐开展作人形,乃女子也。曹殊不畏。忽披发吐舌作缢鬼状。曹笑曰:“犹是发,但稍乱;犹是舌,但稍长,亦何足畏?”忽自摘其首置案上。曹又笑曰:“有首尚不足畏,况无首也。”鬼技穷,倏然。及归途再宿,夜半,门隙又蠕蠕,甫露其首,辄唾曰:“又此败兴物耶?”竟不入。

\chapter*{截竿入城}
\addcontentsline{toc}{chapter}{截竿入城}
\begin{center}
	\textbf{[晋朝]邯郸淳}
\end{center}


鲁有执长竿入城门者,初竖执之,不可入;横执之,亦不可入。计无所出。俄有老父至,曰:“吾非圣人,但见事多矣!何不以锯中截而入?"遂依而截之。

\chapter*{贺进士王参元失火书}
\addcontentsline{toc}{chapter}{贺进士王参元失火书}
\begin{center}
	\textbf{[唐朝]柳宗元}
\end{center}

得杨八书,知足下遇火灾,家无余储。仆始闻而骇,中而疑,终乃大喜。盖将吊而更以贺也。道远言略,犹未能究知其状,若果荡焉泯焉而悉无有,乃吾所以尤贺者也。

足下勤奉养,乐朝夕,惟恬安无事是望也。今乃有焚炀赫烈之虞,以震骇左右,而脂膏滫瀡之具,或以不给,吾是以始而骇也。凡人之言皆曰,盈虚倚伏,去来之不可常。或将大有为也,乃始厄困震悸,于是有水火之孽,有群小之愠。劳苦变动,而后能光明,古之人皆然。斯道辽阔诞漫,虽圣人不能以是必信,是故中而疑也。

以足下读古人书,为文章,善小学,其为多能若是,而进不能出群士之上,以取显贵者,盖无他焉。京城人多言足下家有积货,士之好廉名者,皆畏忌,不敢道足下之善,独自得之心,蓄之衔忍,而不能出诸口。以公道之难明,而世之多嫌也。一出口,则嗤嗤者以为得重赂。仆自贞元十五年,见足下之文章,蓄之者盖六七年未尝言。是仆私一身而负公道久矣,非特负足下也。及为御史尚书郎,自以幸为天子近臣,得奋其舌,思以发明足下之郁塞。然时称道于行列,犹有顾视而窃笑者。仆良恨修己之不亮,素誉之不立,而为世嫌之所加,常与孟几道言而痛之。乃今幸为天火之所涤荡,凡众之疑虑,举为灰埃。黔其庐,赭其垣,以示其无有。而足下之才能,乃可以显白而不污,其实出矣。是祝融、回禄之相吾子也。则仆与几道十年之相知,不若兹火一夕之为足下誉也。宥而彰之,使夫蓄于心者,咸得开其喙;发策决科者,授子而不栗。虽欲如向之蓄缩受侮,其可得乎?于兹吾有望于子,是以终乃大喜也。

古者列国有灾,同位者皆相吊。许不吊灾,君子恶之。今吾之所陈若是,有以异乎古,故将吊而更以贺也。颜、曾之养,其为乐也大矣,又何阙焉?

足下前章要仆文章古书,极不忘,候得数十篇乃并往耳。吴二十一武陵来,言足下为《醉赋》及《对问》,大善,可寄一本。仆近亦好作文,与在京城时颇异,思与足下辈言之,桎梏甚固,未可得也。因人南来,致书访死生。不悉。宗元白。


\chapter*{封建论}
\addcontentsline{toc}{chapter}{封建论}
\begin{center}
	\textbf{[唐朝]柳宗元}
\end{center}


天地果无初乎?吾不得而知之也。生人果有初乎?吾不得而知之也。然则孰为近?曰:有初为近。孰明之?由封建而明之也。彼封建者,更古圣王尧、舜、禹、汤、文、武而莫能去之。盖非不欲去之也,势不可也。势之来,其生人之初乎?不初,无以有封建。封建,非圣人意也。


彼其初与万物皆生,草木榛榛,鹿豕狉狉,人不能搏噬,而且无毛羽,莫克自奉自卫。荀卿有言:“必将假物以为用者也。”夫假物者必争,争而不已,必就其能断曲直者而听命焉。其智而明者,所伏必众,告之以直而不改,必痛之而后畏,由是君长刑政生焉。故近者聚而为群,群之分,其争必大,大而后有兵有德。又有大者,众群之长又就而听命焉,以安其属。于是有诸侯之列,则其争又有大者焉。德又大者,诸侯之列又就而听命焉,以安其封。于是有方伯、连帅之类,则其争又有大者焉。德又大者,方伯、连帅之类又就而听命焉,以安其人,然后天下会于一。是故有里胥而后有县大夫,有县大夫而后有诸侯,有诸侯而后有方伯、连帅,有方伯、连帅而后有天子。自天子至于里胥,其德在人者死,必求其嗣而奉之。故封建非圣人意也,势也。


夫尧、舜、禹、汤之事远矣,及有周而甚详。周有天下,裂土田而瓜分之,设五等,邦群后。布履星罗,四周于天下,轮运而辐集;合为朝觐会同,离为守臣扞城。然而降于夷王,害礼伤尊,下堂而迎觐者。历于宣王,挟中兴复古之德,雄南征北伐之威,卒不能定鲁侯之嗣。陵夷迄于幽、厉,王室东徙,而自列为诸侯。厥后问鼎之轻重者有之,射王中肩者有之,伐凡伯、诛苌弘者有之,天下乖戾,无君君之心。余以为周之丧久矣,徒建空名于公侯之上耳。得非诸侯之盛强,末大不掉之咎欤?遂判为十二,合为七国,威分于陪臣之邦,国殄于后封之秦,则周之败端,其在乎此矣。


秦有天下,裂都会而为之郡邑,废侯卫而为之守宰,据天下之雄图,都六合之上游,摄制四海,运于掌握之内,此其所以为得也。不数载而天下大坏,其有由矣:亟役万人,暴其威刑,竭其货贿,负锄梃谪戍之徒,圜视而合从,大呼而成群,时则有叛人而无叛吏,人怨于下而吏畏于上,天下相合,杀守劫令而并起。咎在人怨,非郡邑之制失也。


汉有天下,矫秦之枉,徇周之制,剖海内而立宗子,封功臣。数年之间,奔命扶伤之不暇,困平城,病流矢,陵迟不救者三代。后乃谋臣献画,而离削自守矣。然而封建之始,郡国居半,时则有叛国而无叛郡,秦制之得亦以明矣。继汉而帝者,虽百代可知也。


唐兴,制州邑,立守宰,此其所以为宜也。然犹桀猾时起,虐害方域者,失不在于州而在于兵,时则有叛将而无叛州。州县之设,固不可革也。


或者曰:“封建者,必私其土,子其人,适其俗,修其理,施化易也。守宰者,苟其心,思迁其秩而已,何能理乎?”余又非之。


周之事迹,断可见矣:列侯骄盈,黩货事戎,大凡乱国多,理国寡,侯伯不得变其政,天子不得变其君,私土子人者,百不有一。失在于制,不在于政,周事然也。


秦之事迹,亦断可见矣:有理人之制,而不委郡邑,是矣。有理人之臣,而不使守宰,是矣。郡邑不得正其制,守宰不得行其理。酷刑苦役,而万人侧目。失在于政,不在于制,秦事然也。


汉兴,天子之政行于郡,不行于国,制其守宰,不制其侯王。侯王虽乱,不可变也,国人虽病,不可除也;及夫大逆不道,然后掩捕而迁之,勒兵而夷之耳。大逆未彰,奸利浚财,怙势作威,大刻于民者,无如之何,及夫郡邑,可谓理且安矣。何以言之?且汉知孟舒于田叔,得魏尚于冯唐,闻黄霸之明审,睹汲黯之简靖,拜之可也,复其位可也,卧而委之以辑一方可也。有罪得以黜,有能得以赏。朝拜而不道,夕斥之矣;夕受而不法,朝斥之矣。设使汉室尽城邑而侯王之,纵令其乱人,戚之而已。孟舒、魏尚之术莫得而施,黄霸、汲黯之化莫得而行;明谴而导之,拜受而退已违矣;下令而削之,缔交合从之谋周于同列,则相顾裂眦,勃然而起;幸而不起,则削其半,削其半,民犹瘁矣,曷若举而移之以全其人乎?汉事然也。


今国家尽制郡邑,连置守宰,其不可变也固矣。善制兵,谨择守,则理平矣。


或者又曰:“夏、商、周、汉封建而延,秦郡邑而促。”尤非所谓知理者也。


魏之承汉也,封爵犹建;晋之承魏也,因循不革;而二姓陵替,不闻延祚。今矫而变之,垂二百祀,大业弥固,何系于诸侯哉?


或者又以为:“殷、周,圣王也,而不革其制,固不当复议也。”是大不然。


夫殷、周之不革者,是不得已也。盖以诸侯归殷者三千焉,资以黜夏,汤不得而废;归周者八百焉,资以胜殷,武王不得而易。徇之以为安,仍之以为俗,汤、武之所不得已也。夫不得已,非公之大者也,私其力于己也,私其卫于子孙也。秦之所以革之者,其为制,公之大者也;其情,私也,私其一己之威也,私其尽臣畜于我也。然而公天下之端自秦始。


夫天下之道,理安斯得人者也。使贤者居上,不肖者居下,而后可以理安。今夫封建者,继世而理;继世而理者,上果贤乎,下果不肖乎?则生人之理乱未可知也。将欲利其社稷以一其人之视听,则又有世大夫世食禄邑,以尽其封略,圣贤生于其时,亦无以立于天下,封建者为之也。岂圣人之制使至于是乎?吾固曰:“非圣人之意也,势也。”









\chapter*{春夜宴从弟桃花园序}
\addcontentsline{toc}{chapter}{春夜宴从弟桃花园序}
\begin{center}
	\textbf{[唐朝]李白}
\end{center}

夫天地者万物之逆旅也;光阴者百代之过客也。而浮生若梦,为欢几何?古人秉烛夜游,良有以也。况阳春召我以烟景,大块假我以文章。会桃花之芳园,序天伦之乐事。群季俊秀,皆为惠连;吾人咏歌,独惭康乐。幽赏未已,高谈转清。开琼筵以坐花,飞羽觞而醉月。不有佳咏,何伸雅怀?如诗不成,罚依金谷酒数。(桃花一作:桃李)


\chapter*{与韩荆州书}
\addcontentsline{toc}{chapter}{与韩荆州书}
\begin{center}
	\textbf{[唐朝]李白}
\end{center}

白闻天下谈士相聚而言曰:“生不用封万户侯,但愿一识韩荆州。”何令人之景慕,一至于此耶!岂不以有周公之风,躬吐握之事,使海内豪俊,奔走而归之,一登龙门,则声价十倍!所以龙蟠凤逸之士,皆欲收名定价于君侯。愿君侯不以富贵而骄之、寒贱而忽之,则三千之中有毛遂,使白得颖脱而出,即其人焉。

白,陇西布衣,流落楚、汉。十五好剑术,遍干诸侯。三十成文章,历抵卿相。虽长不满七尺,而心雄万夫。皆王公大人许与气义。此畴曩心迹,安敢不尽于君侯哉!

君侯制作侔神明,德行动天地,笔参造化,学究天人。幸愿开张心颜,不以长揖见拒。必若接之以高宴,纵之以清谈,请日试万言,倚马可待。今天下以君侯为文章之司命,人物之权衡,一经品题,便作佳士。而君侯何惜阶前盈尺之地,不使白扬眉吐气,激昂青云耶?

昔王子师为豫州,未下车,即辟荀慈明,既下车,又辟孔文举;山涛作冀州,甄拔三十余人,或为侍中、尚书,先代所美。而君侯亦荐一严协律,入为秘书郎,中间崔宗之、房习祖、黎昕、许莹之徒,或以才名见知,或以清白见赏。白每观其衔恩抚躬,忠义奋发,以此感激,知君侯推赤心于诸贤腹中,所以不归他人,而愿委身国士。傥急难有用,敢效微躯。

且人非尧舜,谁能尽善?白谟猷筹画,安能自矜?至于制作,积成卷轴,则欲尘秽视听。恐雕虫小技,不合大人。若赐观刍荛,请给纸墨,兼之书人,然后退扫闲轩,缮写呈上。庶青萍、结绿,长价于薛、卞之门。幸惟下流,大开奖饰,惟君侯图之。


\chapter*{欧阳晔破案}
\addcontentsline{toc}{chapter}{欧阳晔破案}
\begin{center}
	\textbf{[明朝]冯梦龙}
\end{center}

欧阳晔治鄂州,民有争舟而相殴至死者,狱久不决。晔自临其狱,坐囚于庭中,去其桎梏而饮食之,食讫,悉劳而还之狱。独留一人于庭,留者色变而惶顾。晔曰:“杀人者汝也!”囚佯为不知所以。晔曰:“吾观食者皆以右手持箸,而汝独以左。今死者伤在右肋,非汝而谁?”囚无以对。


\chapter*{富人之子}
\addcontentsline{toc}{chapter}{富人之子}
\begin{center}
	\textbf{[宋朝]苏轼}
\end{center}

齐有富人,家累千金。其二子甚愚,其父又不教之。

一日,艾子谓其父曰:“君之子虽美,而不通世务,他日曷能克其家?”

父怒曰:“吾之子敏而且恃多能,岂有不通世务者耶?”

艾子曰:“不须试之他,但问君之子,所食者米,从何来?若知之,吾当妄言之罪。”

父遂呼其子问之。其子嘻然笑曰:“吾岂不知此也?每以布囊取来。”

其父愀然改容曰:“子之愚甚也!彼米不是田中来?”

艾子曰:“非其父不生其子。”


\chapter*{孟冬篇}
\addcontentsline{toc}{chapter}{孟冬篇}
\begin{center}
	\textbf{[三国]曹植}
\end{center}

孟冬十月。阴气厉清。武官诫田。讲旅统兵。元龟袭吉。元光着明。蚩尤跸路。风弭雨停。乘舆启行。鸾鸣幽轧。虎贲采骑。飞象珥鹖。钟鼓铿锵。箫管嘈喝。万骑齐镳。千乘等盖。夷山填谷。平林涤薮。张罗万里。尽其飞走。趯趯狡兔。扬白跳翰。猎以青骹。掩以修竿。韩卢宋鹊。呈才骋足。噬不尽绁。牵麋掎鹿。魏氏发机。养基抚弦。都卢寻高。搜索猴猨。庆忌孟贲。蹈谷超峦。张目决眦。发怒穿冠。顿熊扼虎。蹴豹搏貙。气有余势。负象而趋。获车既盈。日侧乐终。罢役解徒。大飨离宫。乱曰。圣皇临飞轩。论功校猎徒。死禽积如京。流血成沟渠。明诏大劳赐。大官供有无。走马行酒醴。驱车布肉鱼。鸣鼓举觞爵。击钟釂无余。绝纲纵麟麑。弛罩出凤雏。收功在羽校。威灵振鬼区。陛下长欢乐。永世合天符。


\chapter*{谏迎佛骨表}
\addcontentsline{toc}{chapter}{谏迎佛骨表}
\begin{center}
	\textbf{[唐朝]韩愈}
\end{center}

臣某言:伏以佛者,夷狄之一法耳,自后汉时流入中国,上古未尝有也。昔者黄帝在位百年,年百一十岁;少昊在位八十年,年百岁;颛顼在位七十九年,年九十八岁;帝喾在位七十年,年百五岁;帝尧在位九十八年,年百一十八岁;帝舜及禹,年皆百岁。此时天下太平,百姓安乐寿考,然而中国未有佛也。其后殷汤亦年百岁,汤孙太戊在位七十五年,武丁在位五十九年,书史不言其年寿所极,推其年数,盖亦俱不减百岁。周文王年九十七岁,武王年九十三岁,穆王在位百年。此时佛法亦未入中国,非因事佛而致然也。

汉明帝时,始有佛法,明帝在位,才十八年耳。其后乱亡相继,运祚不长。宋、齐、梁、陈、元魏已下,事佛渐谨,年代尤促。惟梁武帝在位四十八年,前后三度舍身施佛,宗庙之祭,不用牲牢,昼日一食,止于菜果,其后竞为侯景所逼,饿死台城,国亦寻灭。事佛求福,乃更得祸。由此观之,佛不足事,亦可知矣。

高祖始受隋禅,则议除之。当时群臣材识不远,不能深知先王之道,古今之宜,推阐圣明,以救斯弊,其事遂止,臣常恨焉。伏维睿圣文武皇帝陛下,神圣英武,数千百年已来,未有伦比。即位之初,即不许度人为僧尼道,又不许创立寺观。臣常以为高祖之志,必行于陛下之手,今纵未能即行,岂可恣之转令盛也?

今闻陛下令群僧迎佛骨于凤翔,御楼以观,舁入大内,又令诸寺递迎供养。臣虽至愚,必知陛下不惑于佛,作此崇奉,以祈福祥也。直以年丰人乐,徇人之心,为京都士庶设诡异之观,戏玩之具耳。安有圣明若此,而肯信此等事哉!然百姓愚冥,易惑难晓,苟见陛下如此,将谓真心事佛,皆云:“天子大圣,犹一心敬信;百姓何人,岂合更惜身命!”焚顶烧指,百十为群,解衣散钱,自朝至暮,转相仿效,惟恐后时,老少奔波,弃其业次。若不即加禁遏,更历诸寺,必有断臂脔身以为供养者。伤风败俗,传笑四方,非细事也。

夫佛本夷狄之人,与中国言语不通,衣服殊制;口不言先王之法言,身不服先王之法服;不知君臣之义,父子之情。假如其身至今尚在,奉其国命,来朝京师,陛下容而接之,不过宣政一见,礼宾一设,赐衣一袭,卫而出之于境,不令惑众也。况其身死已久,枯朽之骨,凶秽之馀,岂宜令入宫禁?

孔子曰:“敬鬼神而远之。”古之诸侯,行吊于其国,尚令巫祝先以桃茹祓除不祥,然后进吊。今无故取朽秽之物,亲临观之,巫祝不先,桃茹不用,群臣不言其非,御史不举其失,臣实耻之。乞以此骨付之有司,投诸水火,永绝根本,断天下之疑,绝后代之惑。使天下之人,知大圣人之所作为,出于寻常万万也。岂不盛哉!岂不快哉!佛如有灵,能作祸祟,凡有殃咎,宜加臣身,上天鉴临,臣不怨悔。无任感激恳悃之至,谨奉表以闻。臣某诚惶诚恐。


\chapter*{圬者王承福传}
\addcontentsline{toc}{chapter}{圬者王承福传}
\begin{center}
	\textbf{[唐朝]韩愈}
\end{center}

圬之为技贱且劳者也。有业之,其色若自得者。听其言,约而尽。问之,王其姓。承福其名。世为京兆长安农夫。天宝之乱,发人为兵。持弓矢十叁年,有官勋,弃之来归。丧其土田,手衣食,馀叁十年。舍于市之主人,而归其屋食之当焉。视时屋食之贵贱,而上下其圬之以偿之;有馀,则以与道路之废疾饿者焉。

又曰:“粟,稼而生者也;若市与帛。必蚕绩而后成者也;其他所以养生之具,皆待人力而后完也;吾皆赖之。然人不可遍为,宜乎各致其能以相生也。故君者,理我所以生者也;而百官者,承君之化者也。任有大小,惟其所能,若器皿焉。食焉而怠其事,必有天殃,故吾不敢一日舍镘以嬉。夫镘易能,可力焉,又诚有功;取其直虽劳无愧,吾心安焉夫力易强而有功也;心难强而有智也。用力者使于人,用心者使人,亦其宜也。吾特择其易为无傀者取焉。

“嘻!吾操镘以入富贵之家有年矣。有一至者焉,又往过之,则为墟矣;有再至、叁至者焉,而往过之,则为墟矣。问之其邻,或曰:“噫!刑戮也。”或曰:“身既死,而其子孙不能有也。”或曰:“死而归之官也。”吾以是观之,非所谓食焉怠其事,而得天殃者邪?非强心以智而不足,不择其才之称否而冒之者邪?非多行可愧,知其不可而强为之者邪?将富贵难守,薄宝而厚飨之者邪?抑丰悴有时,一去一来而不可常者邪?吾之心悯焉,是故择其力之可能者行焉。乐富贵而悲贫贱,我岂异于人哉?”

又曰:“功大者,其所以自奉也博。妻与子,皆养于我者也;吾能薄而功小,不有之可也。又吾所谓劳力者,若立吾家而力不足,则心又劳也。”一身而二任焉,虽圣者石可为也。

愈始闻而惑之,又从而思之,盖所谓“独善其身”者也。然吾有讥焉;谓其自为也过多,其为人也过少。其学杨朱之道者邪?杨之道,不肯拔我一毛而利天下。而夫人以有家为劳心,不肯一动其心以蓄其妻子,其肯劳其心以为人乎哉?虽然,其贤于世者之患不得之,而患失之者,以济其生之欲,贪邪而亡道以丧其身者,其亦远矣!又其言,有可以警余者,故余为之传而自鉴焉。


\chapter*{三人成虎}
\addcontentsline{toc}{chapter}{三人成虎}
\begin{center}
	\textbf{[汉朝]刘向}
\end{center}

庞葱与太子质于邯郸,谓魏王曰:‘今一人言市有虎,王信之乎?’王曰:‘否。’‘二人言市有虎,王信之乎?’王曰:‘寡人疑之矣。’‘三人言市有虎,王信之乎?’王曰:‘寡人信之矣。’庞葱曰:‘夫市之无虎明矣,然而三人言而成虎。今邯郸去大梁也远于市,而议臣者过于三人,愿王察之。’王曰:‘寡人自为知。’于是辞行,而谗言先至。后太子罢质,果不得见。(庞葱一作:庞恭)


\chapter*{枭逢鸠}
\addcontentsline{toc}{chapter}{枭逢鸠}
\begin{center}
	\textbf{[汉朝]刘向}
\end{center}


鸠曰:“子将安之?”

枭曰:“我将东徙。”

鸠曰:“何故?”

枭曰:“乡人皆恶我鸣。以故东徙。”

鸠曰:“子能更鸣,可矣;不能更鸣,东徙,犹恶子之声。”

\chapter*{虎求百兽}
\addcontentsline{toc}{chapter}{虎求百兽}
\begin{center}
	\textbf{[汉朝]刘向}
\end{center}


荆宣王问群臣曰:“吾闻北方之畏昭奚恤也,果诚何如?”群臣莫对。

江乙对曰:“虎求百兽而食之,得狐。狐曰:‘子无敢食我也!天帝使我长百兽。今子食我,是逆天帝命也!子以我为不信,吾为子先行,于随我后,观百兽之见我而敢不走乎?”虎以为然,故遂与之行。兽见之,皆走。虎不知兽畏己而走也,以为畏狐也。

今王之地五千里,带甲百万,而专属之于昭奚恤,故北方之畏奚恤也,其实畏王之甲兵也!犹百兽之畏虎也!”

\chapter*{九叹}
\addcontentsline{toc}{chapter}{九叹}
\begin{center}
	\textbf{[汉朝]刘向}
\end{center}

\section*{逢纷}
\addcontentsline{toc}{section}{逢纷}
\begin{center}
	
	伊伯庸之末胄兮,谅皇直之屈原。
	
	云余肇祖于高阳兮,惟楚怀之婵连。
	
	原生受命于贞节兮,鸿永路有嘉名。
	
	齐名字于天地兮,并光明于列星。
	
	吸精粹而吐氛浊兮,横邪世而不取容。
	
	行叩诚而不阿兮,遂见排而逢谗。
	
	后听虚而黜实兮,不吾理而顺情。
	
	肠愤悁而含怒兮,志迁蹇而左倾。
	
	心戃慌其不我与兮,躬速速其不吾亲。
	
	辞灵修而陨志兮,吟泽畔之江滨。
	
	椒桂罗以颠覆兮,有竭信而归诚。
	
	谗夫蔼蔼而漫著兮,曷其不舒予情?
	
	始结言于庙堂兮,信中涂而叛之。
	
	怀兰蕙与衡芷兮,行中野而散之。
	
	声哀哀而怀高丘兮,心愁愁而思旧邦。
	
	愿承闲而自恃兮,径淫曀而道壅。
	
	颜霉黧以沮败兮,精越裂而衰耄。
	
	裳襜襜而含风兮,衣纳纳而掩露。
	
	赴江湘之湍流兮,顺波凑而下降。
	
	徐徘徊于山阿兮,飘风来之洶洶。
	
	驰余车兮玄石,步余马兮洞庭。
	
	平明发兮苍梧,夕投宿兮石城。
	
	芙蓉盖而菱华车兮,紫贝阙而玉堂。
	
	薜荔饰而陆离荐兮,鱼鳞衣而白蜺裳。
	
	登逢龙而下陨兮,违故都之漫漫。
	
	思南郢之旧俗兮,肠一夕而九运。
	
	扬流波之潢潢兮,体溶溶而东回。
	
	心怊怅以永思兮,意晻晻而日颓。
	
	白露纷以涂涂兮,秋风浏以萧萧。
	
	身永流而不还兮,魂长逝而常愁。
	
	叹曰:
	
	譬彼流水纷扬磕兮,波逢汹涌濆壅滂兮。
	
	揄扬涤荡飘流陨往触崟石兮,
	
	龙卬脟圈缭戾宛转阻相薄兮,
	
	遭纷逢凶蹇离尤兮,垂文扬采遗将来兮。
	
	
\end{center}

\section*{离世}
\addcontentsline{toc}{section}{离世}
\begin{center}
	
	灵怀其不吾知兮,灵怀其不吾闻。
	
	就灵怀之皇祖兮,愬灵怀之鬼神。
	
	灵怀曾不吾与兮,即听夫人之谀辞。
	
	余辞上参于天坠兮,旁引之于四时。
	
	指日月使延照兮,抚招摇以质正。
	
	立师旷俾端辞兮,命咎繇使并听。
	
	兆出名曰正则兮,卦发字曰灵均。
	
	余幼既有此鸿节兮,长愈固而弥纯。
	
	不从俗而诐行兮,直躬指而信志。
	
	不枉绳以追曲兮,屈情素以从事。
	
	端余行其如玉兮,述皇舆之踵迹。
	
	群阿容以晦光兮,皇舆覆以幽辟。
	
	舆中涂以回畔兮,驷马惊而横奔。
	
	执组者不能制兮,必折轭而摧辕。
	
	断镳衔以驰骛兮,暮去次而敢止。
	
	路荡荡其无人兮,遂不禦乎千里。
	
	身衡陷而下沉兮,不可获而复登。
	
	不顾身之卑贱兮,惜皇舆之不兴。
	
	出国门而端指兮,冀壹寤而锡还。
	
	哀仆夫之坎毒兮,屡离忧而逢患。
	
	九年之中不吾反兮,思彭咸之水游。
	
	惜师延之浮渚兮,赴汨罗之长流。
	
	遵江曲之逶移兮,触石碕而衡游。
	
	波澧澧而扬浇兮,顺长濑之浊流。
	
	凌黄沱而下低兮,思还流而复反。
	
	玄舆驰而并集兮,身容与而日远。
	
	棹舟杭以横濿兮,济湘流而南极。
	
	立江界而长吟兮,愁哀哀而累息。
	
	情慌忽以忘归兮,神浮游以高历。
	
	心蛩蛩而怀顾兮,魂眷眷而独逝。
	
	叹曰:
	
	余思旧邦心依违兮,
	
	日暮黄昏羌幽悲兮,
	
	去郢东迁余谁慕兮,
	
	谗夫党旅其以兹故兮,
	
	河水淫淫情所愿兮,
	
	顾瞻郢路终不返兮。
	
	
\end{center}

\section*{怨思}
\addcontentsline{toc}{section}{怨思}
\begin{center}
	
	惟郁郁之忧毒兮,志坎壈而不违。
	
	身憔悴而考旦兮,日黄昏而长悲。
	
	闵空宇之孤子兮,哀枯杨之冤雏。
	
	孤雌吟于高墉兮,鸣鸠栖于桑榆。
	
	玄蝯失于潜林兮,独偏弃而远放。
	
	征夫劳于周行兮,处妇愤而长望。
	
	申诚信而罔违兮,情素洁于纽帛。
	
	光明齐于日月兮,文采耀燿于玉石。
	
	伤压次而不发兮,思沉抑而不扬。
	
	芳懿懿而终败兮,名靡散而不彰。
	
	背玉门以奔骛兮,蹇离尤而干诟。
	
	若龙逢之沉首兮,王子比干之逢醢。
	
	念社稷之几危兮,反为雠而见怨。
	
	思国家之离沮兮,躬获愆而结难。
	
	若青蝇之伪质兮,晋骊姬之反情。
	
	恐登阶之逢殆兮,故退伏于末庭。
	
	孽臣之号咷兮,本朝芜而不治。
	
	犯颜色而触谏兮,反蒙辜而被疑。
	
	菀蘼芜与菌若兮,渐藁本于洿渎。
	
	淹芳芷于腐井兮,弃鸡骇于筐簏。
	
	执棠谿以刜蓬兮,秉干将以割肉。
	
	筐泽泻以豹鞟兮,破荆和以继筑。
	
	时溷浊犹未清兮,世殽乱犹未察。
	
	欲容与以俟时兮,惧年岁之既晏。
	
	顾屈节以从流兮,心巩巩而不夷。
	
	宁浮沅而驰骋兮,下江湘以邅回。
	
	叹曰:
	
	山中槛槛余伤怀兮,征夫皇皇其孰依兮,
	
	经营原野杳冥冥兮,乘骐骋骥舒吾情兮,
	
	归骸旧邦莫谁语兮,长辞远逝乘湘去兮。
	
	
\end{center}

\section*{远逝}
\addcontentsline{toc}{section}{远逝}
\begin{center}
	
	志隐隐而郁怫兮,愁独哀而冤结。
	
	肠纷纭以缭转兮,涕渐渐其若屑。
	
	情慨慨而长怀兮,信上皇而质正。
	
	合五岳与八灵兮,讯九鬿与六神。
	
	指列宿以白情兮,诉五帝以置辞。
	
	北斗为我折中兮,太一为余听之。

	云服阴阳之正道兮,御后土之中和。
	
	佩苍龙之蚴虬兮,带隐虹之逶蛇。
	
	曳彗星之皓旰兮,抚朱爵与鵔鸃。
	
	游清灵之飒戾兮,服云衣之披披。
	
	杖玉策与朱旗兮,垂明月之玄珠。
	
	举霓旌之墆翳兮,建黄纁之总旄。
	
	躬纯粹而罔愆兮,承皇考之妙仪。
	
	惜往事之不合兮,横汨罗而下沥。
	
	乘隆波而南渡兮,逐江湘之顺流。
	
	赴阳侯之潢洋兮,下石濑而登洲。
	
	陆魁堆以蔽视兮,云冥冥而闇前。
	
	山峻高以无垠兮,遂曾闳而迫身。
	
	雪雰雰而薄木兮,云霏霏而陨集。
	
	阜隘狭而幽险兮,石嵾嵯以翳日。
	
	悲故乡而发忿兮,去余邦之弥久。
	
	背龙门而入河兮,登大坟而望夏首。
	
	横舟航而济湘兮,耳聊啾而戃慌。
	
	波淫淫而周流兮,鸿溶溢而滔荡。
	
	路曼曼其无端兮,周容容而无识。
	
	引日月以指极兮,少须臾而释思。
	
	水波远以冥冥兮,眇不睹其东西。
	
	顺风波以南北兮,雾宵晦以纷纷。
	
	日杳杳以西颓兮,路长远而窘迫。
	
	欲酌醴以娱忧兮,蹇骚骚而不释。
	
	叹曰:
	
	飘风蓬龙埃坲坲兮,草木摇落时槁悴兮,
	
	遭倾遇祸不可救兮,长吟永欷涕究究兮,
	
	舒情陈诗冀以自免兮,颓流下陨身日远兮。
	
	
\end{center}

\section*{惜贤}
\addcontentsline{toc}{section}{惜贤}
\begin{center}
	
	览屈氏之离骚兮,心哀哀而怫郁。
	
	声嗷嗷以寂寥兮,顾仆夫之憔悴。
	
	拨谄谀而匡邪兮,切淟涊之流俗。
	
	荡渨涹之奸咎兮,夷蠢蠢之溷浊。
	
	怀芬香而挟蕙兮,佩江蓠之婓婓。
	
	握申椒与杜若兮,冠浮云之峨峨。
	
	登长陵而四望兮,览芷圃之蠡蠡。
	
	游兰皋与蕙林兮,睨玉石之嵾嵯。
	
	扬精华以炫燿兮,芳郁渥而纯美。
	
	结桂树之旖旎兮,纫荃蕙与辛夷。
	
	芳若兹而不御兮,捐林薄而菀死。
	
	驱子侨之奔走兮,申徒狄之赴渊。
	
	若由夷之纯美兮,介子推之隐山。
	
	晋申生之离殃兮,荆和氏之泣血。
	
	吴申胥之抉眼兮,王子比干之横废。
	
	欲卑身而下体兮,心隐恻而不置。
	
	方圜殊而不合兮,钩绳用而异态。
	
	欲俟时于须臾兮,日阴曀其将暮。
	
	时迟迟其日进兮,年忽忽而日度。
	
	妄周容而入世兮,内距闭而不开。
	
	俟时风之清激兮,愈氛雾其如塺。
	
	进雄鸠之耿耿兮,谗介介而蔽之。
	
	默顺风以偃仰兮,尚由由而进之。
	
	心懭悢以冤结兮,情舛错以曼忧。
	
	搴薜荔于山野兮,采撚支于中洲。
	
	望高丘而叹涕兮,悲吸吸而长怀。
	
	孰契契而委栋兮,日晻晻而下颓。
	
	叹曰:
	
	江湘油油长流汩兮,挑揄扬汰荡迅疾兮。
	
	忧心展转愁怫郁兮,冤结未舒长隐忿兮,
	
	丁时逢殃可奈何兮,劳心悁悁涕滂沱兮。
	
	
\end{center}

\section*{忧苦}
\addcontentsline{toc}{section}{忧苦}
\begin{center}
	
	悲余心之悁悁兮,哀故邦之逢殃。
	
	辞九年而不复兮,独茕茕而南行。
	
	思余俗之流风兮,心纷错而不受。
	
	遵野莽以呼风兮,步从容于山廋。
	
	巡陆夷之曲衍兮,幽空虚以寂寞。
	
	倚石岩以流涕兮,忧憔悴而无乐。
	
	登巑岏以长企兮,望南郢而闚之。
	
	山修远其辽辽兮,涂漫漫其无时。
	
	听玄鹤之晨鸣兮,于高冈之峨峨。
	
	独愤积而哀娱兮,翔江洲而安歌。
	
	三鸟飞以自南兮,览其志而欲北。
	
	原寄言于三鸟兮,去飘疾而不可得。
	
	欲迁志而改操兮,心纷结其未离。
	
	外彷徨而游览兮,内恻隐而含哀。
	
	聊须臾以时忘兮,心渐渐其烦错。
	
	原假簧以舒忧兮,志纡郁其难释。
	
	叹《离骚》以扬意兮,犹未殫于《九章》。
	
	长嘘吸以于悒兮,涕横集而成行。
	
	伤明珠之赴泥兮,鱼眼玑之坚藏。
	
	同驽骡与乘駔兮,杂斑駮与阘茸。
	
	葛藟虆于桂树兮,鸱鸮集于木兰。
	
	偓促谈于廊庙兮,律魁放乎山间。
	
	恶虞氏之箫《韶》兮,好遗风之《激楚》。
	
	潜周鼎于江淮兮,爨土鬵于中宇。
	
	且人心之持旧兮,而不可保长。
	
	邅彼南道兮,征夫宵行。
	
	思念郢路兮,还顾睠睠。
	
	涕流交集兮,泣下涟涟。
	
	叹曰:
	
	登山长望中心悲兮,菀彼青青泣如颓兮,
	
	留思北顾涕渐渐兮,折锐摧矜凝氾滥兮,
	
	念我茕茕魂谁求兮,仆夫慌悴散若流兮。
	
	
\end{center}

\section*{愍命}
\addcontentsline{toc}{section}{愍命}
\begin{center}
	
	昔皇考之嘉志兮,喜登能而亮贤。
	
	情纯洁而罔薉兮,姿盛质而无愆。
	
	放佞人与谄谀兮,斥谗夫与便嬖。
	
	亲忠正之悃诚兮,招贞良与明智。
	
	心溶溶其不可量兮,情澹澹其若渊。
	
	回邪辟而不能入兮,诚原藏而不可迁。
	
	逐下袟于後堂兮,迎虙妃于伊雒。
	
	刜谗贼于中廇兮,选吕管于榛薄。
	
	丛林之下无怨士兮,江河之畔无隐夫。
	
	三苗之徒以放逐兮,伊皋之伦以充庐。
	
	今反表以为里兮,颠裳以为衣。
	
	戚宋万于两楹兮,废周邵于遐夷。
	
	却骐骥以转运兮,腾驴骡以驰逐。
	
	蔡女黜而出帷兮,戎妇入而綵绣服。
	
	庆忌囚于阱室兮,陈不占战而赴围。
	
	破伯牙之号钟兮,挟人筝而弹纬。
	
	藏瑉石于金匮兮,捐赤瑾于中庭。
	
	韩信蒙于介胄兮,行夫将而攻城。
	
	莞芎弃于泽洲兮,瓟瓥蠹于筐簏。
	
	麒麟奔于九皋兮,熊罴群而逸囿。
	
	折芳枝与琼华兮,树枳棘与薪柴。
	
	掘荃蕙与射干兮,耘藜藿与蘘荷。
	
	惜今世其何殊兮,远近思而不同。
	
	或沉沦其无所达兮,或清激其无所通。
	
	哀余生之不当兮,独蒙毒而逢尤。
	
	虽謇謇以申志兮,君乖差而屏之。
	
	诚惜芳之菲菲兮,反以兹为腐也。
	
	怀椒聊之蔎蔎兮,乃逢纷以罹诟也。
	
	叹曰:
	
	嘉皇既殁终不返兮,山中幽险郢路远兮。
	
	谗人諓諓孰可愬兮,征夫罔极谁可语兮。
	
	行吟累欷声喟喟兮,怀忧含戚何侘傺兮。
	
	
\end{center}

\section*{思古}
\addcontentsline{toc}{section}{思古}
\begin{center}
	
	冥冥深林兮,树木郁郁。
	
	山参差以崭岩兮,阜杳杳以蔽日。
	
	悲余心之悁悁兮,目眇眇而遗泣。
	
	风骚屑以摇木兮,云吸吸以湫戾。
	
	悲余生之无欢兮,愁倥傯于山陆。
	
	旦徘徊于长阪兮,夕彷徨而独宿。
	
	发披披以鬤鬤兮,躬劬劳而瘏悴。
	
	魂俇俇而南行兮,泣霑襟而濡袂。
	
	心婵媛而无告兮,口噤闭而不言。
	
	违郢都之旧闾兮,回湘、沅而远迁。
	
	念余邦之横陷兮,宗鬼神之无次。
	
	闵先嗣之中绝兮,心惶惑而自悲。
	
	聊浮游于山陿兮,步周流于江畔。
	
	临深水而长啸兮,且倘佯而氾观。
	
	兴离骚之微文兮,冀灵修之壹悟。
	
	还余车于南郢兮,复往轨于初古。
	
	道修远其难迁兮,伤余心之不能已。
	
	背三五之典刑兮,绝洪范之辟纪。
	
	播规矩以背度兮,错权衡而任意。
	
	操绳墨而放弃兮,倾容幸而侍侧。
	
	甘棠枯于丰草兮,藜棘树于中庭。
	
	西施斥于北宫兮,仳倠倚于弥楹。
	
	乌获戚而骖乘兮,燕公操于马圉。
	
	蒯聩登于清府兮,咎繇弃而在野。
	
	盖见兹以永叹兮,欲登阶而狐疑。
	
	乘白水而高骛兮,因徙弛而长词。
	
	叹曰:
	
	倘佯垆阪沼水深兮,容与汉渚涕淫淫兮,
	
	锺牙已死谁为声兮?纤阿不御焉舒情兮,
	
	曾哀悽欷心离离兮,还顾高丘泣如洒兮。
	
	
\end{center}

\section*{远游}
\addcontentsline{toc}{section}{远游}
\begin{center}
	
	悲余性之不可改兮,屡惩艾而不迻。
	
	服觉皓以殊俗兮,貌揭揭以巍巍。
	
	譬若王侨之乘云兮,载赤霄而凌太清。
	
	欲与天地参寿兮,与日月而比荣。
	
	登昆仑而北首兮,悉灵圉而来谒。
	
	选鬼神于太阴兮,登阊阖于玄阙。
	
	回朕车俾西引兮,褰虹旗于玉门。
	
	驰六龙于三危兮,朝西灵于九滨。
	
	结余轸于西山兮,横飞谷以南征。
	
	绝都广以直指兮,历祝融于硃冥。
	
	枉玉衡于炎火兮,委两馆于咸唐。
	
	贯澒濛以东朅兮,维六龙于扶桑。
	
	周流览于四海兮,志升降以高驰。
	
	徵九神于回极兮,建虹采以招指。
	
	驾鸾凤以上游兮,从玄鹤与鹪明。
	
	孔鸟飞而送迎兮,腾群鹤于瑶光。
	
	排帝宫与罗囿兮,升县圃以眩灭。
	
	结琼枝以杂佩兮,立长庚以继日。
	
	凌惊雷以轶骇电兮,缀鬼谷于北辰。
	
	鞭风伯使先驱兮,囚灵玄于虞渊。
	
	遡高风以低佪兮,览周流于朔方。
	
	就颛顼而敶辞兮,考玄冥于空桑。
	
	旋车逝于崇山兮,奏虞舜于苍梧。
	
	济杨舟于会稽兮,就申胥于五湖。
	
	见南郢之流风兮,殒余躬于沅湘。
	
	望旧邦之黯黮兮,时溷浊其犹未央。
	
	怀兰茝之芬芳兮,妒被离而折之。
	
	张绛帷以襜襜兮,风邑邑而蔽之。
	
	日暾暾其西舍兮,阳焱焱而复顾。
	
	聊假日以须臾兮,何骚骚而自故。
	
	叹曰:
	
	譬彼蛟龙乘云浮兮,
	
	汎淫澒溶纷若雾兮。
	
	潺湲轇轕雷动电发馺高举兮。
	
	升虚凌冥沛浊浮清入帝宫兮,
	
	摇翘奋羽驰风骋雨游无穷兮。
	
\end{center}
	

\chapter*{曾子不受邑}
\addcontentsline{toc}{chapter}{曾子不受邑}
\begin{center}
	\textbf{[汉朝]刘向}
\end{center}


曾子衣敝衣以耕。鲁君使人往致邑焉,曰:“请以此修衣。”曾子不受,反,复往,又不受。使者曰:“先生非求于人,人则献之,奚为不受?”曾子曰:“臣闻之,‘受人者畏人;予人者骄人。’纵子有赐,不我骄也,我能勿畏乎?”终不受。孔子闻之,曰:“参之言足以全其节也。”(选自汉·刘向《说苑》)



\chapter*{六亲五法}
\addcontentsline{toc}{chapter}{六亲五法}
\begin{center}
	\textbf{[汉朝]刘向}
\end{center}


以家为乡,乡不可为也;以乡为国,国不可为也;以国为天下,天下不可为也。以家为家,以乡为乡,以国为国,以天下为天下。毋曰不同生,远者不听;毋曰不同乡,远者不行;毋曰不同国,远者不从。如地如天,何私何亲?如月如日,唯君之节!


御民之辔,在上之所贵;道民之门,在上之所先;召民之路,在上之所好恶。故君求之,则臣得之;君嗜之,则臣食之;君好之,则臣服之;君恶之,则臣匿之。毋蔽汝恶,毋异汝度,贤者将不汝助。言室满室,言堂满堂,是谓圣王。城郭沟渠,不足以固守;兵甲强力,不足以应敌;博地多财,不足以有众。惟有道者,能备患於未形也,故祸不萌。


天下不患无臣,患无君以使之;天下不患无财,患无人以分之。故知时者,可立以为长;无私者,可置以为政;审於时而察於用,而能备官者,可奉以为君也。缓者,後於事;吝於财者,失所亲;信小人者,失士。



\chapter*{读书要三到}
\addcontentsline{toc}{chapter}{读书要三到}
\begin{center}
	\textbf{[宋朝]朱熹}
\end{center}

凡读书......须要读得字字响亮,不可误一字,不可少一字,不可多一字,不可倒一字,不可牵强暗记,只是要多诵数遍,自然上口,久远不忘。古人云,“读书百遍,其义自见”。谓读得熟,则不待解说,自晓其义也。余尝谓,读书有三到,谓心到,眼到,口到。心不在此,则眼不看仔细,心眼既不专一,却只漫浪诵读,决不能记,记亦不能久也。三到之中,心到最急。心既到矣,眼口岂不到乎?


\chapter*{司马光好学}
\addcontentsline{toc}{chapter}{司马光好学}
\begin{center}
	\textbf{[宋朝]朱熹}
\end{center}


司马温公幼时,患记问不若人。群居讲习,众兄弟既成诵,游息矣;独下帷绝编,迨能倍诵乃止。用力多者收功远,其所精诵,乃终身不忘也。温公尝言:“书不可不成诵。或在马上,或中夜不寝时,咏其文,思其义,所得多矣。”(选自朱熹编辑的《三朝名臣言行录》)

\chapter*{陈谏议教子}
\addcontentsline{toc}{chapter}{陈谏议教子}
\begin{center}
	\textbf{[宋朝]朱熹}
\end{center}


宋陈谏议家有劣马,性暴,不可驭,蹄啮伤人多矣。一日,谏议入厩,不见是马,因诘仆:“彼马何以不见?”仆言为陈尧咨售之贾人矣。尧咨者,陈谏议之子也。谏议遽召子,曰:“汝为贵臣,家中左右尚不能制,贾人安能蓄之?是移祸于人也!”急命人追贾人取马,而偿其直。戒仆养之终老。时人称陈谏议有古仁之风。

\chapter*{吴起守信}
\addcontentsline{toc}{chapter}{吴起守信}
\begin{center}
	\textbf{[明朝]宋濂}
\end{center}


昔吴起出,遇故人,而止之食。故人曰:“诺,期返而食。”起曰:“待公而食。”故人至暮不来,起不食待之。明日早,令人求故人,故人来,方与之食。起之不食以俟者,恐其自食其言也。其为信若此,宜其能服三军欤?欲服三军,非信不可也!

\chapter*{王冕好学}
\addcontentsline{toc}{chapter}{王冕好学}
\begin{center}
	\textbf{[明朝]宋濂}
\end{center}


王冕者,诸暨人。七八岁时,父命牧牛陇上,窃入学舍,听诸生诵书;听已,辄默记。暮归,忘其牛。或牵牛来责蹊田者。父怒,挞之,已而复如初。母曰:“儿痴如此,曷不听其所为?”冕因去,依僧寺以居。夜潜出,坐佛膝上,执策映长明灯读之,琅琅达旦。佛像多土偶,狞恶可怖;冕小儿,恬若不见。


安阳韩性闻而异之,录为弟子,学遂为通儒。性卒,门人事冕如事性。时冕父已卒,即迎母入越城就养。久之,母思还故里,冕买白牛驾母车,自被古冠服随车后。乡里儿竞遮道讪笑,冕亦笑。选自《元史·王冕传》



\chapter*{心术}
\addcontentsline{toc}{chapter}{心术}
\begin{center}
	\textbf{[宋朝]苏洵}
\end{center}


为将之道,当先治心。泰山崩于前而色不变,麋鹿兴于左而目不瞬,然后可以制利害,可以待敌。


凡兵上义;不义,虽利勿动。非一动之为利害,而他日将有所不可措手足也。夫惟义可以怒士,士以义怒,可与百战。


凡战之道,未战养其财,将战养其力,既战养其气,既胜养其心。谨烽燧,严斥堠,使耕者无所顾忌,所以养其财;丰犒而优游之,所以养其力;小胜益急,小挫益厉,所以养其气;用人不尽其所欲为,所以养其心。故士常蓄其怒、怀其欲而不尽。怒不尽则有馀勇,欲不尽则有馀贪。故虽并天下,而士不厌兵,此黄帝之所以七十战而兵不殆也。不养其心,一战而胜,不可用矣。


凡将欲智而严,凡士欲愚。智则不可测,严则不可犯,故士皆委己而听命,夫安得不愚?夫惟士愚,而后可与之皆死。


凡兵之动,知敌之主,知敌之将,而后可以动于险。邓艾缒兵于蜀中,非刘禅之庸,则百万之师可以坐缚,彼固有所侮而动也。故古之贤将,能以兵尝敌,而又以敌自尝,故去就可以决。


凡主将之道,知理而后可以举兵,知势而后可以加兵,知节而后可以用兵。知理则不屈,知势则不沮,知节则不穷。见小利不动,见小患不避,小利小患,不足以辱吾技也,夫然后有以支大利大患。夫惟养技而自爱者,无敌于天下。故一忍可以支百勇,一静可以制百动。


兵有长短,敌我一也。敢问:“吾之所长,吾出而用之,彼将不与吾校;吾之所短,吾蔽而置之,彼将强与吾角,奈何?”曰:“吾之所短,吾抗而暴之,使之疑而却;吾之所长,吾阴而养之,使之狎而堕其中。此用长短之术也。”


善用兵者,使之无所顾,有所恃。无所顾,则知死之不足惜;有所恃,则知不至于必败。尺箠当猛虎,奋呼而操击;徒手遇蜥蜴,变色而却步,人之情也。知此者,可以将矣。袒裼而案剑,则乌获不敢逼;冠胄衣甲,据兵而寝,则童子弯弓杀之矣。故善用兵者以形固。夫能以形固,则力有馀矣。



\chapter*{黄生借书说}
\addcontentsline{toc}{chapter}{黄生借书说}
\begin{center}
	\textbf{[清朝]袁枚}
\end{center}


黄生允修借书。随园主人授以书,而告之曰:


书非借不能读也。子不闻藏书者乎?七略、四库,天子之书,然天子读书者有几?汗牛塞屋,富贵家之书,然富贵人读书者有几?其他祖父积,子孙弃者无论焉。非独书为然,天下物皆然。非夫人之物而强假焉,必虑人逼取,而惴惴焉摩玩之不已,曰:“今日存,明日去,吾不得而见之矣。”若业为吾所有,必高束焉,庋藏焉,曰“姑俟异日观”云尔。


余幼好书,家贫难致。有张氏藏书甚富。往借,不与,归而形诸梦。其切如是。故有所览辄省记。通籍后,俸去书来,落落大满,素蟫灰丝时蒙卷轴。然后叹借者之用心专,而少时之岁月为可惜也!


今黄生贫类予,其借书亦类予;惟予之公书与张氏之吝书若不相类。然则予固不幸而遇张乎,生固幸而遇予乎?知幸与不幸,则其读书也必专,而其归书也必速。


为一说,使与书俱。



\chapter*{峡江寺飞泉亭记}
\addcontentsline{toc}{chapter}{峡江寺飞泉亭记}
\begin{center}
	\textbf{[清朝]袁枚}
\end{center}


余年来观瀑屡矣,至峡江寺而意难决舍,则飞泉一亭为之也。


凡人之情,其目悦,其体不适,势不能久留。天台之瀑,离寺百步,雁宕瀑旁无寺。他若匡庐,若罗浮,若青田之石门,瀑未尝不奇,而游者皆暴日中,踞危崖,不得从容以观,如倾盖交,虽欢易别。


惟粤东峡山,高不过里许,而磴级纡曲,古松张覆,骄阳不炙。过石桥,有三奇树鼎足立,忽至半空,凝结为一。凡树皆根合而枝分,此独根分而枝合,奇已。


登山大半,飞瀑雷震,从空而下。瀑旁有室,即飞泉亭也。纵横丈馀,八窗明净,闭窗瀑闻,开窗瀑至。人可坐可卧,可箕踞,可偃仰,可放笔研,可瀹茗置饮,以人之逸,待水之劳,取九天银河,置几席间作玩。当时建此亭者,其仙乎!


僧澄波善弈,余命霞裳与之对枰。于是水声、棋声、松声、鸟声,参错并奏。顷之,又有曳杖声从云中来者,则老僧怀远抱诗集尺许,来索余序。于是吟咏之声又复大作。天籁人籁,合同而化。不图观瀑之娱,一至于斯,亭之功大矣!


坐久,日落,不得已下山,宿带玉堂。正对南山,云树蓊郁,中隔长江,风帆往来,妙无一人肯泊岸来此寺者。僧告余曰:“峡江寺俗名飞来寺。”余笑曰:“寺何能飞?惟他日余之魂梦或飞来耳!”僧曰:“无征不信。公爱之,何不记之!”余曰:“诺。”已遂述数行,一以自存,一以与僧。



\chapter*{声无哀乐论}
\addcontentsline{toc}{chapter}{声无哀乐论}
\begin{center}
	\textbf{[三国]嵇康}
\end{center}


有秦客问于东野主人曰:「闻之前论曰:『治世之音安以乐,亡国之音哀以思。』夫治乱在政,而音声应之;故哀思之情,表于金石;安乐之象,形于管弦也。又仲尼闻韶,识虞舜之德;季札听弦,知众国之风。斯已然之事,先贤所不疑也。今子独以为声无哀乐,其理何居?若有嘉讯,今请闻其说。」主人应之曰:「斯义久滞,莫肯拯救,故令历世滥于名实。今蒙启导,将言其一隅焉。夫天地合德,万物贵生,寒暑代往,五行以成。故章为五色,发为五音;音声之作,其犹臭味在于天地之间。其善与不善,虽遭遇浊乱,其体自若而不变也。岂以爱憎易操、哀乐改度哉?及宫商集比,声音克谐,此人心至愿,情欲之所锺。故人知情不可恣,欲不可极故,因其所用,每为之节,使哀不至伤,乐不至淫,斯其大较也。然『乐云乐云,锺鼓云乎哉?哀云哀云,哭泣云乎哉?因兹而言,玉帛非礼敬之实,歌舞非悲哀之主也。何以明之?夫殊方异俗,歌哭不同。使错而用之,或闻哭而欢,或听歌而戚,然而哀乐之情均也。今用均同之情,案,「戚」本作「感」,又脱同字,依《世说·文学篇》注改补。)而发万殊之声,斯非音声之无常哉?然声音和比,感人之最深者也。劳者歌其事,乐者舞其功。夫内有悲痛之心,则激切哀言。言比成诗,声比成音。杂而咏之,聚而听之,心动于和声,情感于苦言。嗟叹未绝,而泣涕流涟矣。夫哀心藏于苦心内,遇和声而后发。和声无象,而哀心有主。夫以有主之哀心,因乎无象之和声,其所觉悟,唯哀而已。岂复知『吹万不同,而使其自已』哉。风俗之流,遂成其政;是故国史明政教之得失,审国风之盛衰,吟咏情性以讽其上,故曰『亡国之音哀以思』也。夫喜、怒、哀、乐、爱、憎、惭、惧,凡此八者,生民所以接物传情,区别有属,而不可溢者也。夫味以甘苦为称,今以甲贤而心爱,以乙愚而情憎,则爱憎宜属我,而贤愚宜属彼也。可以我爱而谓之爱人,我憎而谓之憎人,所喜则谓之喜味,所怒而谓之怒味哉?由此言之,则外内殊用,彼我异名。声音自当以善恶为主,则无关于哀乐;哀乐自当以情感,则无系于声音。名实俱去,则尽然可见矣。且季子在鲁,采《诗》观礼,以别《风》、《雅》,岂徒任声以决臧否哉?又仲尼闻《韶》,叹其一致,是以咨嗟,何必因声以知虞舜之德,然後叹美邪?今粗明其一端,亦可思过半矣。」


秦客难曰:「八方异俗,歌哭万殊,然其哀乐之情,不得不见也。夫心动于中,而声出于心。虽托之于他音,寄之于余声,善听察者,要自觉之不使得过也。昔伯牙理琴而锺子知其所志;隶人击磬而子产识其心哀;鲁人晨哭而颜渊审其生离。夫数子者,岂复假智于常音,借验于曲度哉?心戚者则形为之动,情悲者则声为之哀。此自然相应,不可得逃,唯神明者能精之耳。夫能者不以声众为难,不能者不以声寡为易。今不可以未遇善听,而谓之声无可察之理;见方俗之多变,而谓声音无哀乐也。」又云:「贤不宜言爱,愚不宜言憎。然则有贤然后爱生,有愚然后憎成,但不当共其名耳。哀乐之作,亦有由而然。此为声使我哀,音使我乐也。苟哀乐由声,更为有实,何得名实俱去邪?」又云:「季子采《诗》观礼,以别《风》、《雅》;仲尼叹《韶》音之一致,是以咨嗟。是何言欤?且师襄奏操,而仲尼睹文王之容;师涓进曲,而子野识亡国之音。宁复讲诗而后下言,习礼然后立评哉?斯皆神妙独见,不待留闻积日,而已综其吉凶矣;是以前史以为美谈。今子以区区之近知,齐所见而为限,无乃诬前贤之识微,负夫子之妙察邪?」


主人答曰:「难云:虽歌哭万殊,善听察者要自觉之,不假智于常音,不借验于曲度,锺子之徒云云是也。此为心悲者,虽谈笑鼓舞,情欢者,虽拊膺咨嗟,犹不能御外形以自匿,诳察者于疑似也。以为就令声音之无常,犹谓当有哀乐耳。又曰:「季子听声,以知众国之风;师襄奏操,而仲尼睹文王之容。案如所云,此为文王之功德,与风俗之盛衰,皆可象之于声音:声之轻重,可移于後世;襄涓之巧,能得之于将来。若然者,三皇五帝,可不绝于今日,何独数事哉?若此果然也。则文王之操有常度,韶武之音有定数,不可杂以他变,操以余声也。则向所谓声音之无常,锺子之触类,于是乎踬矣。若音声无常,锺子触类,其果然邪?则仲尼之识微,季札之善听,固亦诬矣。此皆俗儒妄记,欲神其事而追为耳,欲令天下惑声音之道,不言理以尽此,而推使神妙难知,恨不遇奇听于当时,慕古人而自叹,斯所□大罔后生也。夫推类辨物,当先求之自然之理;理已定,然后借古义以明之耳。今未得之于心,而多恃前言以为谈证,自此以往,恐巧历不能纪。」「又难云:「哀乐之作,犹爱憎之由贤愚,此为声使我哀而音使我乐;苟哀乐由声,更为有实矣。夫五色有好丑丑,五声有善恶,此物之自然也。至于爱与不爱,喜与不喜,人情之变,统物之理,唯止于此;然皆无豫于内,待物而成耳。至夫哀乐自以事会,先遘于心,但因和声以自显发。故前论已明其无常,今复假此谈以正名号耳。不为哀乐发于声音,如爱憎之生于贤愚也。然和声之感人心,亦犹酒醴之发人情也。酒以甘苦为主,而醉者以喜怒为用。其见欢戚为声发,而谓声有哀乐,不可见喜怒为酒使,而谓酒有喜怒之理也。」


秦客难曰:「夫观气采色,天下之通用也。心变于内而色应于外,较然可见,故吾子不疑。夫声音,气之激者也。心应感而动,声从变而发。心有盛衰,声亦隆杀。同见役于一身,何独于声便当疑邪!夫喜怒章于色诊,哀乐亦宜形于声音。声音自当有哀乐,但暗者不能识之。至锺子之徒,虽遭无常之声,则颖然独见矣,今蒙瞽面墙而不悟,离娄昭秋毫于百寻,以此言之,则明暗殊能矣。不可守咫尺之度,而疑离娄之察;执中痛之听,而猜锺子之聪;皆谓古人为妄记也。」


主人答曰:「难云:心应感而动,声从变而发,心有盛衰,声亦降杀,哀乐之情,必形于声音,锺子之徒,虽遭无常之声,则颖然独见矣。必若所言,则浊质之饱,首阳之饥,卞和之冤,伯奇之悲,相如之含怒,不占之怖祗,千变百态,使各发一咏之歌,同启数弹之微,则锺子之徒,各审其情矣。尔为听声者不以寡众易思,察情者不以大小为异,同出一身者,期于识之也。设使从下,则子野之徒,亦当复操律鸣管,以考其音,知南风之盛衰,别雅、郑之淫正也?夫食辛之与甚噱,薰目之与哀泣,同用出泪,使狄牙尝之,必不言乐泪甜而哀泪苦,斯可知矣。何者?肌液肉汗,?笮便出,无主于哀乐,犹?酒之囊漉,虽笮具不同,而酒味不变也。声俱一体之所出,何独当含哀乐之理也?且夫《咸池》、《六茎》,《大章》、《韶夏》,此先王之至乐,所以动天地、感鬼神。今必云声音莫不象其体而传其心,此必为至乐不可托之于瞽史,必须圣人理其弦管,尔乃雅音得全也。舜命夔「击石拊石,八音克谐,神人以和。」以此言之,至乐虽待圣人而作,不必圣人自执也。何者?音声有自然之和,而无系于人情。克谐之音,成于金石;至和之声,得于管弦也。夫纤毫自有形可察,故离瞽以明暗异功耳。若乃以水济水,孰异之哉?」


秦客难曰:「虽众喻有隐,足招攻难,然其大理,当有所就。若葛卢闻牛鸣,知其三子为牺;师旷吹律,知南风不竞,楚师必败;羊舌母听闻儿啼,而审其丧家。凡此数事,皆效于上世,是以咸见录载。推此而言,则盛衰吉凶,莫不存乎声音矣。今若复谓之诬罔,则前言往记,皆为弃物,无用之也。以言通论,未之或安。若能明斯所以,显其所由,设二论俱济,愿重闻之。」


主人答曰:「吾谓能反三隅者,得意而忘言,是以前论略而未详。今复烦循环之难,敢不自一竭邪?夫鲁牛能知牺历之丧生,哀三子之不存,含悲经年,诉怨葛卢;此为心与人同,异于兽形耳。此又吾之所疑也。且牛非人类,无道相通,若谓鸣兽皆能有言,葛卢受性独晓之,此为称其语而论其事,犹译传异言耳,不为考声音而知其情,则非所以为难也。若谓知者为当触物而达,无所不知,今且先议其所易者。请问:圣人卒人胡域,当知其所言否乎?难者必曰知之。知之之理何以明之?愿借子之难以立鉴识之域。或当与关接识其言邪?将吹律鸣管校其音邪?观气采色和其心邪?此为知心自由气色,虽自不言,犹将知之,知之之道,可不待言也。若吹律校音以知其心,假令心志于马而误言鹿,察者固当由鹿以知马也。此为心不系于所言,言或不足以证心也。若当关接而知言,此为孺子学言于所师,然后知之,则何贵于聪明哉?夫言,非自然一定之物,五方殊俗,同事异号,举一名以为标识耳。夫圣人穷理,谓自然可寻,无微不照。苟无微不照,理蔽则虽近不见,故异域之言不得强通。推此以往,葛卢之不知牛鸣,得不全乎?」又难云:「师旷吹律,知南风不竞,楚多死声。此又吾之所疑也。请问师旷吹律之时,楚国之风邪,则相去千里,声不足达;若正识楚风来入律中邪,则楚南有吴、越,北有梁、宋,苟不见其原,奚以识之哉?凡阴阳愤激,然后成风。气之相感,触地而发,何得发楚庭,来入晋乎?且又律吕分四时之气耳,时至而气动,律应而灰移,皆自然相待,不假人以为用也。上生下生,所以均五声之和,叙刚柔之分也。然律有一定之声,虽冬吹中吕,其音自满而无损也。今以晋人之气,吹无韵之律,楚风安得来入其中,与为盈缩邪?风无形,声与律不通,则校理之地,无取于风律,不其然乎?岂独师旷多识博物,自有以知胜败之形,欲固众心而托以神微,若伯常骞之许景公寿哉?」又难云:「羊舌母听闻儿啼而审其丧家。复请问何由知之?为神心独悟暗语而当邪?尝闻儿啼若此其大而恶,今之啼声似昔之啼声,故知其丧家邪?若神心独悟暗语之当,非理之所得也。虽曰听啼,无取验于儿声矣。若以尝闻之声为恶,故知今啼当恶,此为以甲声为度,以校乙之啼也。夫声之于音,犹形之于心也。有形同而情乖,貌殊而心均者。何以明之?圣人齐心等德而形状不同也。苟心同而形异,则何言乎观形而知心哉?且口之激气为声,何异于籁?纳气而鸣邪?啼声之善恶,不由儿口吉凶,犹琴瑟之清浊不在操者之工拙也。心能辨理善谈,而不能令内?调利,犹瞽者能善其曲度,而不能令器必清和也。器不假妙瞽而良,?不因惠心而调,然则心之与声,明为二物。二物之诚然,则求情者不留观于形貌,揆心者不借听于声音也。察者欲因声以知心,不亦外乎?今晋母未待之于老成,而专信昨日之声,以证今日之啼,岂不误中于前世好奇者从而称之哉?」


秦客难曰:「吾闻败者不羞走,所以全也。吾心未厌而言,难复更从其馀。今平和之人,听筝笛琵琶,则形躁而志越;闻琴瑟之音,则听静而心闲。同一器之中,曲用每殊,则情随之变:奏秦声则叹羡而慷慨;理齐楚则情一而思专,肆姣弄则欢放而欲惬;心为声变,若此其众。苟躁静由声,则何为限其哀乐,而但云至和之声,无所不感,托大同于声音,归众变于人情?得无知彼不明此哉?」


主人答曰:「难云:琵琶、筝、笛令人躁越。又云:曲用每殊而情随之变。此诚所以使人常感也。琵琶、筝、笛,间促而声高,变众而节数,以高声御数节,故使人形躁而志越。犹铃铎警耳,锺鼓骇心,故『闻鼓鼙之音,思将帅之臣』,盖以声音有大小,故动人有猛静也。琴瑟之体,间辽而音埤,变希而声清,以埤音御希变,不虚心静听,则不尽清和之极,是以听静而心闲也。夫曲用不同,亦犹殊器之音耳。齐楚之曲,多重故情一,变妙故思专。姣弄之音,挹众声之美,会五音之和,其体赡而用博,故心侈于众理;五音会,故欢放而欲惬。然皆以单、复、高、埤、善、恶为体,而人情以躁、静而容端,此为声音之体,尽于舒疾。情之应声,亦止于躁静耳。夫曲用每殊,而情之处变,犹滋味异美,而口辄识之也。五味万殊,而大同于美;曲变虽众,亦大同于和。美有甘,和有乐。然随曲之情,尽于和域;应美之口,绝于甘境,安得哀乐于其间哉?然人情不同,各师所解。则发其所怀;若言平和,哀乐正等,则无所先发,故终得躁静。若有所发,则是有主于内,不为平和也。以此言之,躁静者,声之功也;哀乐者,情之主也。不可见声有躁静之应,因谓哀乐者皆由声音也。且声音虽有猛静,猛静各有一和,和之所感,莫不自发。何以明之?夫会宾盈堂,酒酣奏琴,或忻然而欢,或惨尔泣,非进哀于彼,导乐于此也。其音无变于昔,而欢戚并用,斯非『吹万不同』邪?夫唯无主于喜怒,亦应无主于哀乐,故欢戚俱见。若资偏固之音,含一致之声,其所发明,各当其分,则焉能兼御群理,总发众情邪?由是言之,声音以平和为体,而感物无常;心志以所俟为主,应感而发。然则声之与心,殊涂异轨,不相经纬,焉得染太和于欢戚,缀虚名于哀乐哉?秦客难曰:「论云:猛静之音,各有一和,和之所感,莫不自发,是以酒酣奏琴而欢戚并用。此言偏并之情先积于内,故怀欢者值哀音而发,内戚者遇乐声而感也。夫音声自当有一定之哀乐,但声化迟缓不可仓卒,不能对易。偏重之情,触物而作,故今哀乐同时而应耳;虽二情俱见,则何损于声音有定理邪?主人答曰:「难云:哀乐自有定声,但偏重之情,不可卒移。故怀戚者遇乐声而哀耳。即如所言,声有定分,假使《鹿鸣》重奏,是乐声也。而令戚者遇之,虽声化迟缓,但当不能使变令欢耳,何得更以哀邪?犹一爝之火,虽未能温一室,不宜复增其寒矣。夫火非隆寒之物,乐非增哀之具也。理弦高堂而欢戚并用者,直至和之发滞导情,故令外物所感得自尽耳。难云:偏重之情,触物而作,故令哀乐同时而应耳。夫言哀者,或见机杖而泣,或睹舆服而悲,徒以感人亡而物存,痛事显而形潜,其所以会之,皆自有由,不为触地而生哀,当席而泪出也。今见机杖以致感,听和声而流涕者,斯非和之所感,莫不自发也。」


秦客难曰:「论云:酒酣奏琴而欢戚并用。欲通此言,故答以偏情感物而发耳。今且隐心而言,明之以成效。夫人心不欢则戚,不戚则欢,此情志之大域也。然泣是戚之伤,笑是欢之用。盖闻齐、楚之曲者,唯睹其哀涕之容,而未曾见笑噱之貌。此必齐、楚之曲,以哀为体,故其所感,皆应其度量;岂徒以多重而少变,则致情一而思专邪?若诚能致泣,则声音之有哀乐,断可知矣。」


主人答曰:「虽人情感于哀乐,哀乐各有多少。又哀乐之极,不必同致也。夫小哀容坏,甚悲而泣,哀之方也;小欢颜悦,至乐心喻,乐之理也。何以明之?夫至亲安豫,则恬若自然,所自得也。及在危急,仅然后济,则?不及亻舞。由此言之,亻舞之不若向之自得,岂不然哉?,至夫笑噱虽出于欢情,然自以理成又非自然应声之具也。此为乐之应声,以自得为主;哀之应感,以垂涕为故。垂涕则形动而可觉,自得则神合而无忧,是以观其异而不识其同,别其外而未察其内耳。然笑噱之不显于声音,岂独齐楚之曲邪?今不求乐于自得之域,而以无笑噱谓齐、楚体哀,岂不知哀而不识乐乎?」


秦客问曰:「仲尼有言:『移风易俗,莫善于乐。』即如所论,凡百哀乐,皆不在声,即移风易俗,果以何物邪?又古人慎靡靡之风,抑忄舀耳之声,故曰:『放郑声,远佞人。』然则郑卫之音击鸣球以协神人,敢问郑雅之体,隆弊所极;风俗称易,奚由而济?幸重闻之,以悟所疑。」


主人应之曰:「夫言移风易俗者,必承衰弊之後也。古之王者,承天理物,必崇简易之教,御无为之治,君静于上,臣顺于下,玄化潜通,天人交泰,枯槁之类,浸育灵液,六合之内,沐浴鸿流,荡涤尘垢,群生安逸,自求多福,默然从道,怀忠抱义,而不觉其所以然也。和心足于内,和气见于外,故歌以叙志,亻舞以宣情。然后文之以采章,照之以风雅,播之以八音,感之以太和,导其神气,养而就之。迎其情性,致而明之,使心与理相顺,气与声相应,合乎会通,以济其美。故凯乐之情,见于金石,含弘光大,显于音声也。若以往则万国同风,芳荣济茂,馥如秋兰,不期而信,不谋而诚,穆然相爱,犹舒锦彩,而粲炳可观也。大道之隆,莫盛于兹,太平之业,莫显于此。故曰「『移风易俗,莫善于乐。』乐之为体,以心为主。故无声之乐,民之父母也。至八音会谐,人之所悦,亦总谓之乐,然风俗移易,不在此也。夫音声和比,人情所不能已者也。是以古人知情之不可放,故抑其所遁;知欲之不可绝,故因其所自。为可奉之礼,制可导之乐。口不尽味,乐不极音。揆终始之宜,度贤愚之中。为之检则,使远近同风,用而不竭,亦所以结忠信,著不迁也。故乡校庠塾亦随之变,丝竹与俎豆并存,羽毛与揖让俱用,正言与和声同发。使将听是声也,必闻此言;将观是容也,必崇此礼。礼犹宾主升降,然后酬酢行焉。于是言语之节,声音之度,揖让之仪,动止之数,进退相须,共为一体。君臣用之于朝,庶士用之于家,少而习之,长而不怠,心安志固,从善日迁,然后临之以敬,持之以久而不变,然后化成,此又先王用乐之意也。故朝宴聘享,嘉乐必存。是以国史采风俗之盛衰,寄之乐工,宣之管弦,使言之者无罪,闻之者足以自诫。此又先王用乐之意也。若夫郑声,是音声之至妙。妙音感人,犹美色惑志。耽?荒酒,易以丧业,自非至人,孰能御之?先王恐天下流而不反,故具其八音,不渎其声;绝其大和,不穷其变;捐窈窕之声,使乐而不淫,犹大羹不和,不极勺药之味也。若流俗浅近,则声不足悦,又非所欢也。若上失其道,国丧其纪,男女奔随,淫荒无度,则风以此变,俗以好成。尚其所志,则群能肆之,乐其所习,则何以诛之?托于和声,配而长之,诚动于言,心感于和,风俗一成,因而名之。然所名之声,无中于淫邪也。淫之与正同乎心,雅、郑之体,亦足以观矣。」



\chapter*{乞猫}
\addcontentsline{toc}{chapter}{乞猫}
\begin{center}
	\textbf{[明朝]刘基}
\end{center}


赵人患鼠,乞猫于中山。中山人予之猫,猫善捕鼠及鸡。月余,鼠尽而鸡亦尽。其子患之,告其父曰:“盍去诸?”其父曰:“是非若所知也。吾之患在鼠,不在乎无鸡。夫有鼠,则窃吾食,毁吾衣,穿吾垣墉,毁伤吾器用,吾将饥寒焉,不病于无鸡乎?无鸡者,弗食鸡则已耳,去饥寒犹远,若之何而去夫猫也!”


(选自明·刘基《郁离子·捕鼠》)



\chapter*{若石之死}
\addcontentsline{toc}{chapter}{若石之死}
\begin{center}
	\textbf{[明朝]刘基}
\end{center}


若石居冥山之阴,有虎恒窥其藩。若石帅家人昼夜警:日出而殷钲,日入而举辉,筑墙掘坎以守。卒岁虎不能有获。一日,虎死,若石大喜,自以为虎死无毒己者矣。于是弛其惫,墙坏而不葺。无何,有貙闻其牛羊豕之声而入食焉。若石不知其为貙也,斥之不走。貙人立而爪之毙。人曰:若石知其一而不知其二,其死也宜。

\chapter*{秦西巴纵麑}
\addcontentsline{toc}{chapter}{秦西巴纵麑}
\begin{center}
	\textbf{[秦朝]吕不韦}
\end{center}


孟孙猎而得麑,使秦西巴持归烹之。麑母随之而啼,秦西巴弗忍,纵而与之。孟孙归,求麑安在。秦西巴对曰:“其母随而啼,臣诚弗忍,窃纵而予之。”孟孙怒,逐秦西巴。居一年,取以为子傅。左右曰:“秦西巴有罪于君,今以为子傅,何也?”孟孙曰:“夫一麑不忍,又何况于人乎?”

\chapter*{季梁谏追楚师}
\addcontentsline{toc}{chapter}{季梁谏追楚师}
\begin{center}
	\textbf{[春秋战国]左丘明}
\end{center}


楚武王侵随,使薳章求成焉,军于瑕以待之。随人使少师董成。


斗伯比言于楚子曰:“吾不得志于汉东也,我则使然。我张吾三军而被吾甲兵,以武临之,彼则惧而协以谋我,故难间也。汉东之国,随为大。随张,必弃小国。小国离,楚之利也。少师侈,请羸师以张之。”熊率且比曰:“季梁在,何益?”斗伯比曰:“以为后图。少师得其君。”


王毁军而纳少师。少师归,请追楚师。随侯将许之。


季梁止之曰:“天方授楚。楚之羸,其诱我也,君何急焉?臣闻小之能敌大也,小道大淫。所谓道,忠于民而信于神也。上思利民,忠也;祝史正辞,信也。今民馁而君逞欲,祝史矫举以祭,臣不知其可也。”公曰:“吾牲牷肥腯,粢盛丰备,何则不信?”对曰:“夫民,神之主也。是以圣王先成民,而后致力于神。故奉牲以告曰‘博硕肥腯。’谓民力之普存也,谓其畜之硕大蕃滋也,谓其不疾瘯蠡也,谓其备腯咸有也。奉盛以告曰:‘洁粢丰盛。’谓其三时不害而民和年丰也。奉酒醴以告曰:‘嘉栗旨酒。’谓其上下皆有嘉德而无违心也。所谓馨香,无谗慝也。故务其三时,修其五教,亲其九族,以致其禋祀。于是乎民和而神降之福,故动则有成。今民各有心,而鬼神乏主,君虽独丰,其何福之有?君姑修政而亲兄弟之国,庶免于难。”


随侯惧而修政,楚不敢伐。



\chapter*{子产却楚逆女以兵}
\addcontentsline{toc}{chapter}{子产却楚逆女以兵}
\begin{center}
	\textbf{[春秋战国]左丘明}
\end{center}


楚公子围聘于郑,且娶于公孙段氏。伍举为介。将入馆,郑人恶之。使行人子羽与之言,乃馆于外。


既聘,将以众逆。子产患之,使子羽辞曰:“以敝邑褊小,不足以容从者,请墠听命!”令尹使太宰伯州犁对曰:“君辱贶寡大夫围,谓围:‘将使丰氏抚有而室。’围布几筵,告于庄、共之庙而来。若野赐之,是委君贶于草莽也!是寡大夫不得列于诸卿也!不宁唯是,又使围蒙其先君,将不得为寡君老,其蔑以复矣。唯大夫图之!”子羽曰:“小国无罪,恃实其罪。将恃大国之安靖己,而无乃包藏祸心以图之。小国失恃而惩诸侯,使莫不憾者,距违君命,而有所壅塞不行是惧!不然,敝邑,馆人之属也,其敢爱丰氏之祧?”


伍举知其有备也,请垂櫜而入。许之。



\chapter*{申胥谏许越成}
\addcontentsline{toc}{chapter}{申胥谏许越成}
\begin{center}
	\textbf{[春秋战国]左丘明}
\end{center}


吴王夫差乃告诸大夫曰:“孤将有大志于齐,吾将许越成,而无拂吾虑。若越既改,吾又何求?若其不改,反行,吾振旅焉。”申胥谏曰:“不可许也。夫越非实忠心好吴也,又非慑畏吾甲兵之强也。大夫种勇而善谋,将还玩吴国于股掌之上,以得其志。夫固知君王之盖威以好胜也,故婉约其辞,以从逸王志,使淫乐于诸夏之国,以自伤也。使吾甲兵钝弊,民人离落,而日以憔悴,然后安受吾烬。夫越王好信以爱民,四方归之,年谷时熟,日长炎炎,及吾犹可以战也。为虺弗摧,为蛇将若何?”吴王曰:“大夫奚隆于越?越曾足以为大虞乎?若无越,则吾何以春秋曜吾军士?”乃许之成。


将盟,越王又使诸稽郢辞曰:“以盟为有益乎?前盟口血未乾,足以结信矣。以盟为无益乎?君王舍甲兵之威以临使之,而胡重于鬼神而自轻也。”吴王乃许之,荒成不盟。



\chapter*{曾国藩诫子书}
\addcontentsline{toc}{chapter}{曾国藩诫子书}
\begin{center}
	\textbf{[清朝]曾国藩}
\end{center}


余通籍三十余年,官至极品,而学业一无所成,德行一无许可,老大徒伤,不胜悚惶惭赧。今将永别,特将四条教汝兄弟。


一曰慎独而心安。自修之道,莫难于养心;养心之难,又在慎独。能慎独,册内省不疚,可以对天地质鬼神。人无一内愧之事,则天君泰然。此心常快足宽平,是人生第一自强之道,第一寻乐之方,守身之先务也。


二曰主敬则身强。内而专静纯一,外而整齐严肃。敬之工夫也;出门如见大宾,使民如承大祭,敬之气象也;修己以安百姓,笃恭而天下平,敬之效验也。聪明睿智,皆由此出。庄敬日强,安肆日偷。若人无众寡,事无大小,一一恭敬,不敢怠慢。则身强之强健,又何疑乎?


三曰求仁则人悦。凡人之生,皆得天地之理以成性,得天地之气以成形,我与民物,其大本乃同出一源。若但知私己而不知仁民爱物,是于大本一源之道已悖而失之矣。至于尊官厚禄,高居人上,则有拯民溺救民饥之责。读书学古,粗知大义,既有觉后知觉后觉之责。孔门教人,莫大于求仁,而其最切者,莫要于欲立立人、欲达达人数语。立人达人之人,人有不悦而归之者乎?


四曰习劳则神钦。人一日所着之衣所进之食,与日所行之事所用之力相称,则旁人韪之,鬼神许之,以为彼自食其力也。若农夫织妇终岁勤动,以成数石之粟数尺之布,而富贵之家终岁逸乐,不营一业,而食必珍馐,衣必锦绣,酣豢高眠,一呼百诺,此天下最不平之事,神鬼所不许也,其能久乎?古之圣君贤相,盖无时不以勤劳自励。为一身计,则必操习技艺,磨练筋骨,困知勉行,操心危虑,而后可以增智慧而长见识。为天下计,则必已饥已溺,一夫不获,引为余辜。大禹、墨子皆极俭以奉身而极勤以救民。勤则寿,逸则夭,勤则有材而见用,逸则无劳而见弃,勤则博济斯民而神祇钦仰,逸则无补于人而神鬼不歆。


此四条为余数十年人世之得,汝兄弟记之行之,并传之于子子孙孙,则余曾家可长盛不衰,代有人才。



\chapter*{李遥买杖}
\addcontentsline{toc}{chapter}{李遥买杖}
\begin{center}
	\textbf{[宋朝]沈括}
\end{center}


随州大洪山作人李遥,杀人亡命。逾年,至秭归,因出市,见鬻柱杖者,等闲以数十钱买之。是时,秭归适又有邑民为人所杀,求贼甚急。民之子见遥所操杖,识之,曰:“此吾父杖也。”遂以告官司。吏执遥验之,果邑民之杖也。榜掠备至。遥实买杖,而鬻杖者已不见,卒未有以自明。有司诘其行止来历,势不可隐,乃通随州,而大洪杀人之罪遂败。市人千万而遥适值之,因缘及其隐匿,此亦事之可怪者。

\chapter*{古人铸鉴}
\addcontentsline{toc}{chapter}{古人铸鉴}
\begin{center}
	\textbf{[宋朝]沈括}
\end{center}


此工之巧智,后人不能造。比得古鉴,皆刮磨令平,此师旷所以伤知音也。

世有透光鉴,鉴背有铭文,凡二十字,字极古,莫能读。以鉴承日光,则背文及二十字皆透,在屋壁上了了分明。人有原其理,以谓铸时薄处先冷,唯背文上差厚后冷,而铜缩多。文虽在背,而鉴面隐然有迹,所以于光中现。予观之,理诚如是。然余家有三鉴,又见他家所藏,皆是一样,文画铭字无纤异者,形制甚古。唯此鉴光透,其他鉴虽至薄者,皆莫能透。意古人别自有术。


选自沈括(宋)——《梦溪笔谈》



\chapter*{梁鸿尚节}
\addcontentsline{toc}{chapter}{梁鸿尚节}
\begin{center}
	\textbf{[南北朝]范晔}
\end{center}


(梁鸿)家贫而尚节,博览无不通。而不为章句。学毕,乃牧豕于上林苑中,曾误遗火,延及他舍。鸿乃寻访烧者,问所去失,悉以豕偿之。其主犹以为少。鸿曰:“无他财,愿以身居作。”主人许之。因为执勤,不懈朝夕。邻家耆老见鸿非恒人,乃共责让主人,而称鸿长者。于是,始敬鸿,悉还其豕。鸿不受而去。


(选自南宋·范晔《后汉书.卷八十三》



\chapter*{鲁恭治中牟}
\addcontentsline{toc}{chapter}{鲁恭治中牟}
\begin{center}
	\textbf{[南北朝]范晔}
\end{center}


鲁恭为中牢令,重德化,不任刑罚。袁安闻之,疑其不实,阴使人往视之。随恭行阡陌,俱坐桑下。有雉过,止其旁,旁有儿童。其人曰:“儿何不捕之?”儿言雉方雏,不得捕。其人讶而起,与恭决曰:“所以来者,欲察君之政绩也。今蝗不犯境,此一异也;爱及鸟兽,此二异也;童有仁心,此三异也。久留徒扰贤者耳,吾将速反,以状白安。”

\chapter*{虎求百兽}
\addcontentsline{toc}{chapter}{虎求百兽}
\begin{center}
	\textbf{[汉朝]刘向}
\end{center}


荆宣王问群臣曰:“吾闻北方之畏昭奚恤也,果诚何如?”群臣莫对。

江乙对曰:“虎求百兽而食之,得狐。狐曰:‘子无敢食我也!天帝使我长百兽。今子食我,是逆天帝命也!子以我为不信,吾为子先行,于随我后,观百兽之见我而敢不走乎?”虎以为然,故遂与之行。兽见之,皆走。虎不知兽畏己而走也,以为畏狐也。

今王之地五千里,带甲百万,而专属之于昭奚恤,故北方之畏奚恤也,其实畏王之甲兵也!犹百兽之畏虎也!”

\chapter*{鹬蚌相争}
\addcontentsline{toc}{chapter}{鹬蚌相争}
\begin{center}
	\textbf{[汉朝]刘向}
\end{center}


 赵且伐燕,苏代为燕谓惠王曰:“今者臣来,过易水。蚌方出曝,而鹬啄其肉,蚌合而箝其喙。鹬曰:‘今日不雨,明日不雨,即有死蚌!’蚌亦谓鹬曰:‘今日不出,明日不出,即有死鹬!’两者不肯相舍,渔者得而并禽之。今赵且伐燕,燕赵久相支,以弊大众,臣恐强秦之为渔夫也。故愿王之熟计之也!”惠王曰:“善。”乃止。


\chapter*{曾子不受邑}
\addcontentsline{toc}{chapter}{曾子不受邑}
\begin{center}
	\textbf{[汉朝]刘向}
\end{center}


曾子衣敝衣以耕。鲁君使人往致邑焉,曰:“请以此修衣。”曾子不受,反,复往,又不受。使者曰:“先生非求于人,人则献之,奚为不受?”曾子曰:“臣闻之,‘受人者畏人;予人者骄人。’纵子有赐,不我骄也,我能勿畏乎?”终不受。孔子闻之,曰:“参之言足以全其节也。”(选自汉·刘向《说苑》)



\chapter*{六亲五法}
\addcontentsline{toc}{chapter}{六亲五法}
\begin{center}
	\textbf{[汉朝]刘向}
\end{center}


以家为乡,乡不可为也;以乡为国,国不可为也;以国为天下,天下不可为也。以家为家,以乡为乡,以国为国,以天下为天下。毋曰不同生,远者不听;毋曰不同乡,远者不行;毋曰不同国,远者不从。如地如天,何私何亲?如月如日,唯君之节!


御民之辔,在上之所贵;道民之门,在上之所先;召民之路,在上之所好恶。故君求之,则臣得之;君嗜之,则臣食之;君好之,则臣服之;君恶之,则臣匿之。毋蔽汝恶,毋异汝度,贤者将不汝助。言室满室,言堂满堂,是谓圣王。城郭沟渠,不足以固守;兵甲强力,不足以应敌;博地多财,不足以有众。惟有道者,能备患於未形也,故祸不萌。


天下不患无臣,患无君以使之;天下不患无财,患无人以分之。故知时者,可立以为长;无私者,可置以为政;审於时而察於用,而能备官者,可奉以为君也。缓者,後於事;吝於财者,失所亲;信小人者,失士。



\chapter*{读书要三到}
\addcontentsline{toc}{chapter}{读书要三到}
\begin{center}
	\textbf{[宋朝]朱熹}
\end{center}

凡读书......须要读得字字响亮,不可误一字,不可少一字,不可多一字,不可倒一字,不可牵强暗记,只是要多诵数遍,自然上口,久远不忘。古人云,“读书百遍,其义自见”。谓读得熟,则不待解说,自晓其义也。余尝谓,读书有三到,谓心到,眼到,口到。心不在此,则眼不看仔细,心眼既不专一,却只漫浪诵读,决不能记,记亦不能久也。三到之中,心到最急。心既到矣,眼口岂不到乎?


\chapter*{司马光好学}
\addcontentsline{toc}{chapter}{司马光好学}
\begin{center}
	\textbf{[宋朝]朱熹}
\end{center}


司马温公幼时,患记问不若人。群居讲习,众兄弟既成诵,游息矣;独下帷绝编,迨能倍诵乃止。用力多者收功远,其所精诵,乃终身不忘也。温公尝言:“书不可不成诵。或在马上,或中夜不寝时,咏其文,思其义,所得多矣。”(选自朱熹编辑的《三朝名臣言行录》)

\chapter*{陈谏议教子}
\addcontentsline{toc}{chapter}{陈谏议教子}
\begin{center}
	\textbf{[宋朝]朱熹}
\end{center}


宋陈谏议家有劣马,性暴,不可驭,蹄啮伤人多矣。一日,谏议入厩,不见是马,因诘仆:“彼马何以不见?”仆言为陈尧咨售之贾人矣。尧咨者,陈谏议之子也。谏议遽召子,曰:“汝为贵臣,家中左右尚不能制,贾人安能蓄之?是移祸于人也!”急命人追贾人取马,而偿其直。戒仆养之终老。时人称陈谏议有古仁之风。

\chapter*{吴起守信}
\addcontentsline{toc}{chapter}{吴起守信}
\begin{center}
	\textbf{[明朝]宋濂}
\end{center}


昔吴起出,遇故人,而止之食。故人曰:“诺,期返而食。”起曰:“待公而食。”故人至暮不来,起不食待之。明日早,令人求故人,故人来,方与之食。起之不食以俟者,恐其自食其言也。其为信若此,宜其能服三军欤?欲服三军,非信不可也!

\chapter*{王冕好学}
\addcontentsline{toc}{chapter}{王冕好学}
\begin{center}
	\textbf{[明朝]宋濂}
\end{center}


王冕者,诸暨人。七八岁时,父命牧牛陇上,窃入学舍,听诸生诵书;听已,辄默记。暮归,忘其牛。或牵牛来责蹊田者。父怒,挞之,已而复如初。母曰:“儿痴如此,曷不听其所为?”冕因去,依僧寺以居。夜潜出,坐佛膝上,执策映长明灯读之,琅琅达旦。佛像多土偶,狞恶可怖;冕小儿,恬若不见。


安阳韩性闻而异之,录为弟子,学遂为通儒。性卒,门人事冕如事性。时冕父已卒,即迎母入越城就养。久之,母思还故里,冕买白牛驾母车,自被古冠服随车后。乡里儿竞遮道讪笑,冕亦笑。选自《元史·王冕传》



\chapter*{心术}
\addcontentsline{toc}{chapter}{心术}
\begin{center}
	\textbf{[宋朝]苏洵}
\end{center}


为将之道,当先治心。泰山崩于前而色不变,麋鹿兴于左而目不瞬,然后可以制利害,可以待敌。


凡兵上义;不义,虽利勿动。非一动之为利害,而他日将有所不可措手足也。夫惟义可以怒士,士以义怒,可与百战。


凡战之道,未战养其财,将战养其力,既战养其气,既胜养其心。谨烽燧,严斥堠,使耕者无所顾忌,所以养其财;丰犒而优游之,所以养其力;小胜益急,小挫益厉,所以养其气;用人不尽其所欲为,所以养其心。故士常蓄其怒、怀其欲而不尽。怒不尽则有馀勇,欲不尽则有馀贪。故虽并天下,而士不厌兵,此黄帝之所以七十战而兵不殆也。不养其心,一战而胜,不可用矣。


凡将欲智而严,凡士欲愚。智则不可测,严则不可犯,故士皆委己而听命,夫安得不愚?夫惟士愚,而后可与之皆死。


凡兵之动,知敌之主,知敌之将,而后可以动于险。邓艾缒兵于蜀中,非刘禅之庸,则百万之师可以坐缚,彼固有所侮而动也。故古之贤将,能以兵尝敌,而又以敌自尝,故去就可以决。


凡主将之道,知理而后可以举兵,知势而后可以加兵,知节而后可以用兵。知理则不屈,知势则不沮,知节则不穷。见小利不动,见小患不避,小利小患,不足以辱吾技也,夫然后有以支大利大患。夫惟养技而自爱者,无敌于天下。故一忍可以支百勇,一静可以制百动。


兵有长短,敌我一也。敢问:“吾之所长,吾出而用之,彼将不与吾校;吾之所短,吾蔽而置之,彼将强与吾角,奈何?”曰:“吾之所短,吾抗而暴之,使之疑而却;吾之所长,吾阴而养之,使之狎而堕其中。此用长短之术也。”


善用兵者,使之无所顾,有所恃。无所顾,则知死之不足惜;有所恃,则知不至于必败。尺箠当猛虎,奋呼而操击;徒手遇蜥蜴,变色而却步,人之情也。知此者,可以将矣。袒裼而案剑,则乌获不敢逼;冠胄衣甲,据兵而寝,则童子弯弓杀之矣。故善用兵者以形固。夫能以形固,则力有馀矣。



\chapter*{黄生借书说}
\addcontentsline{toc}{chapter}{黄生借书说}
\begin{center}
	\textbf{[清朝]袁枚}
\end{center}


黄生允修借书。随园主人授以书,而告之曰:


书非借不能读也。子不闻藏书者乎?七略、四库,天子之书,然天子读书者有几?汗牛塞屋,富贵家之书,然富贵人读书者有几?其他祖父积,子孙弃者无论焉。非独书为然,天下物皆然。非夫人之物而强假焉,必虑人逼取,而惴惴焉摩玩之不已,曰:“今日存,明日去,吾不得而见之矣。”若业为吾所有,必高束焉,庋藏焉,曰“姑俟异日观”云尔。


余幼好书,家贫难致。有张氏藏书甚富。往借,不与,归而形诸梦。其切如是。故有所览辄省记。通籍后,俸去书来,落落大满,素蟫灰丝时蒙卷轴。然后叹借者之用心专,而少时之岁月为可惜也!


今黄生贫类予,其借书亦类予;惟予之公书与张氏之吝书若不相类。然则予固不幸而遇张乎,生固幸而遇予乎?知幸与不幸,则其读书也必专,而其归书也必速。


为一说,使与书俱。



\chapter*{峡江寺飞泉亭记}
\addcontentsline{toc}{chapter}{峡江寺飞泉亭记}
\begin{center}
	\textbf{[清朝]袁枚}
\end{center}


余年来观瀑屡矣,至峡江寺而意难决舍,则飞泉一亭为之也。


凡人之情,其目悦,其体不适,势不能久留。天台之瀑,离寺百步,雁宕瀑旁无寺。他若匡庐,若罗浮,若青田之石门,瀑未尝不奇,而游者皆暴日中,踞危崖,不得从容以观,如倾盖交,虽欢易别。


惟粤东峡山,高不过里许,而磴级纡曲,古松张覆,骄阳不炙。过石桥,有三奇树鼎足立,忽至半空,凝结为一。凡树皆根合而枝分,此独根分而枝合,奇已。


登山大半,飞瀑雷震,从空而下。瀑旁有室,即飞泉亭也。纵横丈馀,八窗明净,闭窗瀑闻,开窗瀑至。人可坐可卧,可箕踞,可偃仰,可放笔研,可瀹茗置饮,以人之逸,待水之劳,取九天银河,置几席间作玩。当时建此亭者,其仙乎!


僧澄波善弈,余命霞裳与之对枰。于是水声、棋声、松声、鸟声,参错并奏。顷之,又有曳杖声从云中来者,则老僧怀远抱诗集尺许,来索余序。于是吟咏之声又复大作。天籁人籁,合同而化。不图观瀑之娱,一至于斯,亭之功大矣!


坐久,日落,不得已下山,宿带玉堂。正对南山,云树蓊郁,中隔长江,风帆往来,妙无一人肯泊岸来此寺者。僧告余曰:“峡江寺俗名飞来寺。”余笑曰:“寺何能飞?惟他日余之魂梦或飞来耳!”僧曰:“无征不信。公爱之,何不记之!”余曰:“诺。”已遂述数行,一以自存,一以与僧。



\chapter*{声无哀乐论}
\addcontentsline{toc}{chapter}{声无哀乐论}
\begin{center}
	\textbf{[三国]嵇康}
\end{center}


有秦客问于东野主人曰:「闻之前论曰:『治世之音安以乐,亡国之音哀以思。』夫治乱在政,而音声应之;故哀思之情,表于金石;安乐之象,形于管弦也。又仲尼闻韶,识虞舜之德;季札听弦,知众国之风。斯已然之事,先贤所不疑也。今子独以为声无哀乐,其理何居?若有嘉讯,今请闻其说。」主人应之曰:「斯义久滞,莫肯拯救,故令历世滥于名实。今蒙启导,将言其一隅焉。夫天地合德,万物贵生,寒暑代往,五行以成。故章为五色,发为五音;音声之作,其犹臭味在于天地之间。其善与不善,虽遭遇浊乱,其体自若而不变也。岂以爱憎易操、哀乐改度哉?及宫商集比,声音克谐,此人心至愿,情欲之所锺。故人知情不可恣,欲不可极故,因其所用,每为之节,使哀不至伤,乐不至淫,斯其大较也。然『乐云乐云,锺鼓云乎哉?哀云哀云,哭泣云乎哉?因兹而言,玉帛非礼敬之实,歌舞非悲哀之主也。何以明之?夫殊方异俗,歌哭不同。使错而用之,或闻哭而欢,或听歌而戚,然而哀乐之情均也。今用均同之情,案,「戚」本作「感」,又脱同字,依《世说·文学篇》注改补。)而发万殊之声,斯非音声之无常哉?然声音和比,感人之最深者也。劳者歌其事,乐者舞其功。夫内有悲痛之心,则激切哀言。言比成诗,声比成音。杂而咏之,聚而听之,心动于和声,情感于苦言。嗟叹未绝,而泣涕流涟矣。夫哀心藏于苦心内,遇和声而后发。和声无象,而哀心有主。夫以有主之哀心,因乎无象之和声,其所觉悟,唯哀而已。岂复知『吹万不同,而使其自已』哉。风俗之流,遂成其政;是故国史明政教之得失,审国风之盛衰,吟咏情性以讽其上,故曰『亡国之音哀以思』也。夫喜、怒、哀、乐、爱、憎、惭、惧,凡此八者,生民所以接物传情,区别有属,而不可溢者也。夫味以甘苦为称,今以甲贤而心爱,以乙愚而情憎,则爱憎宜属我,而贤愚宜属彼也。可以我爱而谓之爱人,我憎而谓之憎人,所喜则谓之喜味,所怒而谓之怒味哉?由此言之,则外内殊用,彼我异名。声音自当以善恶为主,则无关于哀乐;哀乐自当以情感,则无系于声音。名实俱去,则尽然可见矣。且季子在鲁,采《诗》观礼,以别《风》、《雅》,岂徒任声以决臧否哉?又仲尼闻《韶》,叹其一致,是以咨嗟,何必因声以知虞舜之德,然後叹美邪?今粗明其一端,亦可思过半矣。」


秦客难曰:「八方异俗,歌哭万殊,然其哀乐之情,不得不见也。夫心动于中,而声出于心。虽托之于他音,寄之于余声,善听察者,要自觉之不使得过也。昔伯牙理琴而锺子知其所志;隶人击磬而子产识其心哀;鲁人晨哭而颜渊审其生离。夫数子者,岂复假智于常音,借验于曲度哉?心戚者则形为之动,情悲者则声为之哀。此自然相应,不可得逃,唯神明者能精之耳。夫能者不以声众为难,不能者不以声寡为易。今不可以未遇善听,而谓之声无可察之理;见方俗之多变,而谓声音无哀乐也。」又云:「贤不宜言爱,愚不宜言憎。然则有贤然后爱生,有愚然后憎成,但不当共其名耳。哀乐之作,亦有由而然。此为声使我哀,音使我乐也。苟哀乐由声,更为有实,何得名实俱去邪?」又云:「季子采《诗》观礼,以别《风》、《雅》;仲尼叹《韶》音之一致,是以咨嗟。是何言欤?且师襄奏操,而仲尼睹文王之容;师涓进曲,而子野识亡国之音。宁复讲诗而后下言,习礼然后立评哉?斯皆神妙独见,不待留闻积日,而已综其吉凶矣;是以前史以为美谈。今子以区区之近知,齐所见而为限,无乃诬前贤之识微,负夫子之妙察邪?」


主人答曰:「难云:虽歌哭万殊,善听察者要自觉之,不假智于常音,不借验于曲度,锺子之徒云云是也。此为心悲者,虽谈笑鼓舞,情欢者,虽拊膺咨嗟,犹不能御外形以自匿,诳察者于疑似也。以为就令声音之无常,犹谓当有哀乐耳。又曰:「季子听声,以知众国之风;师襄奏操,而仲尼睹文王之容。案如所云,此为文王之功德,与风俗之盛衰,皆可象之于声音:声之轻重,可移于後世;襄涓之巧,能得之于将来。若然者,三皇五帝,可不绝于今日,何独数事哉?若此果然也。则文王之操有常度,韶武之音有定数,不可杂以他变,操以余声也。则向所谓声音之无常,锺子之触类,于是乎踬矣。若音声无常,锺子触类,其果然邪?则仲尼之识微,季札之善听,固亦诬矣。此皆俗儒妄记,欲神其事而追为耳,欲令天下惑声音之道,不言理以尽此,而推使神妙难知,恨不遇奇听于当时,慕古人而自叹,斯所□大罔后生也。夫推类辨物,当先求之自然之理;理已定,然后借古义以明之耳。今未得之于心,而多恃前言以为谈证,自此以往,恐巧历不能纪。」「又难云:「哀乐之作,犹爱憎之由贤愚,此为声使我哀而音使我乐;苟哀乐由声,更为有实矣。夫五色有好丑丑,五声有善恶,此物之自然也。至于爱与不爱,喜与不喜,人情之变,统物之理,唯止于此;然皆无豫于内,待物而成耳。至夫哀乐自以事会,先遘于心,但因和声以自显发。故前论已明其无常,今复假此谈以正名号耳。不为哀乐发于声音,如爱憎之生于贤愚也。然和声之感人心,亦犹酒醴之发人情也。酒以甘苦为主,而醉者以喜怒为用。其见欢戚为声发,而谓声有哀乐,不可见喜怒为酒使,而谓酒有喜怒之理也。」


秦客难曰:「夫观气采色,天下之通用也。心变于内而色应于外,较然可见,故吾子不疑。夫声音,气之激者也。心应感而动,声从变而发。心有盛衰,声亦隆杀。同见役于一身,何独于声便当疑邪!夫喜怒章于色诊,哀乐亦宜形于声音。声音自当有哀乐,但暗者不能识之。至锺子之徒,虽遭无常之声,则颖然独见矣,今蒙瞽面墙而不悟,离娄昭秋毫于百寻,以此言之,则明暗殊能矣。不可守咫尺之度,而疑离娄之察;执中痛之听,而猜锺子之聪;皆谓古人为妄记也。」


主人答曰:「难云:心应感而动,声从变而发,心有盛衰,声亦降杀,哀乐之情,必形于声音,锺子之徒,虽遭无常之声,则颖然独见矣。必若所言,则浊质之饱,首阳之饥,卞和之冤,伯奇之悲,相如之含怒,不占之怖祗,千变百态,使各发一咏之歌,同启数弹之微,则锺子之徒,各审其情矣。尔为听声者不以寡众易思,察情者不以大小为异,同出一身者,期于识之也。设使从下,则子野之徒,亦当复操律鸣管,以考其音,知南风之盛衰,别雅、郑之淫正也?夫食辛之与甚噱,薰目之与哀泣,同用出泪,使狄牙尝之,必不言乐泪甜而哀泪苦,斯可知矣。何者?肌液肉汗,?笮便出,无主于哀乐,犹?酒之囊漉,虽笮具不同,而酒味不变也。声俱一体之所出,何独当含哀乐之理也?且夫《咸池》、《六茎》,《大章》、《韶夏》,此先王之至乐,所以动天地、感鬼神。今必云声音莫不象其体而传其心,此必为至乐不可托之于瞽史,必须圣人理其弦管,尔乃雅音得全也。舜命夔「击石拊石,八音克谐,神人以和。」以此言之,至乐虽待圣人而作,不必圣人自执也。何者?音声有自然之和,而无系于人情。克谐之音,成于金石;至和之声,得于管弦也。夫纤毫自有形可察,故离瞽以明暗异功耳。若乃以水济水,孰异之哉?」


秦客难曰:「虽众喻有隐,足招攻难,然其大理,当有所就。若葛卢闻牛鸣,知其三子为牺;师旷吹律,知南风不竞,楚师必败;羊舌母听闻儿啼,而审其丧家。凡此数事,皆效于上世,是以咸见录载。推此而言,则盛衰吉凶,莫不存乎声音矣。今若复谓之诬罔,则前言往记,皆为弃物,无用之也。以言通论,未之或安。若能明斯所以,显其所由,设二论俱济,愿重闻之。」


主人答曰:「吾谓能反三隅者,得意而忘言,是以前论略而未详。今复烦循环之难,敢不自一竭邪?夫鲁牛能知牺历之丧生,哀三子之不存,含悲经年,诉怨葛卢;此为心与人同,异于兽形耳。此又吾之所疑也。且牛非人类,无道相通,若谓鸣兽皆能有言,葛卢受性独晓之,此为称其语而论其事,犹译传异言耳,不为考声音而知其情,则非所以为难也。若谓知者为当触物而达,无所不知,今且先议其所易者。请问:圣人卒人胡域,当知其所言否乎?难者必曰知之。知之之理何以明之?愿借子之难以立鉴识之域。或当与关接识其言邪?将吹律鸣管校其音邪?观气采色和其心邪?此为知心自由气色,虽自不言,犹将知之,知之之道,可不待言也。若吹律校音以知其心,假令心志于马而误言鹿,察者固当由鹿以知马也。此为心不系于所言,言或不足以证心也。若当关接而知言,此为孺子学言于所师,然后知之,则何贵于聪明哉?夫言,非自然一定之物,五方殊俗,同事异号,举一名以为标识耳。夫圣人穷理,谓自然可寻,无微不照。苟无微不照,理蔽则虽近不见,故异域之言不得强通。推此以往,葛卢之不知牛鸣,得不全乎?」又难云:「师旷吹律,知南风不竞,楚多死声。此又吾之所疑也。请问师旷吹律之时,楚国之风邪,则相去千里,声不足达;若正识楚风来入律中邪,则楚南有吴、越,北有梁、宋,苟不见其原,奚以识之哉?凡阴阳愤激,然后成风。气之相感,触地而发,何得发楚庭,来入晋乎?且又律吕分四时之气耳,时至而气动,律应而灰移,皆自然相待,不假人以为用也。上生下生,所以均五声之和,叙刚柔之分也。然律有一定之声,虽冬吹中吕,其音自满而无损也。今以晋人之气,吹无韵之律,楚风安得来入其中,与为盈缩邪?风无形,声与律不通,则校理之地,无取于风律,不其然乎?岂独师旷多识博物,自有以知胜败之形,欲固众心而托以神微,若伯常骞之许景公寿哉?」又难云:「羊舌母听闻儿啼而审其丧家。复请问何由知之?为神心独悟暗语而当邪?尝闻儿啼若此其大而恶,今之啼声似昔之啼声,故知其丧家邪?若神心独悟暗语之当,非理之所得也。虽曰听啼,无取验于儿声矣。若以尝闻之声为恶,故知今啼当恶,此为以甲声为度,以校乙之啼也。夫声之于音,犹形之于心也。有形同而情乖,貌殊而心均者。何以明之?圣人齐心等德而形状不同也。苟心同而形异,则何言乎观形而知心哉?且口之激气为声,何异于籁?纳气而鸣邪?啼声之善恶,不由儿口吉凶,犹琴瑟之清浊不在操者之工拙也。心能辨理善谈,而不能令内?调利,犹瞽者能善其曲度,而不能令器必清和也。器不假妙瞽而良,?不因惠心而调,然则心之与声,明为二物。二物之诚然,则求情者不留观于形貌,揆心者不借听于声音也。察者欲因声以知心,不亦外乎?今晋母未待之于老成,而专信昨日之声,以证今日之啼,岂不误中于前世好奇者从而称之哉?」


秦客难曰:「吾闻败者不羞走,所以全也。吾心未厌而言,难复更从其馀。今平和之人,听筝笛琵琶,则形躁而志越;闻琴瑟之音,则听静而心闲。同一器之中,曲用每殊,则情随之变:奏秦声则叹羡而慷慨;理齐楚则情一而思专,肆姣弄则欢放而欲惬;心为声变,若此其众。苟躁静由声,则何为限其哀乐,而但云至和之声,无所不感,托大同于声音,归众变于人情?得无知彼不明此哉?」


主人答曰:「难云:琵琶、筝、笛令人躁越。又云:曲用每殊而情随之变。此诚所以使人常感也。琵琶、筝、笛,间促而声高,变众而节数,以高声御数节,故使人形躁而志越。犹铃铎警耳,锺鼓骇心,故『闻鼓鼙之音,思将帅之臣』,盖以声音有大小,故动人有猛静也。琴瑟之体,间辽而音埤,变希而声清,以埤音御希变,不虚心静听,则不尽清和之极,是以听静而心闲也。夫曲用不同,亦犹殊器之音耳。齐楚之曲,多重故情一,变妙故思专。姣弄之音,挹众声之美,会五音之和,其体赡而用博,故心侈于众理;五音会,故欢放而欲惬。然皆以单、复、高、埤、善、恶为体,而人情以躁、静而容端,此为声音之体,尽于舒疾。情之应声,亦止于躁静耳。夫曲用每殊,而情之处变,犹滋味异美,而口辄识之也。五味万殊,而大同于美;曲变虽众,亦大同于和。美有甘,和有乐。然随曲之情,尽于和域;应美之口,绝于甘境,安得哀乐于其间哉?然人情不同,各师所解。则发其所怀;若言平和,哀乐正等,则无所先发,故终得躁静。若有所发,则是有主于内,不为平和也。以此言之,躁静者,声之功也;哀乐者,情之主也。不可见声有躁静之应,因谓哀乐者皆由声音也。且声音虽有猛静,猛静各有一和,和之所感,莫不自发。何以明之?夫会宾盈堂,酒酣奏琴,或忻然而欢,或惨尔泣,非进哀于彼,导乐于此也。其音无变于昔,而欢戚并用,斯非『吹万不同』邪?夫唯无主于喜怒,亦应无主于哀乐,故欢戚俱见。若资偏固之音,含一致之声,其所发明,各当其分,则焉能兼御群理,总发众情邪?由是言之,声音以平和为体,而感物无常;心志以所俟为主,应感而发。然则声之与心,殊涂异轨,不相经纬,焉得染太和于欢戚,缀虚名于哀乐哉?秦客难曰:「论云:猛静之音,各有一和,和之所感,莫不自发,是以酒酣奏琴而欢戚并用。此言偏并之情先积于内,故怀欢者值哀音而发,内戚者遇乐声而感也。夫音声自当有一定之哀乐,但声化迟缓不可仓卒,不能对易。偏重之情,触物而作,故今哀乐同时而应耳;虽二情俱见,则何损于声音有定理邪?主人答曰:「难云:哀乐自有定声,但偏重之情,不可卒移。故怀戚者遇乐声而哀耳。即如所言,声有定分,假使《鹿鸣》重奏,是乐声也。而令戚者遇之,虽声化迟缓,但当不能使变令欢耳,何得更以哀邪?犹一爝之火,虽未能温一室,不宜复增其寒矣。夫火非隆寒之物,乐非增哀之具也。理弦高堂而欢戚并用者,直至和之发滞导情,故令外物所感得自尽耳。难云:偏重之情,触物而作,故令哀乐同时而应耳。夫言哀者,或见机杖而泣,或睹舆服而悲,徒以感人亡而物存,痛事显而形潜,其所以会之,皆自有由,不为触地而生哀,当席而泪出也。今见机杖以致感,听和声而流涕者,斯非和之所感,莫不自发也。」


秦客难曰:「论云:酒酣奏琴而欢戚并用。欲通此言,故答以偏情感物而发耳。今且隐心而言,明之以成效。夫人心不欢则戚,不戚则欢,此情志之大域也。然泣是戚之伤,笑是欢之用。盖闻齐、楚之曲者,唯睹其哀涕之容,而未曾见笑噱之貌。此必齐、楚之曲,以哀为体,故其所感,皆应其度量;岂徒以多重而少变,则致情一而思专邪?若诚能致泣,则声音之有哀乐,断可知矣。」


主人答曰:「虽人情感于哀乐,哀乐各有多少。又哀乐之极,不必同致也。夫小哀容坏,甚悲而泣,哀之方也;小欢颜悦,至乐心喻,乐之理也。何以明之?夫至亲安豫,则恬若自然,所自得也。及在危急,仅然后济,则?不及亻舞。由此言之,亻舞之不若向之自得,岂不然哉?,至夫笑噱虽出于欢情,然自以理成又非自然应声之具也。此为乐之应声,以自得为主;哀之应感,以垂涕为故。垂涕则形动而可觉,自得则神合而无忧,是以观其异而不识其同,别其外而未察其内耳。然笑噱之不显于声音,岂独齐楚之曲邪?今不求乐于自得之域,而以无笑噱谓齐、楚体哀,岂不知哀而不识乐乎?」


秦客问曰:「仲尼有言:『移风易俗,莫善于乐。』即如所论,凡百哀乐,皆不在声,即移风易俗,果以何物邪?又古人慎靡靡之风,抑忄舀耳之声,故曰:『放郑声,远佞人。』然则郑卫之音击鸣球以协神人,敢问郑雅之体,隆弊所极;风俗称易,奚由而济?幸重闻之,以悟所疑。」


主人应之曰:「夫言移风易俗者,必承衰弊之後也。古之王者,承天理物,必崇简易之教,御无为之治,君静于上,臣顺于下,玄化潜通,天人交泰,枯槁之类,浸育灵液,六合之内,沐浴鸿流,荡涤尘垢,群生安逸,自求多福,默然从道,怀忠抱义,而不觉其所以然也。和心足于内,和气见于外,故歌以叙志,亻舞以宣情。然后文之以采章,照之以风雅,播之以八音,感之以太和,导其神气,养而就之。迎其情性,致而明之,使心与理相顺,气与声相应,合乎会通,以济其美。故凯乐之情,见于金石,含弘光大,显于音声也。若以往则万国同风,芳荣济茂,馥如秋兰,不期而信,不谋而诚,穆然相爱,犹舒锦彩,而粲炳可观也。大道之隆,莫盛于兹,太平之业,莫显于此。故曰「『移风易俗,莫善于乐。』乐之为体,以心为主。故无声之乐,民之父母也。至八音会谐,人之所悦,亦总谓之乐,然风俗移易,不在此也。夫音声和比,人情所不能已者也。是以古人知情之不可放,故抑其所遁;知欲之不可绝,故因其所自。为可奉之礼,制可导之乐。口不尽味,乐不极音。揆终始之宜,度贤愚之中。为之检则,使远近同风,用而不竭,亦所以结忠信,著不迁也。故乡校庠塾亦随之变,丝竹与俎豆并存,羽毛与揖让俱用,正言与和声同发。使将听是声也,必闻此言;将观是容也,必崇此礼。礼犹宾主升降,然后酬酢行焉。于是言语之节,声音之度,揖让之仪,动止之数,进退相须,共为一体。君臣用之于朝,庶士用之于家,少而习之,长而不怠,心安志固,从善日迁,然后临之以敬,持之以久而不变,然后化成,此又先王用乐之意也。故朝宴聘享,嘉乐必存。是以国史采风俗之盛衰,寄之乐工,宣之管弦,使言之者无罪,闻之者足以自诫。此又先王用乐之意也。若夫郑声,是音声之至妙。妙音感人,犹美色惑志。耽?荒酒,易以丧业,自非至人,孰能御之?先王恐天下流而不反,故具其八音,不渎其声;绝其大和,不穷其变;捐窈窕之声,使乐而不淫,犹大羹不和,不极勺药之味也。若流俗浅近,则声不足悦,又非所欢也。若上失其道,国丧其纪,男女奔随,淫荒无度,则风以此变,俗以好成。尚其所志,则群能肆之,乐其所习,则何以诛之?托于和声,配而长之,诚动于言,心感于和,风俗一成,因而名之。然所名之声,无中于淫邪也。淫之与正同乎心,雅、郑之体,亦足以观矣。」



\chapter*{乞猫}
\addcontentsline{toc}{chapter}{乞猫}
\begin{center}
	\textbf{[明朝]刘基}
\end{center}


赵人患鼠,乞猫于中山。中山人予之猫,猫善捕鼠及鸡。月余,鼠尽而鸡亦尽。其子患之,告其父曰:“盍去诸?”其父曰:“是非若所知也。吾之患在鼠,不在乎无鸡。夫有鼠,则窃吾食,毁吾衣,穿吾垣墉,毁伤吾器用,吾将饥寒焉,不病于无鸡乎?无鸡者,弗食鸡则已耳,去饥寒犹远,若之何而去夫猫也!”


(选自明·刘基《郁离子·捕鼠》)



\chapter*{若石之死}
\addcontentsline{toc}{chapter}{若石之死}
\begin{center}
	\textbf{[明朝]刘基}
\end{center}


若石居冥山之阴,有虎恒窥其藩。若石帅家人昼夜警:日出而殷钲,日入而举辉,筑墙掘坎以守。卒岁虎不能有获。一日,虎死,若石大喜,自以为虎死无毒己者矣。于是弛其惫,墙坏而不葺。无何,有貙闻其牛羊豕之声而入食焉。若石不知其为貙也,斥之不走。貙人立而爪之毙。人曰:若石知其一而不知其二,其死也宜。

\chapter*{秦西巴纵麑}
\addcontentsline{toc}{chapter}{秦西巴纵麑}
\begin{center}
	\textbf{[秦朝]吕不韦}
\end{center}


孟孙猎而得麑,使秦西巴持归烹之。麑母随之而啼,秦西巴弗忍,纵而与之。孟孙归,求麑安在。秦西巴对曰:“其母随而啼,臣诚弗忍,窃纵而予之。”孟孙怒,逐秦西巴。居一年,取以为子傅。左右曰:“秦西巴有罪于君,今以为子傅,何也?”孟孙曰:“夫一麑不忍,又何况于人乎?”

\chapter*{季梁谏追楚师}
\addcontentsline{toc}{chapter}{季梁谏追楚师}
\begin{center}
	\textbf{[春秋战国]左丘明}
\end{center}


楚武王侵随,使薳章求成焉,军于瑕以待之。随人使少师董成。


斗伯比言于楚子曰:“吾不得志于汉东也,我则使然。我张吾三军而被吾甲兵,以武临之,彼则惧而协以谋我,故难间也。汉东之国,随为大。随张,必弃小国。小国离,楚之利也。少师侈,请羸师以张之。”熊率且比曰:“季梁在,何益?”斗伯比曰:“以为后图。少师得其君。”


王毁军而纳少师。少师归,请追楚师。随侯将许之。


季梁止之曰:“天方授楚。楚之羸,其诱我也,君何急焉?臣闻小之能敌大也,小道大淫。所谓道,忠于民而信于神也。上思利民,忠也;祝史正辞,信也。今民馁而君逞欲,祝史矫举以祭,臣不知其可也。”公曰:“吾牲牷肥腯,粢盛丰备,何则不信?”对曰:“夫民,神之主也。是以圣王先成民,而后致力于神。故奉牲以告曰‘博硕肥腯。’谓民力之普存也,谓其畜之硕大蕃滋也,谓其不疾瘯蠡也,谓其备腯咸有也。奉盛以告曰:‘洁粢丰盛。’谓其三时不害而民和年丰也。奉酒醴以告曰:‘嘉栗旨酒。’谓其上下皆有嘉德而无违心也。所谓馨香,无谗慝也。故务其三时,修其五教,亲其九族,以致其禋祀。于是乎民和而神降之福,故动则有成。今民各有心,而鬼神乏主,君虽独丰,其何福之有?君姑修政而亲兄弟之国,庶免于难。”


随侯惧而修政,楚不敢伐。



\chapter*{子产却楚逆女以兵}
\addcontentsline{toc}{chapter}{子产却楚逆女以兵}
\begin{center}
	\textbf{[春秋战国]左丘明}
\end{center}


楚公子围聘于郑,且娶于公孙段氏。伍举为介。将入馆,郑人恶之。使行人子羽与之言,乃馆于外。


既聘,将以众逆。子产患之,使子羽辞曰:“以敝邑褊小,不足以容从者,请墠听命!”令尹使太宰伯州犁对曰:“君辱贶寡大夫围,谓围:‘将使丰氏抚有而室。’围布几筵,告于庄、共之庙而来。若野赐之,是委君贶于草莽也!是寡大夫不得列于诸卿也!不宁唯是,又使围蒙其先君,将不得为寡君老,其蔑以复矣。唯大夫图之!”子羽曰:“小国无罪,恃实其罪。将恃大国之安靖己,而无乃包藏祸心以图之。小国失恃而惩诸侯,使莫不憾者,距违君命,而有所壅塞不行是惧!不然,敝邑,馆人之属也,其敢爱丰氏之祧?”


伍举知其有备也,请垂櫜而入。许之。



\chapter*{申胥谏许越成}
\addcontentsline{toc}{chapter}{申胥谏许越成}
\begin{center}
	\textbf{[春秋战国]左丘明}
\end{center}


吴王夫差乃告诸大夫曰:“孤将有大志于齐,吾将许越成,而无拂吾虑。若越既改,吾又何求?若其不改,反行,吾振旅焉。”申胥谏曰:“不可许也。夫越非实忠心好吴也,又非慑畏吾甲兵之强也。大夫种勇而善谋,将还玩吴国于股掌之上,以得其志。夫固知君王之盖威以好胜也,故婉约其辞,以从逸王志,使淫乐于诸夏之国,以自伤也。使吾甲兵钝弊,民人离落,而日以憔悴,然后安受吾烬。夫越王好信以爱民,四方归之,年谷时熟,日长炎炎,及吾犹可以战也。为虺弗摧,为蛇将若何?”吴王曰:“大夫奚隆于越?越曾足以为大虞乎?若无越,则吾何以春秋曜吾军士?”乃许之成。


将盟,越王又使诸稽郢辞曰:“以盟为有益乎?前盟口血未乾,足以结信矣。以盟为无益乎?君王舍甲兵之威以临使之,而胡重于鬼神而自轻也。”吴王乃许之,荒成不盟。



\chapter*{曾国藩诫子书}
\addcontentsline{toc}{chapter}{曾国藩诫子书}
\begin{center}
	\textbf{[清朝]曾国藩}
\end{center}


余通籍三十余年,官至极品,而学业一无所成,德行一无许可,老大徒伤,不胜悚惶惭赧。今将永别,特将四条教汝兄弟。


一曰慎独而心安。自修之道,莫难于养心;养心之难,又在慎独。能慎独,册内省不疚,可以对天地质鬼神。人无一内愧之事,则天君泰然。此心常快足宽平,是人生第一自强之道,第一寻乐之方,守身之先务也。


二曰主敬则身强。内而专静纯一,外而整齐严肃。敬之工夫也;出门如见大宾,使民如承大祭,敬之气象也;修己以安百姓,笃恭而天下平,敬之效验也。聪明睿智,皆由此出。庄敬日强,安肆日偷。若人无众寡,事无大小,一一恭敬,不敢怠慢。则身强之强健,又何疑乎?


三曰求仁则人悦。凡人之生,皆得天地之理以成性,得天地之气以成形,我与民物,其大本乃同出一源。若但知私己而不知仁民爱物,是于大本一源之道已悖而失之矣。至于尊官厚禄,高居人上,则有拯民溺救民饥之责。读书学古,粗知大义,既有觉后知觉后觉之责。孔门教人,莫大于求仁,而其最切者,莫要于欲立立人、欲达达人数语。立人达人之人,人有不悦而归之者乎?


四曰习劳则神钦。人一日所着之衣所进之食,与日所行之事所用之力相称,则旁人韪之,鬼神许之,以为彼自食其力也。若农夫织妇终岁勤动,以成数石之粟数尺之布,而富贵之家终岁逸乐,不营一业,而食必珍馐,衣必锦绣,酣豢高眠,一呼百诺,此天下最不平之事,神鬼所不许也,其能久乎?古之圣君贤相,盖无时不以勤劳自励。为一身计,则必操习技艺,磨练筋骨,困知勉行,操心危虑,而后可以增智慧而长见识。为天下计,则必已饥已溺,一夫不获,引为余辜。大禹、墨子皆极俭以奉身而极勤以救民。勤则寿,逸则夭,勤则有材而见用,逸则无劳而见弃,勤则博济斯民而神祇钦仰,逸则无补于人而神鬼不歆。


此四条为余数十年人世之得,汝兄弟记之行之,并传之于子子孙孙,则余曾家可长盛不衰,代有人才。



\chapter*{李遥买杖}
\addcontentsline{toc}{chapter}{李遥买杖}
\begin{center}
	\textbf{[宋朝]沈括}
\end{center}


随州大洪山作人李遥,杀人亡命。逾年,至秭归,因出市,见鬻柱杖者,等闲以数十钱买之。是时,秭归适又有邑民为人所杀,求贼甚急。民之子见遥所操杖,识之,曰:“此吾父杖也。”遂以告官司。吏执遥验之,果邑民之杖也。榜掠备至。遥实买杖,而鬻杖者已不见,卒未有以自明。有司诘其行止来历,势不可隐,乃通随州,而大洪杀人之罪遂败。市人千万而遥适值之,因缘及其隐匿,此亦事之可怪者。

\chapter*{古人铸鉴}
\addcontentsline{toc}{chapter}{古人铸鉴}
\begin{center}
	\textbf{[宋朝]沈括}
\end{center}


此工之巧智,后人不能造。比得古鉴,皆刮磨令平,此师旷所以伤知音也。

世有透光鉴,鉴背有铭文,凡二十字,字极古,莫能读。以鉴承日光,则背文及二十字皆透,在屋壁上了了分明。人有原其理,以谓铸时薄处先冷,唯背文上差厚后冷,而铜缩多。文虽在背,而鉴面隐然有迹,所以于光中现。予观之,理诚如是。然余家有三鉴,又见他家所藏,皆是一样,文画铭字无纤异者,形制甚古。唯此鉴光透,其他鉴虽至薄者,皆莫能透。意古人别自有术。


选自沈括(宋)——《梦溪笔谈》



\chapter*{梁鸿尚节}
\addcontentsline{toc}{chapter}{梁鸿尚节}
\begin{center}
	\textbf{[南北朝]范晔}
\end{center}


(梁鸿)家贫而尚节,博览无不通。而不为章句。学毕,乃牧豕于上林苑中,曾误遗火,延及他舍。鸿乃寻访烧者,问所去失,悉以豕偿之。其主犹以为少。鸿曰:“无他财,愿以身居作。”主人许之。因为执勤,不懈朝夕。邻家耆老见鸿非恒人,乃共责让主人,而称鸿长者。于是,始敬鸿,悉还其豕。鸿不受而去。


(选自南宋·范晔《后汉书.卷八十三》



\chapter*{鲁恭治中牟}
\addcontentsline{toc}{chapter}{鲁恭治中牟}
\begin{center}
	\textbf{[南北朝]范晔}
\end{center}


鲁恭为中牢令,重德化,不任刑罚。袁安闻之,疑其不实,阴使人往视之。随恭行阡陌,俱坐桑下。有雉过,止其旁,旁有儿童。其人曰:“儿何不捕之?”儿言雉方雏,不得捕。其人讶而起,与恭决曰:“所以来者,欲察君之政绩也。今蝗不犯境,此一异也;爱及鸟兽,此二异也;童有仁心,此三异也。久留徒扰贤者耳,吾将速反,以状白安。”

\chapter*{守边劝农疏}
\addcontentsline{toc}{chapter}{守边劝农疏}
\begin{center}
	\textbf{[汉朝]晁错}
\end{center}

臣闻秦时北攻胡貉,筑塞河上,南攻杨粤,置戍卒焉。其起兵而攻胡、粤者,非以卫边地而救民死也,贪戾而欲广大也,故功未立而天下乱。且夫起兵而不知其势,战则为人禽,屯则卒积死。夫胡貉之地,积阴之处也,木皮三寸,冰厚六尺,食肉而饮酪,其人密理,鸟兽毳毛,其性能寒。杨粤之地少阴多阳,其人疏理,鸟兽希毛,其性能暑。秦之戍卒不能其水土,戍者死于边,输者偾于道。秦民见行,如往弃市,因以谪发之,名曰「谪戍」。先发吏有谪及赘婿、贾人,后以尝有市籍者,又后以大父母、父母尝有市籍者,后入闾,取其左。发之不顺,行者深恐,有背畔之心。凡民守战至死而不降北者,以计为之也。故战胜守固则有拜爵之赏,攻城屠邑则得其财卤以富家室,故能使其众蒙矢石,赴汤火,视死如生。今秦之发卒也,有万死之害,而亡铢两之报,死事之后不得一算之复,天下明知祸烈及已也。陈胜行戍,至于大泽,为天下先倡,天下从之如流水者,秦以威劫而行之之敝也。

胡人衣食之业不著于地,其势易以扰乱边境。何以明之?胡人食肉饮酪,衣皮毛,非有城郭田宅之归居,如飞鸟走兽于广野,美草甘水则止,草尽水竭则移。以是观之,往来转徙,时至时去,此胡人之生业,而中国之所以离南亩也。今使胡人数处转牧行猎于塞下,或当燕、代,或当上郡、北地、陇西,以候备塞之卒,卒少则入。陛下不救,则边民绝望而有降敌之心;救之,少发则不足,多发,远县才至,则胡又已去。聚而不罢,为费甚大;罢之,则胡复入。如此连年,则中国贫苦而民不安矣。

陛下幸忧边境,遣将吏发卒以治塞,甚大惠也。然令远方之卒守塞,一岁而更,不知胡人之能,不如选常居者,家室田作,且以备之。以便为之高城深堑,具蔺石,布渠答,复为一城其内,城间百五十岁。要害之处,通川之道,调立城邑,毋下千家,为中周虎落。先为室屋,具田器,乃募罪人及免徒复作令居之;不足,募以丁奴婢赎罪及输奴婢欲以拜爵者;不足,乃募民之欲往者。皆赐高爵,复其家。予冬夏衣,廪食,能自给而止。郡县之民得买其爵,以自增至卿。其亡夫若妻者,县官买与之。人情非有匹敌,不能久安其处。塞下之民,禄利不厚,不可使久居危难之地。胡人入驱而能止其所驱者,以其半予之,县官为赎其民。如是,则邑里相救助,赴胡不避死。非以德上也,欲全亲戚而利其财也。此与东方之戍卒不习地势而心畏胡者,功相万也。以陛下之时,徙民实边,使远方亡屯戍之事,塞下之民父子相保,亡系虏之患,利施后世,名称圣明,其与秦之行怨民,相去远矣。


\chapter*{述行赋}
\addcontentsline{toc}{chapter}{述行赋}
\begin{center}
	\textbf{[汉朝]蔡邕}
\end{center}

延熹二年秋,霖雨逾月。是时梁翼新诛,而徐璜、左悺等五侯擅贵于其处。又起显阳苑于城西,人徒冻饿,不得其命者甚众。白马令李云以直言死,鸿胪陈君以救云抵罪。璜以余能鼓琴,白朝廷,敕陈留太守发遣余。到偃师,病比前,得归。心愤此事,遂託所过,述而成赋。

余有行于京洛兮,遘淫雨之经时。塗邅其蹇连兮,潦汙滞而为灾。乘马蹯而不进兮,心郁悒而愤思。聊弘虑以存古兮,宣幽情而属词。

夕宿余于大梁兮,诮无忌之称神。哀晋鄙之无辜兮,忿朱亥之篡军。历中牟之旧城兮,憎佛肸之不臣。问甯越之裔胄兮,藐髣髴而无闻。

经圃田而瞰北境兮,悟卫康之封疆。迄管邑而增感叹兮,愠叔氏之启商。过汉祖之所隘兮,吊纪信于荥阳。

降虎牢之曲阴兮,路丘墟以盘萦。勤诸侯之远戍兮,侈申子之美城。稔涛塗之愎恶兮,陷夫人以大名。登长坂以淩高兮,陟葱山之荛陉;建抚体以立洪高兮,经万世而不倾。迴峭峻以降阻兮,小阜寥其异形。冈岑纡以连属兮,谿谷夐其杳冥。迫嵯峨以乖邪兮,廓严壑以峥嵘。攒棫朴而杂榛楛兮,被浣濯而罗生。步亹菼与台菌兮,缘层崖而结茎。行游目以南望兮,览太室之威灵。顾大河于北垠兮,瞰洛汭之始并。追刘定之攸仪兮,美伯禹之所营。悼太康之失位兮,愍五子之歌声。

寻修轨以增举兮,邈悠悠之未央。山风汩以飙涌兮,气慅慅而厉凉。云郁术而四塞兮,雨濛濛而渐唐。仆夫疲而瘁兮,我马虺隤以玄黄。格莽丘而税驾兮,阴曀曀而不阳。

哀衰周之多故兮,眺濒隈而增感。忿子带之淫逆兮,唁襄王于坛坎。悲宠嬖之为梗兮,心恻怆而怀惨。

乘舫州而湍流兮,浮清波以横厉。想宓妃之灵光兮,神幽隐以潜翳。实熊耳之泉液兮,总伊瀍与涧濑。通渠源于京城兮,引职贡乎荒裔。操吴榜其万艘兮,充王府而纳最。济西溪而容与兮,息鞏都而后逝。愍简公之失师兮,疾子朝之为害。

玄云黯以凝结兮,集零雨之溱溱。路阻败而无轨兮,塗泞溺而难遵。率陵阿以登降兮,赴偃师而释勤。壮田横之奉首兮,义二士之侠坟。淹留以候霁兮,感憂心之殷殷。并日夜而遥思兮,宵不寐以极晨。候风云之体势兮,天牢湍而无文。弥信宿而后阕兮,思逶迤以东运。见阳光之显显兮,怀少弭而有欣。

命仆夫其就驾兮,吾将往乎京邑。皇家赫而天居兮,万方徂而星集。贵宠煽以弥炽兮,佥守利而不戢。前车覆而未远兮,后乘驱而竞及。穷变巧于台榭兮,民露处而寝洷。消嘉榖于禽兽兮,下糠粃而无粒。弘宽裕于便辟兮,纠忠谏其駸急。怀伊吕而黜逐兮,道无因而获人。唐虞渺其既远兮,常俗生于积习。周道鞠为茂草兮,哀正路之日歰。

观风化之得失兮,犹纷挐其多远。无亮采以匡世兮,亦何为乎此畿?甘衡门以宁神兮,詠都人而思归。爰结蹤而迴轨兮,复邦族以自绥。

乱曰:跋涉遐路,艰以阻兮。终其永怀,窘阴雨兮。历观群都,寻前绪兮。考之旧闻,厥事举兮。登高斯赋,义有取兮。则善戒恶,岂云苟兮?翩翩独征,无俦与兮。言旋言复,我心胥兮。


\chapter*{诫兄子严敦书}
\addcontentsline{toc}{chapter}{诫兄子严敦书}
\begin{center}
	\textbf{[汉朝]马援}
\end{center}


援兄子严、敦,并喜讥议,而通轻侠客。援前在交趾,还书诫之曰:“吾欲汝曹闻人过失,如闻父母之名:耳可得闻,口不可得言也。好议论人长短,妄是非正法,此吾所大恶也:宁死,不愿闻子孙有此行也。汝曹知吾恶之甚矣,所以复言者,施衿结缡,申父母之戒,欲使汝曹不忘之耳!


“龙伯高敦厚周慎,口无择言,谦约节俭,廉公有威。吾爱之重之,愿汝曹效之。杜季良豪侠好义,忧人之忧,乐人之乐,清浊无所失。父丧致客,数郡毕至。吾爱之重之,不愿汝曹效也。效伯高不得,犹为谨敕之士,所谓‘刻鹄不成尚类鹜’者也。效季良不得,陷为天下轻薄子,所谓‘画虎不成反类狗’者也。讫今季良尚未可知,郡将下车辄切齿,州郡以为言,吾常为寒心,是以不愿子孙效也。”



\chapter*{牧童逮狼}
\addcontentsline{toc}{chapter}{牧童逮狼}
\begin{center}
	\textbf{[清朝]蒲松龄}
\end{center}


两牧童入山至狼穴,穴中有小狼二。谋分捉之,各登一树,相去数十步。少倾,大狼至,入穴失子,意甚仓皇。童于树上扭小狼蹄、耳,故令嗥。大狼闻声仰视,怒奔树下,且号且抓。其一童嗥又在彼树致小狼鸣急。狼闻声四顾,始望见之;乃舍此趋彼,号抓如前状。前树又鸣,又转奔之。口无停声,足无停趾,数十往复,奔渐迟,声渐弱;既而奄奄僵卧,久之不动。童下视之,气已绝矣。



\chapter*{红毛毡}
\addcontentsline{toc}{chapter}{红毛毡}
\begin{center}
	\textbf{[清朝]蒲松龄}
\end{center}


红毛国,旧许与中国相贸易,边帅见其众,不许登岸。红毛人固请赐一毡地足矣。帅思一毡所容无几,许之。其人置毡岸上,但容二人,拉之容四五人。且拉且登,顷刻毡大亩许,已登百人矣。短刃并发,出于不意,被掠数里而去。



\chapter*{满井游记}
\addcontentsline{toc}{chapter}{满井游记}
\begin{center}
	\textbf{[明朝]袁宏道}
\end{center}

燕地寒,花朝节后,余寒犹厉。冻风时作,作则飞沙走砾。局促一室之内,欲出不得。每冒风驰行,未百步辄返。

廿二日天稍和,偕数友出东直,至满井。高柳夹堤,土膏微润,一望空阔,若脱笼之鹄。于时冰皮始解,波色乍明,鳞浪层层,清澈见底,晶晶然如镜之新开而冷光之乍出于匣也。山峦为晴雪所洗,娟然如拭,鲜妍明媚,如倩女之靧面而髻鬟之始掠也。柳条将舒未舒,柔梢披风,麦田浅鬣寸许。游人虽未盛,泉而茗者,罍而歌者,红装而蹇者,亦时时有。风力虽尚劲,然徒步则汗出浃背。凡曝沙之鸟,呷浪之鳞,悠然自得,毛羽鳞鬣之间皆有喜气。始知郊田之外未始无春,而城居者未之知也。

夫不能以游堕事而潇然于山石草木之间者,惟此官也。而此地适与余近,余之游将自此始,恶能无纪?己亥之二月也。


\chapter*{送江陵薛侯入觐序}
\addcontentsline{toc}{chapter}{送江陵薛侯入觐序}
\begin{center}
	\textbf{[明朝]袁宏道}
\end{center}


当薛侯之初令也,珰而虎者,张甚。郡邑之良,泣而就逮。侯少年甫任事,人皆为侯危。侯笑曰:“不然。此蒙庄氏所谓养虎者也。猝饥则噬人,而猝饱必且负嵎。吾饥之使不至怒;而饱之使不至骄,政在我矣。”已而果就约。至他郡邑,暴横甚,荆则招之亦不至。


而是时适有播酋之变。部使者檄下如雨,计亩而诛,计丁而夫。耕者哭于田,驿者哭于邮。而荆之去川也迩。沮水之余,被江而下,惴惴若不能一日处。侯谕父老曰:“是釜中鱼,何能为?”戒一切勿嚣。且曰,“奈何以一小逆疲吾赤子!”诸征调皆缓其议,未几果平。


余时方使还,闻之叹曰:“今天下为大小吏者皆若此,无忧太平矣。”小民无识,见一二官吏与珰相持而击,则群然誉。故激之名张,而调之功隐。吾务其张而不顾其害,此犹借锋以割耳。自古国家之祸,造于小人,而成于贪功幸名之君子者,十常八九。故自楚、蜀造祸以来,识者之忧,有深于珰与夷者。辟如病人,冀病之速去也,而纯用攻伐之剂,其人不死于病而死于攻。今观侯之治荆,激之耶,抑调之耶?吏侯一日而秉政,其不以贪功幸名之药毒天下也审矣。


侯为人丰颐广额,一见知其巨材。今年秋以试事分校省闱,首取余友元善,次余弟宗郢。元善才识卓绝,其为文骨胜其肌,根极幽彻,非具眼如侯,未有能赏识其俊者。余弟质直温文,其文如其人,能不为师门之辱者。以此二士度一房,奚啻得五?侯可谓神于相士者也。侯之徽政,不可枚举。略述其大者如此。汉庭第治行,讵有能出侯上者?侯行矣。


呜呼。使逆珰时不为激而为调,宁至决裂乎?谁谓文人无奇识,不能烛几于先也。



\chapter*{答王阮亭}
\addcontentsline{toc}{chapter}{答王阮亭}
\begin{center}
	\textbf{[清朝]尤侗}
\end{center}

来书谓仆《清平调》一剧,为吾辈伸眉吐气,第不图肥婢竞远胜冬烘试官,摩诘出公主之门。太白以贵妃上第,乃知世间冬烘试官愧巾帼多矣,读竟太息,又复起舞。

仆谓天下试官皆妇人耳,若闺阁怜才反过试官十倍。太白赋《清平调》、《上清调》,贵妃以玻璃七宣杯酌西凉葡萄酒笑饮,敛绣巾再拜,不正天子门生真为贵妃弟子矣!

假使太白当年果中状元,不过盲宰相作试官耳,不幸出林甫、国忠②之门,耻孰甚焉?何如玉环一顾笑于朱衣万点乎?然仆甫脱稿,即有罪我为骂状元者,昔王渼陂作《杜甫游春》剧,人谓其骂宰相,今仆亦遭此语,何李白、杜甫之不幸,而林甫、力士接踵于世也。此又仆之助公太息者也。


\chapter*{画工弃市}
\addcontentsline{toc}{chapter}{画工弃市}
\begin{center}
	\textbf{[晋朝]葛洪}
\end{center}


元帝后宫既多,不得常见,乃使画工图形,案图召幸之。诸宫人皆赂画工,多者十万,少者亦不减五万。独王嫱不肯,遂不得见。匈奴入朝,求美人为阏氏。于是上案图,以昭君行。及去,召见,貌为后宫第一,善应付,举止优雅。帝悔之,而名籍已定。帝重信于外国,故不复更人。乃穷案其事,画工皆弃市,籍其家,资皆巨万。画工有杜陵毛延寿,为人形,丑好老少,必得其真;安陵陈敞、新丰刘白、龚宽,并工为牛马飞鸟众势,人形好丑,不逮延寿、下杜阳望亦善画,尤善布色,樊育亦善布色:同日弃市。京师画工于是差稀。

\chapter*{欧阳晔破案}
\addcontentsline{toc}{chapter}{欧阳晔破案}
\begin{center}
	\textbf{[明朝]冯梦龙}
\end{center}

欧阳晔治鄂州,民有争舟而相殴至死者,狱久不决。晔自临其狱,坐囚于庭中,去其桎梏而饮食之,食讫,悉劳而还之狱。独留一人于庭,留者色变而惶顾。晔曰:“杀人者汝也!”囚佯为不知所以。晔曰:“吾观食者皆以右手持箸,而汝独以左。今死者伤在右肋,非汝而谁?”囚无以对。


\chapter*{铁杵磨针}
\addcontentsline{toc}{chapter}{铁杵磨针}
\begin{center}
	\textbf{[明朝]郑之珍}
\end{center}

磨针溪,在眉州象耳山下。世传李太白读书山中,未成,弃去。过小溪,逢老媪方磨铁杵,问之,曰:“欲作针。”太白感其意,还卒业。媪自言姓武。今溪旁有武氏岩。


\chapter*{治安疏}
\addcontentsline{toc}{chapter}{治安疏}
\begin{center}
	\textbf{[明朝]海瑞}
\end{center}

户部云南清吏司主事臣海瑞谨奏;为直言天下第一事,以正君道、明臣职,求万世治安事:

君者,天下臣民万物之主也。惟其为天下臣民万物之主,责任至重。凡民生利病,一有所不宜,将有所不称其任。是故事君之道宜无不备,而以其责寄臣工,使之尽言焉。臣工尽言,而君道斯称矣。昔之务为容悦,阿谀曲从,致使灾祸隔绝、主上不闻者,无足言矣。

过为计者则又曰:“君子危明主,忧治世。”夫世则治矣,以不治忧之;主则明矣,以不明危之:无乃使之反求眩瞀,莫知趋舍矣乎!非通论也。

臣受国厚恩矣,请执有犯无隐之义,美曰美,不一毫虚美;过曰过,不一毫讳过。不为悦谀,不暇过计,谨披沥肝胆为陛下言之。

汉贾谊陈政事于文帝曰:“进言者皆曰:天下已安已治矣,臣独以为未也。曰安且治者,非愚则谀。”夫文帝,汉贤君也,贾谊非苛责备也。文帝性颇仁柔,慈恕恭俭,虽有爱民之美,优游退逊、尚多怠废之政。不究其弊所不免,概以安且治当之,愚也。不究其才所不能,概以政之安且治颂之,谀也。

陛下自视,于汉文帝何如?陛下天资英断,睿识绝人,可为尧、舜,可为禹、汤、文、武,下之如汉宣之厉精,光武之大度,唐太宗之英武无敌,宪宗之志平僭乱,宋仁宗之仁恕,举一节可取者,陛下优为之。即位初年,铲除积弊,焕然与天下更始。举其大概:箴敬一以养心,定冠履以定分,除圣贤土木之象,夺宦官内外之权,元世祖毁不与祀,祀孔子推及所生。天下忻忻,以大有作为仰之。识者谓辅相得人,太平指日可期,非虚语也,高汉文帝远甚。然文帝能充其仁恕之性,节用爱人,吕祖谦称其能尽人之才力,诚是也。一时天下虽未可尽以治安予之,然贯朽粟陈,民物康阜,三代后称贤君焉。

陛下则锐精未久,妄念牵之而去矣。反刚明而错用之,谓长生可得,而一意玄修。富有四海不曰民之脂膏在是也,而侈兴土木。二十余年不视朝,纲纪驰矣。数行推广事例,名爵滥矣。二王不相见,人以为薄于父子。以猜疑诽谤戮辱臣下,人以为薄于君臣。乐西苑而不返宫,人以为薄于夫妇。天下吏贪将弱,民不聊生,水旱靡时,盗贼滋炽。自陛下登极初年亦有这,而未甚也。今赋役增常,万方则效。陛下破产礼佛日甚,室如县罄,十余年来极矣。天下因即陛下改元之号而臆之曰:“嘉靖者言家家皆净而无财用也。”

迩者,严嵩罢相,世蕃极刑,差快人意一时称清时焉。然严嵩罢相之后,犹之严嵩未相之先而已,非大清明世界也。不及汉文帝远甚。天下之人不直陛下久矣,内外臣工之所知也。知之,不可谓愚。《诗》去:“衰职有阙,惟仲山甫补之。”今日所赖以弼棐匡救,格非而归之正,诸臣责也。夫圣人岂绝无过举哉?古者设官,亮采惠畴足矣,不必责之以谏。保氏掌谏王恶,不必设也。木绳金砺,圣贤不必言之也,乃修斋建醮,相率进香,天桃天药,相率表贺。建兴宫室,工部极力经营;取香觅宝,户部差求四出。陛下误举,诸臣误顺,无一人为陛下正言焉。都俞吁咈之风,陈善闭邪之义,邈无闻矣;谀之甚也。然愧心馁气,退有后言,以从陛下;昧没本心,以歌颂陛下,欺君之罪何如?

夫天下者,陛下之家也,人未有不顾其家者。内外臣工有官守、有言责,皆所以奠陛下之家而磐石之也。一意玄修,是陛下心之惑也。过于苛断,是陛下情之伪也。而谓陛下不顾其家,人情乎?诸臣顾身家以保一官,多以欺败,以赃败,不事事败,有不足以当陛下之心者。其不然者,君心臣心偶不相值也,遂谓陛下为贱薄臣工。诸臣正心之学微,所言或不免己私,或失详审,诚如胡寅扰乱政事之说,有不足以当陛下之心者。其不然者,君意臣意偶不相值也,遂谓陛下为是己拒谏。执陛下一二事不当之形迹,亿陛下千百事之尽然,陷陛下误终不复,诸臣欺君之罪大矣。《记》曰:“上人疑则百姓惑,下难知则君长劳。”今日之谓也。

为身家心与惧心合,臣职不明,臣以一二事形迹既为诸臣解之矣。求长生心与惑心合,有辞于臣,君道不正,臣请再为陛下开之。

陛下之误多矣,大端在修醮。修醮所以求长生也。自古圣贤止说修身立命,止说顺受其正。盖天地赋予于人而为性命者,此尽之矣。尧、舜、禹、汤、文、武之君,圣之盛也,未能久世不终。下之,亦未见方外士自汉、唐、宋存至今日。使陛下得以访其术者陶仲文,陛下以师呼之,仲文则既死矣。仲文尚不能长生,而陛下独何求之?至谓天赐仙桃药丸,怪妄尤甚。伏羲氏王天下,龙马出河,因则其文以画八卦。禹治水时,神龟负文而列其背,因而第之,以成必畴。河图洛书实有此瑞物,以泄万古不传之秘。天不爱道而显之圣人,借圣人以开示天下,犹之日月星辰之布列,而历数成焉,非虚妄也。宋真宗获天书于乾佑山,孙奭谏曰:“天何言哉?岂有书也?”桃必采而后得,药由人工捣以成者也。兹无因而至,桃药是有足而行耶?天赐之者,有手执而付之耶?陛下玄修多年矣,一无所得。至今日,左右奸人逆陛下玄修妄念,区区桃药之长生,理之所无,而玄修之无益可知矣。

陛下又将谓悬刑赏以督率臣下,分理有人,天下无不可治,而玄修无害矣乎?夫人幼而学,既无致君泽民异事之学,壮而行,亦无致君泽民殊用之心。《太甲》曰:“有言逆于汝志,必求诸道,有言逊于汝志,必求诸非道。”言顺者之未必为道也。即近事观:严嵩有一不顺陛下者乎?昔为贪窃,今为逆本。梁材守道守官,陛下以为逆者也,历任有声,官户部者以有守称之。虽近日严嵩抄没、百官有惕心焉,无用于积贿求迁,稍自洗涤。然严嵩罢相之后,犹严嵩未相之前而已。诸臣宁为严嵩之顺,不为梁材之执。今甚者贪求,未甚者挨日。见称于人者,亦廊庙山林交战热中,鹘突依违,苟举故事。洁己格物,任天下重,使社稷灵长终必赖之者,未见其人焉。得非有所牵制其心,未能纯然精白使然乎?陛下欲诸臣惟予行而莫违也,而责之以效忠;付之以翼为明听也,又欲其顺乎玄修土木之娱:是股肱耳目不为腹心卫也,而自为视听持行之用。有臣如仪、衍焉,可以成“得志与民由之”之业,无是理也。

陛下诚知玄修无益,臣之改行,民之效尤,天下之安与不安、治与不治由之,幡然悟悔,日视正朝,与宰辅、九卿、侍从、言官讲求天下利害,洗数十年君道之误,置其身于尧、舜、禹、汤、文、武之上,使其臣亦得洗数十年阿君之耻,置其身于皋陶、伊、傅之列,相为后先,明良喜起,都俞吁咈。内之宦官宫妾,外之光禄寺厨役,锦衣卫恩荫,诸衙门带俸,举凡无事而官者亦多矣。上之内仓内库,下之户、工部,光禄寺诸厂,段绢、粮料、珠定、器用、木材诸物,多而积于无用,用之非所宜用,亦多矣。诸臣必有为陛下言者。诸臣言之,陛下行之,此则在陛下一节省间而已。京师之一金,田野之百金也。一节省而国有余用,民有盖藏,不知其几也。而陛下何不为之?

官有职掌,先年职守之正、职守之全而未行之。今日职守之废、职守之苟且因循,不认真、不尽法而自以为是。敦本行以端士习,止上纳以清仕途,久任吏将以责成功,练选军士以免召募,驱缁黄游食以归四民,责府州县兼举富教使成礼俗,复屯盐本色以裕边储,均田赋丁差以苏困敝,举天下官之侵渔,将之怯懦,吏之为奸,刑之无少姑息焉。必世之仁,博厚高明悠远之业,诸臣必有陛下言者。诸臣言之,陛下行之,此则在陛下一振作间而已。一振作而诸废具举,百弊铲绝,唐、虞三代之治粲然复兴矣,而陛下何不行之?

节省之,振作之,又非有所劳于陛下也。九卿总其纲,百职分其任,抚按科道纠举肃清之于其间,陛下持大纲、稽治要而责成焉。劳于求贤,逸于任用如天运于上,而四时六气各得其序,恭己无为之道也。天地万物为一体,固有之性也。民物熙洽,熏为太和,而陛下性分中自有真乐矣。可以赞天地之化育,则可与天地参。道与天通,命由我立,而陛下性分中自有真寿矣。此理之所有者,可旋至而立有效者也。若夫服食不终之药,遥望轻举,理之所无者也。理之所无,而切切然散爵禄,竦精神,玄修求之,悬思凿想,系风捕影,终其身如斯而已矣,求之其可得乎?

夫君道不正,臣职不明,此天下第一事也。于此不言,更复何言?大臣持禄而外为谀,小臣畏罪而面为顺,陛下有不得知而改之行之者,臣每恨焉。是以昧死竭忠,惓惓为陛下言之。一反情易向之间,而天下之治与不治,民物之安与不安决焉,伏惟陛下留神,宗社幸甚,天下幸甚。臣不胜战栗恐惧之至,为此具本亲赍,谨具奏闻。


\chapter*{北人食菱}
\addcontentsline{toc}{chapter}{北人食菱}
\begin{center}
	\textbf{[明朝]江盈科}
\end{center}

北人生而不识菱者,仕于南方,席上啖菱,并壳入口。或曰:“食菱须去壳。”其人自护所短,曰:“我非不知,并壳者,欲以去热也。”问者曰:“北土亦有此物否?”答曰:“前山后山,何地不有?”

夫菱生于水而非土产,此坐强不知以为知也。


\chapter*{外科医生}
\addcontentsline{toc}{chapter}{外科医生}
\begin{center}
	\textbf{[明朝]江盈科}
\end{center}

有医者,自称善外科。一裨将阵回,中流矢,深入膜,延使治。乃持并州剪,剪去矢官,跪而请酬。裨将曰:“镞在膜内须亟治。”医曰:“此内科之事,不意并责我。”裨将曰:“呜呼,世直有如是欺诈之徒。”


\chapter*{庸医治驼}
\addcontentsline{toc}{chapter}{庸医治驼}
\begin{center}
	\textbf{[明朝]江盈科}
\end{center}

昔有医人,自媒能治背驼,曰:“如弓者、如虾者、如环者,若延吾治,可朝治而夕如矢矣。”一人信焉,使治曲驼,乃索板二片,以一置地下,卧驼者其上,又以一压焉,又践之。驼者随直,亦随死。其子欲诉诸官。医人曰:“我业治驼,但管人直,不管人死。”呜呼!今之为官,但管钱粮收,不管百姓死,何异于此医也哉!(自媒一作:自诩)


\chapter*{王翱秉公}
\addcontentsline{toc}{chapter}{王翱秉公}
\begin{center}
	\textbf{[明朝]王翱}
\end{center}

王翱一女,嫁于畿辅某官为妻。公夫人甚爱女,每迎女,婿固不遣。恚而语妻曰:“而翁长铨,迁我京职,则汝朝夕侍母;且迁我如振落叶耳,而何吝者?”女寄言于母。夫人一夕置酒,跪白公。公大怒,取案上器击伤夫人,出,驾而宿于朝房,旬乃还第。婿竟不调。


\chapter*{太上感应篇}
\addcontentsline{toc}{chapter}{太上感应篇}
\begin{center}
	\textbf{[宋朝]李昌龄}
\end{center}

太上曰:祸福无门,唯人自召。善恶之报,如影随形。是以天地有司过之神依人所犯轻重,以夺人算。算减则贫耗,多逢忧患,人皆恶之,刑祸随之,吉庆避之,恶星灾之,算尽则死。又有三台北斗神君,在人头上,录人罪恶,夺其纪算。又有三尸神,在人身中,每到庚申日,辄上诣天曹,言人罪过。月晦之日,灶神亦然。凡人有过,大则夺纪,小则夺算。其过大小,有数百事,欲求长生者,先须避之。是道则进,非道则退。不履邪径,不欺暗室。积德累功,慈心於物。忠孝友悌,正己化人,矜孤恤寡,敬老怀幼。昆虫草木,犹不可伤。宜悯人之凶,乐人之善,济人之急,救人之危。见人之得,如己之得。见人之失,如己之失。不彰人短,不炫己长。遏恶扬善,推多取少。受辱不怨,受宠若惊。施恩不求报,与人不追悔。所谓善人,人皆敬之,天道佑之,福禄随之。众邪远之,神灵卫之,所作必成,神仙可冀。

欲求天仙者,当立一千三百善,欲求地仙者,当立三百善;苟或非义而动,背理而行。以恶为能,忍作残害。阴贼良善,暗侮君亲。慢其先生,叛其所事。诳诸无识,谤诸同学。虚诬诈伪,攻讦宗亲。刚强不仁,狠戾自用。是非不当,向背乖宜。虐下取功,谄上希旨。受恩不感,念怨不休。轻蔑天民,扰乱国政。赏及非义,刑及无辜。杀人取财,倾人取位。诛降戮服,贬正排贤。凌孤逼寡,弃法受赂。以直为曲,以曲为直。入轻为重,见杀加怒。知过不改,知善不为。自罪引他,壅塞方术。讪谤贤圣,侵凌道德。射飞逐走,发蛰惊栖,填穴覆巢,伤胎破卵。愿人有失,毁人成功。危人自安,减人自益。以恶易好,以私废公。窃人之能,蔽人之善。形人之丑,讦人之私。耗人货财,离人骨肉。侵人所爱,助人为非,逞志作威,辱人求胜。败人苗稼,破人婚姻。苟富而骄,苟免无耻,认恩推过。嫁祸卖恶。沽买虚誉,包贮险心。挫人所长,护己所短。乘威迫胁,纵暴杀伤。无故剪裁,非礼烹宰。散弃五谷,劳扰众生。破人之家。取其财宝。决水放火,以害民居,紊乱规模,以败人功,损人器物,以穷人用。见他荣贵,愿他流贬。见他富有,愿他破散。见他色美,起心私之。负他货财,原他身死。干求不遂,便生咒恨。见他失便,便说他过。

见他体相不具而笑之。见他才能可称而抑之。埋蛊厌人,用药杀树。恚怒师傅,抵触父兄。强取强求,好侵好夺。掳掠致富,巧诈求迁。赏罚不平,逸乐过节。苛虐其下,恐吓於他。怨天尤人,呵风骂雨。斗合争讼,妄逐朋党。用妻妾语,违父母训。得新忘故。口是心非,贪冒於财,欺罔其上。造作恶语,谗毁平人。毁人称直,骂神称正,弃顺效逆,背亲向疏。

指天地以证鄙怀,引神明而鉴猥事。施与後悔,假借不还。分外营求,力上施设。淫欲过度,心毒貌慈。秽食馁人,左道惑众。短尺狭度,轻秤小升。以伪杂真,采取奸利。压良为贱,谩蓦愚人,贪婪无厌,咒诅求直。嗜酒悖乱,骨肉忿争。男不忠良,女不柔顺。不和其室,不敬其夫。每好矜夸,常行妒忌。

无行於妻子,失礼於舅姑,轻慢先灵,违逆上命。作为无益,怀挟外心。自咒咒他,偏憎偏爱。越井越灶,跳食跳人。损子堕胎,行多隐僻。晦腊歌舞,朔旦号怒。

对北涕唾及溺,对灶吟咏及哭。又以灶火烧香,秽柴作食。夜起裸露,八节行刑。唾流星,指虹霓。辄指三光,久视日月,春月燎猎,对北恶骂。无故杀龟打蛇,如是等罪,司命随其轻重,夺其纪算。算尽则死,死有余责,乃殃及子孙。又诸横取人财者,乃计其妻子家口以当之,渐至死丧。若不死丧,则有水火盗贼,遗亡器物,疾病口舌诸事,以当妄取之直。又枉杀人者,是易刀兵而相杀也。

取非羲之财者,譬如漏脯救饥,鸩酒止渴,非不暂饱,死亦及之。夫心起於善,善虽未为,而吉神已随之。或心起於恶,恶虽未为,而凶神已随之。其有曾行恶事,後自改悔,诸恶莫作,众善奉行。久久必获吉庆,所谓转祸为福也。故吉人语善,视善,行善。一日有三善,三年天必降之福。凶人语恶、视恶、行恶,一日有三恶,三年天必降之祸,胡不勉而行之。


\chapter*{于令仪诲人}
\addcontentsline{toc}{chapter}{于令仪诲人}
\begin{center}
	\textbf{[宋朝]王辟之}
\end{center}

曹州于令仪者,市井人也,长厚不忤物,晚年家颇丰富。一夕,盗入其室,诸子擒之,乃邻子也。令仪曰:“汝素寡悔,何苦而为盗邪?”曰:“迫于贫耳!”问其所欲,曰:“得十千足以衣食。”如其欲与之。既去,复呼之,盗大恐。谓曰:“汝贫甚,夜负十千以归,恐为人所诘。留之,至明使去。"盗大感愧,卒为良民。乡里称君为善士。君择子侄之秀者,起学室,延名儒以掖之,子、侄杰仿举进士第,今为曹南令族。


\chapter*{陈遗至孝}
\addcontentsline{toc}{chapter}{陈遗至孝}
\begin{center}
	\textbf{[南北朝]刘义庆}
\end{center}

陈遗至孝。母好食铛底焦饭,遗作郡主簿,恒装一囊,每煮食,辄贮收焦饭,归以遗母。后值孙恩掠郡,郡守袁山松即日出征。时遗已聚敛得数斗焦饭,未及归家,遂携而从军。与孙恩战,败,军人溃散,遁入山泽,无以为粮,有饥馁而死者。遗独以焦饭得活,时人以为至孝之报也。


\chapter*{棘刺雕猴}
\addcontentsline{toc}{chapter}{棘刺雕猴}
\begin{center}
	\textbf{[春秋战国]韩非}
\end{center}

燕王好微巧,卫人请以棘刺之端为母猴。燕王说之,养之以五乘之奉。王曰:“吾视观客为棘刺之母猴。”客曰:“人主欲观之,必半岁不入宫,不饮酒食肉,雨霁日出,视之晏阴之间,而棘刺之母猴乃可见也。”燕王因养卫人,不能观其母猴。郑有台下之冶者,谓燕王曰:“臣为削者也,诸微物必以削削之,而所削必大于削。今棘刺之端不容削锋,难以治棘刺之端。王试观客之削,能与不能可知也。”王曰:“善。”谓卫人曰:“客为棘刺之母猴也,何以治之?”曰:“以削。”王曰:“吾欲观见之。”客曰:“臣请之舍取之。”因逃。


\chapter*{胠箧}
\addcontentsline{toc}{chapter}{胠箧}
\begin{center}
	\textbf{[春秋战国]庄周}
\end{center}

将为胠箧、探囊、发匮之盗而为守备,则必摄缄縢、固扃鐍;此世俗之所谓知也。然而巨盗至,则负匮、揭箧、担囊而趋;唯恐缄縢扃鐍之不固也。然则乡之所谓知者,不乃为大盗积者也?

故尝试论之,世俗之所谓知者,有不为大盗积者乎?所谓圣者,有不为大盗守者乎?何以知其然邪?昔者齐国邻邑相望,鸡狗之音相闻,罔罟之所布,耒耨之所刺,方二千余里。阖四竟之内,所以立宗庙、社稷,治邑、屋、州、闾、乡、曲者,曷尝不法圣人哉?然而田成子一旦杀齐君而盗其国。所盗者岂独其国邪?并与其圣知之法而盗之。故田成子有乎盗贼之名,而身处尧舜之安,小国不敢非,大国不敢诛,专有齐国。则是不乃窃齐国,并与其圣知之法,以守其盗贼之身乎?

尝试论之,世俗之所谓至知者,有不为大盗积者乎?所谓至圣者,有不为大盗守者乎?何以知其然邪?昔者龙逢斩,比干剖,苌弘胣,子胥靡。故四子之贤而身不免乎戮。故跖之徒问于跖曰:“盗亦有道乎?”跖曰:“何适而无有道邪?”夫妄意室中之藏,圣也;入先,勇也;出后,义也;知可否,知也;分均,仁也。五者不备而能成大盗者,天下未之有也。”由是观之,善人不得圣人之道不立,跖不得圣人之道不行;天下之善人少而不善人多,则圣人之利天下也少,而害天下也多。故曰:唇竭则齿寒,鲁酒薄而邯郸围,圣人生而大盗起。掊击圣人,纵舍盗贼,而天下始治矣!

夫川竭而谷虚,丘夷而渊实。圣人已死,则大盗不起,天下平而无故矣。圣人不死,大盗不止。虽重圣人而治天下,则是重利盗跖也。为之斗斛以量之,则并与斗斛而窃之;为之权衡以称之,则并与权衡而窃之;为之符玺而信之,则并与符玺而窃之;为之仁义以矫之,则并与仁义而窃之。

何以知其然邪?彼窃钩者诛,窃国者为诸侯,诸侯之门而仁义存焉。则是非窃仁义圣知邪?故逐于大盗、揭诸侯、窃仁义并斗斛权衡符玺之利者,虽有轩冕之赏弗能劝,斧钺之威弗能禁。此重利盗跖而使不可禁者,是乃圣人之过也。故曰:“鱼不可脱于渊,国之利器不可以示人。”彼圣人者,天下之利器也,非所以明天下也。

故绝圣弃知,大盗乃止;擿玉毁珠,小盗不起;焚符破玺,而民朴鄙;掊斗折衡,而民不争;殚残天下之圣法,而民始可与论议。擢乱六律,铄绝竽瑟,塞瞽旷之耳,而天下始人含其聪矣;灭文章,散五采,胶离朱之目,而天下始人含其明矣。毁绝钩绳而弃规矩,攦工倕之指,而天下始人含其巧矣。故曰:大巧若拙。削曾史之行,钳杨墨之口,攘弃仁义,而天下之德始玄同矣。

彼人含其明,则天下不铄矣;人含其聪,则天下不累矣;人含其知,则天下不惑矣;人含其德,则天下不僻矣。彼曾、史、杨、墨、师旷、工倕、离朱、皆外立其德而以爚乱天下者也,法之所无用也。

子独不知至德之世乎?昔者容成氏、大庭氏、伯皇氏、中央氏、栗陆氏、骊畜氏、轩辕氏、赫胥氏、尊卢氏、祝融氏、伏牺氏、神农氏,当是时也,民结绳而用之,甘其食,美其服,乐其俗,安其居,邻国相望,鸡狗之音相闻,民至老死而不相往来。若此之时,则至治已。今遂至使民延颈举踵,曰:“某所有贤者,”赢粮而趣之,则内弃其亲,而外弃其主之事;足迹接乎诸侯之境,车轨结乎千里之外,则是上好知之过也。上诚好知而无道,则天下大乱矣!

何以知其然邪?夫弓、弩、毕、弋、机变之知多,则鸟乱于上矣;钩饵、罔罟、罾笱之知多,则鱼乱于水矣;削格、罗落、罝罘之知多,则兽乱于泽矣;知诈渐毒、颉滑坚白、解垢同异之变多,则俗惑于辩矣。故天下每每大乱,罪在于好知。故天下皆知求其所不知,而莫知求其所已知者;皆知非其所不善,而莫知非其所已善者,是以大乱。故上悖日月之明,下烁山川之精,中堕四时之施,惴耎之虫,肖翘之物,莫不失其性。甚矣,夫好知之乱天下也!自三代以下者是已,舍夫种种之民,而悦夫役役之佞,释夫恬淡无为,而悦夫啍啍之意,啍啍已乱天下矣!


\chapter*{宋人及楚人平}
\addcontentsline{toc}{chapter}{宋人及楚人平}
\begin{center}
	\textbf{[春秋战国]公羊高}
\end{center}

外平不书,此何以书?大其平乎己也。何大其平乎己?庄王围宋,军有七日之粮尔!尽此不胜,将去而归尔。于是使司马子反乘堙而窥宋城。宋华元亦乘堙而出见之。司马子反曰:“子之国何如?”华元曰:“惫矣!”曰:“何如?”曰:“易子而食之,析骸而炊之。”司马子反曰:“嘻!甚矣,惫!虽然,吾闻之也,围者柑马而秣之,使肥者应客。是何子之情也?”华元曰:“吾闻之:君子见人之厄则矜之,小人见人之厄则幸之。吾见子之君子也,是以告情于子也。”司马子反曰:“诺,勉之矣!吾军亦有七日之粮尔!尽此不胜,将去而归尔。”揖而去之。

反于庄王。庄王曰:“何如?”司马子反曰:“惫矣!”曰:“何如?”曰:“易子而食之,析骸而炊之。”庄王曰:“嘻!甚矣,惫!虽然,吾今取此,然后而归尔。”司马子反曰:“不可。臣已告之矣,军有七日之粮尔。”庄王怒曰:“吾使子往视之,子曷为告之?”司马子反曰:“以区区之宋,犹有不欺人之臣,可以楚而无乎?是以告之也。”庄王曰:“诺,舍而止。虽然,吾犹取此,然后归尔。”司马子反曰:“然则君请处于此,臣请归尔。”庄王曰:“子去我而归,吾孰与处于此?吾亦从子而归尔。”引师而去之。故君子大其平乎己也。此皆大夫也。其称“人”何?贬。曷为贬?平者在下也。


\chapter*{郑伯克段于鄢}
\addcontentsline{toc}{chapter}{郑伯克段于鄢}
\begin{center}
	\textbf{[春秋战国]谷梁赤}
\end{center}

克者何?能也。何能也?能杀也。何以不言杀?见段之有徒众也。

段,郑伯弟也。何以知其为弟也?杀世子、母弟目君,以其目君知其为弟也。段,弟也,而弗谓弟;公子也,而弗谓公子。贬之也。段失子弟之道矣,贱段而甚郑伯也。何甚乎郑伯?甚郑伯之处心积虑成于杀也。

于鄢,远也,犹曰取之其母之怀之云尔,甚之也。

然则为郑伯者,宜奈何?缓追,逸贼,亲亲之道也。


\chapter*{虞师晋师灭夏阳}
\addcontentsline{toc}{chapter}{虞师晋师灭夏阳}
\begin{center}
	\textbf{[春秋战国]谷梁赤}
\end{center}

非国而曰灭,重夏阳也。虞无师,其曰师,何也?以其先晋,不可以不言师也。其先晋何也?为主乎灭夏阳也。夏阳者,虞、虢之塞邑也。灭夏阳而虞、虢举矣。虞之为主乎灭夏阳何也?晋献公欲伐虢,荀息曰:“君何不以屈产之乘、垂棘之璧,而借道乎虞也?”公曰:“此晋国之宝也。如受吾币而不借吾道,则如之何?”荀息曰:“此小国之所以事大国也。彼不借吾道,必不敢受吾币。如受吾币而借吾道,则是我取之中府,而藏之外府;取之中厩,而置之外厩也。”公曰:“宫之奇存焉,必不使也。”荀息曰:“宫之奇之为人也,达心而懦,又少长于君。达心则其言略,懦则不能强谏;少长于君,则君轻之。且夫玩好在耳目之前,而患在一国之后,此中知以上乃能虑之。臣料虞君中知以下也。”公遂借道而伐虢。宫之奇谏曰:“晋国之使者,其辞卑而币重,必不便于虞。”虞公弗听,遂受其币,而借之道。宫之奇又谏曰:“语曰:‘唇亡齿寒。’其斯之谓与!”挈其妻、子以奔曹。献公亡虢,五年而后举虞。荀息牵马操璧而前曰:“璧则犹是也,而马齿加长矣。”


\chapter*{南岐人之瘿}
\addcontentsline{toc}{chapter}{南岐人之瘿}
\begin{center}
	\textbf{[明朝]刘元卿}
\end{center}

南岐在秦蜀山谷中,其水甘而不良,凡饮之者辄病瘿,故其地之民无一人无瘿者。及见外方人至,则群小妇人聚观而笑之曰:“异哉,人之颈也!焦而不吾类!”外方人曰:“尔垒然凸出于颈者,瘿病之也,不求善药去尔病,反以吾颈为焦耶?”笑者曰:“吾乡之人皆然,焉用去乎哉!”终莫知其为丑。


\chapter*{论佛骨表}
\addcontentsline{toc}{chapter}{论佛骨表}
\begin{center}
	\textbf{[唐朝]韩愈}
\end{center}

臣某言:伏以佛者,夷狄之一法耳,自后汉时流入中国,上古未尝有也。昔者黄帝在位百年,年百一十岁;少昊在位八十年,年百岁;颛顼在位七十九年,年九十八岁;帝喾在位七十年,年百五岁;帝尧在位九十八年,年百一十八岁;帝舜及禹,年皆百岁。此时天下太平,百姓安乐寿考,然而中国未有佛也。其后殷汤亦年百岁,汤孙太戊在位七十五年,武丁在位五十九年,书史不言其年寿所极,推其年数,盖亦俱不减百岁。周文王年九十七岁,武王年九十三岁,穆王在位百年。此时佛法亦未入中国,非因事佛而致然也。



\chapter*{崔山君传}
\addcontentsline{toc}{chapter}{崔山君传}
\begin{center}
	\textbf{[唐朝]韩愈}
\end{center}

谈生之为《崔山君传》,称鹤言者,岂不怪哉!然吾观于人,其能尽其性而不类于禽兽异物者希矣,将愤世嫉邪长往而不来者之所为乎?昔之圣者,其首有若牛者,其形有若蛇者,其喙有若鸟者,其貌有若蒙其者。彼皆貌似而心不同焉,可谓之非人邪?即有平肋曼肤,颜如渥丹,美而很者,貌则人,其心则禽兽,又恶可谓之人邪?然则观貌之是非,不若论其心与其行事之可否为不失也。怪神之事,孔子之徒不言,余将特取其愤世嫉邪而作之,故题之云尔。



\chapter*{词论}
\addcontentsline{toc}{chapter}{词论}
\begin{center}
	\textbf{[宋朝]李清照}
\end{center}

乐府声诗并著,最盛于唐。开元、天宝间,有李八郎者,能歌擅天下。时新及第进士开宴曲江,榜中一名士,先召李,使易服隐姓名,衣冠故敝,精神惨沮,与同之宴所。曰:“表弟愿与坐末。”众皆不顾。既酒行乐作,歌者进,时曹元谦、念奴为冠,歌罢,众皆咨嗟称赏。名士忽指李曰:“请表弟歌。”众皆哂,或有怒者。及转喉发声,歌一曲,众皆泣下。罗拜曰:此李八郎也。”自后郑、卫之声日炽,流糜之变日烦。已有《菩萨蛮》、《春光好》、《莎鸡子》、《更漏子》、《浣溪沙》、《梦江南》、《渔父》等词,不可遍举。五代干戈,四海瓜分豆剖,斯文道息。独江南李氏君臣尚文雅,故有“小楼吹彻玉笙寒”、“吹皱一池春水”之词。语虽甚奇,所谓“亡国之音哀以思”也。逮至本朝,礼乐文武大备。又涵养百余年,始有柳屯田永者,变旧声作新声,出《乐章集》,大得声称于世;虽协音律,而词语尘下。又有张子野、宋子京兄弟,沈唐、元绛、晁次膺辈继出,虽时时有妙语,而破碎何足名家!至晏元献、欧阳永叔、苏子瞻,学际天人,作为小歌词,直如酌蠡水于大海,然皆句读不葺之诗尔。又往往不协音律,何耶?盖诗文分平侧,而歌词分五音,又分五声,又分六律,又分清浊轻重。且如近世所谓《声声慢》、《雨中花》、《喜迁莺》,既押平声韵,又押入声韵;《玉楼春》本押平声韵,有押去声,又押入声。本押仄声韵,如押上声则协;如押入声,则不可歌矣。王介甫、曾子固,文章似西汉,若作一小歌词,则人必绝倒,不可读也。乃知词别是一家,知之者少。后晏叔原、贺方回、秦少游、黄鲁直出,始能知之。又晏苦无铺叙。贺苦少典重。秦即专主情致,而少故实。譬如贫家美女,虽极妍丽丰逸,而终乏富贵态。黄即尚故实而多疵病,譬如良玉有瑕,价自减半矣。



\chapter*{荀卿论}
\addcontentsline{toc}{chapter}{荀卿论}
\begin{center}
	\textbf{[宋朝]苏轼}
\end{center}

尝读《孔子世家》,观其言语文章,循循莫不有规矩,不敢放言高论,言必称先王,然后知圣人忧天下之深也。茫乎不知其畔岸,而非远也;浩乎不知其津涯,而非深也。其所言者,匹夫匹妇之所共知;而所行者,圣人有所不能尽也。呜呼!是亦足矣。使后世有能尽吾说者,虽为圣人无难,而不能者,不失为寡过而已矣。


子路之勇,子贡之辩,冉有之智,此三者,皆天下之所谓难能而可贵者也。然三子者,每不为夫子之所悦。颜渊默然不见其所能,若无以异于众人者,而夫子亟称之。且夫学圣人者,岂必其言之云尔哉?亦观其意之所向而已。夫子以为后世必有不能行其说者矣,必有窃其说而为不义者矣。是故其言平易正直,而不敢为非常可喜之论,要在于不可易也。


昔者常怪李斯事荀卿,既而焚灭其书,大变古先圣王之法,于其师之道,不啻若寇仇。及今观荀卿之书,然后知李斯之所以事秦者皆出于荀卿,而不足怪也。


荀卿者,喜为异说而不让,敢为高论而不顾者也。其言愚人之所惊,小人之所喜也。子思、孟轲,世之所谓贤人君子也。荀卿独曰:“乱天下者,子思、孟轲也。”天下之人,如此其众也;仁人义士,如此其多也。荀卿独曰:“人性恶。桀、纣,性也。尧、舜,伪也。”由是观之,意其为人必也刚愎不逊,而自许太过。彼李斯者,又特甚者耳。


今夫小人之为不善,犹必有所顾忌,是以夏、商之亡,桀、纣之残暴,而先王之法度、礼乐、刑政,犹未至于绝灭而不可考者,是桀、纣犹有所存而不敢尽废也。彼李斯者,独能奋而不顾,焚烧夫子之六经,烹灭三代之诸侯,破坏周公之井田,此亦必有所恃者矣。彼见其师历诋天下之贤人,以自是其愚,以为古先圣王皆无足法者。不知荀卿特以快一时之论,而荀卿亦不知其祸之至于此也。


其父杀人报仇,其子必且行劫。荀卿明王道,述礼乐,而李斯以其学乱天下,其高谈异论有以激之也。孔、孟之论,未尝异也,而天下卒无有及者。苟天下果无有及者,则尚安以求异为哉!



\chapter*{惜誓}
\addcontentsline{toc}{chapter}{惜誓}
\begin{center}
	\textbf{[汉朝]贾谊}
\end{center}

\begin{center}
	
	惜余年老而日衰兮,岁忽忽而不反。
	
	登苍天而高举兮,历众山而日远。
	
	观江河之纡曲兮,离四海之霑濡。
	
	攀北极而一息兮,吸沆瀣以充虚。
	
	飞朱鸟使先驱兮,驾太一之象舆。
	
	苍龙蚴虯于左骖兮,白虎骋而为右騑。
	
	建日月以为盖兮,载玉女于後车。
	
	驰骛于杳冥之中兮,休息虖昆仑之墟。
	
	乐穷极而不厌兮,愿从容虖神明。
	
	涉丹水而驰骋兮,右大夏之遗风。
	
	黄鹄之一举兮,知山川之纡曲。
	
	再举兮,睹天地之圜方。
	
	临中国之众人兮,讬回飙乎尚羊。
	
	乃至少原之野兮,赤松、王乔皆在旁。
	
	二子拥瑟而调均兮,余因称乎清商。
	
	澹然而自乐兮,吸众气而翱翔。
	
	念我长生而久仙兮,不如反余之故乡。
	
	
	黄鹄後时而寄处兮,鸱枭群而制之。
	
	神龙失水而陆居兮,为蝼蚁之所裁。
	
	夫黄鹄神龙犹如此兮,况贤者之逢乱世哉。
	
	寿冉冉而日衰兮,固儃回而不息。
	
	俗流从而不止兮,众枉聚而矫直。
	
	或偷合而苟进兮,或隐居而深藏。
	
	苦称量之不审兮,同权概而就衡。
	
	或推迻而苟容兮,或直言之谔謣。
	
	伤诚是之不察兮,并纫茅丝以为索。
	
	方世俗之幽昏兮,眩白黑之美恶。
	
	放山渊之龟玉兮,相与贵夫砾石。
	
	梅伯数谏而至醢兮,来革顺志而用国。
	
	悲仁人之尽节兮,反为小人之所贼。
	
	比干忠谏而剖心兮,箕子被发而佯狂。
	
	水背流而源竭兮,木去根而不长。
	
	非重躯以虑难兮,惜伤身之无功。
	
	
	已矣哉!
	
	独不见夫鸾凤之高翔兮,乃集大皇之野。
	
	循四极而回周兮,见盛德而後下。
	
	彼圣人之神德兮,远浊世而自藏。
	
	使麒麟可得羁而係兮,又何以异虖犬羊?
	
\end{center}


\chapter*{尚德缓刑书}
\addcontentsline{toc}{chapter}{尚德缓刑书}
\begin{center}
	\textbf{[汉朝]路温舒}
\end{center}

汉昭帝逝世,昌邑王刘贺被废黜,汉宣帝刘询刚刚登上皇位。路温舒呈上奏书,奏书说:

昭帝崩,昌邑王贺废,宣帝初即位,路温舒上书,言宜尚德缓刑。其辞曰:

“臣闻齐有无知之祸,而桓公以兴;晋有骊姬之难,而文公用伯。近世赵王不终,诸吕作乱,而孝文为太宗。由是观之,祸乱之作,将以开圣人也。故桓、文扶微兴坏,尊文、武之业,

泽加百姓,功润诸侯,虽不及三王,天下归仁焉。文帝永思至德,以承天心,崇仁义,省刑罚,通关梁,一远近,敬贤如大宾,爱民如赤子,内恕情之所安而施之于海内,是以囹圄空虚,天下太平。夫继变化之后,必有异旧之恩,此贤圣所以昭天命也。“往者,昭帝即世而无嗣,大臣忧戚,焦心合谋,皆以昌邑尊亲,援而立之。然天不授命,淫乱其心,遂以自亡。深察祸变之故,乃皇天之所以开至圣也。故大将军受命武帝,股肱汉国,披肝胆,决大计,黜亡义,立有德,辅天而行,然后宗庙以安,天下咸宁。臣闻《春秋》正即位,大一统而慎始也。陛下初登至尊,与天合符,宜改前世之失,正始受命之统,涤烦文,除民疾,存亡继绝,以应天意。

“臣闻秦有十失,其一尚存,治狱之吏是也。秦之时,羞文学,好武勇,贱仁义之士,贵治狱之吏,正言者谓之诽谤,遏过者谓之妖言,故盛服先王不用于世⒅,忠良切言皆郁于胸,誉谀之声日满于耳,虚美熏心,实祸蔽塞,此乃秦之所以亡天下也。方今天下,赖陛下恩厚,亡金革之危、饥寒之患,父子夫妻戮力安家,然太平未洽者,狱乱之也。夫狱者,天下之大命也,死者不可复生,绝者不可复属。《书》曰:“与其杀不辜,宁失不经。”今治狱吏则不然,上下相驱,以刻为明,深者获公名,平者多后患。故治狱之吏,皆欲人死,非憎人也,自安之道在人之死。是以死人之血流离于市,被刑之徒比肩而立,大辟之计岁以万数。此仁圣之所以伤也。太平之未洽,凡以此也。夫人情安则乐生,痛则思死,棰楚之下,何求而不得?做囚人不胜痛,则饰词以视之,吏治者利其然,则指道以明之,上奏畏却,则锻练而周内之;盖奏当之成,虽咎繇听之,犹以为死有余辜。何则?成练者众,文致之罪明也。是以狱吏专为深刻,残贼而亡极,媮为一切,不顾国患,此世之大贼也。故俗语曰:“画地为狱议不入;刻木为吏期不对。”此皆疾吏之风,悲痛之辞也。故天下之患,莫深于狱;败法乱正,离亲塞道,莫甚乎治狱之吏,此所谓一尚存者也。”

“臣闻乌鸢之卵不毁,而后凤凰集;诽谤之罪不诛,而后良言进。故古人有言:“山薮臧疾,川泽纳污,瑾瑜匿恶,国君含诟。”唯陛下除诽谤以招切言,开天下之口,广箴谏之路,扫亡秦之失,尊文武之德,省法制,宽刑罚,以废治狱,则太平之风可兴于世,永履和乐,与天亡极,天下幸甚。”

上善其言。


\chapter*{多歧亡羊}
\addcontentsline{toc}{chapter}{多歧亡羊}
\begin{center}
	\textbf{[春秋战国]列子}
\end{center}

杨子之邻人亡羊,既率其党,又请杨子之竖追之。杨子曰:“嘻!亡一羊何追者之众?”邻人曰:“多歧路。”既反,问:“获羊乎?”曰:“亡之矣。”曰:“奚亡之?”曰:“歧路之中又有歧焉。吾不知所之,所以反也。”杨子戚然变容,不言者移时,不笑者竟日。门人怪之,请曰:“羊贱畜,又非夫子之有,而损言笑者何哉?”杨子不答。(追者之众一作:追之者众)

心都子曰:“大道以多歧亡羊,学者以多方丧生。学非本不同,非本不一,而末异若是。唯归同反一,为亡得丧。子长先生之门,习先生之道,而不达先生之况也,哀哉!”


\chapter*{景帝令二千石修职诏}
\addcontentsline{toc}{chapter}{景帝令二千石修职诏}
\begin{center}
	\textbf{[汉朝]刘启}
\end{center}


雕文刻镂,伤农事者也;锦绣纂组,害女红者也。农事伤,则饥之本也;女红害,则寒之原也。夫饥寒并至,而能无为非者寡矣。朕亲耕,后亲桑,以奉宗庙粢盛祭服,为天下先。不受献,减太官,省繇赋,欲天下务农蚕,素有畜积,以备灾害;强毋攘弱,众毋暴寡,老耆以寿终,幼孤得遂长。今岁或不登,民食颇寡,其咎安在?或诈伪为吏,吏以货赂为市,渔夺百姓,侵牟万民。县丞,长吏也,奸法与盗盗,甚无谓也!其令二千石修其职!不事官职耗乱者,丞相以闻,请其罪。布告天下,使明知朕意!

\chapter*{西铭}
\addcontentsline{toc}{chapter}{西铭}
\begin{center}
	\textbf{[晋朝]张载}
\end{center}


乾称父,坤称母;予兹藐焉,乃混然中处。故天地之塞,吾其体;天地之帅,吾其性。民,吾同胞;物,吾与也。


大君者,吾父母宗子;其大臣,宗子之家相也。尊高年,所以长其长;慈孤弱,所以幼其幼;圣,其合德;贤,其秀也。凡天下疲癃、残疾、惸独、鳏寡,皆吾兄弟之颠连而无告者也。


于时保之,子之翼也;乐且不忧,纯乎孝者也。违曰悖德,害仁曰贼,济恶者不才,其践形,惟肖者也。


知化则善述其事,穷神则善继其志。不愧屋漏为无忝,存心养性为匪懈。恶旨酒,崇伯子之顾养;育英才,颍封人之锡类。不弛劳而厎豫,舜其功也;无所逃而待烹,申生其恭也。体其受而归全者,参乎!勇于从而顺令者,伯奇也。


富贵福泽,将厚吾之生也;贫贱忧戚,庸玉汝于成也。存,吾顺事;没,吾宁也。



\chapter*{红桥游记}
\addcontentsline{toc}{chapter}{红桥游记}
\begin{center}
	\textbf{[清朝]王士祯}
\end{center}


出镇淮门,循小秦淮折而北,陂岸起伏多态,竹木蓊郁,清流映带。人家多因水为园亭树石,溪塘幽窃而明瑟,颇尽四时之美。拿小艇,循河西北行,林木尽处,有桥宛然,如垂虹下饮于涧;又如丽人靓妆袨服,流照明镜中,所谓红桥也。


游人登平山堂,率至法海寺,舍舟而陆径,必出红桥下。桥四面触皆人家荷塘。六七月间,菡萏作花,香闻数里,青帘白舫,络绎如织,良谓胜游矣。予数往来北郭,必过红桥,顾而乐之。


登桥四望,忽复徘徊感叹。当哀乐之交乘于中,往往不能自喻其故。王谢冶城之语,景晏牛山之悲,今之视昔,亦有怨耶!壬寅季夏之望,与箨庵、茶村、伯玑诸子,倚歌而和之。箨庵继成一章,予以属和。


嗟乎!丝竹陶写,何必中年;山水清音,自成佳话,予与诸子聚散不恒,良会未易遘,而红桥之名,或反因诸子而得传于后世,增怀古凭吊者之徘徊感叹如予今日,未可知者。



\chapter*{秋水阁记}
\addcontentsline{toc}{chapter}{秋水阁记}
\begin{center}
	\textbf{[明朝]钱谦益}
\end{center}


阁于山与湖之间,山围如屏,湖绕如带,山与湖交相袭也。虞山,嶞山也。蜿蜒西属,至是则如密如防,环拱而不忍去。西湖连延数里,缭如周墙。湖之为陂为寖者,弥望如江流。山与湖之形,经斯地也,若胥变焉。阁屹起平田之中,无垣屋之蔽,无藩离之限,背负云气,胸荡烟水,阴阳晦明,开敛变怪,皆不得遁去豪末。


阁既成,主人与客,登而乐之,谋所以名其阁者。


主人复于客曰:“客亦知河伯之自多于水乎?今吾与子亦犹是也。尝试与子直前楹而望,阳山箭缺,累如重甗。吴王拜郊之台,已为黍离荆棘矣。逦迤而西,江上诸山,参错如眉黛,吴海国、康蕲国之壁垒,亦已荡为江流矣。下上千百年,英雄战争割据,杳然不可以复迹,而况于斯阁欤?又况于吾与子以眇然之躯,寄于斯阁者欤?吾与子登斯阁也,欣然骋望,举酒相属,已不免哑然自笑,而何怪于人世之还而相笑与?”


客曰:“不然。于天地之间有山与湖,于山与湖之间有斯阁,于斯阁之中有吾与子。吾与子相与晞朝阳而浴夕月,钓清流而弋高风,其视人世之区区以井蛙相跨峙而以腐鼠相吓也为何如哉?吾闻之,万物莫不然,莫不非。因其所非而非之,是以小河伯而大海若,少仲尼而轻伯夷,因其所然而然之,则夫夔蚿之相怜,鯈鱼之出游,皆动乎天机而无所待也。吾与子之相乐也,人世之相笑也,皆彼是之两行也,而又何间焉?”


主人曰:“善哉!吾不能辩也。”姑以秋水名阁,而书之以为记。崇祯四年三月初五日。



\chapter*{酒友}
\addcontentsline{toc}{chapter}{酒友}
\begin{center}
	\textbf{[清朝]蒲松龄}
\end{center}

车生者,家不中资[1],而耽饮,夜非浮三白不能寝也[2],以故床头樽常不空[3]。一夜睡醒,转侧间,似有人共卧者,意是覆裳堕耳。摸之,则茸茸有物,似猫而巨;烛之,狐也,酣醉而犬卧[4]。视其瓶,则空矣。因笑曰:“此我酒友也。”不忍惊,覆衣加臂,与之共寝。留烛以观其变,半夜,狐欠伸。生笑曰:“美哉睡乎!”启覆视之,儒冠之俊人也[5]。起拜榻前,谢不杀之恩。生曰:“我癖于曲蘖[6],而人以为痴;卿,我鲍叔也[7]。如不见疑,当为糟丘之良友[8]。”曳登塌,复寝。且言:“卿可常临,无相猜。”狐诺之。生既醒,则狐已去。乃治旨酒一盛[9],专伺狐。

抵夕,果至,促膝欢饮。狐量豪,善谐,于是恨相得晚。狐曰:“屡叨良酝[10],何以报德?”生曰:“斗酒之欢,何置齿颊[11]!”狐曰:“虽然,君贫士,杖头钱大不易。当为君少谋酒资。”明夕,来告曰:“去此东南七里,道侧有遗金,可早取之。”诘旦而往,果得二金,乃市佳肴,以佐夜饮,狐又告曰:“院后有窖藏,宜发之。”如其言,果得钱百余千。喜曰:“囊中已自有,莫漫愁沽矣[13]。”狐曰:“不然。辙中水胡可以久掬?合更谋之。”异日,谓生曰:“市上养价廉[14],此奇货可居[15]。”从之,收荞四十余石。人咸非笑之。未几,大旱,禾豆尽枯,惟荞可种;售种,息十倍[16]。由此益富,治沃田二百亩。但问狐,多种麦则麦收,多种黍则黍收,一切种植之早晚,皆取决于狐。日稔密[17],呼生妻以嫂,视子犹子焉。后生卒,狐遂不复来。 


\chapter*{蝉赋}
\addcontentsline{toc}{chapter}{蝉赋}
\begin{center}
	\textbf{[三国]曹植}
\end{center}

唯夫蝉之清素兮,潜厥类乎太阴。在盛阳之仲夏兮,始游豫乎芳林。实澹泊而寡欲兮,独怡乐而长吟。声皦皦而弥厉兮,似贞士之介心。内含和而弗食兮,与众物而无求。栖高枝而仰首兮,漱朝露之清流。隐柔桑之稠叶兮,快啁号以遁暑。苦黄雀之作害兮,患螳螂之劲斧。冀飘翔而远托兮,毒蜘蛛之网罟。欲降身而卑窜兮,惧草虫之袭予。免众难而弗获兮,遥迁集乎宫宇。依名果之茂阴兮,托修干以静处。有翩翩之狡童兮,步容与于园圃。体离朱之聪视兮,姿才捷于狝猿。条罔叶而不挽兮,树无干而不缘。翳轻躯而奋进兮,跪侧足以自闲。恐余身之惊骇兮,精曾睨而目连。持柔竿之冉冉兮,运微粘而我缠。欲翻飞而逾滞兮,知性命之长捐。委厥体于膳夫。归炎炭而就燔。秋霜纷以宵下,晨风烈其过庭。气(忄替)怛而薄躯,足攀木而失茎。吟嘶哑以沮败,状枯槁以丧(刑)[形]。乱曰:诗叹鸣蜩,声嘒嘒兮,盛阳则来,太阴逝兮。皎皎贞素,侔夷节兮。帝臣是戴,尚其洁兮。


\chapter*{题濠上斋二绝 其一}
\addcontentsline{toc}{chapter}{题濠上斋二绝 其一}
\begin{center}
	\textbf{[宋朝]傅自修}
\end{center}


焉知鱼乐我非鱼,梦里荣枯觉则无。休学痴蝇贪纸穴,小窗烘日谩跏趺。



\chapter*{益州夫子庙碑}
\addcontentsline{toc}{chapter}{益州夫子庙碑}
\begin{center}
	\textbf{[唐朝]王勃}
\end{center}


述夫帝车南指,遁七曜於中阶;华盖西临,藏五?於太甲。虽复星辰荡越,三元之轨躅可寻;雷雨沸腾,六气之经纶有序。然则抚铜浑而观变化,则万象之动不足多也;握瑶镜而临事业,则万机之凑不足大也。故知功有所服,龟龙不能谢鳞介之尊;器有所归,江汉不能窃朝宗之柄。是以朱阳登而九有照,紫泉清而万物睹。粤若皇灵草昧,风骊受河洛之图;帝象权舆,?凤锡乾坤之瑞。高辛尧舜氏没,大夏殷周氏作,达其变遂成天下之文,极其数遂定天下之象。衣冠度律。随鼎器而重光;玉帛讴歌,反宗而大备。洎乎三川失御,九服蒙尘。俎豆丧而王泽竭,钟鼓衰而颂声寝。召陵高会,诸侯轻汉水之威;践土同盟,天子窘河阳之召。三微制度,乘战道而横流;千载英华,与王风而扫地。大业不可以终丧,彝伦不可以遂绝。由是山河联兆,素王开受命之符;天地氤氲,元圣举乘时之策。兴九围之废典,振六合之颓纲。有道存焉,斯文备矣。


夫子姓孔氏,讳邱,字仲尼,鲁国邹人也。帝天乙之灵苗,宋微子之洪绪。自元禽翦夏,俘宝玉於南巢;白马朝周,载旌旗於北面。五迁神器,琮璜高列帝之荣;三命雄图,钟鼎冠承家之礼。商邱诞睿,下属於防山;泗水载灵,遥驰於汶上。礼乐由其委输,人仪所以来苏,排祸乱而构乾元,扫荒屯而树真宰,圣人之大业也。


若乃承百王之丕运,总千圣之殊姿。人灵昭有作之期,岳渎降非常之表。珠衡玉斗,徵象纬於天经;虎踞龙蹲,集风?於地纪。亦犹三阶瞰月,恒星知太紫之宫;八柱冲霄,群岭辨中黄之宅,圣人之至象也。


若乃顺时而动,用晦而明。纡圣哲於常师,混波流於下问。太阳亭午,收爝火於丹衡;沧浪浮天,控涓涔於翠渚。西周捧袂,仙公留紫气之书;东海抠衣,郯子叙青?之秩。接舆非圣,询去就於狂歌;童子何知?屈炎凉於诡问,圣人之降迹也。


若乃参神揆训,录道和倪。辱太白於中都,绊乘黄於下邑。湛无为之迹而众务同并,驰不言之化而群方取则。虽复霓旌羽旆,齐人张夹谷之威;八佾三雍,桓氏逼公宫之制。洎乎历阶而进,宣武备而斩徘优;推义而行,肃刑书而诛正卯。用能使四方知罪,争归旧好之田;三家变色,愿执陪臣之礼,圣人之成务也。


若乃乘机动用,历聘栖遑;神经幽显,志大宇宙。东西南北,推心於暴乱之朝;恭俭温良,授手於危亡之国。道之将行也命,道之将废也命。归齐去鲁,发浩叹於衰周;厄宋围陈,奏悲歌於下蔡,圣人之救时也。


若乃筐篚六艺,笙簧五典。折旋洙泗之间,探赜唐虞之际。三千弟子,攀睿化而升堂;七十门人,奉洪规而入室。从周定礼,宪章知损益之源;反鲁裁诗,雅颂得弦歌之旨。备物而存道,下学而上达。援神叙教,降赤制於南宫;运斗陈经,动元符於北洛,圣人之立教也。


若乃观象设教,法三百八十四爻四十有九;穷神知化,应万一千二百五十策五十有五。成变化而行鬼神,观阴阳而倚天地。以鼓天下之动,以定天下之疑。索众妙於重元,纂群微於太素,圣人之赞易也。


若乃灵襟不测,睿视无涯。石昭集隼之庭,土缶验贲羊之井。稽山南望,识皓骨於封禺;蠡泽东浮,考丹萍於梦渚。麟图鉴远,金编题佐汉之符;凤德钩深,玉策筮亡秦之兆,圣人之观化也。


时义远矣,能事毕矣。然後拂衣方外,脱屣人间,奠楹兴夕梦之灾,负杖起晨歌之迹。挠虹梁於大厦,物莫能宗;摧日观於鲁邱。吾将安仰?明均两曜,不能迁代谢之期;序合四时,不能革盈虚之数。适来夫子时也,适去夫子顺也。为而不有,用九五而长驱;成而勿居,抚?霓而高视,圣人之应化也。


自四教远而微言绝,十哲丧而大义乖。九师争大易之门,五传列春秋之辐;六体分於楚晋,四始派於齐韩。淹中之妙键不追,稷下之高风代起。百家腾跃,攀户牖而同归;万匹驱驰,仰陶钧而其贯。犹使丝簧金石,长悬阙里之堂;荆棘蓬蒿,不入昌平之墓,圣人之遗风也。


导扬十圣,光被六虚,乘素履而保安贞,垂黄裳而获元吉。故能贵而无位,履端於太极之初;高而无名,布政於皇王之首。千秋所不能易,百代所不能移,万乘资以兴衰,四海由其轻重。虽复质文交映,瞻礻龠祀而长存;金火递迁,奉琴书而罔绝。盖《易》曰:「观乎人文,以化成天下。」又云:「圣人以神道设教,而万物服焉。」岂古之聪明睿智神武而不杀者夫?


国家袭宇宙之淳精,据明灵之宝位。高祖武皇帝以黄旗问罪,杖金策以劳华夷;太宗文皇帝以朱翟承天,穆玉衡而正区宇。皇上宣祖宗之累洽,奉文武之重光,稽历数而坐明堂,陈礼容而谒太庙。八神齐飨,停旒太史之宫;六辩同和,驻跸华胥之野。文物隐地,声名动天,乐繁九俗,礼盛三古。冠带混并之所,书轨八;闾阎兼匝之乡,烟火四极。竭河追日,夸父力尽於楹间;越海陵山,竖亥涂穷於庑下。薰腴广被,景贶潜周。乾象著而常文清,坤灵滋而众宝用。溢金膏於紫洞,雨露均华;栖玉烛於元都,风雷顺轨。丹翠菌,藻绘轩庭;凤彩龙姿,激扬池。殊徵,不召而自至;茂祉昭彰,无幽而不洽。虽复帝臣南面,降衢室而无为;岱畎东临,陟名山而有事。灵命不可以辞也,大典不可以推也。由是六戎宵警,横紫殿而金;五校晨驱,蹴元?而喷玉。星罗海运,岳镇川氵亭。登碧单而会神祗,御元坛而礼天地。金箱玉册,益睿算於无疆;玳检银绳,著灵机於不竭。


功既成矣,道既贞矣。历先王之旧国,怀列圣之遗尘。翔赤骥而下?亭,吟翠虬而望邹鲁。泗滨休驾,杳疑汾水之阳;尼岫凝銮,暂似峒山之典。乃下诏曰:「可追赠太师。」托盐梅於异代,鼎路生光;寄舟楫於同时,泉涂改照。咸亨元年,又下诏曰:「宣尼有纵自天,体膺上哲,合两仪之简易,为亿载之师表。顾唯寝庙,义在钦崇。如闻诸州县孔子庙堂及学馆有破坏,并向来未造,生徒无肄业之所,先师阙奠祭之仪,久致飘零,深非敬本。宜令诸州县官司,速加营葺。」


成都县学庙堂者,大唐龙朔三年乡人之所建也。尔其州分化鸟,境属蹲鸱。萦锦室於中区,托铜梁於古地。玉轮斜界,神龙蟠沮泽之?;石镜遥临,宝马蹀禺山之影。天帝会昌之国,上照乾维;英灵秀出之乡,傍清地络。庠序由其纠合,缨弁所以会同。文翁之景化不渝,智士之风猷自远。於是双川旧老,攀帝奖而翘心;三蜀名儒,想成均而变色。探周规於旧宅,询汉制於新都。开基於四会之躔,授矩於三农之隙。土阶无级,就击壤於新欢;茅茨不翦,易层巢於故事。庄坛文杏,即架椽栾;夹谷幽兰,爰疏户牖。仪形莞尔,似闻沂水之歌;列侍訚如,若奉农山之对。缁帷晓辟,横绀带於西河;绛帐宵悬,聚青衿於北海。虽秋礼冬诗之化,已洽於齐人;而宣风观俗之规,实归於上宰。


银青光禄大夫谯国公讳崇义,大武皇帝之支孙,河间大王之长子。高秋九月,振玉[B206]於唐邱;宝算千龄,跃璇蚪於太渚。我国家灵命,东朝抗裘冕之尊;宗子维城,南面袭轩裳之重。析元元之允绪,拥朱虚之禄位,拜玉节於秦京,辉金章於蜀郡。元机应物,潜消水怪之灾;丹笔申冤,俯绝山精之讼。魏文侯之拥,道在而谦尊;董相国之垂帷,风行而俗易。


司马宇文公讳纯,河南洛阳人也。皇根帝绪,列五鼎於三朝;青琐丹梯,跨千寻於十纪。仲举澄清之辔,未极夷涂;士元卿相之材,先登上佐。冰壶精鉴,遥清玉垒之郊;霜镜悬明,下映金城之域。


县令柳公讳明,宇太易,河东人也。梁岳之英,长河之灵。沐?汉之精粹,荷天衢之元亨;旌旗赫奕於中古。组陆离於下叶。凤岩抽律,擢层秀於龙门;骊穴腾姿,吐荣光於贝阙。自朱丝就列,光膺令宰之荣;墨绶驰芬,高践郎官之右。仙凫旦举,影入铜章;乳翟朝飞,声含玉轸。临邛客位,自高文雅之庭;彭泽宾门,犹主壶觞之境。旷怀足以御物,长策足以服人。重泉之惠训大行,单父之讴谣遂远。犹为夏弦春诵,俗化之枢机。西序东胶,政刑之根本。上朝宪,下奉藩维。爰搜复庙之仪,载阐重阎之制。三门四表,焕矣惟新;上哲师宗,肃焉如在。将使圆冠方领,再行邹鲁之风;锐气英声,一变ク渝之俗。於是侍郎幽思,ゼ凤藻於环林;丞相高材,排龙姿於璧沼。遗荣处士,开帘诠孝悌之机;颂德贤臣,持节听中和之乐。其为政也可久,其为志也可大。方当变化台极,仪刑万宇,岂徒偃仰听事,风教一同而已哉?


勃幼乏逸才,少有奇志。虚舟独泛,乘学海之波澜;直辔高驱,践词场之阃阈。观质文之否泰众矣,考圣贤之去就多矣。自生人以来,未有如夫子者也。嗟乎!今古代绝,江湖路远。恨不亲承妙旨,摄齐於游夏之间;躬奉德音,攘袂於天人之际。抚声名而永悼,瞻栋宇而长怀。呜呼哀哉!敢为铭曰:

五帝既没,三王不归。天地震动,阴阳乱飞。山崩海竭,月缺星围。礼乐无主,宗遂微。(其一)

大哉神圣,与时回薄。应运而生,继天而作。龙跃浩荡,鹏飞寥廓。奄有人宗,遂荒天爵。(其二)

尼山降彩,泗滨腾气。志匡六合,神经万类。夹谷登庸,中都历试。睿情贯一,元猷绝四。(其三)

栖遑教迹,寂寞河图。违齐出宋,历楚辞吴。风衰俗坏,礼去朝芜。麟书已卷,凤德终孤。(其四)

杳杳灵命,茫茫天秩。吾道难行,斯文易失。式宣六艺,裁成四术。虚往实归,外堂内室。(其五)

邈矣能仁,悠哉化主。力制群辟,权倾终古。陆离彩粲,蝉联茅土。涉海轻河,登山小鲁。(其六)

皇家载造,神风四极。检玉题祥,绳金署德。聿怀圣迹,同享天则。乃眷台庭,爰升衮职。(其七)

玉津同派,金堤茂版。智士高风,文翁泽远。淳壤沃,声和俗愿。载启仁祠,遂光儒苑。(其八)

沈沈壶奥,肃肃扃除。灵仪若在,列配如初。槐新市密,杏古坛疏。楹疑置奠,壁似藏书。(其九)

泛泛寰中,悠悠天下。徇名则众,知音盖寡。Й石参琼,迷风乱雅。仲尼既没,夫何为者。(其十)



\chapter*{答庄充书}
\addcontentsline{toc}{chapter}{答庄充书}
\begin{center}
	\textbf{[唐朝]杜牧}
\end{center}

某白庄先辈足下。凡为文以意为主,以气为辅,以辞彩章句为之兵卫,未有主强盛而辅不飘逸者,兵卫不华赫而庄整者。四者高下,圆折步骤,随主所指,如鸟随凤,鱼随龙,师众随汤、武,腾天潜泉,横裂天下,无不如意。苟意不先立,止以文彩辞句,绕前捧後,是言愈多而理愈乱,如入圜圚,纷然莫知其谁,暮散而已。是以意全胜者,辞愈朴而文愈高,意不胜者,辞愈华而文愈鄙。是意能遣辞,辞不能成意,大抵为文之旨如此。

观足下所为文百馀篇,实先意气而後辞句,慕古而尚仁义者,苟为之不已,资以学问,则古作者不为难到。今以某无可取,欲命以为序,承当厚意,惕息不安。复观自古序其文者,皆後世宗师其人而为之,《诗》《书》《春秋》《左氏》以降,百家之说,皆是也。古者其身不遇於世,寄志於言,求言遇於後世也。自两汉以来,富贵者千百,自今观之,声势光明,孰若马迁相如贾谊刘向扬雄之徒。

斯人也,岂求知於当世哉!故亲见扬子云著书,欲取覆酱瓿,雄当其时亦未尝自有夸目。况今与足下并生今世,欲序足下未已之文,此固不可也。苟有志,古人不难到,勉之而已。某再拜。


\chapter*{送章起潜序}
\addcontentsline{toc}{chapter}{送章起潜序}
\begin{center}
	\textbf{[明朝]贝琼}
\end{center}

余病天下之士有其位而局于才,不能有所施;有其才而局于位,不得有所施。有其才有其位者宜也,非幸也;有其才无其位者,不幸也。无其才无其位者亦宜也,非不幸也;无其才有其位者,幸也。然君子论其才而不论其位。才浮其位,虽卑冗而与之;位浮其才,虽尊显而斥之。固异乎常人之所见已。常人知有位而已,恶计其才弗才耶?甚矣后世之不古若也。古者度才而官,位必称其才,又何议乎!后世官其所私,而才不称其位,故不得其宜,而有幸不幸存焉。而为士者,往往耻局于位而不得有所施,不耻局于才而不能有所施,何其才而黜、不才而进者多也!呜呼,其亦时之使然与?抑亦有国者不能求才以任之也?

松江儒学史华亭章起潜氏,早岁力学不倦,数游缙绅间。然不得奋于上,其亦不幸而局于位者,特于升斗禄为养。余初未之知,一日耳其议论,下上古今,心已异之。及观所为诗歌,清丽有法,能言人所不能言。惜潜之有其才而无其位,不啻冲霄之羽回翔蓬蒿之下也,余又可以位之卑而易之哉?故乐与之交,久而益笃。盖亦与其才之有过于尊显者也。异日上之人求天下之才,又可遗潜已乎?盈考而去,澄江包君叔蕴、陈君履信,御溪张君梦臣,荆溪蒋君以愚,赋诗以赠之,而求余为之序,于是乎书。


\chapter*{玉带生歌并序}
\addcontentsline{toc}{chapter}{玉带生歌并序}
\begin{center}
	\textbf{[清朝]朱彝尊}
\end{center}


玉带生,文信国所遗砚也。予见之吴下,既摹其铭而装池之,且为之歌曰:


玉带生,吾语汝:汝产自端州,汝来自横浦。幸免事降表,佥名谢道清,亦不识大都承旨赵孟俯。能令信公喜,辟汝置幕府。当年文墨宾,代汝一一数:参军谁?谢皐羽;寮佐谁?邓中甫;弟子谁?王炎午。独汝形躯短小,风貌朴古;步不能趋,口不能语:既无鹳之鹆之活眼睛,兼少犀纹彪纹好眉妩;赖有忠信存,波涛孰敢侮?是时丞相气尚豪,可怜一舟之外无尺土,共汝草檄飞书意良苦。四十四字铭厥背,爱汝心坚刚不吐。自从转战屡丧师,天之所坏不可支。惊心柴市日,慷慨且诵临终诗,疾风蓬勃扬沙时。传有十义士,表以石塔藏公尸。生也亡命何所之?或云西台上,唏发一叟涕涟洏,手击竹如意,生时亦相随。冬青成阴陵骨朽,百年踪迹人莫知。会稽张思廉,逢生赋长句。抱遗老人阁笔看,七客寮中敢(口夭)怒?吾今遇汝沧浪亭,漆匣初开紫衣露,海桑陵谷又经三百秋,以手摩挱尚如故。洗汝池上之寒泉,漂汝林端之霏雾;俾汝畏留天地间,墨花恣洒鹅毛素。

\end{document}